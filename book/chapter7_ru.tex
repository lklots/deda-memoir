\label{121-2}
%\chapter*{Симбирск}
Перенесёмся в Симбирск. В начале 1920-х годов в нашем роду произошли большие события: умер мой прадед Ефроим Берс, отец моей 

\pageimage{page_122}
\label{122-1}
бабушки, умер её муж, мой дед Абрам Величанский, муж тёти Сони — Яков Розенбаум. Все они похоронены в Симбирске. Ефраим Берс умер от глубокой старости, он прожил более ста лет. Фамилию его я назвал точно, когда моя Инна после замужества осталась на своей фамилии Рубинштейн. В разговоре с бабушкой об этом я спросил (догадался!) её девичью фамилию, четко в два слога она произнесла слово Берс. Прадед учительствовал в еврейских школах. По рассказам бабушки, он был очень скромным человеком, жил в мире своих служебных интересов. Став вдовцом, оставшуюся долгую жизнь прожил в семье своей единственной дочери. Абрам Величанский тоже был учителем.

\pageimage{page_123}
\label{123-1}
В какой-то школе он сотрудничал с Берсом, они стали друзьями, а потом и родственниками — тестем и зятем. Интересно в этой более чем столетней предыстории семьи то, что объект родства - моя будущая бабушка Фейга - с радостью восприняла суженного, но вскоре показала, кто в доме генерал: муж-учитель зарабатывал мало, появились дети, перспектива вести жизнь бедной семьи угнетала молодую энергичную женщину, и она, махнув рукой на своего книжника, проявила коммерческую смекалку, организовала в Брест-Литовске в конце 19 века свою мясную торговлю. Она сама вела дело, дом, решала судьбы детей. Вот пример стойкости её характера и воли.

\pageimage{page_124}
\label{124-1}
 Ранняя молодость её пришлась на последнее двадцатилетие царствования русского императора Александра II. Она родилась в 1861 году. Император Александр II сделал много доброго для евреев: отменил армейскую службу для совершеннолетних еврейских мальчиков - кантонистов (прочтите книгу «Берко-Кантонист», советское издание, примерно середина 1930-х годов, автора не помню); евреям-солдатам разрешил дослуживаться до офицерских чинов, разрешил евреям получать юридическое и медицинское образование, лекарям-евреям разрешил служить в армии.

\pageimage{page_125}
\label{125-1}
Он с пониманием относился к положению евреев в Польше. Увидев группы мужчин евреев в национальных религиозных одеждах, Царь высказал недоумение по поводу открытой демонстрации в условиях антисемитизма своей национальной принадлежности. Императора, освободившего крестьян от крепостного права, подготовившего другие демократические преобразования в Российской империи, убили народовольцы в марте 1881 года. К убийству Царя евреи никакого отношения не имели и власть в этом их не обвиняла, однако последовавшая естественная реакция на убийство царя перешла от приемлемых ответных мер к самым крупным погромам евреев в Российской империи.

\pageimage{page_126}
\label{126-1}
Начало периода больших погромов относится к тому же 1881-му и 1882 годам. В городе Кишинёве, в южных губерниях России, бесчинства продолжились до 1903 года, когда император Николай II возложил ответственность за погромы на губернаторов, и всё остановилось (до 1907 года). Обострение антисемитизма привело к росту эмиграции евреев в Новый Свет (Америку); брест-литовский раввин Соловейчик (из потомственного рода раввинов) поднял тему о возвращении в Эрец-Исраэль. Под влиянием погромов состоялась большая нелегальная алия в Эрец-Исраэль из Российской империи. Всё это происходило на фоне идейного разброда в обществе после отмены крепостничества.

\pageimage{page_127}
\label{127-1}
В напряжённой обстановке того времени молодая чета — Абрам и Фейге Величанские принимает решение отправить в Америку старшего сына Шулема: там, в Америке, нет угрозы погромов. (Слово 'шулем' на языке идиш, и шалом на иврите в переводе на русский язык означают мир между людьми, странами, а мир в смысле вселенная обозначен на идише словом велт). Года через полтора Шулем стал проситься обратно: не мог прижиться. Отец дрогнул, но мать категорически возражала и отправила туда второго сына Мойше (Моисея); братья преуспели в период автомобильного бума, но когда наступила в США депрессия, Шулем разорился, с ним случился инсульт. Он умер

\pageimage{page_128}
\label{128-1}
в деятельном возрасте, в самом начале 1930-х годов. Его сын Срул продолжил переписку с бабушкой, его письма читались не по одному разу. В частности, Срул сообщил, что в те же годы он закончил Нью-Йоркский медицинский институт и прислал своё фото (помнится, урология или гинекология). Тогда я не знал значения этих слов, но двор расширил скупые объяснения взрослых. (Имя Срул-уменьшительное 

\pageimage{page_129}
\label{129-1}
от полного имени Исроэл, значение — борющийся). Вскоре связи с зарубежными родственниками стали обрываться навсегда. Власть отгораживала советский народ от бытового общения с гражданами других стран. Бабушка тяжело переживала возникшее неведение, но никогда не высказывала сожаления по поводу того, что настояла на отъезде двух старших мальчиков в Америку; на семейных

\pageimage{page_130}
\label{130-1}
посиделках иногда говорилось кое-что по поводу острой темы — о сожалении. Бабушка как-то сказала, что "тогда было тяжело, но теперь намного хуже,и я не сожалею о своих соображениях тех лет". Смысл бабушкиных слов передан правильно, но образ её не мелькнул, потому что в моих воспоминаниях бабушка не ассоциируется с русской реальностью, скорее — с польской. Это был не единственный факт твёрдости 

\pageimage{page_131}
\label{131-1}
бабушкиного характера. В данном случае обстановка в стране придавала факту черты убеждённости: на фоне интенсивной индустриализации, выполнения за четыре года первого пятилетнего плана развития народного хозяйства СССР, в то же время при продовольственной недостаточности и прекрасно поставленной пропаганде проходили крупные политические судебные процессы. Вчерашних видных деятелей государства сегодня

\pageimage{page_132}
\label{132-1}
на организованных собраниях трудящихся клеймили позором, называли врагами народа и т.п. Все эти разнонаправленные явления вызывали недоумение и более волнующие чувства и мысли. О чём же могла сожалеть старая Фейге? Только в послесталинское время приоткрылись факты, правда об этих судебных процессах. 

Возвращаюсь в Симбирск (Ульяновск) нэповского периода.

\pageimage{page_133}
\label{133-1}
В центральной части главной улицы города, Гончаровской, близко от кинотеатра „Палас", мой отец открыл небольшой промтоварный магазин. Помню его по одному эпизоду в кондитерской, которая была рядом с папиным магазином. Ульяновск моего раннего детства неизменно дорог и памятен мне четырьмя-пятью эпизодами, в которых тоже высвечивается та эпоха. Папа один работал в магазине, мама вела всю документацию,

\pageimage{page_134}
\label{134-1}
ходила в финотдел и т.п. Когда папа уезжал в Москву или Самару заказывать товары, мама работала в магазине, Гриша и Давид ходили в школу, по домашнему хозяйству маме помогала пожилая крестьянка, которой я, возможно, обязан жизнью. 

Семья жила на улице Фридриха Энгельса, 56, на втором этаже, состоявшем из неказистой надстройки, такой же, но большой, одной квартиры. Помню холодные сени, широкую приземистую входную дверь и огромные клубы морозного "пара", когда зимой открывали дверь. Вела в просторное неразгороженное помещение объединившее 
\pageimage{page_135}
\label{135-1}
прихожую, кухню с печкой, столовую, были ещё двери в комнаты, их я не помню. Любопытно: повторение в городе типовой планировки крестьянской избы нашло выражение в современном домостроении - кухни-салоны. Семья моих родителей жила близко от пересечения с Гончаровской. На противоположной стороне - «Палас» и папин магазин. От угла кинотеатра переулок вверх вёл в самую престижную часть старого города - на Венец; Венец - вершина высокого крутого берега реки Волга, он украшен аллеями, посаженными деревьями, богатыми особняками "бывших".

\pageimage{page_136}
\label{136-1}
среди путей

Среди особняков стоял дом известного русского писателя Гончарова Ивана Александровича (1812-1891). Это был подарок, рассказывают, симбирского купечества земляку после его кругосветного плавания на фрегате "Паллада" и выхода в свет книги одноимённого названия. С Венца Волга видится далеко внизу: железнодорожный мост, пароходы, баржи кажутся чуть ли не игрушечными. На противоположном берегу реки - ровные, широченные, бескрайние поля Заволжья. Это и есть то, что называют русское раздолье. Далее по Венцу влево, уже после особняков, за аллеями, Карамзинский сад: на высоком 

\pageimage{page_137}
\label{137-1}
гранитном пьедестале бюст гуманиста, видного историка и писателя Карамзина Николая Михаиловича (1766-1826).Вокруг пьедестала большая песочница для малышей. Из сада по улице вниз — бывшая гимназия, а потом школа, в ней учились Гриша и Давид. С няней иду встречать их, они пулей вылетают на улицу, бегут навстречу, хватают меня за подмышки, я с радостью поджимаю ноги, и мы мчимся до пологого спуска на Гончаровскую. Все  залито полуденным солнцем ещё тёплого 

\pageimage{page_138}
\label{138-1}
осеннего дня. Таков замкнутый эллипс первой дороги моей жизни, дороги раннего и счастливого детства. 

Раннее детство счастливо потому, что оно не знает тяжёлых проблем жизни, его оберегают от них, но когда проблемы врываются в жизнь ребёнка, кончается счастье возраста. Так случилось на пятом году моей жизни, в 1929 году, но пока не об этом.Вернёмся на несколько лет назад. Успешность НЭПа привела простых людей к мнению, что экономические методы

\pageimage{page_139}
\label{139-1}
хозяйственных взаимоотношений и есть форма управления новой властью народным хозяйством. Пережив мытарства 1914-1924 годов, народ поверил в НЭП, и это сказалось на добровольном (подчёркиваю это слово) и активном трудовом оживлении в стране. Жизнь, как обильный базар, предлагала своё разнообразие, и мамин брат Хаим в 1923 году уехал в Москву. Сразу же или в последующий визит он 

\pageimage{page_140}
\label{140-1}
забрал в Москву младшего брата Янкеля (Яшу), и помог ему самостоятельно войти в жизнь. Неприметный, на первый взгляд, факт передал главную черту характера Хаима: помочь родственному человеку, оказавшемуся в тяжёлых условиях. Когда наступили суровые времена, Хаим-Ефим Александрович Величанский (1900-1972) поддержал многих родственников, но об этом потом, по мере хода истории страны. Мои родители наладили свою жизнь в

\pageimage{page_141}
\label{141-1}
Симбирске но в 1924 году моя мама, не обратившая внимание на легкие заболевания, добегалась до плеврита; это было на последних сроках её беременности. О рождении Давида, о чём вы уже прочитали, и о моем,  рассказывалась  "семейная агада". Я расскажу ее и вам, в ней тоже отражено время. Роды прошли очень тяжело, мама еле выжила жила. Я родился в соответствующем роженице состоянии, однако от религиозно- национальной традиции отказаться не посмели: Мне дали

\pageimage{page_142}
\label{142-1}
имя прадеда Ефроим. Взаимо-отталкивающие физические состояния роженицы и ребёнка заставили срочно искать кормилицу. Надо сказать, что в те годы в СССР ещё не было широкой сети молочных кухонь для новорожденных, и во многих семьях судьбы ослабленных плачущих детей зависели от семьи. Кормилицу мне нашли, и жившая 

\pageimage{page_143}
\label{143-1}
в нашей семье пожилая крестьянка стала носить меня на кормёжки. Я был настолько безнадёжен, что она, видимо, сжалившись, решила по-своему помочь. Папа после работы, вечером на руках укачивала меня, и в один из таких вечеров она отозвала папу и сказала: «Абрам, если хочешь, чтоб мальчик жил, купи церковного вина и давай по нескольку капель на ночь. Только старухе (т.е. бабушке) не говори."

\pageimage{page_144}
\label{144-1}
Родители идею красного вина приняли, и папа каждый вечер макал тряпочный жгутик в вино. Я стал поправляться, крепнуть. Убедившись, это я закрепился в жизни, мое рождениезарегистрировали в ЗАГСе не 28-го августа, что соответствовало факту, а 2 декабря. Не могу не сказать о совпадении чисел: 28 августа, но девятью годами

\pageimage{page_145}
\label{145-1}
470
раньше, родился Давид. 2 декабря 1955 года родился Женичка. 24 июля 1944 года я был тяжело ранен, и через 6 лет в тот же день родился Боричка.
Уровень медицины начала 20-х годов был ниже возможного уровня: многие врачи погибли или были изувечены в войнах и революциях. Другие попали в ольшое число 

\pageimage{page_146}
\label{146-1}
интеллектуальной элиты, которую выслали из советской России. Некоторые сами уехали в другие страны, другие приспособились к новым условиям. Лекарств и медикаментов не хватало. Многие лечились народными средствами. Помню рассказ мамы о том, как её лечили растопленным собачьим жиром, чтобы обезопасить лёгкие. Между тем, жизнь семьи постепенно нормализовалась. Как вы поняли, речь идет о второй половине 1920-х годов, о которых частично сказано в предыдущем изложении.