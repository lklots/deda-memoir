\label{121-2}
Перенесёмся в Симбирск. В начале 1920-х годов в нашем роду произошли большие события: умер мой пра-дед Ефраим Берс, отец моей ба-

\pageimage{page_122}
\label{122-1}
бабушки, умер её муж 122 мой дед-по имени Величанский, муж тёти Сони — Яков Розен тоже. Соня и Яков тоже, всё похоронены в Симбирске. Ефраим Берс умер от глубокой старости, он прожил более ста лет. Фамилию его я назвал точно, когда моя Инна после замужества осталась на своей фамилии Рубинштейн. В разговоре с бабушкой об этом я спросил (догадался её девичью фамилию, четко в два слога она произнесла слово Берс. Предположительно учительствовал в еврейских школах. По рассказам бабушки, он был очень скромным человеком, жил в мире своих служебных интересов. Став вдовцом, оставшуюся долгую жизнь прожил в семье своей единственной дочери; Абрам Величанский.

\pageimage{page_123}
\label{123-1}
В какой-то школе сотрудничал с Берсом, стали друзьями, а потом родственниками — тестем и зятем; интересное в этой более чем столетней тоже был учителем, предыстории семьи то, что объект родства моя будущая бабушка Фейга с радостью восприняла свекра, но вскоре показала, кто в доме генерал: муж-учитель зарабатывал мало, появились дети, перспектива вести жизнь бедной семьи угнетала молодую энергичную женщину, и она, махнув рукой на своего книжника, проявив коммерческую смекалку, организовала в Брест-Литовске в конце 19 века свою мясную торговлю, сама же вела дело, дом, решала судьбы детей. Вот пример стойкости.

\pageimage{page_124}
\label{124-1}
её характера и воли. Ранняя молодость её пришлась на последнее двадцатилетие царствования русского императора Александра II. Она родилась в 1861 году. Император Александр II сделал много доброго для евреев: отменил армейскую службу для совершеннолетних еврейских мальчиков - кантонистов (прочтите книгу «Кантонисты», советское издание, примерно середина 1930-х годов, автора не помню); евреям-солдатам разрешил дослуживаться до офицерских чинов, разрешил евреям получать юридическое и медицинское образование, лекарям-евреям разрешил служить в армии.

\pageimage{page_125}
\label{125-1}
С пониманием относился к положению евреев в Польше, увидев группы мужчин евреев в национальных религиозных одеждах, Царь высказал недоумение по поводу открытой демонстрации в условиях антисемитизма своей национальной принадлежности. Императора, освободившего крестьян от крепостного права, подготовившего другие демократические преобразования в Российской империи, убили народовольцы в марте 1881 года. К убийству Царя евреи никакого отношения не имели и власть в этом их не обвиняла, однако последовавшая естественная реакция на убийство царя перешла от приемлемых ответных мер к самым крупным погромам евреев в Российской империи.

\pageimage{page_126}
\label{126-1}
империи; начало периода больших погромов относится к тому же 1881-му и 1882 годам. В городе Кишинёве, в южных губерниях России, бесчинства продолжились до 1903 года, когда император Николай II возложил ответственность за погромы на губернаторов, и всё остановилось (до 1907 года). Обострение антисемитизма привело к росту эмиграции евреев в Новый Свет (Америка); брестлитовский раввин Соловейчик (из потомственного рода раввинов) поднял тему о возвращении в Эрец-Исраэль. Влиянием погромов состоялась большая нелегальная алия в Эрец-Исраэль из Российской империи. Всё это происходило на фоне идейного разброда в обществе после отмены крепостничества.

\pageimage{page_127}
\label{127-1}
В напряжённой обстановке того времени молодая чета — Абрам и Фейге Величанские принимает решение отправить в Америку старшего сына Шулема, там нет угрозы погромов. Слово шулем на языке идиш, и шалом на иврите в переводе на русский язык означают мир между людьми, странами, а мир в смысле вселенная обозначен на идише словом велт. Года через полтора Шулем стал проситься обратно: не мог прижиться; отец дрогнул, мать категорически возражала и отправила туда второго сына Мойше (Моисея); братья преуспели в период автомобильного бума, но когда наступила в США депрессия, Шулем разорился, с ним случился инсульт, он умер.

\pageimage{page_128}
\label{128-1}
в деятельном возрасте, в самом начале 1930-х годов, его сын Срул продолжил переписку с бабушкой, его письма читались не по одному разу. В частности, Срул сообщил, что в те же годы он закончил Нью-Йоркский медицинский институт и прислал своё фото (помнится, урология или гинекология; тогда я не знал значения этих слов, но двор расширил скупые объяснения взрослых). Тема Срул-уменьши-

\pageimage{page_129}
\label{129-1}
тельное от паного 129. имени Исроэл, значение — борющийся. Вскоре связи с зарубежными родственниками стали обрываться навсегда. Власть отгораживала советский народ от бытового общения с гражданами других стран. Бабушка тяжело переживала возникшее неведение, но никогда не высказывала сожаления по поводу того, что настояла на отъезде двух старших мальчиков в Америку; на семей-

\pageimage{page_130}
\label{130-1}
тых посиделках иногда говорилось кое-что по поводу острой темы — о сожалении. Бабушка как-то сказала, что где было тяжело, но теперь намного хуже, я не сожалею о своих соображениях тех лет. Смысл бабушкиных слов передан правильно, но образ её не мелькнул, потому что в моих воспоминаниях бабушка не ассоциируется с русской реальностью, скорее — с польской. Это был не единственный факт твёрдости ба-

\pageimage{page_131}
\label{131-1}
бушкиного характера; в данном случае обстановка в стране придавала факту карты убеждённости: на фоне интенсивной индустриализации, выполнения за четыре года первого пятилетнего плана развития народного хозяйства СССР, в то же время при продовольственной недостаточности и прекрасно поставленной пропаганде проходили крупные политические судебные процессы, вчерашних видных деятелей государства сего.

\pageimage{page_132}
\label{132-1}
дня на организованных собраниях трудящихся клеймили позором, называли врагами народа и т.п.; все эти разнонаправленные явления вызывали недоумение и более волнующие чувства. Мысль о чём же могла сожалеть старая Фейге? Только в послесталинское время приоткрылись факты. Правда об этих вечерах судебных процессах. Возвращаюсь в Симбирск (Ульяновск) нэповского периода.

\pageimage{page_133}
\label{133-1}
центральной части главной улицы 133. города Гончаровской, близко от кинотеатра „Палас", мой отец открыл небольшой промтоварный магазин. Помню его по одному эпизоду в кондитерской, которая была рядом с папиным магазином. Ульяновск моего раннего детства неизменно дороги, памятен мне четырьмя-пятью эпизодами, в которых тоже высвечивается та эпоха. Папа один работал в магазине, мама вела всю документацию.

\pageimage{page_134}
\label{134-1}
ментацию, ходи- 134. ла в финотдел и т.п.; когда папа уезжал в Москву или Самару заказывать товары, мама работала в магазине, Гриша и Давид ходили в школу, по домашнему хозяйству маме помогала пожилая крестьянка, которой я, возможно, обязан жизнью; семья жила на улице Фридриха Энгельса, 56 на втором этаже, состоявшем из неказистой надстройки, такой же, но большой, одной квартиры.

Помню холодные сени, широкую приземистую входную дверь и огромные клубы морозного "пара", когда зимой открывали дверь. Вела в просторное неразгороженное помещение она.

\pageimage{page_135}
\label{135-1}
ние, об окне - ещё прихожую, кухню с печкой, столовую, были ещё двери в комнаты, их я не помню. Любопытно: повторение в городе типовой планировки крестьянской избы нашло выражение в современном домостроении - кухни-салоны. Семья моих родителей жила близко от пересечения Сконча-ровской, на противоположной стороне - «Палас» и папин магазин; от угла кинотеатра переулок вверх вёл в самую престижную часть старого города - на Венец; Венец - вершина высокого крутого берега реки Волга, он украшен аллеями, посаженными деревьями, богатыми особняками и бывшими.

\pageimage{page_136}
\label{136-1}
среди путей

дом известного русского писателя Гончарова Ивана Александровича (1812-1891), это подарок, рассказывают, симбирского купечества земдяку после его кругосветного плавания на фрегате "Паллада" и выхода в свет книги одноимённого названия с Венца. Волга видится далеко внизу: железнодорожный мост, пароходы, баржи кажутся чуть ли не игрушечными, на противоположном берегу реки - ровные, широченные, бескрайние поля Заволжья, это и есть то, что называют русское раздолье; далее по Венцу влево, уже после особняков, за аллеями, Карамзинский сад;

\pageimage{page_137}
\label{137-1}
КЫСОКОМ гранитном пьедестале 137. не бюст гуманиста, видного историка и писателя Карамзина Николая вокруг пьедестала большая песочница для малышей из сада по улице вниз — бывшая гимназия, а потом школа, в ней учились Гриша и Давид; с няней иду встречать их, они пулей вылетают на улицу, бегут навстречу, хватают меня за подмышки, я с радостью поджимаю ноги, и мы мчимся до пологого спуска. Гончаровская вся залита полуденным солнцем ещё тёплого Михайловия.

\pageimage{page_138}
\label{138-1}
осеннего дня; таков замкнутый эллипс подвиг дороги моей жизни, дороги раннего и счастливого детства. Раннее детство счастливо потому, что оно не знает тяжёлых проблем жизни, его оберегают от них, но когда проблемы врываются в жизнь ребёнка, кончается счастье возраста. Так случилось на пятом году моей жизни, в 1929 году, но пока не об этом, вернёмся на несколько лет назад. Успешность непа привела простых людей к мнению, что экономические методы...

\pageimage{page_139}
\label{139-1}
оды хозяйственных 139. Взаимоотношений и есть форма управления новой властью народным хозяйством. Пережив мытарства 1914-1924 годов, народ поверил в НЭП, и это сказалось на добровольном (подчёркиваю это слово) и активном трудовом оживлении в стране; жизнь, как обильный базар, предлагала своё разнообразие, и мамин брат Хаим в 1923 году уехал в Москву. Сразу же или в последующий ви-

\pageimage{page_140}
\label{140-1}
в зифон заорал о В Москву младшего Янкеля брата, юношу шуби помог ему самостоятельно войти в жизнь те приметный, на первый взгляд факт передал живо 196-ную черту характера — помочь родственному человеку, оказавшемуся в тяжёлых условиях; когда наступили, времена, Хаим-Ефим Александрович Величанский (1900-1972) поддержал многих родственников, но об этом потом по мере хода истории страны. Мои родители наладили свою жизнь в

\pageimage{page_141}
\label{141-1}
имбирске по 81924 Анна Ленина во моя будущая мама, Анна, добегалась с хорошей на заболевала: было на последних её беременности рождении свидания, о чём вы уже упоминали, способа рассказать о моем разная семейная стадо расскажу и вам, в ней тоже отражено время Ро- очень тяжело ды прошли бы жила, а ло, мама еле родился в соответствующем роженице состоянии, однако от религиозно национальных традиций отказаться не посмели; мне дели

\pageimage{page_142}
\label{142-1}
имя прадеда
Еф по имени; взаимоотталкивающие физические состояния роженицы и ребёнка заставили срочно искать кормилицу. Надо сказать, что в те годы в СССР ещё не было широкой сети молочных кухонь для новорожденных, и во многих семьях судьбы ослабленных плачущих детей зависели от кормилицы.

\pageimage{page_143}
\label{143-1}
в нашей семье пожилая крестьянка стала носить меня на кормёжки; я был настолько безнадёжен, что она, видимо, сжалившись, решила по-своему помочь; она после работы, вечером на руках укачивала меня, и в один из таких вечеров она отозвала папу и сказала: «Абрам, если хочешь, чтоб мальчик жил, купи церковного вина и давай по нескольку капель новатолько старухе».

\pageimage{page_144}
\label{144-1}
Родители идею красного вина приняли, и папа каждый вечер макал тряпочный жгутик в вино: я стол поправил, поняв, что, убедившись, это я закрепился в жизни, мое рок зарегистрировали. в загсе 28-го августа. 2 декабря. не могу не сказать о владении 24:28 августа, но девятью годами.

\pageimage{page_145}
\label{145-1}
470
Ами раньше, родился Давид: 2 декабря 1955. Вова родился женичка 24 июля 1944 года, и он тяжело ранен, и через блетк в тот же день родился мальчик.
Уровень медицины начала 20-х годов был ниже возможного уровня: многие врачи не жили, были изувечены в войнах и революциях; другие попали вне-большое число инте.

\pageimage{page_146}
\label{146-1}
интеллектуальной элиты, которую выслали из советской России; некоторые сами уехали в другие страны, другие приспособились к новым условиям. Лекарств и медикаментов не хватало. Многие лечились народными средствами. Помню рассказ мамы о том, как её лечили растопленным собачьим жиром, чтобы обезопасить лёгкие. Но жизнь семьи постепенно нормализовалась. Как вы поняли, речь идет о второй половине 1920-х годов, о которых частично сказано в предыдущем изложении.