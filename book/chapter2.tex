\pageimage{page_002}{
\translate{
Чем дальше уходят годы Великой Отечественной войны, тем важнее, мне представляется, показывать связь военных эпизодов с решениями высшего командования, с фактами жизни страны на всех стадиях войны. В период определившегося всеобщего наступления Советской Армии действия даже взвода были частью стратегического тактического плана победы. В такой связи - один из залогов успеха и более ясного представления о происходившем. Поэтому приведу фрагменты из книги Мартына Мержанова «Солдат, генерал, маршал (о Баграмяне и др.)». Изд-во полит. литературы, М., 1974.}
{
The further the years of the Great Patriotic War go, the more important it seems to me to show the connection of military episodes with the decisions of the high command, with the facts of the country's life at all stages of the war. During the period of the determined general offensive of the Soviet Army, the actions of even a platoon were part of the strategic tactical plan of victory. In such a connection lies one of the keys to success and a clearer understanding of what happened. Therefore, I will provide fragments from the book by Martyn Merzhanov "Soldier, General, Marshal (about Bagramyan and others)". Politizdat, Moscow, 1974.}

\translate{
В июле 1944 года «Правда» писала:}
{
In July 1944, "Pravda" wrote:}

\begin{quote}
\translate{
«По центральным улицам Москвы под конвоем советских солдат прошли 57 600 человек пленных, захваченных в Белоруссии. Впереди гигантской колонны, опустив головы, двигались германские генералы и офицеры... Они победно промаршировали через многие столицы Европы: Варшаву, Париж, Прагу, Белград, Афины, Амстердам, Брюссель и Копенгаген. Их мечтой было так пройти по Москве. И вот они шагали по ней, но не как победители, а как побежденные...

Шагали пленные мимо молчавших, гневных москвичей, плотными рядами стоявших на тротуарах. (с. 81)»}
{
"Along the central streets of Moscow, under the escort of Soviet soldiers, 57,600 prisoners captured in Belarus marched. At the head of the giant column, with their heads down, moved German generals and officers... They triumphantly marched through many capitals of Europe: Warsaw, Paris, Prague, Belgrade, Athens, Amsterdam, Brussels, and Copenhagen. Their dream was to march through Moscow in the same way. And here they were walking through it, but not as victors, but as the defeated...

The prisoners marched past the silent, angry Muscovites, who stood in dense rows on the sidewalks. (p. 81)"}
\end{quote}

\translate{
Прошедшие пленные, по численности, составляли более четырёх стрелковых дивизий. Внушительная сила. Так что июльское шествие было зримым свидетельством грядущей победы, моральным стимулом измученному войной советскому народу. Тогда 9 мая было не более, чем число в календаре. Несмотря на определившийся явный перевес в пользу Советской Армии, война была ожесточённой, враг пытался переломить ситуацию.}
{
The prisoners who passed by, in terms of numbers, constituted more than four rifle divisions. An impressive force. So the July march was a visible testimony of the coming victory, a moral stimulus for the war-weary Soviet people. At that time, May 9 was nothing more than a date on the calendar. Despite the clear advantage in favor of the Soviet Army, the war was fierce, and the enemy tried to turn the tide.}

\translate{
Баграмян исходил из того, что Прибалтику гитлеровцы будут удерживать до последних возможностей. Эта их задача имела не только стратегический, но и политический смысл. Опираясь на созданные здесь хорошо оборудованные в инженерном отношении рубежи, командование группы армий «Север» могло высвободить достаточное количество войск для противодействия войскам 1-го Прибалтийского фронта. (с. 82)}
{
Bagramyan assumed that the Germans would hold the Baltics to the last. This task had not only strategic but also political significance for them. Relying on the well-equipped defensive lines created here, the command of the Army Group "North" could free up a sufficient number of troops to counteract the forces of the 1st Baltic Front. (p. 82)}

\translate{
Предстоящая операция была связана с трудностями, которые возникли в связи с контратаками гитлеровцев. 20 июля 1944 года началось наступление ударной группировки фронта на Шяуляйском направлении. Войска сравнительно быстро преодолели первый оборонительный рубеж и двинулись вперед. (с. 83, 85)}
{
The upcoming operation was associated with difficulties that arose due to the counterattacks of the Germans. On July 20, 1944, the offensive of the front's strike group began in the Šiauliai direction. The troops relatively quickly overcame the first defensive line and moved forward. (pp. 83, 85)}

\translate{
В книге М. Мошинова.}
{
In the book by M. Moshinov.}
}
