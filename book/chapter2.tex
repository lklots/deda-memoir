\translatechapter{Фронтовые воспоминания}{Frontline Memories}
\begin{paracol}{2}

% continued from page_002
\translate{
Чем дальше уходят годы Великой Отечественной войны, тем важнее, мне представляется, показывать связь военных эпизодов с решениями высшего командования, с фактами жизни страны на всех стадиях войны. В период определившегося всеобщего наступления Советской Армии действия даже взвода были частью стратегически-тактического плана победы. В такой связи - один из залогов успеха и более ясного представления о происходившем. Поэтому приведу фрагменты из книги Мартына Мержанова «Солдат, генерал, маршал (о Баграмяне и др.)». Из-во полит. литературы, М., 1974.}
{
The further the years of the Great Patriotic War go, the more important it seems to me to show the connection of military episodes with the decisions of the high command, with the facts of the country's life at all stages of the war. During the period of the determined general offensive of the Soviet Army, the actions of even a platoon were part of the strategic tactical plan of victory. In such a connection lies one of the keys to success and a clearer understanding of what happened. Therefore, I will provide fragments from the book by Martyn Merzhanov "Soldier, General, Marshal (about Bagramyan and others)". Politizdat, Moscow, 1974.}

\translate{В июле 1944 года «Правда» писала:}
{In July 1944, "Pravda" wrote:}

\translate{
«По центральным улицам Москвы под конвоем советских солдат прошли 57 600 человек пленных.}
{
"Along the central streets of Moscow, under the escort of Soviet soldiers, 57,600 prisoners captured in Belarus marched.}

\pageimage{page_003}{
\translate{
хваченных в Белоруссии. Впереди гигантской колонны, опустив головы, двигались германские генералы и офицеры... Они победно промаршировали через многие столицы Европы — Варшаву, Париж, Прагу, Белград, Афины, Амстердам, Брюссель и Копенгаген. Их мечтой было так пройти по Москве. И вот они шагали по ней, но не как победители, а как побежденные... Шагали пленные мимо молчавших, гневных москвичей, плотными рядами стоявших на тротуарах. с. 81. 3. Прошедшие пленные, по численности, составляли более четырёх стрелковых дивизий. Внушительная сила. Так что июльское шествие было зримым свидетельством грядущей победы, моральным стимулом измученному войной советскому народу. Тогда 9 мая было не более, чем число в календаре. Несмотря на определившийся явный перевес в пользу Советской Армии, война оставалась ожесточённой, враг пытался переломить ситуацию. 4. Баграмян исходил из того, что Прибалтику гитлеровцы будут удерживать до последних возможностей. Эта их задача имела не только стратегический, но и политический смысл. Опираясь на созданные здесь хорошо оборудованные в инженерном отношении рубежи, командование группы армий «Север» могло высвободить достаточное количество войск для противодействия войскам 1-го Прибалтийского фронта. с. 82. 14. Предстоящая операция была связана с трудностями, которые возникли в связи с контратаками гитлеровцев. 20 июля 1944 года началось наступление ударной группировки фронта на Шяуляйском направлении. Войска сравнительно быстро преодолели первый оборонительный рубеж и двинулись вперед. с. 83, 85. В книге М. Мошинов. с. 86.}
{
captured in Belarus. At the head of the giant column, with their heads down, moved German generals and officers... They triumphantly marched through many capitals of Europe: Warsaw, Paris, Prague, Belgrade, Athens, Amsterdam, Brussels, and Copenhagen. Their dream was to march through Moscow in the same way. And here they were walking through it, but not as victors, but as the defeated... The prisoners marched past the silent, angry Muscovites, who stood in dense rows on the sidewalks. (p. 81) The prisoners who passed by, in terms of numbers, constituted more than four rifle divisions. An impressive force. So the July march was a visible testimony of the coming victory, a moral stimulus for the war-weary Soviet people. At that time, May 9 was nothing more than a date on the calendar. Despite the clear advantage in favor of the Soviet Army, the war was fierce, and the enemy tried to turn the tide. Bagramyan assumed that the Germans would hold the Baltics to the last. This task had not only strategic but also political significance for them. Relying on the well-equipped defensive lines created here, the command of the Army Group "North" could free up a sufficient number of troops to counteract the forces of the 1st Baltic Front. (p. 82) The upcoming operation was associated with difficulties that arose due to the counterattacks of the Germans. On July 20, 1944, the offensive of the front's strike group began in the Šiauliai direction. The troops relatively quickly overcame the first defensive line and moved forward. (pp. 83, 85) In the book by M. Moshinov. (p. 86)}
}
\pageimage{page_004}{
\translate{
Обратим внимание ещё на один значительный факт, произошедший в тот же день, 20 июля, но по ту сторону фронта. В ту ночь, когда командующий фронтом генерал Баграмян и начальник штаба генерал Курасов окончательно отшлифовали все детали шяуляйской операции, в Восточной Пруссии, в районе Растенбурга, в ставке Гитлера - так называемом «Вольфшанце» («волчье логово») - в 300 километрах от штаба 1-го Прибалтийского фронта, тоже над картой сидел начальник оперативного отдела генерального штаба генерал А. Хойзингер. Он готовился к докладу фюреру, назначенному на 20 июля в полдень... Именно в эту секунду раздался оглушительный взрыв. Гитлер вылез из-под упавшего на него стола. Он получил ожоги и легкие ранения. (с. 85-86) Наши войска шли на запад. Гитлеровцы чувствовали близкую Восточную Пруссию и край земли - берег Балтийского моря. На второй день после начала наступления был освобожден город Паневежис, вскоре и Шяуляй. В тот же день после упорных боев войска 2-го Прибалтийского фронта и 6-й гвардейской армии генерала Чистякова овладели Даугавпилсом. После освобождения Шяуляя противник лихорадочно усиливал сопротивление. (с. 86) Череда масштабных событий приближала мою особую дату, 24 июля 1944 года, о которой рассказываю. Раннее утро. Передовая. Вернулась разведка. Она проверила, в частности, положение на нейтральном участке, откуда предстояло продолжить наступление. Незадолго до этого 6-я гвардейская армия входила в состав 1-го Прибалтийского фронта. В данной операции она взаимодействовала с войсками 2-го Прибалтийского фронта.}
{
Let us pay attention to another significant fact that happened on the same day, July 20, but on the other side of the front. That night, when the front commander General Bagramyan and Chief of Staff General Kurasov finally polished all the details of the Šiauliai operation, in East Prussia, in the Rastenburg area, at Hitler's headquarters - the so-called "Wolf's Lair" - 300 kilometers from the headquarters of the 1st Baltic Front, the head of the operations department of the general staff, General A. Heusinger, was also sitting over a map. He was preparing to report to the Führer, scheduled for noon on July 20... At that very second, a deafening explosion occurred. Hitler crawled out from under the table that had fallen on him. He received burns and minor injuries. (pp. 85-86) Our troops were moving west. The Germans felt the proximity of East Prussia and the edge of the earth - the shore of the Baltic Sea. On the second day after the start of the offensive, the city of Panevėžys was liberated, and soon after, Šiauliai. On the same day, after fierce battles, the troops of the 2nd Baltic Front and the 6th Guards Army of General Chistyakov captured Daugavpils. After the liberation of Šiauliai, the enemy feverishly intensified resistance. (p. 86) A series of large-scale events was approaching my special date, July 24, 1944, which I am telling about. Early morning. The front line. The reconnaissance returned. It checked, in particular, the situation in the neutral area from which the offensive was to continue. Shortly before that, the 6th Guards Army had joined the 1st Baltic Front. In this operation, it interacted with the troops of the 2nd Baltic Front.
}
}

\pageimage{page_005}{
\translate{
Возвращения разведки нам принесли в наплечных термосах горячую вкусную кашу с маслом, сладкий чай, хлеб. Мы не торопясь наелись, ещё раз проверили оружие, кто-то осторожно закурил, все замолчали; ждали приказа на выдвижение к исходным позициям. Я был младшим лейтенантом (после окончания годичного Тульского пулемётно-миномётного училища), командиром стрелкового взвода - 517 стрелковый Краснознамённый полк, 166 гвардейская дивизия, 8-я гвардейская армия. Обстановка на уровне стрелкового-пехотного взвода была такова: Нейтральная полоса проходила как под углом. Линия нашего переднего края возвышалась слегка над узким полем, которое упиралось противоположной стороной в молоденькое редколесье. Предстояло спуститься в лесок, пройти его поперёк и залечь за низким земляным валом - рубежом атаки. Впритык к валу, с внешней от нас стороны, узкая мелиоративная канава. За валом, напрямую в сторону немцев, тоже было небольшое поле длиной - по направлению нашего наступления - метров, примерно 120.
}{
The returning scouts brought us hot, delicious porridge with butter, sweet tea, and bread in shoulder thermoses. We ate unhurriedly, checked our weapons again, someone carefully lit a cigarette, and everyone fell silent; we were waiting for the order to move to the starting positions. I was a junior lieutenant (after graduating from the one-year Tula Machine Gun and Mortar School), commander of a rifle platoon - 517th Rifle Red Banner Regiment, 166th Guards Division, 8th Guards Army. The situation at the level of the rifle-infantry platoon was as follows: The neutral zone ran at an angle. The line of our front edge was slightly elevated above a narrow field, which on the opposite side abutted a young sparse forest. We had to descend into the forest, cross it, and lie down behind a low earthen rampart - the attack line. Close to the rampart, on the side away from us, was a narrow drainage ditch. Beyond the rampart, directly towards the Germans, there was also a small field about 120 meters long in the direction of our advance.
}
}

\pageimage{page_006}{
\translate{
Это поле было под слабым наклоном, но в нашу сторону (для наступающей пехоты это плохо). На немецкой стороне поле граничило с кустарником, старыми деревьями, боковой стеной длинного бревенчатого хуторского строения (амбар или сарай), а немного правее — с большой ёмкостью на высоких металлических опорах. Два объекта — строение и ёмкость — можно было превратить за короткое время в хорошо защищённые огневые точки. Это входило в защитную полосу немецкой обороны. Такова была диспозиция, место нашего взвода в бою, показываю всё в плане.
}{
This field had a slight slope, but towards us (which is bad for advancing infantry). On the German side, the field bordered with bushes, old trees, the side wall of a long log farm building (barn or shed), and a little to the right - a large tank on high metal supports. These two objects - the building and the tank - could be quickly turned into well-protected firing points. This was part of the German defense line. This was the disposition, the place of our platoon in the battle, I show everything in the plan.
}
}

\pageimage{page_007}{
\translate{
Выдвижение на исходные позиции прошло быстро, спокойно. В молоденький лесок вошли в полной боевой готовности. Когда приближались к земляному валу, противник открыл огонь, как слышалось, по широкому фронту наступавшей армии. Всем взводом мы нырнули к валу. Никто не пострадал. Мгновенно открыли ответный огонь. Сила нажатия на спусковой курок, убойная сила пули, казалось, умножались на силу ненависти к фашизму, Гитлеру. Автоматными очередями мы били по местам наиболее вероятных, надёжных для врага, огневых точек. По обстановке мы хорошо подготовились к атаке. В соседнем взводе находился командир роты (значит, там участок был сложнее). Они тоже как следует прополоскали позицию врага, и мы одновременно поднялись в атаку: мигом перемахнули через земляной вал, канаву и устремились вперёд. Особенность боя состояла в том, что расстояние между нашими и немецкими позициями, повторяю, было минимальное. Это значило, что нам надо было сбить уверенность врага в надёжности обороны, а главное — усилением огня ближнего боя выбить его с занимаемых позиций. Причины такого вида пехотных атак возникали потому,
}{
The advance to the starting positions went quickly and calmly. We entered the young forest in full combat readiness. As we approached the earthen rampart, the enemy opened fire, as it seemed, along the wide front of the advancing army. Our entire platoon dived to the rampart. No one was hurt. We instantly returned fire. The force of pressing the trigger, the killing power of the bullet, seemed to multiply by the force of hatred for fascism, for Hitler. We fired bursts at the most likely, reliable enemy firing points. We were well prepared for the attack. In the neighboring platoon was the company commander (which meant that section was more difficult). They also thoroughly rinsed the enemy's position, and we simultaneously rose to the attack: in an instant, we jumped over the earthen rampart, the ditch, and rushed forward. The peculiarity of the battle was that the distance between our and the German positions, I repeat, was minimal. This meant that we had to shake the enemy's confidence in the reliability of their defense, and most importantly, by intensifying close-range fire, drive them out of their positions. The reasons for this type of infantry attack arose because,
}
}

\pageimage{page_008}{
\translate{
Что если враг выдерживал ад ударов сгруппированной огневой мощи дивизий, армий, то он закреплялся на промежуточных линиях обороны и пытался сдержать натиск наступающей Советской Армии. Для этого он поспешно выстраивал защиту, и на переднюю линию своей обороны выдвигал (обычно стрелковые подразделения) человека с ружьём. Немцы отступали, но не бежали, поэтому — не дать закрепиться врагу на промежуточных позициях, выбить человека с ружьём и отогнать противника к рубежам и ко времени, установленным в оперативных разработках высшего командования. Такова задача. В подобных операциях последнюю точку ставила пехота. В этом была её незаменимость, сила и жертвенность. Мы атаковали и на бегу вели автоматный огонь короткими очередями, чтобы ограничить ответный огонь противника и не снижать темпа атаки. (Пояснение: на коротком расстоянии, да ещё на открытой местности под наклоном даже слабым, но...
}{
if the enemy withstood the hell of the concentrated firepower of divisions and armies, they entrenched themselves on intermediate defense lines and tried to hold back the onslaught of the advancing Soviet Army. For this, they hastily built defenses and pushed a man with a rifle (usually rifle units) to the front line of their defense. The Germans retreated but did not flee, so the task was to prevent the enemy from consolidating on intermediate positions, to drive out the man with the rifle, and to push the enemy to the lines and times set in the operational plans of the high command. This was the task. In such operations, the infantry put the final point. This was its irreplaceability, strength, and sacrifice. We attacked and fired short bursts of automatic fire on the run to limit the enemy's return fire and not slow down the pace of the attack. (Explanation: at a short distance, and even on open terrain with a slope, even a slight one, but...
}
}

\pageimage{page_009}{
\translate{
Выходным противнику, "классическая" атака — залегания и короткие перебежки с ведением огня — практически исключена. Атакующие хорошо видны, малоподвижны. Нужен быстрый бросок и сконцентрированный огонь по неширокой полосе атакуемой взводом линии. Так и было. По мере быстрого преодоления короткого расстояния огонь нарастал, но хуже тому, возле их позиций находится атакующая пехота. Однако в общем шуме боя пули выводили из строя наших бойцов. На каком-то десятке метров что-то плоское, жёсткое очень сильно толкнуло меня справа, сбило с ног. Толчок был такой силы, что в моём сознании не зафиксировался момент падения.
}{
for the enemy, a "classic" attack - lying down and short dashes with firing - is practically excluded. The attackers are well visible, immobile. A quick dash and concentrated fire on a narrow strip of the line attacked by the platoon are needed. And so it was. As the short distance was quickly overcome, the fire increased, but worse for them, the attacking infantry was near their positions. However, in the general noise of the battle, bullets took our soldiers out of action. At some ten meters, something flat, hard hit me very hard on the right, knocked me off my feet. The blow was so strong that the moment of falling was not recorded in my consciousness.
}

\translate{
Самоощущение начало возвращаться непрерывным гулом, мраком и своеобразным ветром в голове. Всё это как бы вытягивало меня в какую-то даль, к светлому пятну. В какой-то миг открылись глаза. Я лежал ничком, в луже крови. Жгущая боль во рту, на лице, в правом плече. У головы...
}{
The sensation began to return with a continuous hum, darkness, and a peculiar wind in my head. All this seemed to pull me into some distance, to a bright spot. At some point, my eyes opened. I was lying face down in a pool of blood. Burning pain in my mouth, on my face, in my right shoulder. By the head...
}
}

\pageimage{page_010}{
\translate{
Лежал мой автомат ППШ (пистолет-пулемёт Шпагина), боковым зрением увидел куски рваной щеки, с которых стекали сгустки крови; правого предплечья не было: месиво мяса и крови держало лоб на куске разорванного рукава гимнастёрки, и на темно-синей жилке ужас охватил меня: неужели умираю, неужели не увижу мою Москву?!
}{
lay my PPSh submachine gun (Shpagin submachine gun), with peripheral vision I saw pieces of torn cheek, from which clots of blood were dripping; there was no right forearm: a mess of meat and blood held the forehead on a piece of torn sleeve of the tunic, and on the dark blue vein horror gripped me: am I really dying, will I really not see my Moscow?!
}
}
\end{paracol}