
\pageimage{page_242}
\label{242-1}
Снабжение продовольствием заметно ухудшалось, неслучайно зачастило словечко "недоедание": постоянное физическое ощущение его смыслового значения пригибало плечи. Штрих: когда папа и я изредка навещали близко живших от нас его друзей Болотиных, мы брали с собой два кусочка сахара для двух стаканов чая. Покушаться на сахар хозяина дома было кощунственно. С подобными контурами жизни. Мы вошли в морозную, голодную, тяжёлую в военном отношении зиму 1941/1942 года.

- Итак, после тревожных дней середины октября Красная Армия вошла в тяжёлые,

\pageimage{page_243}
\label{243-1}
жертвенные бои за каждый метр подмосковной земли; как потом стало известно, это было продолжением Московской битвы, начатой ещё 30 сентября 1941 года, закончилась она нашей победой 20 апреля 1942 года. В результате, немецкие войска были отогнаны от Москвы на десятки километров; в отдельных случаях до 200 км. Из сопоставлений дат и расстояний можно понять, как велики были потери, каково значение Московской битвы для жизни столицы; значение также и в том, что это было первое поражение наземных немецко-фашистских войск в Европе после начала Второй мировой войны. До этого поражения они входили в столицы захватываемых стран в парадных

\pageimage{page_244}
\label{244-1}
мундирах. Победу Красной Армии в Московской битве можно сравнить с ярким лучом прожектора во мраке ночи: свет — это надежда на победу, мрак — это продолжающееся наползание немецко-фашистских войск на территорию СССР. Они воспринималось как одно из форм идеологического давления фашизма. Сознание не покидало реальное ощущение этого давления: захватывались земли твоей раодины и других стран, порабощались народы для принуждения их работать на фашизм; из порабощенных народов фашизм выделял и собирал евреев для специально организованного их (нас) физического уничтожения.

\pageimage{page_245}
\label{245-1}
Из рассказанного вам о довоенной антифашистской пропаганде и об убийстве в Минском гетто родственной мне еврейской семьи можете - хотя бы по названным фактам - почувствовать, чем был в реальной жизни нацизм. Мировая документалистика содержит огромный материал о преступлениях германского национал-социализма.
В наши дни, когда отдельные последыши фашизма отрицают Холокост, не лишне напомнить о документе Международного суда. Он издан под названием "Нюрнбергский процесс" (два тома, Москва, 1954 г.).

\pageimage{page_246}
\label{246-1}
Когда всё, ныне документально зафиксированное,было тragическим действием на захваченных немцами территориях, то ещё тогда стала совершенно ясной чрезвычайная, смертельная — в буквальном смысле слова — опасность, которую нес германский фашизм каждому еврею.
Смириться с положением обреченного на смерть — исключалось напрочь. Наоборот, чем больше было фактов бесчинств нацистов на захватываемых территориях Советского Союза, тем яростнее становилась ненависть к врагу. Так было не только в моей душе.

\pageimage{page_247}
\label{247-1}
Обзавто
Послевоенная статистика даёт конкретное представление об участии евреев в Великой Отечественной войне. Например, по количеству Героев Советского Союза евреи заняли пятое место (1 - русские, 2 - украинцы, 3 - белорусы, 4 - татары, 5 - евреи и т.д.). В условиях зыбкого положения на значительной части линии фронта первой половины 1942 года повышался градус патриотизма: усиленно развивалось партизанское движение в тылу врага, шире становились ряды тех, кто добровольно уходил в армию. Я тоже решил уйти в армию.

\pageimage{page_248}
\label{248-1}
По условиям военного времени юноши допризывного возраста проходили курс всеобщего военного обучения при местных районных военных комиссариатах; я - при Тимирязевском РВК г. Москвы. Моё окончание курса совпало по времени с завершением Московской битвы и с очередным весенним призывом в армию (апрель 1942 года). Всеобуч заключался в том, что нас знакомили с устройством, методами ухода и использования стрелкового оружия; по учебным плакатам изучали артиллерию малого и среднего калибров; было интересно. Строевой подготовки не было, видимо, потому что в округе домов по Дмитровскому шоссе, где помещался РВК, не оказалось необходимой площадки.

* * *

\pageimage{page_249}
\label{249-1}
На всеобуч я смотрел как на техническую подготовку к решённому мною вопросу - досрочному уходу в армию; об этом я сказал руководителю курса,и уже после окончания всеобуча 9 мая 1942 года я стоял перед призывной комиссией. После того, как я назвал себя, председатель комиссии спросил, в каких войсках хотел бы я служить. Мой ответ был краток: хочу быть на войне лётчиком-истребителем. Председатель внимательно посмотрел мне в лицо, взял листок из стопочки бумаг на столе, долго , как мне показалось, читал, потом сказал: "Молодой человек, для лётчика-истребителя вы ещё слишком молоды, давайте-ка поедем во второе ЛАУ, хорошев Ленинградское 

\pageimage{page_250}
\label{250-1}
артиллерийское училище. Согласны?" Огорчённый такой неожиданностью, я мотнул головой в знак согласия. 
В г. Кострому, где находилось эвакуированное из Ленинграда училище, я поехал в составе небольшой группы призывников под командой ефрейтора. Поезд прибыл ближе к полдню, погода была отменная, и мы с удовольствием прошли по улицам незнакомого, старой русской архитектуры, города. На территории училища мы остались «щипать травку» возле КПП (контрольного пропускного пункта). Ефрейтор ушел с папочкой к начальству. Он очень долго отсутствовал, а когда вернулся, удивил:

\pageimage{page_251}
\label{251-1}
в училище нас не приняли. Причина — новички не нагонят пройденного материала. Вот те на! Кто из нас мог подумать о такой нестыковке решений, да ещё во время войны. В Москве, в военкомате, никто, похоже, не удивился нашему возвращению. Нам сказали, что когда понадобимся, вызовут повесткой, можем разъезжаться по домам. Опять пришла весна, опять весенний призыв в армию. Я получил повестку как призывник призываемого возраста, и уже без всяких собеседований меня включили в список лиц, направляемых в Тульское пулемётно-миномётное училище (ТПУ).

\pageimage{page_252}
\label{252-1}
За время, прошедшее между двумя названными призывами, у меня произошла личная трагедия. 1 октября 1942 года умер мой папа. Ему стало плохо на работе, на «Скорой помощи» его отвезли в больницу на площади Восстания, рядом с территорией планетария, куда мы, кстати, нередко ходили слушать популярные лекции по астрономии и смотреть тематические спектакли; папа умер как бы возле близкого душе его дома. В то утро я приехал в больницу около 11 утра, чтобы после обхода поговорить с врачом. Я взбежал на второй этаж, открыл дверь длинного коридора и увидел перед собой нянечку

\pageimage{page_253}
\label{253-1}
среди лужи воды: она мыла пол. Я аккуратненько обошёл лужу и направился в палату. Вдруг слышу: "Мальчик! Ты куда? Твой отец утром помер. Иди вон туда, — и протянутой рукой, в которой держала обильно смоченную тряпку, показала на противоположный конец коридора. Я впервые услышал это страшное цлово. Моё сердце тормознуло, как тормозит машина на скорости. Куда? Зачем? Я растерялся, остановился. Нянечка всё поняла, бросила тряпку в ведро, быстро обтерла руки о халат, взяла меня за руку и тихо сказала: "Пойдём к доктору". Этому эпизоду придаю некое философское значение: в столкновении смерти и человечности побеждает человечность.