%%[[page_198]]
1989

Война пришла через весьма прозаичные факты. Сухой, июньский выходной день оказался кстати для навеленца Пореска. В доме папа вынес на улицу зимнюю одежду и развесил её для просушки, как сказала мама, протер где-то в доме и начал мыть пол. Окна были раскрыты, радиотарелка включена, когда диктор сообщил, что будет говорить товарищ Молотов.

- Неживость поля дневной радиопередачи - я насторожился. После первых слов В.М. Молотова я громко крикнул папе, что Гитлер напал на нас. Папа ответил кивком головы; многие жильцы нашего дома в тот момент были на улице. Никакой паники, никакого смятения, только раньше обычного разошлись по домам. Тема войны с германским фашизмом, как я рассказал вам на предыдущих страницах, давно висела в воздухе. Папа предложил поехать в центр города. От станции метро Маяковская по улице Горького.

%%[[page_199]]
мы медленно пошли 199. к Манежной площади: Тоже движение людей, машин, замерли открыты, никаких эмоциональных проявлений после плохой вести; только на противоположной стороне улицы Горького у входа в сберегательную кассу суетно толпилась плотная кучка людей; Поля осуждающе оценил этот факт и медленно, как наш шаг, стал рассказывать мне о своём понимании начавшегося сегодня сверхсложного времени. С того дня сегодня шло более толп, однако первый день войны и наша попытка уловить самое начало понятия её, - особенно мною.

%%[[page_200]]
Помнятся как семи 200. Мутная явь. Папа начал рассуждения с эпизода у сберкассы как одном из первых признаков начала Большой войны: с огромным интересом, присущим неведающему, но любознательному человеку, я, как говорится, разинув рот, слушал папу. Я впервые понял силу маленьких, рыженьких рублёвок и зелёненьких трёшек. На примере Гражданской войны и разрухи, когда цены достигали фантастических размеров, была скрытно разъяснена суть глубокой инфляции и т. д. Папа говорил размеренно, бытовым языком.

%%[[page_201]]
201 ком, да ещё с вкраплениями еврейских слов, в светском понимании. Малограмотный Левец он не знал политэкономической терминологии, но до объяснения многих явлений дошёл своим умом. Это была последняя длительная прогулка, какие мы совершали после смерти мамы: нас гнало из дому её отсутствие, вечерами, когда папа приходил с работы, мы кушали и довольно часто выходили бродить по тропинке.

%%[[page_202]]
вам нашего невзрачного посёлка Коптева я брал папу под руку, за что он всегда благодарил и был рад заслужить папино спасибо; после относительно долгого молчания и под медленный шаг разговор начинал папа с вопроса - на всякий случай, в товарищеском тоне, о школе: он часто просил рассказать что-нибудь интересное, необычное из услышанного на уроках и спрашивал моё мнение о учителе, но сам никогда не пересказывал, однако на родительские собрания не ходил, дневник не смотрел, мою учёбу возложил на меня же. Это была одна из форм доверительных отношений, какие установил между нами после смерти мамы. Он старался воспитать меня ответственным человеком.

%%[[page_203]]
203. Повышение уровня доверительности не означает панибратства, да я к этому и не стремлюсь. После смерти мамы я почувствовал некое утончение слоя защиты и непроизвольно пытался поглубже забиться под папино крыло; папа по-иному видел ситуацию и по-своему стал вводить меня в многообразие реальной жизни. Мои домоводческие функции постепенно расширялись: сперва я стал ответственным за наличие керосина, по своевременную оплату коммунальных услуг; изредка приходилось ездить в домоуправление, где работали в основном женщины; они знали мою маму, поэтому ко мне относились немного внимательнее, заботливее, в инструкциях малозначимых слов пробивалась же недоступное.

%%[[page_204]]
204

для меня материнское тепло. Редкие, душевные эпизоды запомнились, видимо, потому, что они совпадали с моим настроением после смерти мамы и контрастировали с обычным отношением. В таких неродственных случаях, потому-то вспоминая предвоенный период, я вижу в копилке ценностей моей памяти этих милых, обыкновенных женщин. Из того же домоуправления привезли лопаты, грабли вскоре после начала войны, кто-то из приехавших сделал разметку земли: траншея должна вместить всех жильцов восьмиквартирного дома в случае сигнала тревоги.

%%[[page_205]]
205. Взялись за работу. Трудно было снять верхний слой земли, дёрн, потом работа пошла легче, но когда уже прилично углубились в землю, стала проступать обильная вода. Чтобы кто-либо в потемках не угодил в большую яму с водой, мы засыпали её. Пришло распоряжение занавешивать окна так, чтобы с улицы не был виден вечерний свет окон, а на стёкла оконных рам приклеить полосы бумаги, чтобы возможные при взрывах осколки повисали на бумаге. После первой попытки немецких бомбардировщиков прорваться в небо Москвы...

%%[[page_206]]
206.
Жильцы дома решили ещё раз попытаться врезаться (лопатами!) в землю, но, понятно, на другом месте: попробовали, с остервенением и мотом — не помогло, опять вода. Все указания по безопасности населения, и разумные, и наивные, — выполнялись; в поздние часы город погружался в полутьму. На Сталина смотрели с надеждой.
Однако решение главной проблемы все эти меры не помогали, сообщения были тревожные, выработался радиоштамп: «после упорных боёв наши войска шли» (и всё ближе к Москве).

%%[[page_207]]
Когда наступил короткий период воздушных тревог, многие соседи по дому выходили на крыльцо и оставались стоять под навесом от дождя, вспоминая о траншее. Все — с немалым напряжением — смотрели на темное, ночное небо; вдруг, где-то далеко от нас, в облако уперся сильнояркий, голубоватый белый луч прожектора, появился второй луч, они сошлись, видимо, врага нашли. Четвертый этаж и крыша моей школы помешали досмотреть происходившее далеко на горизонте. Так обычно проводили воздушные тревоги в нашем кооперативе на окраине Москвы. Впрочем, было за что переживать и в нашей округе.

%%[[page_208]]
На близком от нас расстоянии (для автомобиля) находились важные объекты: гурт политейный завод имени Войкова; военный завод, путь Рижской железной дорогой и мосты через неё, Братцевская птицефабрика - первое в стране предприятие поточного производства птицеводческой продукции (на базе американской технологии).

И в трагическое иногда вмешивается смешное. В третьей комнате нашей коммуналки жила цыганка Катя с мужем-китайцем. Она работала уборщицей, он - в китайской прачечной (они были такие в Москве). С работы.

%%[[page_209]]
209.
Они приходили вместе и поздно, кухней почему-то не пользовались, даже ведро с водой держали в комнате; люди они были очень скромные и малообщительные. Но когда начались воздушные тревоги, сосед стал совершенно другим человеком: по объявлению воздушной тревоги он раскрывал дверь комнаты и стучал рядом. При первых звуках сообщения об отмене тревоги он выбегал на лестничную клетку и с радостной интонацией выкрикивал: «Тревога миновала, тревога миновала!» Напряжённое ожидание сменилось незаметной улыбкой. Сегодня пронесло. Некоторые, уходившие с крыльца мужики, дружелюбно похлопывали.

%%[[page_210]]
его по плечу, а неко 210. Торые особы (бывало, только бывало!) позволяли себе ущипнуть его.... В один из дней того же периода я откуда-то приехал в конце полуденных часов, соседка сказала, что приходили ребята из школы, меня там ждут, я мигом в школу — она уже закрыта, я — на квартиру к директору (в школах в новостройках была запроектирована квартира для семьи директора школы); открыл дверь сам Иван Дмитриевич, предложил войти, я извинился и объяснил причину прихода в школу; он немного призадумался и с большими паузами между словами сказал: «Клоц. В этой войне тебе еще достанется, иди домой и никого не ищи». Новый учебный год 1.9.41 в школе не начался; я поступил в Московский авиационный техникум.
