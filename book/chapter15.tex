%%[[page_224]]
2. Попытаюсь дать вам почувствовать Москву того времени. Для этого расскажу о том, как я однажды оказался во власти строгостей комендантского часа. После посещения театра на нем тоже был отблеск войны. Признаки столичного города сохранялись по мере возможного. Например, большинство ведущих театров эвакуировали, но Московский академический музыкальный театр имени Станиславского и Немировича-Данченко оставили в Москве. Я очень любил этот театр и остановлюсь немного на воспоминаниях о нём (в попутчики приглашаю ваше воображение). Гладкий, большой прямоугольник светло-жёлтого фасада здания театра смело вписался в ряд мрачных, богатых домов капиталистической застройки на одной из старейших улиц.

%%[[page_225]]
225.н. и центральных улиц Москвы — на Большой Дмитровке (в советское время улица Пушкинская). Новое здание театра как бы дразнит стариков незамысловатостью своих простых, ясных линий, скромностью главного входа. Этот сдвоенные двери обозначены, с улицы, в меру большими фонарями конусной формы, порог входных дверей — вровень с тротуаром, никаких колонн, парадных лестниц. Идет человек по тротуару, шаг в сторону, перешагнул порог, и он в театре (своеобразная архитектурная форма выражения стремления к демократии); перешагнём порог. В просторном кассовом зале свет приглушен, стоят единичные зрители в ожидании своих пар. Ярко светится только окошко кассы, кассир в зимнем пальто и головном уборе, покупаю самый дешевый билет.

%%[[page_226]]
226 Н. Женщина-контролёр тоже в пледе большим шерстяным платком поверх служебного пиджака. — Спускаюсь в гардеробный зал. Зачем? Я же знаю, что в эту зиму (1941-42 гг.) многие общественные здания в Москве не отапливаются. Я сразу почувствовал и увидел, что театр в их числе, но как можно нарушить традицию — в холодный месяц не зайти в раздевалки, не ощутить зарождения праздничного вечера. Ещё недавно, до войны, в этом зале женщины поспешно меняли уличную обувь на выходную, поспешно, не обращая внимания на чужих мужчин, подправляли прически, подкрашивали губы, а то, условно отвернувшись, быстренько выровняли колготки, тогда ещё носили чулки, нефис выровняли по центру юбки, платья. Мужчины поторапливали дам поскорее выйти, чтобы перекусить после работы, но непривычный галстук оказывался съехавшим вбок, и дома умелой рукой наводили порядок. «Ну, пошли все дружелюбно, в хорошем настроении.»

%%[[page_227]]
зау 227 н поднятое настроение вечера дополняли гардеробщики: они, как пульсирующая жила, протянутая вдоль длинного раздевочного стола, заученно метались между зрителями и вешалками: они старались побыстрее вручить номерок и, по возможности, бинокль. театральная публика, разогретая своим, несколько сумбурным, приготовлением к встрече с искусством, постепенно переместилась в большое фойе, где вход в зрительный зал, медленно прогуливаясь по периметру фойе, зрители с невидимым нетерпением ждали первого звонка. театральный звонок - это запоминающаяся кульминация надежд актёров и зрителей.

%%[[page_228]]
Незабываемы микросценки у выходов из станций метро «Охотный ряд», пл. Свердлова, пл. Революции, недалеко от которых расположены на Театральной площади театры Большой, Малый, Центральный детский и далее вверх по Пушкинской филиал Большого театра, филиал МХАТа, Музыкальный, в котором я мысленно нахожусь, а ещё дальше Ленком. Поднимаешься вечером на улицу и обязательно услышишь многократно повторяемый вопрос: «Есть билетик?», и чем ближе к театру, тем слышнее вопрос. Не есть ли это наивысшая оценка театру?

%%[[page_229]]
я 1229н. затянуть — рассказанное о театре медленно складывалось из отрывочных воспоминаний, когда я долго стоял в том же гардеробном зале, но в зимний вечер 422. В сыровато-холодном зале было безлюдно, изредка хлопала дверь туалета. Раздевалочные стойки, за исключением одной, были невзрачным материалом, и только несколько мужчин-иностранцев сдали на вешалку свои полувоенные покрой пальто. Те гости были приметны высоким ростом, необыкновенной худобой, явно военной выправкой. Я остался созерцать давно знакомый зал — он казался чем-то родным, страдающим от отсутствия прихорашивающейся к празднику публики, от обнаружившейся наготы огромного простора.

Своего

%%[[page_230]]
230 Н.
Неоживлённо было и на втором этаже, в большом фойе: почти все зрители в пальто, кругом по которому пробуждаются зрители в ожидании звонка, не сложился. Зрителей мало. Буфет никого, ничего, стулья составлены стопкой и припёрты к стенам помещения, как говорится, до лучших времён; люди беспорядочно бродят по фойе, смотрят неприметно друг на друга и, видимо, одинаково постигают особенности театра военного времени. Наконец, первый звонок, открываются высокие, нарочито красивые двери зрительного зала. Элегантный рисунок дверей, полагаю, символизирует последний рубеж перед входом в мир прекрасного. В партер я вошёл вместе с теми, у кого были дешёвые билеты, свободных мест не было.

%%[[page_231]]
231.H.

Кресел оказалось много, и галёрочники осели в партере; в зрительном зале было очень зябко, а холод, я заметил, создаёт ложно расширительное представление о расстоянии, кубатуре ёмких помещений. Поэтому знакомый зал впервые показался мне больше, чем всегда; впрочем, оптический обман только усилил впечатление от чётких функциональных деталей архитектуры, не обременённых украшательствами. В таком скромном зале дышится легче.

В составлявшиеся минуты до начала оперы-буфф Ксенофонта Прекрасная Елена подумалось: как в таком холоде артисты, играющие легенду Древней Греции, смогут выступать с оголёнными спинами, руками, да чуть ли...

%%[[page_232]]
Алем себя 232H. и смы не с голыми ногами? Смогли. Содержание слова "смогли" в то время включало в себя организации сопротивления советского народа немецким фашистским захватчикам: война нарастала, грушила условия жизни во всех её сферах, стремление к сохранению нормальной организации труда - это было элементом сопротивления; преодоление постановочных проблем театром - это тоже сложная задача времени, которую успешно решали. На мой зрительский взгляд, спектакль прошёл на высоком уровне: оркестр, исполнители, декорации в реалистическом стиле и впечатление от них, усиленное осветительной техникой - всё это работало на сто и не нуждалось в скидках на военное время. Моё следующее посещение театра состоялось через несколько лет, уже после войны.

%%[[page_233]]
233 н.
- Что давали, не запомнил, как играли - не видел, как звучал оркестр - не слышал: в том же затемнённом зрительном зале я неотрывно смотрел на изящный, с чуть приведёрнутым носиком профиль любимой, моей невесты. Не вернусь к «Прекрасной Елене».
От театра до дому - полтора часа, поэтому в условиях комендантского часа вернуться домой к ночи невозможно. Сработал мальчишеский задор: с одной стороны, хотелось немного побыть в уголке давленной жизни, с другой, посмотреть, ощутить готовность к войне.

%%[[page_234]]
1234 н.
Московского метрополитена — технического чуда эпохи. Такое восприятие метро подкреплялось пейзажем московских улиц того времени. Ещё можно было видеть неухоженных, понурых лошадей, тянувших тяжело нагруженные телеги.
Почему я так много говорю о метрополитене? Почему я выбрал его одним из показателей готовности к войне? Отвечаю на ваш возможный вопрос. Вырывающийся на скорости и с шумом из туннеля сверкающий осветительными лампами, никелем поезд был той новой реальностью, мощью, которая, в числе подобных, создавалась страной.

%%[[page_235]]
Должен сказать, что организация работы метрополитена положительно повлияла на образ жизни деловой Москвы давленного периода, когда метро воспринималось как новинка. Шестидесяти-пятидесяти секундные интервалы в часы пик между уходящим поездом и подъезжающим к платформе поездом долго потрясали пассажиров. Технические новации метрополитена были предметом одобрительных обсуждений, он быстро стал незабываемой частью города. Воспоминания о нём у многих, убеждён, стали лучше понимать цену времени.

Итак, после спектакля я, в соответствии со своим замыслом, пошёл ночевать в бомбоубежище на станцию метро «Маяковская». Воздушной тревоги не было, тихо улицы.

%%[[page_236]]
236н. — Пустынно, людей очень мало, все спешат, скоро начало комендантского часа, машин почти нет; давно знакомые здания главной улицы Москвы смотрятся как-то иначе: вроде бы их фасады одухотворились и насупились, ощетинились даже. Не признаёт меня театр юного зрителя! Во всей этой смешанной провинности наступающей ночи сердце переходит на тревожный ритм, рождает неприятные ощущения и желание поскорее уйти из этой необычной тишины, уйти от этого безлюдья. Когда я подошёл к «Маяковской», комендантский час уже действовал: патруль решал переход на другую сторону площади, улицы, всем предлагали войти в метро на ночь.

%%[[page_237]]
237 н.
Другая функция станции метрополитена изменила её облик: все три эскалатора работали только на спуск, загрузка их была полной; с верхних ступеней эскалатора видно было, что среди публики очень много маленьких детей и старых людей, пестрели матрасики, от кро...
Поток младенцев, складные стульчики, скамеечки. Этой «прослойке общества» отвели огромный, красивый зал, расположенный между платформами; большинство граждан должно было спуститься по деревянным трапам, прикреплённым к платформам на рельсы железнодорожного пути. Мне помнятся тяжёлые картины укладывания детей спать: на каменный пол развернутые листы газет, на них матрасик, рядом — согнутый в коленях родитель, пытающийся...

%%[[page_238]]
в этих напряжённых условиях успокоить и уложить ребёнка. Пристроиться возле него, хотя бы вздремнуть; ничего не получается, а утром на работу. Зачем эти пытания, если нет воздушной тревоги? Она может быть и среди ночи, тогда ещё хуже. Многие живут в старых домах, в полуподвальных комнатах, поэтому взрыв авиабомбы, воздушная волна могут обрушить дом, завалить подвальные помещения, а вот метро — надёга. Да, война проникла во все поры жизни, даже в кроватку младенца. Я спустился по трапу на рельсы. На уровне их высоты лежали, поперёк линий, деревянные настилы, чтобы люди могли лечь. Охватило желание ознакомиться, в туннеле меня бы сказал, с магией скоростей и ритмов движения в метро.

%%[[page_239]]
239 K. диаметр. Труба туннель по всей длине пестрела красным сигналом миниатюрных светофоров, прикреплённых к правой полукруглой стене туннеля. Короткие промежутки между светофорами, видимо, и есть одно из средств регулирования скорости. Отключение высоковольтной линии и достаточная освещённость стимулировали любопытство, и я пошёл немного вперёд, в сторону станции Свердловская. Вскоре я увидел железнодорожную ветку, соединяющую, при необходимости, противоположные пути. А пройдя ещё немного, увидел ярко освещённое помещение. Я оказался в просторном, ухоженном санузле, множество туалетных кабин и раковин с водой.

%%[[page_240]]
239 H. диаметр. Труба туннель по всей длине пестрела красным сигналом миниатюрных светофоров, прикреплённых к правой полукруглой стене туннеля; короткие промежутки между светофорами, видимо, и есть одно из средств регулирования скорости отключения высоковольтной линии и достаточная освещённость стимулировали любопытство, и я пошёл немного вперёд, в сторону станции пл. Свердлова. Вскоре я увидел железнодорожную ветку, соединяющую, при необходимости, противоположные пути, а пройдя ещё немного, увидел ярко освещённое помещение. Оказался в просторном, ухоженном санузле, множество туалетных кабин и раковин с водой.

%%[[page_241]]
241н

Одно дело смотреть на блоки через окно вагона, другое - прикоснуться, вообразить силу, сопротивление стали, мощь её толщины. Всё это применительно к войне, к обеспечению стойкой обороны!

7. Прошедший вечер в Театре, уходящая нога в метро, наполненные сопоставлениями впечатления, размышления, да ещё при пустом желудке - свалили меня. В 5 часов утром те же три эскалатора на ветке работали только на подъём, и через 802 в своём транспортном режиме; Комендантский час закончился. Я убедился в целесообразности своего замысла и в многоликости четкого сопротивления врагу.
