% continue from page_022
\translate{В узких семейных родственных отношениях относились к меняющейся ситуации с тревогой и пониманием. % changed "с тревожным пониманием" to "с тревогой и пониманием" for clarity
Сталин правильно сделал, что подписал мирный договор (так в быту называли пакт): мы проиграли соперничество с Гитлером в Испании, он показал силу в Австрии, в Чехословакии, у него поддержка в Италии, Испании, есть сильная Япония; так что всё верно, нам надо укрепляться. Однако последовавшие действия встревожили. Зачем мы влезли в Польшу, зачем война с Финляндией (зима 1939/40 года)? % changed "влезли" to "влезли" for consistency
С чего это вдруг тихая малолюдная Финляндия стала угрозой Ленинграду? Куда ведут все эти неожиданности? Ответа не было.}
{In close family relationships, the changing situation was met with anxious understanding. Stalin did the right thing by signing the peace treaty (as the pact was commonly called): we lost the rivalry with Hitler in Spain, he showed strength in Austria, in Czechoslovakia, he has support in Italy, Spain, there is a strong Japan; so everything is correct, we need to strengthen ourselves. However, the subsequent actions were alarming. Why did we get involved in Poland, why the war with Finland (winter of 1939/40)? How did quiet, sparsely populated Finland suddenly become a threat to Leningrad? Where are all these surprises leading? There was no answer.}

%%[[page_023]]
23. Фактический раздел Польши вызвал особую тревогу в нашей семье: в Варшаве жила семья родной сестры моего папы, там было четверо маленьких внуков. Из антифашистской делегатовской пропаганды мы уже знали о кострах из книг, сжигаемых нацистами (в центральных газетах печатали фотографии таких безумств); о законах Розенберга (Нюрнбергские антиеврейские законы 1935 года); мы знали из информационных сообщений печати и радио о хрустальной ночи (1938 год) — все-германском погроме евреев: об убитых, о разграблении собственности, об отправленных в концентрационные лагеря Бухенвальд, Дахау, Заксенхаузен. Эти страшные слова уже были на слуху у нас в СССР, а значит и в других странах. Некое зримое понятие о бесчинствах в фашистской Германии дал советский фильм «Профессор Мамлок»; потом вышел фильм «Семья Оппенгейм». Хочу подчеркнуть: в довоенные годы в политике Советского руководства антисемитизма не было, достаточно обратить внимание на кадровую политику в стране, на возможности образования; наоборот, была атмосфера интернационализма, СССР был антиподом фашистской Германии. Из элементов наглядной агитации того времени помню огромный...

%%[[page_024]]
24. Транспарант и плакат, которые были вывешены на улице Горького в Москве, если не ошибаюсь, на фасаде здания Центрального телеграфа. На плакате мужчина в белой рубахе с украинским национальным прямоугольным орнаментом на груди динамичным движением руки протягивает букет цветов красноармейцу в каске и с винтовкой с примкнутым штыком. Содержание плаката не соответствовало неожиданности навалившихся событий, настороженному ожиданию неясных перспектив. Всё это ввело нас, школьников, в пионерское лето 1940 года, которое я назвал трудным, потому что события продолжались — неясности оставались. Мы были достаточно взрослыми и воспринимали всю жизнь, но и слишком молоды, чтобы отказываться от юности; пионерское лето продолжалось в обычных интересных делах. Постепенно подошла вторая половина августа. Солнце уже не припекало, к концу дня оно приобретало оранжевый окрас, и этим тёплым приглушённым тоном накрывало всю видимую глазу округу, становилось ещё красивее и почему-то ещё тише. Только протяжные гудки паровоза доносились.

%%[[page_025]]
до лагеря. Мы невольно стали прислушиваться к гудкам, ведь скоро домой, новый учебный год. Сборы к отъезду затягивались; поотрядное построение сопровождалось долгой проверкой списков лиц каждого отряда, докладами начальнику Пионерского лагеря. Затем была торжественная прощальная пионерская линейка, спуск флага. В напутственном слове начальник напомнил об интересных темах кружковых работ, поблагодарил педагогов, а нас — за активное участие во всех мероприятиях, за то, что не было серьезных нарушений дисциплины, пожелал нам хорошо учиться и на будущее лето опять собраться здесь. После перерыва — в путь. Старшие отряды шли в конце колонны, чтобы не задавать быстрого темпа ходьбы младшим. Но постепенно все смешались в одну толпу и весело, хотя и подустали, дошли до станции. Появился, наконец, паровоз, и весь лагерь с криком, визгом, свистом взял на абордаж два первых вагона поезда; абордажная прыть сохранялась до самой Москвы. Младшеклассников встречали родители, старшеклассников — далеко не всех, был рабочий день. Запросто, по-ребячьи скупо, мы попрощались и разошлись. Яшу встретила мама, у меня мамы уже не было. Никто не мог подумать тогда, что на пригородной платформе Киевского вокзала закончился путь нашей юности: через 10 месяцев началась Великая Отечественная война.

%%[[page_026]]
26

Роль идей, понятия морального, нравственного и общественного значения; прояснила отношения между людьми; усилила материальную зыбкость большого слоя людей, к которому принадлежала моя родня; война — это не только боевые действия, но и отношение к ней населения. Разность состояний времён закрепила в моей памяти происходившее в последний довоенный период. Была ли война (Вторая мировая и Великая Отечественная) неизбежна? Да, главная причина в том, что Гитлер, возглавлявший с 1921 года национал-социалистическую рабочую партию Германии (наци), германский фашизм в целом хотели привести мир к третьему рейху (третьей Германской империи) на основе расизма — официальной идеологии фашизма. Её основные положения: вождизм (фюрерство), тоталитаризм (государства с тоталитарным управлением); пангерманизм через геополитику, военная экспансия; антисемитизм; физическая и психологическая неравноценность человеческих рас и влияние этого на историю и культуру общества; высшие расы — создатели цивилизации должны господствовать, а низшие расы — быть эксплуатируемыми; использование идей социализма для привлечения народных масс к нацизму, фашизму.

75

Историческую справку о германских рейхах даю в приложении.

%%[[page_027]]
27. Противостоять этому могла только объединённая демократическая Европа вместе с Советским Союзом, но по ряду причин этого не произошло. Объединение цивилизованных демократических стран мира в антифашистскую, антигитлеровскую коалицию произошло уже в огне Второй мировой войны. Опоздание дало возможность германскому фашизму полностью вооружиться, что обошлось человечеству в десятки миллионов погибших; наибольшие потери в годы Великой Отечественной войны понёс Советский Союз: по разным оценкам, в войне погибло от 20 до 37 миллионов человек. Германские нацисты и их помощники-преступники в оккупированных немцами странах и на захваченной территории СССР успели совершить целенаправленную "Катастрофу" (Шоа) европейского еврейства: по данным международных комиссий, организаций уничтожено 6 миллионов евреев. Наша родня представлена во многих ликах войны, начиная с первого её дня. Два родных брата — мои двоюродные братья Розенбаумы Николай и Алексей, москвичи, жили в посёлке Алексеевское, ныне район напротив станции метро "Щербаковская" (в сторону ВДНХ). Розенбаум Николай (Калмэн) Яковлевич, 1920 года, рабочий завода "Калибр", был призван...

%%[[page_028]]
28. В армию ещё до начала Великой Отечественной войны служил красноармейцем под Львовом. Его мать — старшая сестра моей мамы — получила от него только одно письмо вскоре после начала войны, и всё. То письмо я помню. Он писал, что идут большие бои, успокаивал мать... Розенбаум Алексей Яковлевич, р. 1922 года, учился в техникуме, был призван в армию в первые декады после начала войны. Со дня призыва — ни весточки. Другой мой двоюродный брат — Розенбаум Александр Яковлевич (Шура), р. 1915 года, родной старший брат погибших Николая и Алексея, на фронте был солдатом-связистом. Он «тянул катушку», то есть прокладывал временные телефонные линии связи между воинскими подразделениями, таская тяжёлый от веса провода, но легко вращающийся ручной барабан, по форме и методу пользования напоминавший обыкновенную катушку ниток. Отсюда солдатская ирония — «тянул катушку»... Шура с войны вернулся, но прожил всего несколько лет; своей семьи у него не было. Трижды солдатская мать — Софья Абрамовна Розенбаум, моя тётя Соня, когда-то высокая, красивая женщина, совсем рухнула после смерти третьего сына. У старой женщины обострился диабет, ей отняли ногу. Судьба её сложилась тяжёлая. В начале 1920-х годов умер её муж. Она осталась с пятью детьми. Самый старший сын, Мотя (Матвей Яковлевич), родившийся до Первой мировой войны, после смерти отца оставил дом...

%%[[page_029]]
уехал из Ульяновска в Москву и начал работать; поддерживал мать. С её родственниками в тесных контактах не был. После смерти Шуры мог, его жена Нина - обаятельная женщина, их дочери Слава и Роза звали Соню жить у них, но она отказалась. Осталась с внучкой Бебой, которую воспитывала после смерти своей дочери Розы. Нашему роду не везёт на имя Роза. Из трёх женщин по имени Роза две умерли в молодом возрасте и от одной болезни - порок сердца. Это дочь тёти, а также Роза Величанская - тоже моя двоюродная сестра. К сожалению, давно ничего не знаю о троюродной племяннице Бебе Кауфман, о двоюродном брате Мотеле и его семье.

Не могу остановить трагические, хотя и краткие, воспоминания о Розенбаумах, не сказав о некоторых других фактах жизни моей родной тёти. У неё не было специального профессионального образования, как у большинства евреев её возраста, поэтому она могла быть в основном на неквалифицированных, низкооплачиваемых работах; в этом статусе её удерживало и плохое знание русского языка. Между тем у Сони было еврейское, преимущественно религиозное образование. Родители моей мамы успели до начала Первой мировой войны дать старшим детям такое образование. Соня изучала иудаизм, историю своего народа, знала еврейскую литературу; была религиозна. В Советской России это не могло быть источником существования. В конце 1920-х годов Соня с младшими детьми переселилась в Москву.

%%[[page_030]]
на свои финансовые крохи купила лачугу, буквально лачугу, в селе Алексеевском. Село Алексеевское и подобные ему окраинные захолустные районы расширявшейся Москвы представляли собой обиталища бедноты - новых москвичей. Это был начальный период индустриализации страны, первой пятилетки, расширения и реконструкции заводов и фабрик Москвы, строительство новых предприятий. Нужна была на всё это большая масса рабочей силы, столица страны всегда обладала особой притягательностью. В алексеевских селились люди из бесперспективных провинций, а также уехавшие от голода на Украине; крестьяне, сумевшие каким-то образом избежать коллективизации; разорённые непманы; отдельные люди из низших рядов церковной иерархии, уцелевшие в период идейного и физического разгрома религии; прочий люд, утративший свою социальную среду, а вместе с этим - материальную базу существования. Таким был замеслом нараставшего рабочего класса для ускоренного развития индустрии, в частности в Москве. Алексеевское представляло собой посёлок, застроенный бараками, чуть поодаль от них, ниже, начинались так называемые засыпные постройки и пристройки, самое дешёвое в то время жилье в Москве. Это вот что: плотники сколачивают двухрядный каркас жилья из досок. Промежутки каркаса заполняют сухой землёй, перемешанной с сухим песком. Внутренние стены обивают фанерой и оклеивают обоями. Такое жилье было только одноэтажным.

%%[[page_031]]
можно бы

Толок, он же крыша. Примечательно, доски, вплотную примыкавшие к потолку, «пришивали» так, чтобы через какое-то время их можно было отодвинуть и дополнить осевший слой земли, (чем ни «женщина в песках»?).

С родителями летом, в выходной день, я иногда ездил в гости к Розенбаумам. Запомнился путь от трамвайной остановки: надо было пройти через весь барачный посёлок. В выходной день жизнь кипела на улице: гармошки, частушки, песни, плясовые, конечно, «Барыня», всё это - возле разных бараков, потому сливалось в одну нестройную, удалую песенно-гармошечную мелодию; кто-то полоскал бельё возле водонапорной колонки; другой в резиновых калошах пробирался к туалету. Асфальта на окраинах тогда не было, в лучшем случае - деревянные настилы по длине барачной улицы, а чаще всего - утоптанная земля. Встречал нас Коля, высокий, изящный, интеллигентный юноша: у него был характерный еврейский нос: узкий, с большой горбинкой. Он работал на заводе и учился в системе фабрично-заводского обучения (ФЗО - ФЗУ-училища; на их базе создали техникумы). Лёша на вид был грубоватый парень, широкоплечий, физически очень сильный; больше всего любил сидеть дома и читать. Помню, Соня жаловалась моей маме: нет сил заставить его куда-нибудь сходить. Ко мне они относились, как к «сильно младшему», подчёркнуто внимательно: всегда занимали рассказом о чём-то из прочитанного. Встречались мы не часто, поэтому мне всегда было интересно с ними. Близ-

%%[[page_032]]
кий родственник как-то связал твой лёшка вместо нижней рубахи книжки на теле носит, работал бы лучше. Соня разозлилась: не твоё депо; если ты остался неучем, не значит, что мои дети должны быть такими же; помощи у тебя не прошу. Соня пожинала семейную встречу. Упрёком «книжки» можно было только гордиться, обидел тётю намёк на её нуждаемость, которую она переносила сама. Я объелся Сониного угощения: большая миска с тёплыми картофельными оладушками и сладкий чай, да ещё с вареньем. В своей засыпушке она прожила до послевоенных лет, а когда вдоль Ярославского шоссе началось грандиозное жилищное строительство, Соне дали двухкомнатную квартиру, в которой она жила с Бебой. Моя редкая (упрекаю себя) и возможно последняя встреча с Соней была незадолго до её смерти. Встреча оставила огромное впечатление, и я решил, после многих колебаний, рассказать о ней. Моя тётя своей фигурой, своей сущностью всегда воспринималась как монолит, как нечто духовно цельное; все знали, что она религиозна, но не все знали, что в религии она находила духовную опору. В тот день я в этом убедился. От Сони осталось только имя: исхудавшее, измученное страданиями лицо, отощавшие, обессиленные руки, которые плохо справлялись с костылями. Беба сказала, что бабушка молится, но это было больше (на мой взгляд), больше, чем прочтение напечатанной молитвы, это был преисполненный доверия, искренний монолог, обра...

%%[[page_033]]
33. щённый к Всевышнему, она говорила в таких обычных разговорных интонациях, будто сидит напротив собеседника и в чём-то его убеждает. Мы молча сидели на кухоньке и слушали. Она часто произносила готеню (в языке идиш окончание Ю при обращении к кому-либо имеет значение исключительности, ласкательно-превосходящую форму). Соня спокойно, без эмоционального надрыва, размеренными убеждающими тонами благодарила его за то, что он дал ей силы перенести смерть детей (я сразу обратил внимание на слово ибертруђен — перенести; она не сказала иберлейбен — пережить, смысловая разница очевидна); Соня благодарила за данные ей силы и крепость души (штарка нешума), которые помогали ей справляться с жизненными невзгодами. А сейчас моя душа растерзана (ди нешумэ ист церисен), я обессилена, всем тяжело со мной (алемэн ист швер мит мир), у меня нет будущего на этом свете (их обништ кайн офенунг аф дейм велт), прошу Тебя, забери меня из этой жизни, пожалей меня; Ты всегда в моём сердце (Дубист эйбиг ин майн арц). На последних словах Соня разрыдалась. Я вошёл в её комнату после бебы. (Не знаю, передалась ли в кратком воспроизведении, да ещё по памяти, вся религиозность мольбы тёти Сони). В селе Богородском далеко за Сокольниками в районе завода «Красный богатырь» в однокомнатной квартирке барачного дома жила семья младшего брата моей ма...

%%[[page_034]]
мы, моего родного дяди Симы. Октябрь 1941 года, особенно его первая половина, был решающим в судьбе Москвы, возможно и всей страны. Вермахт — вооружённые силы фашистской Германии медленно, фактически беспрепятственно продвигались к Москве. Государственная и промышленная Москва эвакуировалась; положение москвичей ухудшалось. Медленное продвижение вермахта объясняли потом тремя причинами: опасениями западни, ловушки со стороны Красной Армии; отставанием в продвижении вспомогательных служб немецкой армии; переброской части сил Вермахта с Московского направления на северокавказское. В один из опаснейших для Москвы дней сам Сталин стоял в думах перед готовым к отправлению спецпоездом. Если бы Сталин покинул Москву, это могло бы, скорее всего, означать сдачу города.

Если предположение о западне действительно имело место, то оно, к нашему счастью, оказалось совершенно ошибочным: в октябре (и до этого) Красная Армия могла вести только оборонительные бои; будущий маршал Советского Союза Г.К. Жуков по заданию Сталина выезжал на некоторые подмосковные командные пункты высших командиров, но не всегда находил их там. Хотя мы не знали тогда о спешке и разочарованиях Жукова, всё равно — ситуация осложнялась. Москва быстро становилась прифронтовым городом: на больших улицах и магистралях устанавливали противотанковые ежи, создавали баррикады, преимущественно из мешков с песком, землей, такими же материалами.

%%[[page_035]]
35. шками закладывали нижние этажи некоторых административных зданий и крупных магазинов. Назначение двойное: защита от обстрелов, а при уличных боях — укреплённые огневые точки. Недалеко от нашего барочного дома в Коптево в районе завода имени Войкова, в створе с школой № 204, в которой училась Зоя Космодемьянская, на большом колхозном картофельном поле установили одну из зенитных батарей; четыре удлинённых ствола мощных 85-миллиметровых пушек грозно и обнадеживающе смотрели в небо, иногда ночью оно прощупывалось прожекторами; в поздние вечерние часы над Москвой стали поднимать первые аэростаты воздушного заграждения. Приказано было сдать на хранение радиоприёмники, чтобы враг не мог настроить свои радиопередатчики на радиоволну советской радиостанции. Самая известная тогда была — имени Коминтерна. Наш новый «СИ-235» я отнёс на почту — это был первый советский бытовой приёмник с прекрасным звучанием. Всё слушали по прямой трансляции через репродуктор — настенную тарелку. В сводках новостей говорилось о городах, которые Красная Армия оставила после упорных боёв, о зарождении партизанского движения. Известные драматические актёры часто читали «Бородино» Лермонтова, большие отрывки из «Войны и мира» Льва Толстого об оставленной Москве, организованных пожарах — обо всём, что соответствовало духу двух отечественных войн. Всё звучало актуально, часто передавали шестую симфонию Чайковского.

%%[[page_036]]
рваться

Было невозможно: беспокойный, мятущийся, а временами призывный образ, создаваемый симфонией, не успокаивал, потому что ударные смысловые фрагменты ассоциировались с ударами войны осени 41-го. Москву явно готовили к различным поворотам её судьбы. Никто не выключал радиорепродукторы на ночь. В помощь армии начали создавать народное ополчение; в Москве и области организовали 16 таких дивизий. В одну из них 5 октября 1941 года был призван мой родной дядя Величанский Семён Александрович (Шимон), р. 1903 года. Первые пять дней ополченцы находились и обучались военному делу в здании школы в самом же посёлке Богородском. Отца навещала старшая дочь Рая; младшей было 2,5 годика. На шестой день моя двоюродная сестра Райка пришла навестить отца, но школа была безлюдна: ополченцы уехали на фронт. Ни одной весточки от дяди Симы семья не получила — он погиб или, того хуже, попал в плен, как и большая часть ополчения в боях под городом Ельня. В похоронке (почтовой открытке) говорилось, что пропал без вести; спустя время выяснилось, что так писали об известных убитых. Ельня — примерно 300 км на юго-запад от Москвы в сторону города Смоленска.

О числах октября: 5 октября + 6 дней = 11 октября; из Москвы ополченцев до определённого пункта в сторону фронта доставили транспортом, потом начались сложности фронтовых дорог, включая бомбёжки вражеской авиацией движущихся армейских колонн (в то время в воздухе госту).

%%[[page_037]]
подступала немецкая авиация); остановки для укрытия от авианалётов, остановки привалы по другим причинам; необходимость вовремя прибыть к месту назначения осложнённая реальной скоростью движения людей сверхпризывного возраста, повышенная требовательность в связи с этим к солдату (не отставать), общая нервозность; неотложное по тем условиям введение ополченческих частей в боевые действия; думаю, что ополченческая часть из посёлка Богородское прибыла на фронт не позднее 13 октября; к 15 октября вермахт разбил плохо подготовленные ополченческие части, оборонявшие Москву, поэтому 16 октября 41-го было для Москвы критическим, можно сказать паническим, днём, один из дней периода 13-15 октября, чего считаю днём гибели дяди Симы. 2. Сын дяди Симы, мой двоюродный брат - Величанский Борис Семёнович (8.11.1924-7.06.1994) после призыва в армию был направлен в город Дзержинск, Горьковской области на военное техническое обучение. Как бывало в первой поло...

%%[[page_038]]
вине войны, поток курсантов снимали с обучения и отправляли на фронт. Профиль обучения Бори я не знаю. (В редких послевоенных встречах мы обсуждали немалые текущие проблемы. Возможно, что и он оказался в такой же ситуации. В год призыва, в 1942-м, он был шофёром 65-го автополка; 28 июля 1943 года под хутором Западный Боря был ранен. (Это Орловско-Белгородская операция: 5 августа 1943 года Советская Армия освободила областные центры, города Орёл и Белгород). По найденному обгоревшему Бориному нагрудному медальону решили, что он убит, сообщили матери. Представьте мать семейства с двумя похоронками и двумя детьми. Когда Боря оказался в стационарном госпитале в городе Муроме, он написал домой. Рая причалась к брату. После выздоровления - опять фронт (1944/45 год), 36-й автополк. Боря Величанский награждён орденом «Отечественная война» второй степени; медалью «За Победу над Германией» и другими медалями. После окончания войны и демобилизации его личная жизнь, к сожалению, сложилась труднее обычного: отец погиб, мать - в тяжёлом материальном положении; две сестры.

%%[[page_039]]
то

рички-младшей Люсе всего семь лет, в солдатском кармане — дырка, одежды — всего-то в чём ушёл из армии, от государства социальной поддержки — никакой. С чего начинать? Здесь стоит сказать о Борином характере. С детства его отличало высокое самолюбие, почвой для которого было постоянное стремление самостоятельно решать все свои проблемы; независимость он ценил превыше всего; до властности, деспотизма не дошёл. Но его мама, передав сыну эти характерные для неё качества, пыталась бороться с ними в сыне: она пыталась подчинить его своей сильной воле. В детстве и позже, до войны, мы дружили, помню схватки характеров; у мальчика не хватало образовательной аргументации для убедительного противостояния не доказательству, а своеобразному насилию, и его, тоже упорный, характер подсказывал отрицание. По мере взросления Бори столкновения становились реже, но личных отношений не повысился до баланса теплосемейного уровня.

У Бори была хорошая черта: он никого не считал себе обязанным; она ярко прояви...

%%[[page_040]]
лась в трудный послевоенный период. У моего брата были три варианта пути к будущему: закончить школу, получить аттестат зрелости и поступить в институт. Не на что было жить, тем более что надо было помочь вырастить младшую сестренку; учиться и работать такой нагрузки не позволяло здоровье, ослабевшее после войны. Оставалось работать по специальности, которую получил в армии: он пошёл водителем автобусов московских городских линий. Многолетний труд закончился инфарктом, Борю перевели на диспетчерскую службу до пенсии. Со временем водители московских автобусов стали хорошо зарабатывать, и Боря «поднялся». Женился он поздно, зато очень удачно: Елизавета Файбишевна - уравновешенная, рассудительная и хозяйственная женщина. Они прожили в согласии и спокойствии. Наконец, Боря пришёл к такому состоянию; помню их ухоженную квартирку в Тушино, своим видом она дополняла характеристику их жизни, рано оборванной Бориной смертью. Семья дяди Симы была несколько в стороне от обычного, не навязчивого в нашем...

%%[[page_041]]
родственного общения. Причина тому - пламенная страсть и последующее известно, что Сима был самым красивым, пользовался успехом у женщин, но голову не терял, кроме единственного случая, с которого началось... Как в жизни Сима становится отцом, он поден энтузиазм работы, можно хорошо зарабатывать и развивать семью (не торопись, парень). По мере вхождения в бытовое русло, естественного проявления характеров и взглядов супруги стали духовно отдаляться друг от друга. Сима родился и большую часть жизни (на тот период) прожил в большой еврейской семье; он был в те годы носителем

%%[[page_042]]
телем еврейской ментальности, в которой забота о семье, о душевном внутрисемейном тепле - не на последнем месте; Российская провинция, русский язык, новации времени были для него новым образом жизни, который молодой человек только осваивал. Можно сказать так: он был эмигрантом в молодой Советской России из западных районов бывшей Российской империи. У Сименой Доры - всё наоборот, она была коренной жительницей России, легче разбиралась в ситуациях того сложного времени; своё преимущество сделала средством управления, а не помощи мужу. Еврейские традиции, ментальность Доре были, как говорится, побоку, хотя не отрицала как факт, имеющий к ней какое-то отношение: родившемуся сыночку брис (брит милу) сделали, ещё в 1994 году не сделать брис - это вызов, его она не хотела. Косенности складывающихся отношений в молодой семье родня улавливала, но никто не вмешивался и симпатий к родственнице не.

%%[[page_043]]
4.3. накапливал, а она, в ответ, со всеми была ровна, тактична, симпатии - далеко не всем. Дора Борисовна Величанская (Шуф) была пикантна, невысока ростом, хорошо фигурой; когда встречался прямым взглядом её красивыми чёрными глазами, то возникало ощущение, что они насквозь прочитывают меня: она была начитана и даже самообразована, жила пониманием времени, что сочеталось у неё с идеями феминизма; ко всему она ещё курила, правда, очень изящно. Исходя из названных и других фактов, могу с определённой долей иронии утверждать, что если бы феминизм - движение за права женщин - не пришёл с Запада, то он пришёл бы туда из Посёлка Богородское; экспериментальной площадкой для отработки прав женщин была, конечно, семья, поэтому дядя Сима давно и безнадёжно махнул рукой на папиросный дым идей своей жены, никогда не спорил, а утыкался с головой в газету, когда же ему осточертевали выкладки, он вскакивал и, как бы спохватившись, говорил: "Ох, забыл воду на завтра принести", или не...

%%[[page_044]]
что подобное, и уходил. Помню это потому, что в дни школьных каникул нередко гостил у них по 3-4 дня. Боря и Рая были душевные, открытые ребята, с удовольствием общался с ними (иначе бы не приезжал); у них были увлекательные игры и юношеская классика, которую тётя Дора приносила из библиотеки, где работала. На одном дыхании, вслух, помнится, мы прочитали «Всадник без головы»; Кассий Калохаин всё ещё сидит в моей голове. Нарасхват была газета «Пионерская Правда», в которой печатались отрывки ещё не вышедшей отдельной книгой повести «Тайна двух океанов». В период 30-х годов входила в широкий школьный быт также романтическая приключенческая литература Запада, большими тиражами выходили произведения Фенимора Купера, Жюля Верна, Альфонса Додэ, Александра Дюма, Джонатана Свифта, Джека Лондона и т.д. То было время развития культа книги, социалистической культуры в целом, как и время последовательного укрепления политического культа Сталина; культы взаимодействовали.

%%[[page_045]]
Памятными в гостевании усадьбы Симы остались для меня и беседы с тётей Дорой, хорошо помню обстановку, условия, в которых проходили беседы. Семья жила в одной комнате, угловой и очень светлой. Стена, к которой примыкала входная дверь, выполняла функцию кухни: вплотную к ней стоял небольшой кухонный столик, над ним полки, закрытые тряпочной занавеской, а рядом со столиком низкая скамеечка, на которой стояли два ведра с водой. Ещё ближе к двери - табуретка для сидения - моё место, когда тетушка стояла у столика и на единственной керосинке готовила еду на предстоящие два-три рабочих дня. Холодильником служили несколько ровных кирпичей, сложенных в два ряда, они лежали в уголке холодных сеней. Тут же стоял прочный сосновый ящик, сколоченный дядей, для хранения бутыли с керосином. Стояла и старая раскладушка, которую ненавидели. На гвоздиках висели полотняные мешочки с запасом овощей, зимой лежал запасец дров на пару топок. Войдёшь, бывало, и улавливаешь запахи скудной жизни. Итак,

%%[[page_046]]
Я добрался до своей табуретки. Впрочем, этому частенько мешал ребячий крик с улицы: «Борька, выходи, Фимка, лапту играть». Фимка иногда великодушно отпускал Борьку, а сам всё ж таки усаживался на табурет. Разговор начинался, например, так: «Тётя, там вашими очками проложена книжка. Книга Гамсуна. Имя слышал, а больше ничего не знаю, расскажите». Если очки лежали ближе к концу книги, значит, она одобрена, и пересказ, и особенно смысловая, морализующая часть будет соответствовать какой-либо актуальной идее, поддерживаемой тётушкой. Кнут Гамсун — известный норвежский писатель, лауреат Нобелевской премии 1920 года, почитаемый еврейкой Дорой, как говорят по-русски, скурвился: он отказался от своих демократических взглядов, а в период Второй мировой войны примкнул к фашизму, за что после победы был отдан под суд и осуждён. Тётя Дора не была обывательницей на этой стороне жизни, ей не везло, отчасти подводило самомнение, даже самонадеянность, источником которых...

%%[[page_047]]
Помимо характера был большой духовный мир, и в этом состояла её привлекающая сила. Она, как и многие люди нашего окружения, верила (до периода больших разочарований) в идею социализма; некоторые из них были особенно близки, и она пропускала их через свой творческий ум. Присущемуся в ней феминизму она привязывала идею дружбы народов СССР, а дружбу связывала с идеей культуры — социалистической по содержанию и национальной по форме, причём социалистическое считала фактором постоянным, национальное — временным, уходящим. Дядя Сима, хорошо знавший историю еврейства и откликавшийся на разговоры о национальном, утверждал обратное: национальное глубоко сидит в душе человека, приводил примеры-сопоставления, оно никогда не умрёт. Дора не отрицала историю, но была убеждена, что победой идей социализма отомрёт значение истории народов и в социалистической стране с новым типом государственного устройства все народы СССР сольются в один новый социалистический народ.

%%[[page_048]]
не соглашался: народы одной страны могут объединяться в общих интересах ее развития совместной работы, но ни один народ не откажется от своей истории, создавшей национальную традицию, воспитавшей национальный характер; оскорбление национального достоинства любого человека остро задевает его самолюбие. Я обобщил содержание разных мнений, чтобы сказать об особой их актуальности в связи с тем, что к середине 30-х годов прошлого века Гитлер, его человеконенавистнические и антиеврейские намерения уже были известны миру, поэтому популярная идея дружбы народов СССР была серьёзным антитезисом нацистской пропаганде. Отойдя от этих лет немного вперед, должен с сожалением сказать, что идея дружбы народов начала рушиться в первые же дни войны, когда в захваченных регионах СССР немцы нашли пособников и соучастников по уничтожению евреев. Поездки в Богородское становились реже после смерти моей мамы. Помню случаи, когда рано утром вместе с дядей Симой уезжал домой: на трамвае - до Сокольников,

%%[[page_049]]
шли на остановку напротив пожарной каланчи, вместе дожидались нужного мне номера трамвая, я продирался к окну кондуктора, махал дяде, он отвечал скупым взмахом руки и уходил на работу. Уходящим, со спины вижу первым кадром своего дядю, когда вспоминаю его.

В ноябре 2005 года Люся прислала мне письмо и фотографию с изображением трех мужчин, фото было из альбома матери; Люся спрашивала, кто из троих её отец. Интуиция не подвела её, однако хотела знать точно. В фотомастерской вскоре я получил три прекрасно выполненных экземпляра отдельного портрета отца Люси — моего дяди; техническая сторона заданного вопроса решилась просто. Недоумение вызвали соотношения фактов во времени: как можно было годами держать фотографию мужа и не показать её взрослой младшей дочери, которая по причине войны никогда не видела родного отца? Ответ неожиданный и достаточный следует из содержания того же письма, хочу его процитировать:

%%[[page_050]]
«Трагедии судеб сквозят меж его строк. «Посылаю тебе фотографию (с возвратом), может, ты знаешь, кто на ней? Мне кажется, что первый справа (как смотришь) — мой папа. Кто остальные, может, ты знаешь. На обороте фотографии я написала, она на что-то была наклеена, не я отрывала, я её обнаружила в альбоме у мамы. Мама никогда её не показывала, а может, я не помню? Мама не любила рассказывать о прошлом, поэтому я ничего не знаю о своих предках, да и Рая, думаю, тоже. Я выросла русской по паспорту, еврейкой по душе. Сейчас я часто хожу в ближайшую синагогу, заказываю там йорцайт по маме и папе каждый год, слушаю молитвы, пение кантора. Завершается служба всегда молитвой за Государство Израиль и пением «Атиквы»... Приятно душевно. Таких, как я, пятеро. Приходят в синагогу 50-70 англо-еврейских женщин. Вот так.»

%%[[page_051]]
Почему такойtragический финал отношения к моему дяде? Это же отрицание его присутствия в семье, забвение памяти о нём. Я категорически против, не думаю, что причина в разных мнениях на отдельные общественные кампании, проходившие в стране. Из своего «табуреточного сидения» помню их расхождения по поводу атеизма. Несмотря на начальное образование (двенадцатилетний мальчик кончил учиться с началом Первой мировой войны), дядя Сима не был религиозным человеком, но и не соглашался с атеизмом; тётя Дора была атеисткой, однако осуждала осуществлявшийся в СССР разгром религии и пропагандировавшуюся неприязнь к ней, поэтому тётушка заинтересовалась и научным атеизмом; от неё я впервые услышал о Гольбахе и Фейербахе. Все обсуждения расхождения проходили только в стенах своего дома, возникали чаще всего после статьи в газете или радиопередачи.

%%[[page_052]]
52.
ная семейная ситуация, когда по серьезным и даже второстепенным вопросам нет единого мнения. Что же было в семье моего дяди? А было то, чего не было — ни любви, ни дружбы, возникли две ошибки молодости и следствие — ответственность перед детьми; третья трагедия — сама Люся: только на седьмом десятке жизни она удостоверилась в подлинном облике своего отца; душевный и духовный дискомфорт, который она многие годы испытывала в связи с разностью между записью в паспорте и чувством национального достоинства, был по причине того, что Дора сделала дочке русский паспорт; родня знала и не одобряла. Возникают и другие вопросы, но главное в том, чтобы выполнить моральную обязанность, которую я сам возложил на себя: снять несправедливое забвение с памяти о родном саде, с памяти о погибшем солдате.

%%[[page_053]]
Из редких послевоенных встреч с тётей Дорой запомнилась поездка к ней с моим старшим братом. Приехали в город Дмитров (Подмосковье), где она жила и воспитывала маленького внука — крепыша Стасика, сына Люси. Мы привезли ему пару заводных игрушек, чем обеспечили ему большую занятость, но содержательности общения взрослых это не помогло. Встретились с искренней радостью, сердечностью. Замечательно стареющая тётя Дора мало изменилась в своём внешнем образе, в голосе, чувствовалось и в твёрдости характера; это ещё больше приближало далёкие времена, далёкие не столько по количеству лет, сколько по их грандиозному значению. Всё было ясно, незачем теребить воспоминания о рухнувших надеждах, идеалах. Говорили о «сегодняшнем дне», быте; бедненькая квартирка тётушки и она сама как напоминание о широко мыслящем человеке — всё это ассоциировалось с музеем: экспонаты достоверны, но уже не действуют, они омертвели, осталась только память о человеке и о времени, обманувшем его.
