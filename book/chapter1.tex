\pageimage{page_001}{
\translate{
Течение лет вывело
меня на ту ступень жизни,
когда человек может сказать о себе, что он самый
старший по возрасту и единственный ещё
живой, кто хорошо знал многих представителей
когда-то большого рода. К сожалению, я знаю не всё и
не всех. Мне стыдно за то, что не знаю даже имени
своей бабушки — мамы моего папы. О подобных явлениях —
дальше. Здесь, в вводной части, считаю нужным отметить,
что написал только о том, что помню сам и о
слышанном в семье; оценки некоторым общественно-
политическим событиям в понимании
времени, к которому они относились, а отдельные
подробности
для лучшего представления о ситуациях.}
{
The passage of years has brought me to that stage of life
when a person can say about themselves that they are the oldest by age and the only one still
alive who knew many representatives of what was once a large family. Unfortunately, I do not know everything and everyone. I am ashamed that I do not even know the name
of my grandmother - my father's mother. More about such phenomena - later. Here, in the introduction, I consider it necessary to note
that I wrote only about what I remember myself and what I heard in the family; assessments of some socio-political events in the understanding
of the time to which they belonged, and some details
for a better understanding of the situations.}

\par
\noindent

\translate
{Воспоминания написал по многим причинам. Цепочки жизни большинства еврейских семей разорваны вынужденной диаспорой, многие звенья цепочки безвозвратно утеряны, поэтому надо передавать потомкам ещё известное; мои, к счастью, этим интересуются.Я давно хотел заняться семейной тематикой, но текущие работы никак не кончаются.}
{I wrote the memoirs for many reasons. The chains of life of most Jewish families are broken by forced diaspora, many links of the chain are irretrievably lost, so it is necessary to pass on what is still known to descendants; fortunately, mine are interested in this. I have long wanted to engage in family topics, but current work never ends.}
}

\translate{
Коррективы в последовательность работ внёс мой самый младший внучок Лёвочка: он проявил большой интерес к истории семьи. Однако приближалась шестидесятая годовщина Победы над фашистской Германией. Мои друзья предложили откликнуться на эту очень важную дату, и я начал воспоминания о семье с себя, написав «Метры пехоты». Замысел темы не нарушен: война неумолимыми стальными скребками прошлась по жизням людей и из нашего рода.}
{
The sequence of work was corrected by my youngest grandson Lyovochka: he showed great interest in the history of the family. However, the sixtieth anniversary of the Victory over Nazi Germany was approaching. My friends suggested responding to this very important date, and I began my family memoirs with myself, writing "Meters of Infantry". The theme's concept is not violated: the war passed through the lives of people from our family with relentless steel scrapers.}
