
%\translatechapter{Метры пехоты}{Meters of Infantry}
\pageimage{page_002}{
\translate{
Чем дальше уходят годы Великой Отечественной войны, тем важнее, мне представляется, показывать связь военных эпизодов с решениями высшего командования, с фактами жизни страны на всех стадиях войны. В период определившегося всеобщего наступления Советской Армии действия даже взвода были частью стратегически-тактического плана победы. В такой связи - один из залогов успеха и более ясного представления о происходившем. Поэтому приведу фрагменты из книги Мартына Мержанова «Солдат, генерал, маршал (о Баграмяне и др.)». Из-во полит. литературы, М., 1974.}
{
The further the years of the Great Patriotic War go, the more important it seems to me to show the connection of military episodes with the decisions of the high command, with the facts of the country's life at all stages of the war. During the period of the determined general offensive of the Soviet Army, the actions of even a platoon were part of the strategic tactical plan of victory. In such a connection lies one of the keys to success and a clearer understanding of what happened. Therefore, I will provide fragments from the book by Martyn Merzhanov "Soldier, General, Marshal (about Bagramyan and others)". Politizdat, Moscow, 1974.}


\translate{В июле 1944 года «Правда» писала:}
{In July 1944, "Pravda" wrote:}

\translate{
«По центральным улицам Москвы под конвоем советских солдат прошли 57 600 человек пленных.}
{
"Along the central streets of Moscow, under the escort of Soviet soldiers, 57,600 prisoners captured in Belarus marched.}
}
\pageimage{page_003}{
\translate{
хваченных в Белоруссии. Впереди гигантской колонны, опустив головы, двигались германские генералы и офицеры... Они победно промаршировали через многие столицы Европы — Варшаву, Париж, Прагу, Белград, Афины, Амстердам, Брюссель и Копенгаген. Их мечтой было так пройти по Москве. И вот они шагали по ней, но не как победители, а как побежденные... Шагали пленные мимо молчавших, гневных москвичей, плотными рядами стоявших на тротуарах. с. 81. 3. Прошедшие пленные, по численности, составляли более четырёх стрелковых дивизий. Внушительная сила. Так что июльское шествие было зримым свидетельством грядущей победы, моральным стимулом измученному войной советскому народу. Тогда 9 мая было не более, чем число в календаре. Несмотря на определившийся явный перевес в пользу Советской Армии, война оставалась ожесточённой, враг пытался переломить ситуацию. 4. Баграмян исходил из того, что Прибалтику гитлеровцы будут удерживать до последних возможностей. Эта их задача имела не только стратегический, но и политический смысл. Опираясь на созданные здесь хорошо оборудованные в инженерном отношении рубежи, командование группы армий «Север» могло высвободить достаточное количество войск для противодействия войскам 1-го Прибалтийского фронта. с. 82. 14. Предстоящая операция была связана с трудностями, которые возникли в связи с контратаками гитлеровцев. 20 июля 1944 года началось наступление ударной группировки фронта на Шяуляйском направлении. Войска сравнительно быстро преодолели первый оборонительный рубеж и двинулись вперед. с. 83, 85. В книге М. Мошинов. с. 86.}
{
captured in Belarus. At the head of the giant column, with their heads down, moved German generals and officers... They triumphantly marched through many capitals of Europe: Warsaw, Paris, Prague, Belgrade, Athens, Amsterdam, Brussels, and Copenhagen. Their dream was to march through Moscow in the same way. And here they were walking through it, but not as victors, but as the defeated... The prisoners marched past the silent, angry Muscovites, who stood in dense rows on the sidewalks. (p. 81) The prisoners who passed by, in terms of numbers, constituted more than four rifle divisions. An impressive force. So the July march was a visible testimony of the coming victory, a moral stimulus for the war-weary Soviet people. At that time, May 9 was nothing more than a date on the calendar. Despite the clear advantage in favor of the Soviet Army, the war was fierce, and the enemy tried to turn the tide. Bagramyan assumed that the Germans would hold the Baltics to the last. This task had not only strategic but also political significance for them. Relying on the well-equipped defensive lines created here, the command of the Army Group "North" could free up a sufficient number of troops to counteract the forces of the 1st Baltic Front. (p. 82) The upcoming operation was associated with difficulties that arose due to the counterattacks of the Germans. On July 20, 1944, the offensive of the front's strike group began in the Šiauliai direction. The troops relatively quickly overcame the first defensive line and moved forward. (pp. 83, 85) In the book by M. Moshinov. (p. 86)}
}
\pageimage{page_004}{
\translate{
Обратим внимание ещё на один значительный факт, произошедший в тот же день, 20 июля, но по ту сторону фронта. В ту ночь, когда командующий фронтом генерал Баграмян и начальник штаба генерал Курасов окончательно отшлифовали все детали шяуляйской операции, в Восточной Пруссии, в районе Растенбурга, в ставке Гитлера - так называемом «Вольфшанце» («волчье логово») - в 300 километрах от штаба 1-го Прибалтийского фронта, тоже над картой сидел начальник оперативного отдела генерального штаба генерал А. Хойзингер. Он готовился к докладу фюреру, назначенному на 20 июля в полдень... Именно в эту секунду раздался оглушительный взрыв. Гитлер вылез из-под упавшего на него стола. Он получил ожоги и легкие ранения. (с. 85-86) Наши войска шли на запад. Гитлеровцы чувствовали близкую Восточную Пруссию и край земли - берег Балтийского моря. На второй день после начала наступления был освобожден город Паневежис, вскоре и Шяуляй. В тот же день после упорных боев войска 2-го Прибалтийского фронта и 6-й гвардейской армии генерала Чистякова овладели Даугавпилсом. После освобождения Шяуляя противник лихорадочно усиливал сопротивление. (с. 86) Череда масштабных событий приближала мою особую дату, 24 июля 1944 года, о которой рассказываю. Раннее утро. Передовая. Вернулась разведка. Она проверила, в частности, положение на нейтральном участке, откуда предстояло продолжить наступление. Незадолго до этого 6-я гвардейская армия входила в состав 1-го Прибалтийского фронта. В данной операции она взаимодействовала с войсками 2-го Прибалтийского фронта.}
{
Let us pay attention to another significant fact that happened on the same day, July 20, but on the other side of the front. That night, when the front commander General Bagramyan and Chief of Staff General Kurasov finally polished all the details of the Šiauliai operation, in East Prussia, in the Rastenburg area, at Hitler's headquarters - the so-called "Wolf's Lair" - 300 kilometers from the headquarters of the 1st Baltic Front, the head of the operations department of the general staff, General A. Heusinger, was also sitting over a map. He was preparing to report to the Führer, scheduled for noon on July 20... At that very second, a deafening explosion occurred. Hitler crawled out from under the table that had fallen on him. He received burns and minor injuries. (pp. 85-86) Our troops were moving west. The Germans felt the proximity of East Prussia and the edge of the earth - the shore of the Baltic Sea. On the second day after the start of the offensive, the city of Panevėžys was liberated, and soon after, Šiauliai. On the same day, after fierce battles, the troops of the 2nd Baltic Front and the 6th Guards Army of General Chistyakov captured Daugavpils. After the liberation of Šiauliai, the enemy feverishly intensified resistance. (p. 86) A series of large-scale events was approaching my special date, July 24, 1944, which I am telling about. Early morning. The front line. The reconnaissance returned. It checked, in particular, the situation in the neutral area from which the offensive was to continue. Shortly before that, the 6th Guards Army had joined the 1st Baltic Front. In this operation, it interacted with the troops of the 2nd Baltic Front.
}
}

\pageimage{page_005}{
\translate{
Возвращения разведки нам принесли в наплечных термосах горячую вкусную кашу с маслом, сладкий чай, хлеб. Мы не торопясь наелись, ещё раз проверили оружие, кто-то осторожно закурил, все замолчали; ждали приказа на выдвижение к исходным позициям. Я был младшим лейтенантом (после окончания годичного Тульского пулемётно-миномётного училища), командиром стрелкового взвода - 517 стрелковый Краснознамённый полк, 166 гвардейская дивизия, 8-я гвардейская армия. Обстановка на уровне стрелкового-пехотного взвода была такова: Нейтральная полоса проходила как под углом. Линия нашего переднего края возвышалась слегка над узким полем, которое упиралось противоположной стороной в молоденькое редколесье. Предстояло спуститься в лесок, пройти его поперёк и залечь за низким земляным валом - рубежом атаки. Впритык к валу, с внешней от нас стороны, узкая мелиоративная канава. За валом, напрямую в сторону немцев, тоже было небольшое поле длиной - по направлению нашего наступления - метров, примерно 120.
}{
The returning scouts brought us hot, delicious porridge with butter, sweet tea, and bread in shoulder thermoses. We ate unhurriedly, checked our weapons again, someone carefully lit a cigarette, and everyone fell silent; we were waiting for the order to move to the starting positions. I was a junior lieutenant (after graduating from the one-year Tula Machine Gun and Mortar School), commander of a rifle platoon - 517th Rifle Red Banner Regiment, 166th Guards Division, 8th Guards Army. The situation at the level of the rifle-infantry platoon was as follows: The neutral zone ran at an angle. The line of our front edge was slightly elevated above a narrow field, which on the opposite side abutted a young sparse forest. We had to descend into the forest, cross it, and lie down behind a low earthen rampart - the attack line. Close to the rampart, on the side away from us, was a narrow drainage ditch. Beyond the rampart, directly towards the Germans, there was also a small field about 120 meters long in the direction of our advance.
}
}

\pageimage{page_006}{
\translate{
Это поле было под слабым наклоном, но в нашу сторону (для наступающей пехоты это плохо). На немецкой стороне поле граничило с кустарником, старыми деревьями, боковой стеной длинного бревенчатого хуторского строения (амбар или сарай), а немного правее — с большой ёмкостью на высоких металлических опорах. Два объекта — строение и ёмкость — можно было превратить за короткое время в хорошо защищённые огневые точки. Это входило в защитную полосу немецкой обороны. Такова была диспозиция, место нашего взвода в бою, показываю всё в плане.
}{
This field had a slight slope, but towards us (which is bad for advancing infantry). On the German side, the field bordered with bushes, old trees, the side wall of a long log farm building (barn or shed), and a little to the right - a large tank on high metal supports. These two objects - the building and the tank - could be quickly turned into well-protected firing points. This was part of the German defense line. This was the disposition, the place of our platoon in the battle, I show everything in the plan.
}
}

\pageimage{page_007}{
\translate{
Выдвижение на исходные позиции прошло быстро, спокойно. В молоденький лесок вошли в полной боевой готовности. Когда приближались к земляному валу, противник открыл огонь, как слышалось, по широкому фронту наступавшей армии. Всем взводом мы нырнули к валу. Никто не пострадал. Мгновенно открыли ответный огонь. Сила нажатия на спусковой курок, убойная сила пули, казалось, умножались на силу ненависти к фашизму, Гитлеру. Автоматными очередями мы били по местам наиболее вероятных, надёжных для врага, огневых точек. По обстановке мы хорошо подготовились к атаке. В соседнем взводе находился командир роты (значит, там участок был сложнее). Они тоже как следует прополоскали позицию врага, и мы одновременно поднялись в атаку: мигом перемахнули через земляной вал, канаву и устремились вперёд. Особенность боя состояла в том, что расстояние между нашими и немецкими позициями, повторяю, было минимальное. Это значило, что нам надо было сбить уверенность врага в надёжности обороны, а главное — усилением огня ближнего боя выбить его с занимаемых позиций. Причины такого вида пехотных атак возникали потому,
}{
The advance to the starting positions went quickly and calmly. We entered the young forest in full combat readiness. As we approached the earthen rampart, the enemy opened fire, as it seemed, along the wide front of the advancing army. Our entire platoon dived to the rampart. No one was hurt. We instantly returned fire. The force of pressing the trigger, the killing power of the bullet, seemed to multiply by the force of hatred for fascism, for Hitler. We fired bursts at the most likely, reliable enemy firing points. We were well prepared for the attack. In the neighboring platoon was the company commander (which meant that section was more difficult). They also thoroughly rinsed the enemy's position, and we simultaneously rose to the attack: in an instant, we jumped over the earthen rampart, the ditch, and rushed forward. The peculiarity of the battle was that the distance between our and the German positions, I repeat, was minimal. This meant that we had to shake the enemy's confidence in the reliability of their defense, and most importantly, by intensifying close-range fire, drive them out of their positions. The reasons for this type of infantry attack arose because,
}
}

\pageimage{page_008}{
\translate{
Что если враг выдерживал ад ударов сгруппированной огневой мощи дивизий, армий, то он закреплялся на промежуточных линиях обороны и пытался сдержать натиск наступающей Советской Армии. Для этого он поспешно выстраивал защиту, и на переднюю линию своей обороны выдвигал (обычно стрелковые подразделения) человека с ружьём. Немцы отступали, но не бежали, поэтому — не дать закрепиться врагу на промежуточных позициях, выбить человека с ружьём и отогнать противника к рубежам и ко времени, установленным в оперативных разработках высшего командования. Такова задача. В подобных операциях последнюю точку ставила пехота. В этом была её незаменимость, сила и жертвенность. Мы атаковали и на бегу вели автоматный огонь короткими очередями, чтобы ограничить ответный огонь противника и не снижать темпа атаки. (Пояснение: на коротком расстоянии, да ещё на открытой местности под наклоном даже слабым, но...
}{
if the enemy withstood the hell of the concentrated firepower of divisions and armies, they entrenched themselves on intermediate defense lines and tried to hold back the onslaught of the advancing Soviet Army. For this, they hastily built defenses and pushed a man with a rifle (usually rifle units) to the front line of their defense. The Germans retreated but did not flee, so the task was to prevent the enemy from consolidating on intermediate positions, to drive out the man with the rifle, and to push the enemy to the lines and times set in the operational plans of the high command. This was the task. In such operations, the infantry put the final point. This was its irreplaceability, strength, and sacrifice. We attacked and fired short bursts of automatic fire on the run to limit the enemy's return fire and not slow down the pace of the attack. (Explanation: at a short distance, and even on open terrain with a slope, even a slight one, but...
}
}

\pageimage{page_009}{
\translate{
Выходным противнику, "классическая" атака — залегания и короткие перебежки с ведением огня — практически исключена. Атакующие хорошо видны, малоподвижны. Нужен быстрый бросок и сконцентрированный огонь по неширокой полосе атакуемой взводом линии. Так и было. По мере быстрого преодоления короткого расстояния огонь нарастал, но хуже тому, возле их позиций находится атакующая пехота. Однако в общем шуме боя пули выводили из строя наших бойцов. На каком-то десятке метров что-то плоское, жёсткое очень сильно толкнуло меня справа, сбило с ног. Толчок был такой силы, что в моём сознании не зафиксировался момент падения.
}{
for the enemy, a "classic" attack - lying down and short dashes with firing - is practically excluded. The attackers are well visible, immobile. A quick dash and concentrated fire on a narrow strip of the line attacked by the platoon are needed. And so it was. As the short distance was quickly overcome, the fire increased, but worse for them, the attacking infantry was near their positions. However, in the general noise of the battle, bullets took our soldiers out of action. At some ten meters, something flat, hard hit me very hard on the right, knocked me off my feet. The blow was so strong that the moment of falling was not recorded in my consciousness.
}

\translate{
Самоощущение начало возвращаться непрерывным гулом, мраком и своеобразным ветром в голове. Всё это как бы вытягивало меня в какую-то даль, к светлому пятну. В какой-то миг открылись глаза. Я лежал ничком, в луже крови. Жгущая боль во рту, на лице, в правом плече. У головы...
}{
The sensation began to return with a continuous hum, darkness, and a peculiar wind in my head. All this seemed to pull me into some distance, to a bright spot. At some point, my eyes opened. I was lying face down in a pool of blood. Burning pain in my mouth, on my face, in my right shoulder. By the head...
}
}

\pageimage{page_010}{
\translate{
Лежал мой автомат ППШ (пистолет-пулемёт Шпагина), боковым зрением увидел куски рваной щеки, с которых стекали сгустки крови; правого предплечья не было: месиво мяса и крови держало лоб на куске разорванного рукава гимнастёрки, и на темно-синей жилке ужас охватил меня: неужели умираю, неужели не увижу мою Москву?!
}{
lay my PPSh submachine gun (Shpagin submachine gun), with peripheral vision I saw pieces of torn cheek, from which clots of blood were dripping; there was no right forearm: a mess of meat and blood held the forehead on a piece of torn sleeve of the tunic, and on the dark blue vein horror gripped me: am I really dying, will I really not see my Moscow?!
}


\translate{
Москва представилась не родственниками и не родным домом - мне было 19 лет, мои родители к тому времени умерли. Москва представилась в тот миг уютным уголком Пушкинской площади, где стояли памятник великому поэту (на исконном месте) и гипсовая статуя балерины на крыше нового тогда углового дома по улице Горького.
}{
Moscow did not appear as relatives or a native home - I was 19 years old, my parents had died by that time. Moscow appeared at that moment as a cozy corner of Pushkin Square, where the monument to the great poet (in its original place) and the plaster statue of a ballerina on the roof of the then new corner house on Gorky Street stood.
}

\translate{
Растерянность, шок сменились восстановлением сознания, хотя в голове сильно шумело и стучало в висках, сгустки крови давили горло, я беспрерывно заглатывал их. Однако страх смерти сменился стремлением выжить. Я осторожно развернулся на левом плече, опёрся на левую часть спины, левой рукой подтянул правую руку, потом протащил полевую сумку и ремнём обмотал оторванную руку; в поиске некоторого облегчения боли я попытался снять с лица жгущую тяжесть, но пальцы коснулись языка. Я решил выбираться своими силами. Не успел поеду.
}{
Confusion, shock were replaced by the restoration of consciousness, although there was a loud noise in my head and pounding in my temples, clots of blood pressed on my throat, I continuously swallowed them. However, the fear of death was replaced by the desire to survive. I carefully turned on my left shoulder, leaned on the left part of my back, pulled my right hand with my left hand, then dragged the field bag and wrapped the torn hand with the strap; in search of some relief from the pain, I tried to remove the burning weight from my face, but my fingers touched my tongue. I decided to get out on my own. I didn't have time to...
}
}

\pageimage{page_011}{
\translate{
свои действия, как через секунды я увидел возле себя медицинскую сестру. С колен, нагнувшись без слов, она начала бинтовать моё лицо, потом дважды пыталась наложить жгут на правое предплечье, жгут не держался. «Придётся тугую повязку», — сказала она тихо ни мне, ни себе и опять открыла санитарную сумку. Пока сестра бинтовала, я пытался внятно произнести слово «пить». Она, конечно, всё понимала и без моего мычания, и когда кончила бинтовать, приподняла две фляги, слегка потрясла ими и тихо, с большим сочувствием сказала: «Раненые выпили. Потерпи, родной, скоро заберём тебя». По-прежнему оставаясь низко пригнувшейся к земле, сестра отползла. Тугие повязки приглушили ощущения острой боли, а жажда мучала: казалось, что во рту у меня провонявшая, перепревшая портянка, которую я вынужден отсасывать... Полулёжа на спине, я подтянул повыше, к груди, полевую сумку с примотанной рукой, зацепил ремень автомата и, опираясь на левый локоть, стал ползти в сторону лесочка воз-
}{
my actions, when seconds later I saw a nurse next to me. Kneeling, bending over without a word, she began to bandage my face, then twice tried to apply a tourniquet to my right forearm, the tourniquet did not hold. "We'll have to use a tight bandage," she said quietly, neither to me nor to herself, and opened the medical bag again. While the nurse was bandaging, I tried to clearly say the word "drink." She, of course, understood everything without my mumbling, and when she finished bandaging, she lifted two flasks, shook them slightly, and quietly, with great sympathy, said: "The wounded drank. Hold on, dear, we'll take you soon." Still staying low to the ground, the nurse crawled away. The tight bandages muffled the sensations of sharp pain, but the thirst tormented me: it seemed that there was a stinking, overripe footcloth in my mouth, which I was forced to suck... Half-lying on my back, I pulled the field bag with the strapped hand higher to my chest, hooked the strap of the submachine gun, and, leaning on my left elbow, began to crawl towards the forest...
}
}

\pageimage{page_012}{
\translate{
Можно, если бы я попил воды, остался бы ждать помощи. Близкий лесочек уже казался далёким, однако я дополз. Помню себя уже в санроте: значит, сестра вернулась, меня вынесли. В санроте наполнили незабываемой прелести сырой, прохладной водой, сделали обязательный для всех раненых противостолбнячный укол. Потом - операционная медсанбата.
}{
Maybe if I had drunk water, I would have stayed to wait for help. The nearby forest already seemed far away, but I crawled. I remember myself already in the medical company: it means the nurse returned, they carried me out. In the medical company, they filled me with the unforgettable charm of raw, cool water, gave the mandatory tetanus shot for all the wounded. Then - the operating room of the medical battalion.
}

\translate{
- Санитарная рота - ближайший к боевым действиям медицинский пункт; располагалась на передовых позициях; её функция - неотложная помощь раненому и отправка его в медсанбат.
}{
- The medical company is the nearest medical point to the combat actions; it was located at the front positions; its function was to provide emergency assistance to the wounded and send them to the medical battalion.
}

\translate{
Медицинско-санитарный батальон - развёрнутый в боевых условиях полноценный, по мере продвижения армии стационарный больничный комплекс; располагался в глубине фронтовой полосы на уровне вспомогательных подразделений дивизий, армии. Из медсанбата прооперированных тяжело раненых бойцов отправляли по цепочке эвакогоспиталей (ЭГ) до ближайшей восстановленной железной дороги, куда подходили санитарные поезда. Они развозили раненых по госпиталям страны; в ЭГ раненые находились по 2-4 дня, в зависимости от процесса лечения и их физического состояния. Перевозили раненых и самолётами.
}{
The medical battalion is a full-fledged hospital complex deployed in combat conditions, becoming stationary as the army advanced; it was located deep in the front line at the level of auxiliary units of divisions and the army. From the medical battalion, operated-on severely wounded soldiers were sent through a chain of evacuation hospitals (EH) to the nearest restored railway, where hospital trains arrived. They transported the wounded to hospitals across the country; in the EH, the wounded stayed for 2-4 days, depending on the treatment process and their physical condition. The wounded were also transported by planes.
}
}

\pageimage{page_013}{
\translate{
Была ночь, когда я проснулся от наркоза. Увидел себя в большой армейской палатке. На центральном опорном шесте слабо светилась лампочка. За небольшой тумбочкой сидела медицинская сестра. Много коек, на всех раненые. Полог входа в палатку отброшен, тянет запахом леса. Особая тишина ночи. Ни выстрела, как будто нет войны. Моя челюсть стянута бинтами, по горло накрыт простыней. Рука?! Я резко сдёрнул простыню. Широкие бинты опоясывали грудь, а правое плечо было замотано вкруговую толстой бинтовой повязкой. «Всё. Руки не будет», — холодно сказал я себе и понял свою неновую будущность. Сестра услышала какое-то движение, подошла, заново накрыла простыней, говорила что-то обнадеживающее, но я уже не вслушивался, как утром. Оказалось, я не всё знал тогда об этом дне — была и третья пуля. В городе Иваново, где я лежал в челюстном госпитале, впервые после ранения взял свою полевую сумку и ахнул: в краешке её левого уголка был пулевой вход (попросту говоря, маленькая дырка), а на вылете, в задней утолщённой стенке — большой разрез, как ножом; на донышке сумки лежали разороченная латунная оболочка пули и...
}{
It was night when I woke up from anesthesia. I saw myself in a large army tent. A dim light bulb glowed on the central support pole. A nurse sat at a small bedside table. Many cots, all with wounded. The flap of the tent entrance was thrown back, the smell of the forest wafted in. A special silence of the night. Not a shot, as if there was no war. My jaw was bandaged, covered with a sheet up to my neck. My arm?! I sharply pulled off the sheet. Wide bandages wrapped around my chest, and my right shoulder was wrapped in a thick bandage. "That's it. There will be no arm," I coldly said to myself and understood my new future. The nurse heard some movement, approached, covered me with the sheet again, said something encouraging, but I was no longer listening, as in the morning. It turned out I didn't know everything about that day - there was a third bullet. In the city of Ivanovo, where I was lying in a maxillofacial hospital, I first took my field bag after the injury and gasped: in the corner of its left edge was a bullet entry (simply put, a small hole), and at the exit, in the back thickened wall - a large cut, like with a knife; at the bottom of the bag lay a torn brass bullet casing and...
}
}

\pageimage{page_014}{
\translate{
Рассплющенный кусочек свинца, так рвёт разрывная пуля. По сей день храню эти предметы, ставшие для меня реликвиями Великой Отечественной войны. Дороги к победе над германским нацизмом были крутыми. Слушая чистые звуки фанфар победителям, отдадим честь и тем, чья молодость навечно застыла в той тяжёлой войне. Я награждён двумя орденами Отечественной войны, первой и второй степени, а также медалью «За Победу над Германией» и памятными юбилейными медалями СССР, РФ и Израиля.
}{
a flattened piece of lead, that's how an explosive bullet tears. To this day, I keep these items, which have become relics of the Great Patriotic War for me. The roads to victory over German Nazism were steep. Listening to the pure sounds of fanfares for the victors, let us honor those whose youth was forever frozen in that heavy war. I was awarded two Orders of the Patriotic War, first and second class, as well as the medal "For Victory over Germany" and commemorative jubilee medals of the USSR, RF, and Israel.
}

\translate{
Сокращённо опубликовано в еженедельнике «Кстати» (Калифорния) 1-7 июня 2006 года. №585, стр. 35. http://www.kstati.net
}{
Abbreviated published in the weekly "Kstati" (California) June 1-7, 2006. No. 585, p. 35. http://www.kstati.net
}

\translate{
Продолжение начну с важного для меня эпизода из госпитального периода. Я задался вопросом: как мне жить без руки? Сложности были очевидны, да суровые условия жизни военных лет не утешали. Были и такие инвалиды войны, которые не выдерживали свалившихся на них бед и скатывались на дно жизни; это вовсе не была закономерность, следовавшая из их прошлого. Я понял, что помощь, силу надо искать в себе же; уже в госпитале надо начинать готовить себя к новой жизни. Но как?
}{
I will continue with an important episode from the hospital period. I asked myself: how do I live without an arm? The difficulties were obvious, and the harsh living conditions of the war years did not console. There were also such war invalids who could not withstand the misfortunes that befell them and slid to the bottom of life; this was not at all a pattern that followed from their past. I realized that help, strength must be sought within oneself; already in the hospital, one must begin to prepare oneself for a new life. But how?
}

\translate{
Первое испытание подсказалось необходимостью написать братьям о своём ранении. Я по-
}{
The first test was prompted by the need to write to my brothers about my injury. I...
}
}

\pageimage{page_015}{
\translate{
буквы. Когда решил, что достиг успеха в каллиграфии, попросил медицинскую сестру поискать твёрдую доску или фанерку, чтобы на согнутых ногах мог написать два коротких письма. Как сейчас помню её удивление и предложение написать письма под мою диктовку. Я поблагодарил и отказался, объяснив, что у меня никого нет, надо учиться всё делать самому. Буквы получились корявые, адреса написала сестра. Главное было в том, что я не принял соблазнительную, иногда разрушающую волю человека помощь. Это был мой первый шаг к самоутверждению в сложившейся реальности.
}{
letters. When I decided that I had achieved success in calligraphy, I asked the nurse to find a hard board or plywood so that I could write two short letters on my bent legs. I still remember her surprise and suggestion to write the letters under my dictation. I thanked her and refused, explaining that I had no one, I had to learn to do everything myself. The letters turned out crooked, the nurse wrote the addresses. The main thing was that I did not accept the tempting, sometimes destructive help that destroys a person's will. This was my first step towards self-affirmation in the new reality.
}

\translate{
Среди погибших мой друг — Заславский Яков Михайлович, р. 1923 года, москвич, жил на Сретенке, ул. Хмелева, 14. Мы познакомились задолго до войны в пионерском лагере. Яша закончил (если не ошибаюсь, Самаркандское) танковое училище. Был ранен, после госпиталя вернулся на фронт, но не танкистом. В звании старшего лейтенанта, на 2-м Прибалтийском фронте, 25 июля 1944 года, то есть на следующий день после моего ранения, Яшу тяжело ранило в живот, и в тот же день он скончался.
}{
Among the dead was my friend - Zaslavsky Yakov Mikhailovich, born in 1923, a Muscovite, lived on Sretenka, Khmeleva St., 14. We met long before the war in a pioneer camp. Yasha graduated (if I'm not mistaken, from the Samarkand) tank school. He was wounded, after the hospital he returned to the front, but not as a tanker. As a senior lieutenant, on the 2nd Baltic Front, on July 25, 1944, that is, the day after my injury, Yasha was seriously wounded in the stomach, and on the same day he died.
}

\translate{
После войны фронтовой товарищ Яши прислал его родителям полевую карту, на которой точно помечена могила Яши. При содействии военкоматов родителям разрешили перезахоронить своего единственного сына на подмосковном еврейском кладбище «Востряковское». Яшина могила в правом углу от главных ворот. Подборка его писем с фронта опубликована в журнале «Знамя». Отдаю честь двоюродной сестре моего друга — Люсе Зимоненко, которая многое сделала для сохранения памяти о погибшем брате.
}{
After the war, Yasha's front-line comrade sent his parents a field map on which Yasha's grave was precisely marked. With the assistance of military commissariats, the parents were allowed to reburial their only son in the Vostryakovskoye Jewish cemetery near Moscow. Yasha's grave is in the right corner from the main gate. A selection of his letters from the front was published in the magazine "Znamya". I pay tribute to my friend's cousin - Lyusya Zimonenko, who did a lot to preserve the memory of her deceased brother.
}
}


%%[[page_016]]
\translate{
В той же могиле покоится его мать — Мильда Яковлевна. С Ящей связано воспоминание о последнем довоенном пионерском лете, о 1940 годе. Всегда можно было пробыть за городом две смены, да родители освобождались от забот о детях. Уровень интереса определялся в лагере квалификацией школьных педагогов, которые вели кружковую работу. Кружки были: художественного слова, ботанический, умелые руки, военизированного направления — авиамодельный, географический. Вечером, когда уже смеркалось, разжигали костёр. Ребята, преимущественно девочки, читали — кто наизусть, кто по книге — стихи, небольшие рассказы русско-советских классиков. Потом педагог-руководитель кружка помогал разобраться в подлинном содержании прочитанного, найти смысловые ударения. Такие необязательные теоретизированные уроки легко запоминались и помогали нам в школе получать хорошие отметки по литературе.
В предвоенные годы пригородные электрички по Киевской железной дороге ещё не ходили. Только на участке Северный (Ярославский) вокзал — город Александров поезда тянули электровозы. Приезжавшие с восторгом рассказывали о разности впечатлений от езды. Ну, а если такое выпадало на счастье знакомых ребят, то ты становился завистливым свидетелем потока хвастовства; ребята чувствовали себя героями, будто они и есть те самые электровозы.
}{
In the same grave rests his mother — Milda Yakovlevna. A memory associated with Yasha is about the last pre-war pioneer summer, in 1940. You could always stay out of town for two shifts, and parents were freed from the worries of children. The level of interest in the camp was determined by the qualifications of school teachers who led club activities. The clubs included: literary, botanical, handicrafts, military-oriented — model aircraft, geographical. In the evening, when it was already getting dark, a campfire was lit. The children, mostly girls, read — some from memory, some from books — poems, short stories by Russian-Soviet classics. Then the club leader helped to understand the true content of what was read, to find the semantic emphasis. Such optional theoretical lessons were easily remembered and helped us get good grades in literature at school.
In the pre-war years, suburban electric trains on the Kiev railway did not yet run. Only on the section from the Northern (Yaroslavsky) station to the city of Alexandrov did the trains run on electric locomotives. Those who came back told with delight about the different impressions of the ride. Well, if such a thing happened to familiar guys, you became an envious witness to the flow of boasting; the guys felt like heroes, as if they were those very electric locomotives.
}

%%[[page_017]]
17
\translate{
Так что и летом 1940 года до станции Суково по Киевской, где размещался пионерский лагерь, мы ехали в старых, громыхавших, переваливавшихся сбоку на бок вагонах. Состав тащил тоже старый, приспособленный для пригородного движения, паровоз. Зато он был надраен до сверкающего блеска, и это придавало какой-то шарм путешествию. Паровоз прерывисто гудел, с шумом выпускал пар; колёса монотонно и часто стучали на стыках рельс; угольная гарь из трубы паровоза летела в открытые окна вагонов; при торможении вагоны с лязгом и силой сталкивались буферами, поэтому ребята, которые неустойчиво стояли или сидели на краешках сидений, могли упасть. Условия езды веселили нас, мы сравнивали их с относительно недавно (лето 1935 года) открывшейся впервые в СССР линией Московского метрополитена имени Л.М. Кагановича («Сокольники — Парк культуры имени А. М. Горького»). Несмотря на пыхтение паровоза, пейзаж медленно проплывал мимо окон поезда. Суково казалось по отношению к Москве несусветной далью, и на родительские дни мало кто приезжал, даже к младшим классникам. Отдалённость объединяла всех. Мы с лёгкостью принимали предложения руководителей. Так, педагог по ботанике предложила присоединиться к обору.
}
{
So, in the summer of 1940, we traveled to the Sukovo station on the Kiev railway, where the pioneer camp was located, in old, rattling, swaying side-to-side carriages. The train was also pulled by an old steam locomotive adapted for suburban traffic. But it was polished to a sparkling shine, which added some charm to the journey. The steam locomotive whistled intermittently, released steam with noise; the wheels monotonously and frequently clattered at the rail joints; coal soot from the steam locomotive's chimney flew into the open windows of the carriages; when braking, the carriages collided with a clang and force, so the children who stood unsteadily or sat on the edges of the seats could fall. The travel conditions amused us, we compared them with the relatively recently (summer of 1935) opened first line of the Moscow Metro named after L.M. Kaganovich ("Sokolniki — Gorky Park"). Despite the puffing of the steam locomotive, the landscape slowly floated past the train windows. Sukovo seemed an incredible distance from Moscow, and few people came on parents' days, even to the younger students. The remoteness united everyone. We easily accepted the leaders' suggestions. For example, the botany teacher suggested joining the collection.
}

\translate{
и оформлению гербариев для среднеклассников. За таким занятие меня сфотографировал Яша: с букетиком полевых цветов в правой руке (это оказалось её последним фото). С большим интересом мы ходили в походы по азимуту. Ту двум группам пионеров географ давал схемы разных маршрутов движения к общему сборному пункту. После тренировок на территории лагеря мы частенько уходили на весь день в увлекательные походы. После завтрака получали сухие пайки, географ присоединялся к группе, в которой преобладали новички, и в путь. Отношение к длительным походам было по-мальчишески серьёзное, у девочек тоже. Даже в тех случаях, когда выходили на знакомую местность, всё равно сверяли заданное в схеме с нашим движением, в этом был особый штурманский фарс. На сборный пункт сходились к концу дня. Пришедшие первыми занимались костром. У огня доедали продзапасы, обсуждали примечательности маршрутов, задавали вопросы, внимательно слушали руководителя кружка, который к прямому ответу обычно добавлял какую-нибудь байку. Было весело, непринужденно, но кульминацией всему была... картошка. Мы клали её на угольки, с краю костра: на пределе терпения...
}{
and preparation of herbariums for middle school students. Yasha photographed me doing this: with a bouquet of wildflowers in my right hand (this turned out to be her last photo). We went on azimuth hikes with great interest. The geographer gave two groups of pioneers diagrams of different routes to a common meeting point. After training on the camp grounds, we often went on exciting day-long hikes. After breakfast, we received dry rations, the geographer joined the group, which was mostly made up of newcomers, and off we went. The attitude towards long hikes was serious for both boys and girls. Even in cases when we went to familiar terrain, we still compared the given route with our movement, there was a special navigator's farce in this. We converged at the meeting point by the end of the day. Those who arrived first took care of the campfire. By the fire, we finished our food supplies, discussed the highlights of the routes, asked questions, and listened attentively to the club leader, who usually added some story to the direct answer. It was fun and relaxed, but the culmination of everything was... potatoes. We placed them on the coals, at the edge of the fire: at the limit of patience...
}

\translate{
Ченную картофелину с руки на руку, обжигаясь, с наслаждением пожирали королеву похода. Маршруты завершены, недогоревший костёр залит под пристальным оком руководителя, вещмешочки - за спины и домой, в Пионерский лагерь, где ждёт ужин. Если возвращались очень поздно, и на небе светились Большая Медведица и Полярная звезда, то на каких-то участках пути они были нашими путеводными звёздами. Географ интересно рассказывал также о странах, где эти звёзды видны плохо, об ориентировании на море. От насыщенности впечатлениями большого дня, вконец уставшие, мы крепко засыпали. В самом страшном сне никому не могло присниться, что некоторые из старших ребят немногим более чем через год будут пытаться выйти из немецкого окружения, используя навыки пионерских походов.
}{
Passing the hot potato from hand to hand, burning ourselves, we devoured the queen of the hike with pleasure. The routes were completed, the smoldering campfire was extinguished under the watchful eye of the leader, backpacks on our backs, and home to the Pioneer camp, where dinner awaited. If we returned very late, and the Big Dipper and the North Star shone in the sky, they were our guiding stars on some parts of the way. The geographer also told interesting stories about countries where these stars are poorly visible, and about navigation at sea. Filled with impressions of the big day, completely exhausted, we fell asleep soundly. In the worst nightmare, no one could have dreamed that some of the older guys would be trying to escape from the German encirclement using the skills of pioneer hikes in just over a year.
}

\translate{
Но мы пока в пионерском лете сложного 1940 года. Помимо школы Яша занимался в студии живописи на Сретенке. Больше всего он увлекался портретом. К сожалению, у меня не сохранился его рисунок; было бы интересно посмотреть, насколько психологичны его портреты. Мне ещё помнятся характерные выражения глаз на отдельных изображениях некоторых общих знакомых. Меня рисование никогда не привлекало. Единственный раз, в начальной школе, я со всем старанием нарисовал свою кошку, и то маме пришлось добавить зверю пушистости. Понятно, Яша много рисовал; я ходил на авиамоделирование. Мой выбор не был случайным. Даже сейчас, когда вижу летящий самолёт, провожаю его взглядом до тех пор, пока он не сольётся с глубиной неба. Конечно, авиационная романтика убита в воздушных боях, террористами. Но меня всегда восхищали, и я легко понимал физические законы, которые объединяются словом самолёт, поэтому школьная физика была моим любимым предметом. Кроме того, мой старший брат Гриша работал на авиационном заводе и учился на вечернем отделении Московского авиационного института; у нас бывали увлекательные беседы. После заключения Пакта о ненападении между фашистской Германией и СССР (август 1939 года) Гриша иногда приносил мне хорошо иллюстрированные книги.
}{
But we are still in the pioneer summer of the difficult year 1940. Besides school, Yasha attended a painting studio on Sretenka. He was most interested in portraiture. Unfortunately, I did not keep his drawing; it would be interesting to see how psychological his portraits were. I still remember the characteristic expressions of the eyes in some images of our mutual acquaintances. Drawing never attracted me. The only time, in elementary school, I drew my cat with all my effort, and even then my mother had to add fluffiness to the animal. Clearly, Yasha drew a lot; I attended model aircraft building. My choice was not accidental. Even now, when I see a flying airplane, I follow it with my eyes until it merges with the depth of the sky. Of course, aviation romance has been killed in air battles, by terrorists. But I was always fascinated, and I easily understood the physical laws that are combined in the word airplane, so school physics was my favorite subject. In addition, my older brother Grisha worked at an aircraft factory and studied in the evening department of the Moscow Aviation Institute; we had fascinating conversations. After the signing of the Non-Aggression Pact between Nazi Germany and the USSR (August 1939), Grisha sometimes brought me well-illustrated books.
}
\iffalse
%%[[page_021]]
рованный технический журнал германской авиационной промышленности. Тогда я впервые увидел «Юнкерсы», «Мессершмиты» и узнал некоторые их технические характеристики. Немцы, очевидно, невысоко оценивали советскую авиацию и её защиту, если присылали журнал, из данных которого специалисты могли многое понять, а может быть, это было элементарным запугиванием.

Интересное пионерское лето вовсе не отрывало нас от больших событий в стране и мире, а было их навалом. К лету 1940 года войска фашистской Германии через Бельгию, обойдя с севера французскую оборонительную линию Мажино, беспрепятственно вторглись во Францию и 14 июня без боя вошли в Париж. В том же июне Советский Союз присоединил Эстонию, Латвию, Литву, Бессарабию.

В молодёжных научно-популярных изданиях было много интересной, увлекательной информации из разных областей знаний. Помнится, в журнале «Наука и жизнь» читал статью о надёжности линии Мажино. Это было, когда Советский Союз вёл мощную, разоблачающую фашизм пропаганду. В подобных статьях ненавязчиво говорилось о военной мощи наших потенциальных солидных союзников по предстоящей борьбе с фашизмом. Но такой потенциал — Францию немцы положили на лопатки с первого захвата. У нас, школьников, были и свои вехи определения перемен: наши животы надрывались от смеха на про-

%%[[page_022]]
смотрах фильмов Чарли Чаплина «Новые времена» и «Огни большого города», мы уже предвкушали веселье на анонсировавшемся фильме Чарли Чаплина «Диктатор»; вдруг об этом фильме замолчали, не стало в газетах антифашистского содержания карикатур Бор. Ефимова, Кукрыниксов (аббревиатура фамилий трёх известных советских художников Куприянова М.В., Крылова П.Н., Соколова Н.А.), не стало и других агитационных антифашистских материалов. Но вскоре в центральных газетах появилась фотография Иоахима Риббентропа, министра иностранных дел фашистской Германии, приехавшего на подписание Пакта о ненападении, и впрямь, наступили новые времена.

В узких семейных родственных отношениях относились к меняющейся ситуации с тревожным пониманием. Сталин правильно сделал, что подписал мирный договор (так в быту называли пакт): мы проиграли соперничество с Гитлером в Испании, он показал силу в Австрии, в Чехословакии, у него поддержка в Италии, Испании, есть сильная Япония; так что всё верно, нам надо укрепляться. Однако последовавшие действия встревожили. Зачем мы влезли в Польшу, зачем война с Финляндией (зима 1939/40 года)? С чего это вдруг тихая малолюдная Финляндия стала угрозой Ленинграду? Куда ведут все эти неожиданности? Ответа не было.

%%[[page_023]]
23. Фактический раздел Польши вызвал особую тревогу в нашей семье: в Варшаве жила семья родной сестры моего папы, там было четверо маленьких внуков. Из антифашистской делегатовской пропаганды мы уже знали о кострах из книг, сжигаемых нацистами (в центральных газетах печатали фотографии таких безумств); о законах Розенберга (Нюрнбергские антиеврейские законы 1935 года); мы знали из информационных сообщений печати и радио о хрустальной ночи (1938 год) — все-германском погроме евреев: об убитых, о разграблении собственности, об отправленных в концентрационные лагеря Бухенвальд, Дахау, Заксенхаузен. Эти страшные слова уже были на слуху у нас в СССР, а значит и в других странах. Некое зримое понятие о бесчинствах в фашистской Германии дал советский фильм «Профессор Мамлок»; потом вышел фильм «Семья Оппенгейм». Хочу подчеркнуть: в довоенные годы в политике Советского руководства антисемитизма не было, достаточно обратить внимание на кадровую политику в стране, на возможности образования; наоборот, была атмосфера интернационализма, СССР был антиподом фашистской Германии. Из элементов наглядной агитации того времени помню огромный...

%%[[page_024]]
24. Транспарант и плакат, которые были вывешены на улице Горького в Москве, если не ошибаюсь, на фасаде здания Центрального телеграфа. На плакате мужчина в белой рубахе с украинским национальным прямоугольным орнаментом на груди динамичным движением руки протягивает букет цветов красноармейцу в каске и с винтовкой с примкнутым штыком. Содержание плаката не соответствовало неожиданности навалившихся событий, настороженному ожиданию неясных перспектив. Всё это ввело нас, школьников, в пионерское лето 1940 года, которое я назвал трудным, потому что события продолжались — неясности оставались. Мы были достаточно взрослыми и воспринимали всю жизнь, но и слишком молоды, чтобы отказываться от юности; пионерское лето продолжалось в обычных интересных делах. Постепенно подошла вторая половина августа. Солнце уже не припекало, к концу дня оно приобретало оранжевый окрас, и этим тёплым приглушённым тоном накрывало всю видимую глазу округу, становилось ещё красивее и почему-то ещё тише. Только протяжные гудки паровоза доносились.

%%[[page_025]]
до лагеря. Мы невольно стали прислушиваться к гудкам, ведь скоро домой, новый учебный год. Сборы к отъезду затягивались; поотрядное построение сопровождалось долгой проверкой списков лиц каждого отряда, докладами начальнику Пионерского лагеря. Затем была торжественная прощальная пионерская линейка, спуск флага. В напутственном слове начальник напомнил об интересных темах кружковых работ, поблагодарил педагогов, а нас — за активное участие во всех мероприятиях, за то, что не было серьезных нарушений дисциплины, пожелал нам хорошо учиться и на будущее лето опять собраться здесь. После перерыва — в путь. Старшие отряды шли в конце колонны, чтобы не задавать быстрого темпа ходьбы младшим. Но постепенно все смешались в одну толпу и весело, хотя и подустали, дошли до станции. Появился, наконец, паровоз, и весь лагерь с криком, визгом, свистом взял на абордаж два первых вагона поезда; абордажная прыть сохранялась до самой Москвы. Младшеклассников встречали родители, старшеклассников — далеко не всех, был рабочий день. Запросто, по-ребячьи скупо, мы попрощались и разошлись. Яшу встретила мама, у меня мамы уже не было. Никто не мог подумать тогда, что на пригородной платформе Киевского вокзала закончился путь нашей юности: через 10 месяцев началась Великая Отечественная война.

%%[[page_026]]
26

Роль идей, понятия морального, нравственного и общественного значения; прояснила отношения между людьми; усилила материальную зыбкость большого слоя людей, к которому принадлежала моя родня; война — это не только боевые действия, но и отношение к ней населения. Разность состояний времён закрепила в моей памяти происходившее в последний довоенный период. Была ли война (Вторая мировая и Великая Отечественная) неизбежна? Да, главная причина в том, что Гитлер, возглавлявший с 1921 года национал-социалистическую рабочую партию Германии (наци), германский фашизм в целом хотели привести мир к третьему рейху (третьей Германской империи) на основе расизма — официальной идеологии фашизма. Её основные положения: вождизм (фюрерство), тоталитаризм (государства с тоталитарным управлением); пангерманизм через геополитику, военная экспансия; антисемитизм; физическая и психологическая неравноценность человеческих рас и влияние этого на историю и культуру общества; высшие расы — создатели цивилизации должны господствовать, а низшие расы — быть эксплуатируемыми; использование идей социализма для привлечения народных масс к нацизму, фашизму.

75

Историческую справку о германских рейхах даю в приложении.

%%[[page_027]]
27. Противостоять этому могла только объединённая демократическая Европа вместе с Советским Союзом, но по ряду причин этого не произошло. Объединение цивилизованных демократических стран мира в антифашистскую, антигитлеровскую коалицию произошло уже в огне Второй мировой войны. Опоздание дало возможность германскому фашизму полностью вооружиться, что обошлось человечеству в десятки миллионов погибших; наибольшие потери в годы Великой Отечественной войны понёс Советский Союз: по разным оценкам, в войне погибло от 20 до 37 миллионов человек. Германские нацисты и их помощники-преступники в оккупированных немцами странах и на захваченной территории СССР успели совершить целенаправленную "Катастрофу" (Шоа) европейского еврейства: по данным международных комиссий, организаций уничтожено 6 миллионов евреев. Наша родня представлена во многих ликах войны, начиная с первого её дня. Два родных брата — мои двоюродные братья Розенбаумы Николай и Алексей, москвичи, жили в посёлке Алексеевское, ныне район напротив станции метро "Щербаковская" (в сторону ВДНХ). Розенбаум Николай (Калмэн) Яковлевич, 1920 года, рабочий завода "Калибр", был призван...

%%[[page_028]]
28. В армию ещё до начала Великой Отечественной войны служил красноармейцем под Львовом. Его мать — старшая сестра моей мамы — получила от него только одно письмо вскоре после начала войны, и всё. То письмо я помню. Он писал, что идут большие бои, успокаивал мать... Розенбаум Алексей Яковлевич, р. 1922 года, учился в техникуме, был призван в армию в первые декады после начала войны. Со дня призыва — ни весточки. Другой мой двоюродный брат — Розенбаум Александр Яковлевич (Шура), р. 1915 года, родной старший брат погибших Николая и Алексея, на фронте был солдатом-связистом. Он «тянул катушку», то есть прокладывал временные телефонные линии связи между воинскими подразделениями, таская тяжёлый от веса провода, но легко вращающийся ручной барабан, по форме и методу пользования напоминавший обыкновенную катушку ниток. Отсюда солдатская ирония — «тянул катушку»... Шура с войны вернулся, но прожил всего несколько лет; своей семьи у него не было. Трижды солдатская мать — Софья Абрамовна Розенбаум, моя тётя Соня, когда-то высокая, красивая женщина, совсем рухнула после смерти третьего сына. У старой женщины обострился диабет, ей отняли ногу. Судьба её сложилась тяжёлая. В начале 1920-х годов умер её муж. Она осталась с пятью детьми. Самый старший сын, Мотя (Матвей Яковлевич), родившийся до Первой мировой войны, после смерти отца оставил дом...

%%[[page_029]]
уехал из Ульяновска в Москву и начал работать; поддерживал мать. С её родственниками в тесных контактах не был. После смерти Шуры мог, его жена Нина - обаятельная женщина, их дочери Слава и Роза звали Соню жить у них, но она отказалась. Осталась с внучкой Бебой, которую воспитывала после смерти своей дочери Розы. Нашему роду не везёт на имя Роза. Из трёх женщин по имени Роза две умерли в молодом возрасте и от одной болезни - порок сердца. Это дочь тёти, а также Роза Величанская - тоже моя двоюродная сестра. К сожалению, давно ничего не знаю о троюродной племяннице Бебе Кауфман, о двоюродном брате Мотеле и его семье.

Не могу остановить трагические, хотя и краткие, воспоминания о Розенбаумах, не сказав о некоторых других фактах жизни моей родной тёти. У неё не было специального профессионального образования, как у большинства евреев её возраста, поэтому она могла быть в основном на неквалифицированных, низкооплачиваемых работах; в этом статусе её удерживало и плохое знание русского языка. Между тем у Сони было еврейское, преимущественно религиозное образование. Родители моей мамы успели до начала Первой мировой войны дать старшим детям такое образование. Соня изучала иудаизм, историю своего народа, знала еврейскую литературу; была религиозна. В Советской России это не могло быть источником существования. В конце 1920-х годов Соня с младшими детьми переселилась в Москву.

%%[[page_030]]
на свои финансовые крохи купила лачугу, буквально лачугу, в селе Алексеевском. Село Алексеевское и подобные ему окраинные захолустные районы расширявшейся Москвы представляли собой обиталища бедноты - новых москвичей. Это был начальный период индустриализации страны, первой пятилетки, расширения и реконструкции заводов и фабрик Москвы, строительство новых предприятий. Нужна была на всё это большая масса рабочей силы, столица страны всегда обладала особой притягательностью. В алексеевских селились люди из бесперспективных провинций, а также уехавшие от голода на Украине; крестьяне, сумевшие каким-то образом избежать коллективизации; разорённые непманы; отдельные люди из низших рядов церковной иерархии, уцелевшие в период идейного и физического разгрома религии; прочий люд, утративший свою социальную среду, а вместе с этим - материальную базу существования. Таким был замеслом нараставшего рабочего класса для ускоренного развития индустрии, в частности в Москве. Алексеевское представляло собой посёлок, застроенный бараками, чуть поодаль от них, ниже, начинались так называемые засыпные постройки и пристройки, самое дешёвое в то время жилье в Москве. Это вот что: плотники сколачивают двухрядный каркас жилья из досок. Промежутки каркаса заполняют сухой землёй, перемешанной с сухим песком. Внутренние стены обивают фанерой и оклеивают обоями. Такое жилье было только одноэтажным.

%%[[page_031]]
можно бы

Толок, он же крыша. Примечательно, доски, вплотную примыкавшие к потолку, «пришивали» так, чтобы через какое-то время их можно было отодвинуть и дополнить осевший слой земли, (чем ни «женщина в песках»?).

С родителями летом, в выходной день, я иногда ездил в гости к Розенбаумам. Запомнился путь от трамвайной остановки: надо было пройти через весь барачный посёлок. В выходной день жизнь кипела на улице: гармошки, частушки, песни, плясовые, конечно, «Барыня», всё это - возле разных бараков, потому сливалось в одну нестройную, удалую песенно-гармошечную мелодию; кто-то полоскал бельё возле водонапорной колонки; другой в резиновых калошах пробирался к туалету. Асфальта на окраинах тогда не было, в лучшем случае - деревянные настилы по длине барачной улицы, а чаще всего - утоптанная земля. Встречал нас Коля, высокий, изящный, интеллигентный юноша: у него был характерный еврейский нос: узкий, с большой горбинкой. Он работал на заводе и учился в системе фабрично-заводского обучения (ФЗО - ФЗУ-училища; на их базе создали техникумы). Лёша на вид был грубоватый парень, широкоплечий, физически очень сильный; больше всего любил сидеть дома и читать. Помню, Соня жаловалась моей маме: нет сил заставить его куда-нибудь сходить. Ко мне они относились, как к «сильно младшему», подчёркнуто внимательно: всегда занимали рассказом о чём-то из прочитанного. Встречались мы не часто, поэтому мне всегда было интересно с ними. Близ-

%%[[page_032]]
кий родственник как-то связал твой лёшка вместо нижней рубахи книжки на теле носит, работал бы лучше. Соня разозлилась: не твоё депо; если ты остался неучем, не значит, что мои дети должны быть такими же; помощи у тебя не прошу. Соня пожинала семейную встречу. Упрёком «книжки» можно было только гордиться, обидел тётю намёк на её нуждаемость, которую она переносила сама. Я объелся Сониного угощения: большая миска с тёплыми картофельными оладушками и сладкий чай, да ещё с вареньем. В своей засыпушке она прожила до послевоенных лет, а когда вдоль Ярославского шоссе началось грандиозное жилищное строительство, Соне дали двухкомнатную квартиру, в которой она жила с Бебой. Моя редкая (упрекаю себя) и возможно последняя встреча с Соней была незадолго до её смерти. Встреча оставила огромное впечатление, и я решил, после многих колебаний, рассказать о ней. Моя тётя своей фигурой, своей сущностью всегда воспринималась как монолит, как нечто духовно цельное; все знали, что она религиозна, но не все знали, что в религии она находила духовную опору. В тот день я в этом убедился. От Сони осталось только имя: исхудавшее, измученное страданиями лицо, отощавшие, обессиленные руки, которые плохо справлялись с костылями. Беба сказала, что бабушка молится, но это было больше (на мой взгляд), больше, чем прочтение напечатанной молитвы, это был преисполненный доверия, искренний монолог, обра...

%%[[page_033]]
33. щённый к Всевышнему, она говорила в таких обычных разговорных интонациях, будто сидит напротив собеседника и в чём-то его убеждает. Мы молча сидели на кухоньке и слушали. Она часто произносила готеню (в языке идиш окончание Ю при обращении к кому-либо имеет значение исключительности, ласкательно-превосходящую форму). Соня спокойно, без эмоционального надрыва, размеренными убеждающими тонами благодарила его за то, что он дал ей силы перенести смерть детей (я сразу обратил внимание на слово ибертруђен — перенести; она не сказала иберлейбен — пережить, смысловая разница очевидна); Соня благодарила за данные ей силы и крепость души (штарка нешума), которые помогали ей справляться с жизненными невзгодами. А сейчас моя душа растерзана (ди нешумэ ист церисен), я обессилена, всем тяжело со мной (алемэн ист швер мит мир), у меня нет будущего на этом свете (их обништ кайн офенунг аф дейм велт), прошу Тебя, забери меня из этой жизни, пожалей меня; Ты всегда в моём сердце (Дубист эйбиг ин майн арц). На последних словах Соня разрыдалась. Я вошёл в её комнату после бебы. (Не знаю, передалась ли в кратком воспроизведении, да ещё по памяти, вся религиозность мольбы тёти Сони). В селе Богородском далеко за Сокольниками в районе завода «Красный богатырь» в однокомнатной квартирке барачного дома жила семья младшего брата моей ма...

%%[[page_034]]
мы, моего родного дяди Симы. Октябрь 1941 года, особенно его первая половина, был решающим в судьбе Москвы, возможно и всей страны. Вермахт — вооружённые силы фашистской Германии медленно, фактически беспрепятственно продвигались к Москве. Государственная и промышленная Москва эвакуировалась; положение москвичей ухудшалось. Медленное продвижение вермахта объясняли потом тремя причинами: опасениями западни, ловушки со стороны Красной Армии; отставанием в продвижении вспомогательных служб немецкой армии; переброской части сил Вермахта с Московского направления на северокавказское. В один из опаснейших для Москвы дней сам Сталин стоял в думах перед готовым к отправлению спецпоездом. Если бы Сталин покинул Москву, это могло бы, скорее всего, означать сдачу города.

Если предположение о западне действительно имело место, то оно, к нашему счастью, оказалось совершенно ошибочным: в октябре (и до этого) Красная Армия могла вести только оборонительные бои; будущий маршал Советского Союза Г.К. Жуков по заданию Сталина выезжал на некоторые подмосковные командные пункты высших командиров, но не всегда находил их там. Хотя мы не знали тогда о спешке и разочарованиях Жукова, всё равно — ситуация осложнялась. Москва быстро становилась прифронтовым городом: на больших улицах и магистралях устанавливали противотанковые ежи, создавали баррикады, преимущественно из мешков с песком, землей, такими же материалами.

%%[[page_035]]
35. шками закладывали нижние этажи некоторых административных зданий и крупных магазинов. Назначение двойное: защита от обстрелов, а при уличных боях — укреплённые огневые точки. Недалеко от нашего барочного дома в Коптево в районе завода имени Войкова, в створе с школой № 204, в которой училась Зоя Космодемьянская, на большом колхозном картофельном поле установили одну из зенитных батарей; четыре удлинённых ствола мощных 85-миллиметровых пушек грозно и обнадеживающе смотрели в небо, иногда ночью оно прощупывалось прожекторами; в поздние вечерние часы над Москвой стали поднимать первые аэростаты воздушного заграждения. Приказано было сдать на хранение радиоприёмники, чтобы враг не мог настроить свои радиопередатчики на радиоволну советской радиостанции. Самая известная тогда была — имени Коминтерна. Наш новый «СИ-235» я отнёс на почту — это был первый советский бытовой приёмник с прекрасным звучанием. Всё слушали по прямой трансляции через репродуктор — настенную тарелку. В сводках новостей говорилось о городах, которые Красная Армия оставила после упорных боёв, о зарождении партизанского движения. Известные драматические актёры часто читали «Бородино» Лермонтова, большие отрывки из «Войны и мира» Льва Толстого об оставленной Москве, организованных пожарах — обо всём, что соответствовало духу двух отечественных войн. Всё звучало актуально, часто передавали шестую симфонию Чайковского.

%%[[page_036]]
рваться

Было невозможно: беспокойный, мятущийся, а временами призывный образ, создаваемый симфонией, не успокаивал, потому что ударные смысловые фрагменты ассоциировались с ударами войны осени 41-го. Москву явно готовили к различным поворотам её судьбы. Никто не выключал радиорепродукторы на ночь. В помощь армии начали создавать народное ополчение; в Москве и области организовали 16 таких дивизий. В одну из них 5 октября 1941 года был призван мой родной дядя Величанский Семён Александрович (Шимон), р. 1903 года. Первые пять дней ополченцы находились и обучались военному делу в здании школы в самом же посёлке Богородском. Отца навещала старшая дочь Рая; младшей было 2,5 годика. На шестой день моя двоюродная сестра Райка пришла навестить отца, но школа была безлюдна: ополченцы уехали на фронт. Ни одной весточки от дяди Симы семья не получила — он погиб или, того хуже, попал в плен, как и большая часть ополчения в боях под городом Ельня. В похоронке (почтовой открытке) говорилось, что пропал без вести; спустя время выяснилось, что так писали об известных убитых. Ельня — примерно 300 км на юго-запад от Москвы в сторону города Смоленска.

О числах октября: 5 октября + 6 дней = 11 октября; из Москвы ополченцев до определённого пункта в сторону фронта доставили транспортом, потом начались сложности фронтовых дорог, включая бомбёжки вражеской авиацией движущихся армейских колонн (в то время в воздухе госту).

%%[[page_037]]
подступала немецкая авиация); остановки для укрытия от авианалётов, остановки привалы по другим причинам; необходимость вовремя прибыть к месту назначения осложнённая реальной скоростью движения людей сверхпризывного возраста, повышенная требовательность в связи с этим к солдату (не отставать), общая нервозность; неотложное по тем условиям введение ополченческих частей в боевые действия; думаю, что ополченческая часть из посёлка Богородское прибыла на фронт не позднее 13 октября; к 15 октября вермахт разбил плохо подготовленные ополченческие части, оборонявшие Москву, поэтому 16 октября 41-го было для Москвы критическим, можно сказать паническим, днём, один из дней периода 13-15 октября, чего считаю днём гибели дяди Симы. 2. Сын дяди Симы, мой двоюродный брат - Величанский Борис Семёнович (8.11.1924-7.06.1994) после призыва в армию был направлен в город Дзержинск, Горьковской области на военное техническое обучение. Как бывало в первой поло...

%%[[page_038]]
вине войны, поток курсантов снимали с обучения и отправляли на фронт. Профиль обучения Бори я не знаю. (В редких послевоенных встречах мы обсуждали немалые текущие проблемы. Возможно, что и он оказался в такой же ситуации. В год призыва, в 1942-м, он был шофёром 65-го автополка; 28 июля 1943 года под хутором Западный Боря был ранен. (Это Орловско-Белгородская операция: 5 августа 1943 года Советская Армия освободила областные центры, города Орёл и Белгород). По найденному обгоревшему Бориному нагрудному медальону решили, что он убит, сообщили матери. Представьте мать семейства с двумя похоронками и двумя детьми. Когда Боря оказался в стационарном госпитале в городе Муроме, он написал домой. Рая причалась к брату. После выздоровления - опять фронт (1944/45 год), 36-й автополк. Боря Величанский награждён орденом «Отечественная война» второй степени; медалью «За Победу над Германией» и другими медалями. После окончания войны и демобилизации его личная жизнь, к сожалению, сложилась труднее обычного: отец погиб, мать - в тяжёлом материальном положении; две сестры.

%%[[page_039]]
то

рички-младшей Люсе всего семь лет, в солдатском кармане — дырка, одежды — всего-то в чём ушёл из армии, от государства социальной поддержки — никакой. С чего начинать? Здесь стоит сказать о Борином характере. С детства его отличало высокое самолюбие, почвой для которого было постоянное стремление самостоятельно решать все свои проблемы; независимость он ценил превыше всего; до властности, деспотизма не дошёл. Но его мама, передав сыну эти характерные для неё качества, пыталась бороться с ними в сыне: она пыталась подчинить его своей сильной воле. В детстве и позже, до войны, мы дружили, помню схватки характеров; у мальчика не хватало образовательной аргументации для убедительного противостояния не доказательству, а своеобразному насилию, и его, тоже упорный, характер подсказывал отрицание. По мере взросления Бори столкновения становились реже, но личных отношений не повысился до баланса теплосемейного уровня.

У Бори была хорошая черта: он никого не считал себе обязанным; она ярко прояви...

%%[[page_040]]
лась в трудный послевоенный период. У моего брата были три варианта пути к будущему: закончить школу, получить аттестат зрелости и поступить в институт. Не на что было жить, тем более что надо было помочь вырастить младшую сестренку; учиться и работать такой нагрузки не позволяло здоровье, ослабевшее после войны. Оставалось работать по специальности, которую получил в армии: он пошёл водителем автобусов московских городских линий. Многолетний труд закончился инфарктом, Борю перевели на диспетчерскую службу до пенсии. Со временем водители московских автобусов стали хорошо зарабатывать, и Боря «поднялся». Женился он поздно, зато очень удачно: Елизавета Файбишевна - уравновешенная, рассудительная и хозяйственная женщина. Они прожили в согласии и спокойствии. Наконец, Боря пришёл к такому состоянию; помню их ухоженную квартирку в Тушино, своим видом она дополняла характеристику их жизни, рано оборванной Бориной смертью. Семья дяди Симы была несколько в стороне от обычного, не навязчивого в нашем...

%%[[page_041]]
родственного общения. Причина тому - пламенная страсть и последующее известно, что Сима был самым красивым, пользовался успехом у женщин, но голову не терял, кроме единственного случая, с которого началось... Как в жизни Сима становится отцом, он поден энтузиазм работы, можно хорошо зарабатывать и развивать семью (не торопись, парень). По мере вхождения в бытовое русло, естественного проявления характеров и взглядов супруги стали духовно отдаляться друг от друга. Сима родился и большую часть жизни (на тот период) прожил в большой еврейской семье; он был в те годы носителем

%%[[page_042]]
телем еврейской ментальности, в которой забота о семье, о душевном внутрисемейном тепле - не на последнем месте; Российская провинция, русский язык, новации времени были для него новым образом жизни, который молодой человек только осваивал. Можно сказать так: он был эмигрантом в молодой Советской России из западных районов бывшей Российской империи. У Сименой Доры - всё наоборот, она была коренной жительницей России, легче разбиралась в ситуациях того сложного времени; своё преимущество сделала средством управления, а не помощи мужу. Еврейские традиции, ментальность Доре были, как говорится, побоку, хотя не отрицала как факт, имеющий к ней какое-то отношение: родившемуся сыночку брис (брит милу) сделали, ещё в 1994 году не сделать брис - это вызов, его она не хотела. Косенности складывающихся отношений в молодой семье родня улавливала, но никто не вмешивался и симпатий к родственнице не.

%%[[page_043]]
4.3. накапливал, а она, в ответ, со всеми была ровна, тактична, симпатии - далеко не всем. Дора Борисовна Величанская (Шуф) была пикантна, невысока ростом, хорошо фигурой; когда встречался прямым взглядом её красивыми чёрными глазами, то возникало ощущение, что они насквозь прочитывают меня: она была начитана и даже самообразована, жила пониманием времени, что сочеталось у неё с идеями феминизма; ко всему она ещё курила, правда, очень изящно. Исходя из названных и других фактов, могу с определённой долей иронии утверждать, что если бы феминизм - движение за права женщин - не пришёл с Запада, то он пришёл бы туда из Посёлка Богородское; экспериментальной площадкой для отработки прав женщин была, конечно, семья, поэтому дядя Сима давно и безнадёжно махнул рукой на папиросный дым идей своей жены, никогда не спорил, а утыкался с головой в газету, когда же ему осточертевали выкладки, он вскакивал и, как бы спохватившись, говорил: "Ох, забыл воду на завтра принести", или не...

%%[[page_044]]
что подобное, и уходил. Помню это потому, что в дни школьных каникул нередко гостил у них по 3-4 дня. Боря и Рая были душевные, открытые ребята, с удовольствием общался с ними (иначе бы не приезжал); у них были увлекательные игры и юношеская классика, которую тётя Дора приносила из библиотеки, где работала. На одном дыхании, вслух, помнится, мы прочитали «Всадник без головы»; Кассий Калохаин всё ещё сидит в моей голове. Нарасхват была газета «Пионерская Правда», в которой печатались отрывки ещё не вышедшей отдельной книгой повести «Тайна двух океанов». В период 30-х годов входила в широкий школьный быт также романтическая приключенческая литература Запада, большими тиражами выходили произведения Фенимора Купера, Жюля Верна, Альфонса Додэ, Александра Дюма, Джонатана Свифта, Джека Лондона и т.д. То было время развития культа книги, социалистической культуры в целом, как и время последовательного укрепления политического культа Сталина; культы взаимодействовали.

%%[[page_045]]
Памятными в гостевании усадьбы Симы остались для меня и беседы с тётей Дорой, хорошо помню обстановку, условия, в которых проходили беседы. Семья жила в одной комнате, угловой и очень светлой. Стена, к которой примыкала входная дверь, выполняла функцию кухни: вплотную к ней стоял небольшой кухонный столик, над ним полки, закрытые тряпочной занавеской, а рядом со столиком низкая скамеечка, на которой стояли два ведра с водой. Ещё ближе к двери - табуретка для сидения - моё место, когда тетушка стояла у столика и на единственной керосинке готовила еду на предстоящие два-три рабочих дня. Холодильником служили несколько ровных кирпичей, сложенных в два ряда, они лежали в уголке холодных сеней. Тут же стоял прочный сосновый ящик, сколоченный дядей, для хранения бутыли с керосином. Стояла и старая раскладушка, которую ненавидели. На гвоздиках висели полотняные мешочки с запасом овощей, зимой лежал запасец дров на пару топок. Войдёшь, бывало, и улавливаешь запахи скудной жизни. Итак,

%%[[page_046]]
Я добрался до своей табуретки. Впрочем, этому частенько мешал ребячий крик с улицы: «Борька, выходи, Фимка, лапту играть». Фимка иногда великодушно отпускал Борьку, а сам всё ж таки усаживался на табурет. Разговор начинался, например, так: «Тётя, там вашими очками проложена книжка. Книга Гамсуна. Имя слышал, а больше ничего не знаю, расскажите». Если очки лежали ближе к концу книги, значит, она одобрена, и пересказ, и особенно смысловая, морализующая часть будет соответствовать какой-либо актуальной идее, поддерживаемой тётушкой. Кнут Гамсун — известный норвежский писатель, лауреат Нобелевской премии 1920 года, почитаемый еврейкой Дорой, как говорят по-русски, скурвился: он отказался от своих демократических взглядов, а в период Второй мировой войны примкнул к фашизму, за что после победы был отдан под суд и осуждён. Тётя Дора не была обывательницей на этой стороне жизни, ей не везло, отчасти подводило самомнение, даже самонадеянность, источником которых...

%%[[page_047]]
Помимо характера был большой духовный мир, и в этом состояла её привлекающая сила. Она, как и многие люди нашего окружения, верила (до периода больших разочарований) в идею социализма; некоторые из них были особенно близки, и она пропускала их через свой творческий ум. Присущемуся в ней феминизму она привязывала идею дружбы народов СССР, а дружбу связывала с идеей культуры — социалистической по содержанию и национальной по форме, причём социалистическое считала фактором постоянным, национальное — временным, уходящим. Дядя Сима, хорошо знавший историю еврейства и откликавшийся на разговоры о национальном, утверждал обратное: национальное глубоко сидит в душе человека, приводил примеры-сопоставления, оно никогда не умрёт. Дора не отрицала историю, но была убеждена, что победой идей социализма отомрёт значение истории народов и в социалистической стране с новым типом государственного устройства все народы СССР сольются в один новый социалистический народ.

%%[[page_048]]
не соглашался: народы одной страны могут объединяться в общих интересах ее развития совместной работы, но ни один народ не откажется от своей истории, создавшей национальную традицию, воспитавшей национальный характер; оскорбление национального достоинства любого человека остро задевает его самолюбие. Я обобщил содержание разных мнений, чтобы сказать об особой их актуальности в связи с тем, что к середине 30-х годов прошлого века Гитлер, его человеконенавистнические и антиеврейские намерения уже были известны миру, поэтому популярная идея дружбы народов СССР была серьёзным антитезисом нацистской пропаганде. Отойдя от этих лет немного вперед, должен с сожалением сказать, что идея дружбы народов начала рушиться в первые же дни войны, когда в захваченных регионах СССР немцы нашли пособников и соучастников по уничтожению евреев. Поездки в Богородское становились реже после смерти моей мамы. Помню случаи, когда рано утром вместе с дядей Симой уезжал домой: на трамвае - до Сокольников,

%%[[page_049]]
шли на остановку напротив пожарной каланчи, вместе дожидались нужного мне номера трамвая, я продирался к окну кондуктора, махал дяде, он отвечал скупым взмахом руки и уходил на работу. Уходящим, со спины вижу первым кадром своего дядю, когда вспоминаю его.

В ноябре 2005 года Люся прислала мне письмо и фотографию с изображением трех мужчин, фото было из альбома матери; Люся спрашивала, кто из троих её отец. Интуиция не подвела её, однако хотела знать точно. В фотомастерской вскоре я получил три прекрасно выполненных экземпляра отдельного портрета отца Люси — моего дяди; техническая сторона заданного вопроса решилась просто. Недоумение вызвали соотношения фактов во времени: как можно было годами держать фотографию мужа и не показать её взрослой младшей дочери, которая по причине войны никогда не видела родного отца? Ответ неожиданный и достаточный следует из содержания того же письма, хочу его процитировать:

%%[[page_050]]
«Трагедии судеб сквозят меж его строк. «Посылаю тебе фотографию (с возвратом), может, ты знаешь, кто на ней? Мне кажется, что первый справа (как смотришь) — мой папа. Кто остальные, может, ты знаешь. На обороте фотографии я написала, она на что-то была наклеена, не я отрывала, я её обнаружила в альбоме у мамы. Мама никогда её не показывала, а может, я не помню? Мама не любила рассказывать о прошлом, поэтому я ничего не знаю о своих предках, да и Рая, думаю, тоже. Я выросла русской по паспорту, еврейкой по душе. Сейчас я часто хожу в ближайшую синагогу, заказываю там йорцайт по маме и папе каждый год, слушаю молитвы, пение кантора. Завершается служба всегда молитвой за Государство Израиль и пением «Атиквы»... Приятно душевно. Таких, как я, пятеро. Приходят в синагогу 50-70 англо-еврейских женщин. Вот так.»

%%[[page_051]]
Почему такойtragический финал отношения к моему дяде? Это же отрицание его присутствия в семье, забвение памяти о нём. Я категорически против, не думаю, что причина в разных мнениях на отдельные общественные кампании, проходившие в стране. Из своего «табуреточного сидения» помню их расхождения по поводу атеизма. Несмотря на начальное образование (двенадцатилетний мальчик кончил учиться с началом Первой мировой войны), дядя Сима не был религиозным человеком, но и не соглашался с атеизмом; тётя Дора была атеисткой, однако осуждала осуществлявшийся в СССР разгром религии и пропагандировавшуюся неприязнь к ней, поэтому тётушка заинтересовалась и научным атеизмом; от неё я впервые услышал о Гольбахе и Фейербахе. Все обсуждения расхождения проходили только в стенах своего дома, возникали чаще всего после статьи в газете или радиопередачи.

%%[[page_052]]
52.
ная семейная ситуация, когда по серьезным и даже второстепенным вопросам нет единого мнения. Что же было в семье моего дяди? А было то, чего не было — ни любви, ни дружбы, возникли две ошибки молодости и следствие — ответственность перед детьми; третья трагедия — сама Люся: только на седьмом десятке жизни она удостоверилась в подлинном облике своего отца; душевный и духовный дискомфорт, который она многие годы испытывала в связи с разностью между записью в паспорте и чувством национального достоинства, был по причине того, что Дора сделала дочке русский паспорт; родня знала и не одобряла. Возникают и другие вопросы, но главное в том, чтобы выполнить моральную обязанность, которую я сам возложил на себя: снять несправедливое забвение с памяти о родном саде, с памяти о погибшем солдате.

%%[[page_053]]
Из редких послевоенных встреч с тётей Дорой запомнилась поездка к ней с моим старшим братом. Приехали в город Дмитров (Подмосковье), где она жила и воспитывала маленького внука — крепыша Стасика, сына Люси. Мы привезли ему пару заводных игрушек, чем обеспечили ему большую занятость, но содержательности общения взрослых это не помогло. Встретились с искренней радостью, сердечностью. Замечательно стареющая тётя Дора мало изменилась в своём внешнем образе, в голосе, чувствовалось и в твёрдости характера; это ещё больше приближало далёкие времена, далёкие не столько по количеству лет, сколько по их грандиозному значению. Всё было ясно, незачем теребить воспоминания о рухнувших надеждах, идеалах. Говорили о «сегодняшнем дне», быте; бедненькая квартирка тётушки и она сама как напоминание о широко мыслящем человеке — всё это ассоциировалось с музеем: экспонаты достоверны, но уже не действуют, они омертвели, осталась только память о человеке и о времени, обманувшем его.

%%[[page_054]]
На фронте погиб ещё один мой двоюродный брат Величанский Лев Наумович. Он родился в городе Ульяновске в конце 1925 года (или в начале 1926 года), он был моложе меня значительно, как мне казалось. Старший брат моей мамы, Наум (ноях так называла его вся родня), из Ульяновска перевёз свою семью в Смоленск, а потом в Тамбов, где жил самый младший брат Мотя. Из города Тамбова Лёвочка ушел в армию, он был танкистом; «Погиб в бою», — сказано в книге памяти воинов-евреев, павших в боях с нацизмом. 1941-1945», том 5, стр. 376. Строкой ниже сказано о дяде Си...

Вышло девять томов книги памяти..., предполагается десятый. Сколько же погибло воинов-евреев, если понадобилось издать пять томов, чтобы дойти всего лишь до третьей буквы русского алфавита, с которой начинаются фамилии с буквенного сочетания "Велич"? Оказалось, что в это многотомье не включены...

%%[[page_055]]
Все имена павших воинов-евреев, например, нет сведений о моих братьях Розенбаумах. Данное издание - израильско-российское: Публикации в девяти томах подтверждены Центральным архивом Советской Армии при Министерстве обороны Российской Федерации. По вопросам гибели советских воинов в годы Великой Отечественной войны следует обращаться в этот архив. Лёвиному отцу - Науму Абрамовичу Величанскому к началу Великой Отечественной войны было уже за 50 лет, однако его мобилизовали, служил он во вспомогательных прифронтовых частях; вернулся с войны невредимым. Его старший сын Абрам Наумович Величанский (Котик) был старше меня, служил офицером на Дальнем Востоке, затем в Северной Корее; после демобилизации жил в Москве. Нарицательное имя Котик дала своему первенцу мать; оно прижилось в кругу родни и утратило сущностное значение. Жена Мила пыталась переименовать мужа в Костю, но безуспешно. С Лёвочкой я общался редко и только в довоенные годы: один раз летом ездил с бабушкой в Смоленск, потом дядя Ноях несколько раз приезжал в Москву с сыновьями. В моей памяти Лёвочка остался любознательным, очень подвижным, добродушным взрослым мальчиком.

%%[[page_056]]
В 1931 году наша семья породнилась с большой семьей Хаима и Эстер Нибощиков. Мы жили на Новосущёвской улице, 29, а Нибощики — почти напротив, в старой двухэтажной развалюхе, на втором этаже, куда вела мрачная, не освещавшаяся лестница. (На месте того дома и подобных ему на той стороне улицы ещё до войны построили один из корпусов расширявшегося Московского института железнодорожного транспорта.) С двумя сыновьями и тремя дочерьми Нибощики переехали в Москву из Минска; там остались самые старшие дети — Рафаил и Рахель. Хаим работал наборщиком в типографии какой-то крестьянской газеты; Эстер, будучи профессиональной дамской портнихой, подрабатывала дома. На этой почве моя бабушка познакомилась с ней, и вскоре обе старушки (в моём тогдашнем понимании) стали близкими подружками. Я очень хорошо помню мадам Эстер, потому что у моей бабушки было плохое зрение, она остерегалась...

%%[[page_057]]
57. Тёмной лестницы и часто просила меня проводить её. Нас встречала немного суетливая, радушная, высокая, полноватая женщина в больших круглых очках и с портновским сантиметром на шее. Швейная машина была открыта, лежали какие-то материалы; машина стояла между окном и маленьким буфетиком. Вскоре мадам Эстер, так называли её моя мама и тётя Лия - жена младшего маминого брата Ефима, открывала малюсенькую дверцу буфетика, клала на блюдце несколько карамельных конфеток «Подушечка» и добрым жестом руки угощала меня. Если Люся - самая младшая дочка Нибощиков, моя ровесница, была дома, мы отходили в недалёкий угол небольшой комнаты, о чём-то болтали и слопывали подушечки - роскошь бедноты. (Люся доживает свой век в доме для престарелых; я бы хотел, чтобы ей прочитали всё, что относится к жизни семьи её родителей. Думаю, она подтвердит достоверность моих воспоминаний. Единственное, что уточнил имя Эстер: с десятилетиями оно трансформировалось в моём сознании в имя Эсфирь. Однако это разная транскрипция одного и того же имени одной и той же женщины еврейки, персидской царицы, которой мы обязаны праздником Пурим).

%%[[page_058]]
Итак, старушки решили женить своих детей. Моя бабушка опасалась, что её двадцатидвухлетний сын Янкель — весёлый, работящий парень — попадёт в плохие руки, а мадам Эстер беспокоилась, что у её старшей дочери Берты — элегантной и красивой девушки — нет достойного жениха. Старушки "взвесили" каждого и пришли к согласию. Семьи познакомились поближе, бабушка запустила в ход свою команду: мою маму как средство внимания к слову любимой сестры, тетю Лию — искусного мастера тонкой женской дипломатии. Как рассказывала мама, ребятам долго гулять не дали. И вот, тёплым летним вечером 1931 года я присутствую на настоящей еврейской свадьбе в старой деревянной синагоге в Марьиной Роще. Старушки скроили замечательную семью, с которой мы прошли всю жизнь, о чём непременно расскажу. А пока из послесвадебного периода отмечу факты: 2 мая 1933 года у молодых родился мальчик. К началу войны в Германии Шурику было 8 лет. Соня, младшая сестра Берты, за пару лет до войны вышла замуж, родила мальчика и в первой половине июня 1941 года повезла его в Минск показать родственникам. Берта тоже собирала Шурика в поездку с Соней, но у Берты не было денег на покупку сыну нового костюмчика, и она отменила его поездку.

%%[[page_059]]
покунученика у тинькой родне. Небожчик Рафаил, р. 1905 года - старший брат моей тёти, жил в Минске, был призван в Красную Армию. В 1941 году пропал без вести на фронте Великой Отечественной. О сёстрах Рафаила в частности, я писал в январе 2003 года в Институт мемориал «Яд ва-Шем». Фрагмент письма: Отделу регистрации евреев, погибших от нацизма в годы Второй мировой войны, сообщаю имена моих близких и дальних родственников: Рахель Гольдблат, рождения 1907 года, минчанка. Погибла в Минске 28 июля 1942 года; Соня Гольдблат, рождения 1918 года, москвичка. Погибла в Минске 26 июля 1942 года; её годовалый младенец Сашенька Гольдблат, рождения 1940 года, москвич. Погиб в Минске 28 июля 1942 года. За несколько дней до начала войны Соня приехала в Минск показать своего первенца родителям мужа, сестре и др. родственникам минчанам. Сёстры-однофамилицы потому, что их мужья были между собой родными братьями. Бабушка и дедушка Сашеньки тоже погибли в Минске 28 июля 1942 года... После проверки фактов имена погибших внесли в сайт «Яд ва-Шем»: www.yadvashem.org.il names@yad-vashem.org.il Спустя месяцы после этого звонит мне женщина из Канады. После объяснений трудностей в поиске моего телефона...

%%[[page_060]]
на и выяснения цепочки родственных связей она рассказала, что является дочерью от второго брака бывшего мужа Сони: он вернулся с войны, 17 лет жил один, потом женился. В разговорах новой семьи бытуют воспоминания о Соне, Сашеньке. Сейчас папа очень стар, дочь не хочет тревожить его полученной информацией. Элла сообщила мне свой номер телефона и уточнила: малышу было тогда два годика; родной племянник Сони - Шурик Величанский созвонился с Эмой. На этом ограничить воспоминания о Минской трагедии нельзя, потому что о бесчеловечном завершении её мы узнаём, даже трудно представить, от Хайма Нибощика. Однако сперва несколько слов в память

%%[[page_061]]
на и выяснения цепочки родственных связей она рассказала, что является дочерью от второго брака бывшего мужа Сони: он вернулся с войны, 17 лет жил один, потом женился. В разговорах новой семьи бытуют воспоминания о Соне, Сашеньке. Сейчас папа очень стар, дочь не хочет тревожить его полученной информацией; Элла сообщила мне свой номер телефона и уточнила: малышу было тогда два годика; родной племянник Сони - Шурик Величанский созвонился с Эллой. На этом ограничить воспоминания о Минской трагедии нельзя, потому что о бесчеловечном завершении её мы узнаём, даже трудно представить, от Хайма Нибощика. Однако сперва несколько слов в память

%%[[page_062]]
об этом милейшем старом человеке: Своим тактом, всегда к месту найденным словом, советом, почерпнутым из традиционной еврейской мудрости, ощущением покоя, которое появлялось при разговоре с ним, - всё это сделало Хаима Нибощика безусловно уважаемым человеком всей новой родней. Худой, можно сказать, - отощавший, сгорбленный старик, в неизменно старом тёмно-сером костюме с узелком галстука, завязанным, очевидно, ещё живой Эстер, нёс в себе груз больших утрат. Через несколько лет после войны, когда миграция населения как-то отрегулировалась, он решил поехать в Минск попытаться узнать о гибели близких. Ему удалось разыскать дальнего родственника Лазара Чарного, который

%%[[page_063]]
скать дальнего род 62 служил в партизанском отряде, базировавшемся в пригородах Минска. Рассказ Лазаря дедушка Хаим пересказал своему родному внуку Мурику Велиганскому, когда он стал уже зрелым юношей, а Шурик — мне, 217. Лазарь проникал в еврейское гетто Минска, чтобы организовывать единичные побеги в партизанские отряды. Он встретился с Рахелью и Соней, предложил им попытку побега в партизанский отряд (Рахели было 35 лет, Соне — 24), но при условии, что сына Соня оставит в гетто: плач или крик малыша могут помочь немцам обнаружить отряд. Соня сказала, что не сможет своими руками отдать ребёнка фашистам; Лазарь предложил побег Рахели одной.

%%[[page_064]]
Ответ: в этой беде она Соне мать и останется с нею до конца. Он наступил в виде грузовиков-фургонов. Укрупненные команды нацистов и их пособников насилием загоняли евреев в металлические фургоны, набивали их людьми до отказа, захлопнутые двери ставили на запоры с внешней стороны, включали двигатели грузовиков и душили евреев выхлопным газом, поступавшим в кузова фургонов по шлангам, протянутым от выхлопных труб. Можно полагать, что бывший партизан пожалел старика и не рассказал ему, что облазело людьми, когда они воочию увидели распахнутые перед собой двери в Смерть. Считаю справедливым сперва сказать о погибших, а потом хронологии жизни рода; и здесь,

%%[[page_065]]
не без войн, причем, всякого толка.
Клоц Давид Абрамович (28.8.1915-08.10.1971). Мой родной, старший брат. На факте рождения младенца отразилась история евреев Российской империи. Местом традиционного проживания моих предков был город Брест-Литовск. [город известен по Брестской унии 1596 года - о признании православной церковью Украины и Беларуси своим главой Папы римского, но при сохранении православной обрядности. Уния расторгнута в 1946 году; по Брестскому миру - 1918 год; Брестской крепости = герою - начало Вел. Отеч. войны.]
По тем понятиям город был большой. Весьма значительную часть его составляло польско-литовское еврейство. Мои родители женились в 1911 году и счастливо прожили до конца лета 1914 года, до начала Первой мировой войны.
Ряду неудач на русско-германском фронте предшествовали относительно недавнее поражение России в русско-японской войне 1904/5 года, а в связи с поражением - обострившаяся революционная обстановка в 1905-1907 годах и последующие рецидивные события.
Трудно понять, как можно было после потрясений в Российской Империи за период после 1904 года дать втянуть себя (через короткое время!) в новую, изначально большую, войну; протянуть руку своей гибели. О том времени впечатляюще рассказано в забытом, по существу документальном, романе А. С. Новикова-Прибоя "Цусима". За потрясения,

%%[[page_066]]
политические провалы власти взыскали с евреев в многочисленных погромах. И в неблагоприятной военной обстановке 1914-1915 годов тоже зацепили евреев тем, которые жили или работали в прифронтовой полосе, куда входил и Брест-Литовск, приказано было покинуть места обитания. На сей раз без насилия обошлось.

В придирках к евреям первая причина была величиной постоянной - никем не защищаемая нация; вторая величиной переменной - повод, над поводом власть не задумывалась, его не назвали и в требовании покинуть Брест-Литовск. В еврейской среде предполагали, как рассказывал мой папа, что поводом стал бытовой еврейский язык - идиш, который сродни немецкому. Следовательно, евреи могут передавать немцам сведения о русской армии... (Религиозным языком является древнееврейский, иврит). Среди местных жителей - поляков, литовцев, русских и даже полуоседлых цыган некоторые знали идиш. Большие базары, деловые и личные общения были в те времена своеобразными инязами. Однако выселяли евреев. Предполагалось также, что слух об идише подкинула сама власть.

Наступило время собираться в дорогу. Продать имущество - дом, дело (т.е. бизнес), хозяйство, корову, козу, всякую утварь было почти невозможно: уезжала большая часть города и пригорода; всех настораживал факт войны, некоторые не торопились покупать, понимая, что евреям так или иначе придётся всё бросить. В то же время еврейские семьи начали покупать лошадей, фуры, телеги, большие брички; многие стали учиться у извозчиков обращаться с лошадьми.

%%[[page_067]]
Детьми, упряжью и т.п. В один из августовских дней 1915 года семья Величанских, к которой по рождению принадлежала моя мама — Гита Абрамовна Клон (1892-8.10.1939), тронулась в неведомый путь, на восток, вглубь Российской империи. Плетёные корзины — сундучки (прообразы чемоданов); скатерти, занавеси, простыни, превращённые в узлы с домашним скарбом, положили в подводы. Поверх клали узлы с мягкими вещами, чтобы на них посадить слабых. Таких в большой семье было немало: мой прадед Ефраим (по маминой линии), которому было далеко за 90; старшая мамина сестра Соня (Сура) с маленьким мальчиком и с несколькомесячным младенцем (Шурой Розенбаумом); два самых младших, шестилетних, сына-близнеца моей бабушки — Янкэлэ и Мотя; моя мама — двадцатитрёхлетняя Гита на последнем месяце беременности и с двухлетним сыночком Эршэлэ (Тришей). Дождаться родов в своём брест-литовском доме уже не представлялось возможным.

Мобильной силой семьи были мои дедушка и бабушка, папа; Яков Розенбаум — муж тёти Сони, старший брат моей мамы — Ноях (Наум) и ещё два младших родных брата моей мамы — Шимэн (Семён) и Хаим (Ефим). Для большей безопасности и взаимопомощи в дорогу двинулись большими группами. В них входили семьи родственников, друзей. Вы поняли, что о финале жизни некоторых названных здесь людей сказано на предыдущих страницах.

%%[[page_068]]
купить изразцы помнили составы групп. За многие дни пути выработался ритм движения: в жаркие августовские дни группы двигались с утра до полуденных часов, потом большой привал, где как, а с вечера до ночи — дальше, как позволяли свет луны, дорога, потом опять остановка до утра. Неизвестность не манила беженцев. Они двигались со скоростью лошадиного шага и с длительными остановками. Тлелась надежда, что может быть ещё разрешат вернуться в Брест-Литовск, домой. Под конец одного из ночных переходов у Гиты начались родовые схватки. Хотя в принципе это ожидалось и родные как-то готовились, но в ночи, на краю леса, без света и тёплой воды? К счастью, кто-то из ребят разглядел контуры крестьянских изб. Папа и Ноях помчались в ту сторону. Встретившийся им человек указал избу, в которую надо обратиться; там две женщины выслушали папу и сказали, чтобы скорее везли роженицу. Оказалось, это были повитухи (крестьянки, которые в далёких деревнях выполняли функции акушерок). С роженицей оставили в избе только бабушку. На рассвете 28 августа 1915 года моя мама благополучно родила мальчика. В этой избе они прожили три дня, дальше — опять телеги. На восьмой день после рождения устроили брис, под открытым небом, мальчику дали имя Дувэд (Давид). И представьте, через 30 лет, уже после разгрома нацистской Германии, воинская часть, в которой служил Давид Абрамович, базировалась в Польше, не очень далеко от мест, откуда изгнали его предков, родителей, да, собственно, и его самого. Надежда вернуться в свой Брест-Литовск всем угасла, и семья двигалась всё дальше на вос-

%%[[page_069]]
Ток. На базарах и мелким перекупщикам продавали содержимое корзин, узлов; клади становилось всё меньше, а лошадям — всё легче. Политическая ситуация в стране осложнялась, а положение бездомных беженцев — в ещё большей мере: чем дальше на восток, тем меньше можно было пользоваться польским и литовским языками, а собственно в России оставался слабый русский. В обществе слышались пораженческие настроения, Россия проигрывала Первую мировую войну. Население было взбудоражено и политической борьбой внутри страны. Наступало время двух революций — февральской и октябрьской; Гражданской войны; проигрышной советско-польской войны 1920 года. К Польше отошли (до 1939 года) Западная Украина, Западная Белоруссия, в том числе город Брест-Литовск. Была еще интервенция. Большие потрясения, произошедшие за короткое время на территории бывшей Российской империи, привели к разрухе, голоду, сиротству, гибели людей. В этом переборе осязаемых бед сказалась, как щепка в штормовом море, большая, традиционная, приближённая к патриархально-религиозному укладу жизни, еврейская семья; в тех условиях она острее ощущала значение насильственного изгнания из своего дома, что усиливало тоску по нему, но, разумеется, не по старой власти. Осложнённая малыми детьми, моя родня искала приемлемого пристанища. Выбор пал на небольшой, расположенный на высоком берегу Волги, красивый город Симбирск (Ульяновск).

%%[[page_070]]
Последовательность изложения прерываю потому, что Люся Величанская прочитала предшествовавший текст. С волнением, признательностью пожелала успешного завершения моего замысла и сообщила факт, которому я придаю значение документального свидетельства начала войны. С фронта дядя Сима прислал домой две записки, о которых Рая рассказала Люсе. В последней, уже из Вязьмы, отец писал: «Положение наших ополченцев плохое, у нас одна винтовка на трёх человек, вины. В таких условиях, наверное, не вернусь. Дора, сбереги детей». Записка-предчувствие скорой гибели: ладонью пу-

%%[[page_071]]
ле не противостоишь; записка - прощание мужественного человека не сохранится. Уместно вспомнить глубокий смысл афоризма М. Булгакова: «Рукописи не горят»; значительное сохраняет память, она же воссоздаёт сказанное. Шура Величенский тоже сделал дополнения. После войны Лазарь Сарный остался в Минске, Шура приезжал к нему, а он - в Москву, конец жизни прожил в Израиле. О партизане Лазаре Чарном рассказано в книге «Мстители гетто». Через московских друзей попытаюсь найти эту книгу. Вернёмся на несколько десятилетий назад, в город Симбирск. В глухой провинциальный город все потрясе...

%%[[page_072]]
ния в стране доходили +1. в виде ослабленной волны. Это облегчало обустройство в новой, во всех смыслах, жизни. Свободного жилья в городе было много, проблема - в его оплате в условиях обесценивания денег. Впрочем, это обесценивание - одна из причин рынка свободного жилья: владельцы его, фэкспроприированные новым государством их собственности, сдавали или продавали жилье и таким образом справлялись с трудностями. Вся взрослая и юношеская часть родни бралась за любую работу: жили скученно, напряженно. С началом НЭПа все начало меняться, как и во всей стране, в лучшую сторону. Незавершенный план В.И. Ленина «Новая экономическая политика (НЭП)» был уникальным явлением в управ-

%%[[page_073]]
Влении народным хозяйством. Мои ранние детские впечатления об окружающей жизни начались именно в этот период; они закрепились в памяти потому, что мой папа примкнул к НЭПу. В итоге это печально сказалось на судьбе семьи; впрочем, с ней произошло то же, что со всей новой экономической политикой — разгром. Поскольку участие в НЭПе — звено в цепи жизни семьи, и это малоизвестная вам часть истории государства, остановлюсь на связке проблем. К концу 1920 г. - началу 1921 годов как бы определились последствия сложных общественно-политических, межнациональных и череды военных событий, начавшихся всего 6 лет назад, в 1914 году — Первая Мировая война.

%%[[page_074]]
Бойна. События сотрясали бывшую Российскую империю, а затем молодую Советскую Россию; на предыдущих страницах названы эти события; каждое из них явление в тяжёлой истории народа. О многом из этого рассказано в высокой российско-советской литературе 1920-ых-1930-ых годов; широкий доступ к ней читатель впервые получил в годы «оттепели» Н.С. Хрущёва. В частности, имею в виду произведения М.А. Булгакова, К.П. Платонова: отдельно говорю о шедевре киноискусства, о художественном фильме «Бег» и фильме «Белое солнце пустыни». Многие произведения названных писателей переведены на языки иврит и английский. Что же привело В.И. Ленина менее, чем через 4 года...

%%[[page_075]]
Волюции 1917 года к решению использовать буржуазно-капиталистические методы управления экономикой страны? Как и другая новая власть, установившаяся Советская власть делала всё, чтобы удержаться. В тех условиях главным было снять проблему затянувшегося голода, которая сфокусировалась своей остротой к концу Гражданской войны (1918 - 1920 годы). Произошло это по ряду причин; начну в определённой мере с объективно-естественной - засухи. В журнале «Вопросы экономики» (1969 год, № 7; издание Института экономики АН СССР) опубликована моя статья, в которой даю таблицу, составленную по собранным мною данным о засухах, начиная с 1885 года - первого года введе-

%%[[page_076]]
сной службы в России - по 1400 год; привожу часто таблицы: 1905-1909 гг. 1906 г. 1925-1929 гг. нет 1920-1924 гг. 1920 г., 1921 г., 1924 г. 1910-1914 гг. 1911 г., 1914 г. 1930-1934 гг. 1915-1919 гг. 1917 г. 1935-1939 гг. 1936 г., 1938 г., 1940-1944 гг. 1939 г. 1943 г. 1931 г., 1934 г.

* Указана засуха только в Поволжье, значение которого в обеспечении страны хлебом всегда остается большим. В годы Великой Отечественной войны роль этого района необычайно возросла, так как из-за временной немецко-фашистской оккупации части нашей страны и разрухи в освобожденных районах посевные площади в стране в 1943 г. составляли 94,1 млн. га, или всего 63% посевной площади 1940 г.

Как видим, события интересующего нас периода совпали с годами засух (1914, 1917, 1920, 1924). Прямой связи между явлением природы - засухой и политическими явлениями в стране - революцией, войной, конечно, нет, но есть, или вполне возможна, прямая зависимость между силой воздействия засухи на общество и происходящими в нём процессами. Когда страна живёт в своих привычных условиях, то убытки урожая в засушливые годы проходят по установившимся (Стр. 81).

%%[[page_077]]
традициям: в крестьянской стране, какой была Россия в те времена, когда малочисленный рабочий класс сам был вчерашним крестьянином с большой деревенской родней, существовал тёплый, надёжный обычай - в годы засух родственники из городов приезжали в деревню помочь поскорее убрать урожай. Ударным является слово «поскорее». Дело в том, что при благоприятных погодных условиях уборка урожая, возьмём зерновые, которые, исходя из их значимости, называют также хлебами – уборка урожая, например, пшеницы после полной её спелости должна завершиться за 8-10 дней: потом начинается естественное осыпание зерна - выпадение зёрен из колосьев.

%%[[page_078]]
са; иными словами, начинается безвозвратная потеря урожая. Столько дней, всего 8-10, поля полноценных колосьев пшеницы восхищают вас густотой стояния, поклонами массы поля от дуновения ветра, отблеском Солнцу родственным ему цветом, поклонами колосья как бы благодарят Солнце за данную им энергию, а Человека - за приложенный труд, который поднял их из почвы, колосья как бы говорят: возьми нас человек, не дай погибнуть, наберись нашей силы, чтобы под Солнцем будущего года мы опять наполнили друг друга. Советую: съездите на зерно-производящую ферму, поживите там, попросите хозяина рассказать о циклах производства.

%%[[page_079]]
серое кольцо ишелищ, полути те его на ладонь, всмотритесь: помимо покалывающего тепла колоса вы с неожиданным восторгом воспримите изящество его контура, Гармонию Геометрических симметрий, линий, форм. Эта строгая выразительность одного из источников энергии нашей жизни не оставляла меня равнодушным, когда во время командировок доводилось бывать на таких полях; убеждён, что и вы испытаете такое же чувство, поймёте, что рядом с вами есть возвышенный мир Гармонии и дела, который не менее интересен далёких стран. В годы засух всё не так: колос слабый, поэтому зерно начинает выпадать раньше обычного срока, чтобы убрать без потерь хотя бы то, что выросло, надо спешить с уборкой; в те годы, о которых мы говорим, ситуация ослож-

%%[[page_080]]
часть крестьян и тех, кто приезжал помогать, были в солдатских шинелях разных армий: кто в красной, кто в белой, кто в бандитских формированиях. А хлеб нужен всем, и его брали у крестьян в меру своей морали. Как нетрудно понять, главного деятеля крестьянского двора не было, всё легло на плечи женщин, подростков, родителей, управиться с уборкой урожая в отведенный природой срок, особенно в засуху, практически невозможно. А из собранного надо оставить на питание семьи, на семена и сколько-то продать, чтобы купить необходимое для семьи и в хозяйство. При таких условиях в массе крестьянских хозяйств всё на минимуме. Итак, первая причина, думаю, прояснилась.

%%[[page_081]]
свежую продукцию по согласованным ценам, а само пыталось обеспечить запросы деревни в различных промышленных товарах тоже по ценам, приемлемым для крестьян; однако обменное равновесие власть поддерживала с большим трудом, потому что боевые действия, точнее, результаты участия Российской империи в Первой мировой войне перетекли в Февральскую революцию, из неё - Октябрьскую, а затем в Гражданскую войну, тогда к тому же противостояния политических партий, различных организованных сил, стремления каждого - победить, дало быстрый, первый, самый ощутимый результат для всех - разрушение структуры управления экономикой страны, воцарилась хозяйственная разруха: промышленность еле дышала, и то - не вся.

%%[[page_082]]
главный показатель 82. состояния экономики - провал торговли пришёл в упадок: товаров всё меньше, а потому цены — всё выше, происходит обесценивание денег, понятно, что в условиях недействующей экономики страны в казну её не поступают законные отчисления от различных отраслей, бюджет крайне ограничен. При таких условиях Советское правительство предложило твёрдые цены (фактически ниже рыночных или согласованных договорных) на закупаемую у крестьян продукцию, но крестьянство отказалось, оно само находилось по всем названным причинам в тяжёлом хозяйственном и материальном положении. Вопрос о ценах осложнился тем, что в марте 1918 года началась интервенция стран-союзников Российской империи по первой мировой.

%%[[page_083]]
войне, их цель была — поддержать борьбу против советской власти сил, верных старому режиму, поэтому Гражданская война обострялась, а с ней проблемы продовольствия, особенно зерна (хлеба), и советское руководство, которое возглавлял В. И. Ленин, в июне того же 1918 года пошло на первую насильственную меру по отношению ко всему советскому крестьянству — на создание в деревнях комитетов бедноты (комбедов); из данных комбедам прав следует, что Советская власть доверила свою функцию взаимоотношений с крестьянством этим комитетам.

%%[[page_084]]
ношении с крестьянством его низшему бездеятельному слою: в комбеды входили действительно бедные, несчастные по тем или иным причинам люди, бывшие батраки. Примазывались лодыри, горлопаны и всякая подлость; комбеды распределяли средства производства и отдельные виды товаров между крестьянскими семьями, определяли за них их личную потребность в хлебе, в семенах, а остальное зерно предлагали продать по твёрдым ценам государству; к экономическому прессу добавлялось и то, что отдельные комбедовцы, пользуясь положением, сводили личные счёты с односельчанами.

%%[[page_085]]
Давление Советской власти на крестьянство продолжилось потому, что летом 1918 года 3/4 территории Советской страны было в руках её врагов; положение сложилось гибельное; на местах уже начали создавать белогвардейские и националистические правительства; чтобы спасти положение, Советское правительство принимает решительные меры, оно вводит всеобщее военное обучение (всеобуч), всеобщую трудовую повинность, в сельском хозяйстве - с января 1919 года вводит продовольственную

%%[[page_086]]
разверстку и организуем продовольственные отряды для облегчения изъятия продовольствия; это был период военного коммунизма в деревне, когда, исходя из расчётных данных, каждому крестьянскому двору заранее предопределяли обязательный объём сдачи (продажи) государству сельскохозяйственной продукции, и «излишки» изымали; и эти условия ужесточались: в зависимости от обстоятельств, власть уменьшала семенной фонд, убавляла количество оставляемого продовольствия на личное питание крестьянской семьи;

%%[[page_087]]
Потом крестьянам запретили торговать своей продукцией. В ответ на жесткие меры ленинского правительства они сократили посевные площади (всё равно отнимут, сидеть голодными), в результате деревня нищала ещё больше. По мере приближения исхода Гражданской войны напряжение нарастало. В.И. Ленин тоже объяснял потом причины усиленного давления на крестьян в годы Гражданской войны; он, как увидим, понимал, судя по его последующим оценкам первых результатов Кэпа, что репрессивными мерами невозможно за короткое время вывести страну из состояния голода и хозяйственной разрухи, а времени было мало: ситуация угрожала существованию советской власти, и В.И. Ленин решает послевоенные проблемы с позиций новой экономической политики.

%%[[page_088]]
разработал; программу утвердил десятый съезд партии в марте 1921 года. Главная цель НЭПа — вернуть крестьянина к материальной заинтересованности в своём труде и таким образом преодолеть голод была достигнута грамотными экономическими решениями: вместо насильственных изъятий хлеба власть ввела натуральный подоходный налог на количество произведённых основных видов сельскохозяйственной продукции (900 баллов), остальной, наибольшей частью её крестьянин распоряжался сам — сколько оставить на потребление и воспроизводство, сколько продать; деловая жизнь в деревне пробуждалась; в ряде случаев крестьяне кооперировались по видам отдельных работ, по использо-

%%[[page_089]]
ванию средств производства, а позже - и по их собственному приобретению: значение этих фактов и поощрения государством расширения посевных площадей станет вам понятным, если узнаете, что к 1924 году они сократились до 90,3 млн. га пашни, а в 1913 году было 105 млн. Процент снижения урожайности был ещё выше, плюс падёж скота (продовольственные трудности в стране утяжелялись бескомпромиссным, истребительным отношением советской власти к кулачейству; в городе разрешили, говоря современным языком, малый и средний бизнесы, а также ограниченное использование частного капитала в рамках НЭПа, но естественно, государство контролировало соблюдение условий новой политики и оберегало свои монопольные интересы; результат

%%[[page_090]]
Получился хороший: действующие на то время промышленные предприятия страны увеличили выпуск средств производства для сельского хозяйства и таким образом Государство обеспечило себе господствующее положение в экономических связях с деревней (уровни цен, объём и направления поставок и т.п.); частные мелкие предприятия, артели, кустарная частная торговля тоже были вовлечены в оборот с деревней, потому что она нуждалась во всём - от мелкого инвентаря до иголки; новые условия взаимоотношений продолжились в небывалом росте сельскохозяйственного производства. Ожившую трудовую сельскую жизнь того периода, охвативший людей энтузиазм, показал Андрей Платонов в рассказе «Лампочка Ильича»; рассказ датирован 1926 годом (через два года после

%%[[page_091]]
смерти Ленина), писатель, видимо уловил тогда другие веяния, потому финал рассказа символичен. Причиной его написания? Уже в 1923 году острота голода спала. Об этом я и по отрывочным разговорам в семье, то есть и на бытовом уровне в среде простых людей, подтверждалось положительное значение начатого процесса. Всего лишь за первые два года осуществления НЭПа Советская власть убедилась в преимуществе экономических динамичных методов управления; после первых значительных успехов в сельскохозяйственном производстве Советское правительство видоизменило в пользу крестьян условия переходного периода и в 1923 году отменило продналог, государство перешло на закупки зерна и другой сельхозпродукции.

%%[[page_092]]
рынке; получив дополнительные производственные свободы, крестьяне в том же (1923) году расширили посевные площади на 12,4 миллиона гектаров, или на 18,7%. [Чтобы вы не запутались в цифрах: пашня — составная часть посевных площадей, а посевные площади — составная часть сельскохозяйственных угодий]; результаты экономического стимулирования выразились в прогрессирующем развитии сельского хозяйства вплоть до ликвидации НЭПа в 1928 году. Вот «узловые» данные: довоенный, благополучный 1913 год примем по всем показателям за 100%, в критическом 1921 году производство валовой продукции сельского хозяйства упало до 60%, продукции земледелия — до 55%, а в 1928 году подошло (соответственно показателям к

%%[[page_093]]
уровни даже 1913 года, эти соотношения наглядны в графике, 1928 130 120 1913 год. 100% до 65 1991 год 50 Валовая продукция сельского хозяйства - Продукция земледелия Легко понять, в какую яму падало сельское хозяйство, а с ним и вся экономика России, в период с осени 1914 по 1921 годы (при ускорении в годы Гражданской войны).

%%[[page_094]]
P63. 94. Отмечу: до утверждения партийным съездом Новой экономической политики она обсуждалась на различных форумах. Несмотря на горький опыт военного коммунизма, были мнения о продолжении принудительной массовой организации производства. Ленин, ссылаясь на тот же опыт, опровергал подобные взгляды. Полагаю, что уверенность в правильности альтернативной позиции придавали ему ещё два факта: подавление возмущения, или восстания Тамбовских крестьян в 1920-1921 годах действиями Советской власти в деревне. Руководил восстанием начальник уездного (районного) отделения милиции А.С. Антонов (1888-1922), отсюда — «антоновщина». По датам восстания и по высокой должности А.С. Антонова можно предположить, что он был коммунистом-большевиком; восстания по той же причине были и в других местах.

%%[[page_095]]
95. губерниях; что самый факт - мятеж на военной морской базе «Кронштадт», и здесь звучала крестьянская тема, поскольку подавляющее число восставших моряков и солдат - это были крестьянские парни, тесно связанные с деревней; углублять разрыв с крестьянством - вовсе не было задачей, как вы уже знаете, нового государства, наоборот, требовалось восстановить значение декрета о земле, чтобы крестьяне быстрее наладили производство товарной продукции; так и получилось: единоличные крестьянские хозяйства, поддержанные государственным экономическим регулированием, учитывавшим интересы обеих сторон, сняли за два года проблему голода и накормили народ; этот факт свидетельствует о том, что никакой необычности во взаимоотношениях социалистического

%%[[page_096]]
96. Государства с единоличным, частнособственническим крестьянством не было, именно в таком соотношении В.И. Ленин видел развитие сельского хозяйства страны на длительную перспективу. Объяснял он это так: крестьяне составляют гигантскую часть всего населения и всей экономики; преобладает мелкотоварная форма ведения крестьянского хозяйства, она будет существовать в течение длительного исторического периода в рамках переходной экономики, потому что для социалистического преобразования мелкотоварного крестьянского хозяйства нужна мощная материально-техническая база. В те и во многие последующие годы такой базы не было.

%%[[page_097]]
Крестьянин с его фотопными орудиями труда, работой по старинке, с психологией собственника, пересаженный в коллективное хозяйство (колхоз), не будет заинтересован в труде, так как не ощутит его конкретного результата. Поэтому длительным путём к колхозу является кооперация - не административная, а добровольная. Разъяснительная фантазёры те коммунисты, которые думали, что за три года можно переделать экономическую базу, экономические корни мелкого земледелия.

Двухлетняя практика действия Новой экономической политики В.И. Ленина подтвердила, воп.

%%[[page_098]]
реки высказывавшимся несогласиям, правильность его взглядов на аграрное развитие страны: рассуждая о государственном капитализме в условиях социалистического строительства, он пришёл к важному выводу о том, что кооперация выполняет роль государственного контроля и надзора за частным капиталом, кооперация помогает государственному регулированию овладеть стихией мелкотоварного производства, поэтому необходимо направить развитие капитализма в деревне в русло кооперативного capitalismo» (формулировка В.И. Ленина). Чтобы вы поняли значение приведенного текста, объясню различие между членством в кооперации и членством в коллективном хозяйстве.

%%[[page_099]]
ОК (Коллод), крестьянскую кооперацию. Крестьянник: единоличник вступал добровольно, чтобы совместно, а, как правило, и легче, решить однозначную и для дружных односельчан задачу. В условиях ускоренного роста сельскохозяйственного производства расширялись хозяйственные потребности крестьян, и специфика коопераций соответствовала этому, она охватывала всю инфраструктуру сельской жизни. Всё это делалось при активном, преимущественно экономическом, участии государства; главная цель его была в том, чтобы через кооперацию вытащить крестьянина-единоличника из его психологической и хозяйственной замкнутости, показать ему преимущества совместных решений текущих, повседневных задач.

%%[[page_100]]
лилии но: был ли крестьянин постоянным или временным членом кооперации, или не входил в неё вообще, - в любом случае он оставался собственником участка земли, который закрепила за ним Советская Власть после национализации всей земли и в соответствии с Декретом о земле, то есть: Крестьянин-единоличник вёл частное хозяйство (в этом смысле он оставался «капиталистом») на государственной, социалистической земле; в то же время он ощущал «локоть» кооперации и знал о преимуществах вхождения в неё (некоторые более выгодные условия от совместных дел, цены, финансовые кредиты, также консультации агрономов, ветеринаров и пр.); исто-

%%[[page_101]]
Источник возможностей для внутрикооперационной взаимопомощи было отчасти само кооперационное объединение людей, средств, но главным источником было государство: организационно, пропагандой, своевременно принятыми экономическими решениями оно поддерживало и укрепляло «кооперативный капитализм»; таким образом государство пыталось на широкой, доказательной практике «переделать экономическую базу, экономические корни мелкого земледелия». В результате такой политики сельское хозяйство развивалось активно, чтобы усилить этот процесс, крестьянам в 1925 году разрешили использовать наёмный труд для работ на арендованных ими землях.

%%[[page_102]]
кто-либо из вас, - Вскоре после революции возрождение отдельных условий, из-за которых громили помещичьи и кулацкие хозяйства. Отступление? Никакого отступления не было, дело в том, какие экономические и социальные условия стоят за понятиями аренда земли и наёмный труд. К успеху НЭПа надо отнести и тот факт, что в те годы кулацкие хозяйства, где - разорённые, где - полуразорённые, а кое-где - ещё нетронутые советской властью, получили права в занятии сельскохозяйственным производством. Ленинское толкование кооперации, очередность направлений аграрного развития; что НЭП — не экономический манёвр, а долгосрочная стратегия и вводится всерьёз и надолго — эта концепция просуществовала всего

%%[[page_103]]
1927 года, когда пятнадцатый съезд партии, вопреки предостережениям В.И. Ленина, принял решение о развертывании на селе коллективизации, то есть фактически речь шла о создании в деревне таких условий, при которых экономически поднявшееся, неповское крестьянство вынуждено было бы вступать в колхозы, и это в условиях, когда более половины площадей под хлебами крестьяне засеивали вручную и также убирали урожай, ручной труд преобладал и в животноводстве. Этому, исходя из ленинских оценок ситуации ещё в 1925-1927 годах, преобладало мнение, что не колхозы являются столбовой дорогой к социализму, а кооперация. (По книге В.В. Милосердова - академика Россельхозакадемии).

%%[[page_104]]
после смерти
В.И. Ленина (24 января 1924 года) защиту его идей на партийных обсуждениях и в полемике с И.В. Сталиным взял на себя известный партийный и государственный деятель того времени Бухарин Николай Иванович. Суть расхождений во мнениях между ними на перспективные решения в следующем. В.И. Ленин умер на взлёте НЭПа. С 1920 года осуществлялась также инициированная В.И. Лениным электрификация страны (ГОЭЛРО), которая, значимая сама по себе, являлась одной из опор индустриализации СССР; развивать равнозначно три направления народного хозяйства государство не могло: не было ни материальных, ни финансовых средств.

%%[[page_105]]
тому выбрали самое жизненно важное - сельское хозяйство, чтобы ликвидировать голод. Как вы уже знаете, эта проблема решалась успешно: продовольствия было полно, цены - низкие. В деревне накапливались заработанные деньги, крестьяне стали больше покупать разных товаров, спрос деревни опережал предложение города (промышленности), нарушился паритет вздутых низких цен, промышленные товары дорожали. Вопрос о паритете цен - обычный и решаемый (и тогда была возможность благополучно его снять: например, даже в медленно восстанавливаемой после разрухи промышленности можно было перераспределить номенклатуру выпускаемых товаров, в том числе и средств производства для села) - вопрос завис, вместо его решения.

%%[[page_106]]
106

на уровне государственного экономического регулирования. Крестьяне уже восприняли за несколько лет НЭПа, они ощутили влияние начинающегося ускорения темпов индустриализации в виде требований продавать зерно и другую сельхозпродукцию государству по сложившимся низким рыночным ценам. Крестьяне, как и перед НЭПом, ответили сдерживанием объёмов продаж хлеба, сокращением посевных площадей. С индустриализацией возник, на неопытный взгляд, заколдованный круг: ускорение её развития вовлекало всё больше промышленных центров, увеличивалось городское население, всех надо было кормить, возрастала

%%[[page_107]]
Потребность в расширении сельскохозяйственного производства, особенно зерна, восстановленное и хорошо развитое за годы НЭПа сельское хозяйство в силах было справиться с этой задачей даже в тех условиях, когда государство большую часть средств вкладывало в индустриализацию; средства, деньги на ускорение сельскохозяйственного производства были в самой отрасли: приведенный условный пример можно тематически продолжить вариантом, по которому государство могло скупить у крестьян через торгово-сбытовые кооперации удерживаемое зерно по согласованным ценам, но в кредит под Госбанковскую гарантию на повышенную часть цены; примерно также можно было договориться.

%%[[page_108]]
106. Мы кооперируемся по поводу урожая предстоящего года и спокойно, не нарушая производственного цикла, увеличиваем экспортную массу зерна, валютная прибыль от которого шла на закупки техники для ускоренно развивавшейся тяжёлой промышленности. Дело, конечно, не в поиске решения возникших противоречий из-за цен, всё заключалось в позиции И.В. Сталина. В статье «Сталин» в «Советском энциклопедическом словаре» (М., «Советская энциклопедия», 1980) сказано, в частности, «допускал теоретические и политические ошибки» (с.1275). В составляющие эту оценку вошло, можно считать, и отношение И.В. Сталина к новой экономической политике.

%%[[page_109]]
тупал за ускоренную индустриализацию страны, но против сохранения тех экономических связей, методов управления и их результатов, которые обеспечивал НЭП, поскольку они не удовлетворяли растущих запросов индустриализации; исходя из этого своего взгляда, а также имея сторонников в партии, И.В. Сталин считал возможным применять насильственные меры по отношению к крестьянству, чтобы изъять у него дополнительные количества зерна в целях поддержания высоких темпов индустриализации (острота вопросов концентрировалась вокруг зерна потому, что в те годы зерно было единственным экспортным товаром СССР); И.В. Сталин считал, что неповские капиталисты — средний и малый бизнесы — будут мешать социалистическому развитию.

%%[[page_110]]
по-алистической реконструкции страны, в частности расширению производственных связей между предприятиями на основе государственного планирования, и он призывал «к активному наступлению на капиталистические элементы» (надуманность повода для наступления состояла в том, что все заводы, фабрики, земля, железные дороги, источники сырья и другие стержневые ветви экономики были после Октябрьской революции в руках государства; неповские капиталисты — это разрешённый правительством ремесленно-артельный уровень частного производства, который работал на купленных отходах заводских, фабричных производств, на отходах других типов предприятий и на сельскохозяйственном сырье. Торговцы — напманы — того же экономического ряда, но они знали бытовые запросы населения и заказывали производителям нужные товары. Весь этот ремесленный и торговый люд платил установленные налоги, прилично жили, избавлял правительство от забот по обеспечению населения большим набором.

%%[[page_111]]
ром бытовых товаров; 131, никакой политической силы эти люди не представляли, так как никаких партийных групп не создавали, это были обычные обыватели, перенесшие две войны, две революции, разруху, голод, они ухватились за возможность работать и наладить свою жизнь; вся политическая и экономическая сила была у руководства страны, но громкая, модная тогда фразеология понадобилась, чтобы обосновать ликвидацию НЭПа. На партийных форумах Н.И. Бухарин выступал против насильственных мер по отношению к крестьянству, они ведут,

%%[[page_112]]
как известно, к сопротивлению и падению производства (Так, в приведенном выше графике на данных за 1928 год уже сказались вновь начавшиеся протестные меры крестьянства); причины нарастающих потребностей, особенно в зерне, Бухарин видел в неимоверных темпах индустриализации. Как получать дополнительное зерно для всесторонних нужд индустриализации он предлагал не путём насилия над крестьянством, а путём расширения аренды земли и прав на наём рабочей силы, дальнейшего развития товарно-денежных отношений. Главными принципами в аграрных отношениях Бухарин считал нормальные хозрасчетные отношения, эквивалентный обмен между городом и деревней.

%%[[page_113]]
ом и деревней, у него 115. были предложения и по текущим вопросам, но он возражал против преждевременной коллективизации, которая, утверждал Н.И. Бухарин, не сможет оказать существенного влияния на развитие сельского хозяйства. И.В. Сталин и его единомышленники придерживались другого мнения: экономический механизм нэпа не обеспечит перекачку необходимых средств в развитие тяжёлой индустрии, поэтому при возникающем неэквивалентном обмене невозможен рынок (в широком политэкономическом смысле: речь идет о преимущественно равновыгодном торговом обмене продукцией разных сфер производства).

%%[[page_114]]
льку в те годы только аграрная отрасль могла стать основным донором индустриализации, то при отрицании возможности попытаться снять возникшие производственно-экономические разногласия с крестьянством начали складываться экономические неравенства отраслей. Можно полагать, что идея неизбежности неэквивалентного обмена и сохранение условий, способствующих этому, укрепляли позиции сторонников преждевременной коллективизации: правительство получало фактически неограниченное право брать из сельского хозяйства продукции столько и так, как посчитает нужным руководство страны, и таким образом перекачивать средства в индустриализацию, лишая крестьян возможности выражать действенными мерами.

%%[[page_115]]
отдельными решениями правительства. В партийной полемике верх одержала позиция И.В. Сталина, и в 1929 году началась, повторюсь, коллективизация. Точки зрения взглядов в члени- на материально-технически и морально (без влияния кооперации) неподготовленная, а потому преждевременная, коллективизация. В реальной жизни это означало отказ от управления экономическими методами и замена их пропагандой призывно: карающими лозунгами типа: «Кто не идет в колхоз, тот враг Советской власти». узаконенное партийным решением 6 ноября 1929 года соревнование по наращиванию темпов коллективизации, затем соревнование рец.

%%[[page_116]]
116. онов за скорейшее завершение коллективизации; переход к применению репрессивных, карательных мер, в отдельных случаях и военной силы, для выполнения «темпов», «соревнований», о которых благополучное в массе своей неповское крестьянство и понятия не имело, привел к тому, что насильно включённые в колхозы люди обязаны были передать колхозам свои права на пользование землёй, отдать рабочий и продуктивный скот, орудия сельскохозяйственного труда, семена и т.п.; крестьянин-индивидуалист, лишаемый своей производственной опоры и товарно-денежных отношений,

%%[[page_117]]
в которые укладывались его хозяйственные и личные интересы, в то же время обязанный принять новые формы организации труда, сбитый с толку навалившимися на него переменами, оставшись без поддержки правительства и видя теперь в нём своего противника (на фоне безобразных фактов коллективизации), стал резко сокращать свою хозяйственную деятельность, в частности сверхмерно забивать скот, чтобы получить какие-то деньги за отбираемую собственность. Как начавшиеся перемены быстро сказались на разорении единоличных крестьянских хозяйств, в данном случае на снижении численности скота, видно из сопоставимых

%%[[page_118]]
1928 года по 1 января 1954. Гораздо меньше голов во всех категориях хозяйства при значительном преобладании индивидуальных хозяйств:

1928 год

Крупный рогатый скот Свиньи Овцы

8,9 27,7 104,2 36,1

50,6 14,2 85,5 31,0

Численность поголовья заметно снижалась и в последующие несколько лет; тенденция падения производства по отношению к уровням 1928 года была явная во всех главных отраслях сельского хозяйства: Итак, принудительная коллективизация, мародёрские методы изъятий дополнительного зерна.

%%[[page_119]]
нарастающих потребностей Экспорта — опять, как в годы Гражданской войны, насильно отнятое зерно, которое было предназначено для питания семьи и на семена, усиление административного произвола в деревне, вся эта беззащитность привела к массовому выражению протеста тысячами крестьянских восстаний, зарегистрированных только в одном 1929 году: чтобы неповадно было бунтовать, усилили темпы коллективизации, и к середине 1931 года в колхозы уже было включено 52,7% единоличных крестьянских хозяйств, преднамеренный отказ послеленинского руководства от Новой экономической политики завершился крахом.

%%[[page_120]]
хозяйства: восстановленное и давшее замечательные производственные результаты в годы НЭПа, оно развалилось под воздействием названных притеснений крестьянства. Коллективизация не обеспечила необходимого количества зерна для нужд индустриализации, в том числе экспорта, поэтому подключили продажу за рубеж произведений искусства. В розах стали открывать магазины «Торгсин» (торговля с иностранцами - там обменивали на специальные талоны драгоценности и иностранную валюту, на талоны можно было приобрести продовольствие и промтовары); в то же время власть ещё сильнее придавила крестьянство.

%%[[page_121]]
и низовой аппарат управления на селе в результате несоразмерных с возможностями требований власть допустила голод в деревнях и селах, унесший не менее трёх миллионов жизней крестьян в благоприятные для сельского хозяйства годы — 1932-1933; на Украине это голод называют голодомором. Таковы финальные аккорды реквиема по новой экономической политике в деревне, диезы и бемоли в условиях городского исполнения покажу на примере моего отца, а потом поделюсь своими личными обобщениями. Перенесёмся в Симбирск. В начале 1920-х годов в нашем роду произошли большие события: умер мой пра-дед Ефраим Берс, отец моей ба-

%%[[page_122]]
бабушки, умер её муж 122 мой дед-по имени Величанский, муж тёти Сони — Яков Розен тоже. Соня и Яков тоже, всё похоронены в Симбирске. Ефраим Берс умер от глубокой старости, он прожил более ста лет. Фамилию его я назвал точно, когда моя Инна после замужества осталась на своей фамилии Рубинштейн. В разговоре с бабушкой об этом я спросил (догадался её девичью фамилию, четко в два слога она произнесла слово Берс. Предположительно учительствовал в еврейских школах. По рассказам бабушки, он был очень скромным человеком, жил в мире своих служебных интересов. Став вдовцом, оставшуюся долгую жизнь прожил в семье своей единственной дочери; Абрам Величанский.

%%[[page_123]]
193 В какой-то школе сотрудничал с Берсом, стали друзьями, а потом родственниками — тестем и зятем; интересное в этой более чем столетней тоже был учителем, предыстории семьи то, что объект родства моя будущая бабушка Фейга с радостью восприняла свекра, но вскоре показала, кто в доме генерал: муж-учитель зарабатывал мало, появились дети, перспектива вести жизнь бедной семьи угнетала молодую энергичную женщину, и она, махнув рукой на своего книжника, проявив коммерческую смекалку, организовала в Брест-Литовске в конце 19 века свою мясную торговлю, сама же вела дело, дом, решала судьбы детей. Вот пример стойкости.

%%[[page_124]]
её характера и воли. Ранняя молодость её пришлась на последнее двадцатилетие царствования русского императора Александра II. Она родилась в 1861 году. Император Александр II сделал много доброго для евреев: отменил армейскую службу для совершеннолетних еврейских мальчиков - кантонистов (прочтите книгу «Кантонисты», советское издание, примерно середина 1930-х годов, автора не помню); евреям-солдатам разрешил дослуживаться до офицерских чинов, разрешил евреям получать юридическое и медицинское образование, лекарям-евреям разрешил служить в армии.

%%[[page_125]]
С пониманием относился к положению евреев в Польше, увидев группы мужчин евреев в национальных религиозных одеждах, Царь высказал недоумение по поводу открытой демонстрации в условиях антисемитизма своей национальной принадлежности. Императора, освободившего крестьян от крепостного права, подготовившего другие демократические преобразования в Российской империи, убили народовольцы в марте 1881 года. К убийству Царя евреи никакого отношения не имели и власть в этом их не обвиняла, однако последовавшая естественная реакция на убийство царя перешла от приемлемых ответных мер к самым крупным погромам евреев в Российской империи.

%%[[page_126]]
империи; начало периода больших погромов относится к тому же 1881-му и 1882 годам. В городе Кишинёве, в южных губерниях России, бесчинства продолжились до 1903 года, когда император Николай II возложил ответственность за погромы на губернаторов, и всё остановилось (до 1907 года). Обострение антисемитизма привело к росту эмиграции евреев в Новый Свет (Америка); брестлитовский раввин Соловейчик (из потомственного рода раввинов) поднял тему о возвращении в Эрец-Исраэль. Влиянием погромов состоялась большая нелегальная алия в Эрец-Исраэль из Российской империи. Всё это происходило на фоне идейного разброда в обществе после отмены крепостничества.

%%[[page_127]]
В напряжённой обстановке того времени молодая чета — Абрам и Фейге Величанские принимает решение отправить в Америку старшего сына Шулема, там нет угрозы погромов. Слово шулем на языке идиш, и шалом на иврите в переводе на русский язык означают мир между людьми, странами, а мир в смысле вселенная обозначен на идише словом велт. Года через полтора Шулем стал проситься обратно: не мог прижиться; отец дрогнул, мать категорически возражала и отправила туда второго сына Мойше (Моисея); братья преуспели в период автомобильного бума, но когда наступила в США депрессия, Шулем разорился, с ним случился инсульт, он умер.

%%[[page_128]]
в деятельном возрасте, в самом начале 1930-х годов, его сын Срул продолжил переписку с бабушкой, его письма читались не по одному разу. В частности, Срул сообщил, что в те же годы он закончил Нью-Йоркский медицинский институт и прислал своё фото (помнится, урология или гинекология; тогда я не знал значения этих слов, но двор расширил скупые объяснения взрослых). Тема Срул-уменьши-

%%[[page_129]]
тельное от паного 129. имени Исроэл, значение — борющийся. Вскоре связи с зарубежными родственниками стали обрываться навсегда. Власть отгораживала советский народ от бытового общения с гражданами других стран. Бабушка тяжело переживала возникшее неведение, но никогда не высказывала сожаления по поводу того, что настояла на отъезде двух старших мальчиков в Америку; на семей-

%%[[page_130]]
тых посиделках иногда говорилось кое-что по поводу острой темы — о сожалении. Бабушка как-то сказала, что где было тяжело, но теперь намного хуже, я не сожалею о своих соображениях тех лет. Смысл бабушкиных слов передан правильно, но образ её не мелькнул, потому что в моих воспоминаниях бабушка не ассоциируется с русской реальностью, скорее — с польской. Это был не единственный факт твёрдости ба-

%%[[page_131]]
бушкиного характера; в данном случае обстановка в стране придавала факту карты убеждённости: на фоне интенсивной индустриализации, выполнения за четыре года первого пятилетнего плана развития народного хозяйства СССР, в то же время при продовольственной недостаточности и прекрасно поставленной пропаганде проходили крупные политические судебные процессы, вчерашних видных деятелей государства сего.

%%[[page_132]]
132. дня на организованных собраниях трудящихся клеймили позором, называли врагами народа и т.п.; все эти разнонаправленные явления вызывали недоумение и более волнующие чувства. Мысль о чём же могла сожалеть старая Фейге? Только в послесталинское время приоткрылись факты. Правда об этих вечерах судебных процессах. Возвращаюсь в Симбирск (Ульяновск) нэповского периода.

%%[[page_133]]
центральной части главной улицы 133. города Гончаровской, близко от кинотеатра „Палас", мой отец открыл небольшой промтоварный магазин. Помню его по одному эпизоду в кондитерской, которая была рядом с папиным магазином. Ульяновск моего раннего детства неизменно дороги, памятен мне четырьмя-пятью эпизодами, в которых тоже высвечивается та эпоха. Папа один работал в магазине, мама вела всю документацию.

%%[[page_134]]
ментацию, ходи- 134. ла в финотдел и т.п.; когда папа уезжал в Москву или Самару заказывать товары, мама работала в магазине, Гриша и Давид ходили в школу, по домашнему хозяйству маме помогала пожилая крестьянка, которой я, возможно, обязан жизнью; семья жила на улице Фридриха Энгельса, 56 на втором этаже, состоявшем из неказистой надстройки, такой же, но большой, одной квартиры.

Помню холодные сени, широкую приземистую входную дверь и огромные клубы морозного "пара", когда зимой открывали дверь. Вела в просторное неразгороженное помещение она.

%%[[page_135]]
ние, об окне - ещё прихожую, кухню с печкой, столовую, были ещё двери в комнаты, их я не помню. Любопытно: повторение в городе типовой планировки крестьянской избы нашло выражение в современном домостроении - кухни-салоны. Семья моих родителей жила близко от пересечения Сконча-ровской, на противоположной стороне - «Палас» и папин магазин; от угла кинотеатра переулок вверх вёл в самую престижную часть старого города - на Венец; Венец - вершина высокого крутого берега реки Волга, он украшен аллеями, посаженными деревьями, богатыми особняками и бывшими.

%%[[page_136]]
среди путей

дом известного русского писателя Гончарова Ивана Александровича (1812-1891), это подарок, рассказывают, симбирского купечества земдяку после его кругосветного плавания на фрегате "Паллада" и выхода в свет книги одноимённого названия с Венца. Волга видится далеко внизу: железнодорожный мост, пароходы, баржи кажутся чуть ли не игрушечными, на противоположном берегу реки - ровные, широченные, бескрайние поля Заволжья, это и есть то, что называют русское раздолье; далее по Венцу влево, уже после особняков, за аллеями, Карамзинский сад;

%%[[page_137]]
КЫСОКОМ гранитном пьедестале 137. не бюст гуманиста, видного историка и писателя Карамзина Николая вокруг пьедестала большая песочница для малышей из сада по улице вниз — бывшая гимназия, а потом школа, в ней учились Гриша и Давид; с няней иду встречать их, они пулей вылетают на улицу, бегут навстречу, хватают меня за подмышки, я с радостью поджимаю ноги, и мы мчимся до пологого спуска. Гончаровская вся залита полуденным солнцем ещё тёплого Михайловия.

%%[[page_138]]
осеннего дня; таков замкнутый эллипс подвиг дороги моей жизни, дороги раннего и счастливого детства. Раннее детство счастливо потому, что оно не знает тяжёлых проблем жизни, его оберегают от них, но когда проблемы врываются в жизнь ребёнка, кончается счастье возраста. Так случилось на пятом году моей жизни, в 1929 году, но пока не об этом, вернёмся на несколько лет назад. Успешность непа привела простых людей к мнению, что экономические методы...

%%[[page_139]]
оды хозяйственных 139. Взаимоотношений и есть форма управления новой властью народным хозяйством. Пережив мытарства 1914-1924 годов, народ поверил в НЭП, и это сказалось на добровольном (подчёркиваю это слово) и активном трудовом оживлении в стране; жизнь, как обильный базар, предлагала своё разнообразие, и мамин брат Хаим в 1923 году уехал в Москву. Сразу же или в последующий ви-

%%[[page_140]]
в зифон заорал о В Москву младшего Янкеля брата, юношу шуби помог ему самостоятельно войти в жизнь те приметный, на первый взгляд факт передал живо 196-ную черту характера — помочь родственному человеку, оказавшемуся в тяжёлых условиях; когда наступили, времена, Хаим-Ефим Александрович Величанский (1900-1972) поддержал многих родственников, но об этом потом по мере хода истории страны. Мои родители наладили свою жизнь в

%%[[page_141]]
имбирске по 81924 Анна Ленина во моя будущая мама, Анна, добегалась с хорошей на заболевала: было на последних её беременности рождении свидания, о чём вы уже упоминали, способа рассказать о моем разная семейная стадо расскажу и вам, в ней тоже отражено время Ро- очень тяжело ды прошли бы жила, а ло, мама еле родился в соответствующем роженице состоянии, однако от религиозно национальных традиций отказаться не посмели; мне дели

%%[[page_142]]
имя прадеда
Еф по имени; взаимоотталкивающие физические состояния роженицы и ребёнка заставили срочно искать кормилицу. Надо сказать, что в те годы в СССР ещё не было широкой сети молочных кухонь для новорожденных, и во многих семьях судьбы ослабленных плачущих детей зависели от кормилицы.

%%[[page_143]]
в нашей семье пожилая крестьянка стала носить меня на кормёжки; я был настолько безнадёжен, что она, видимо, сжалившись, решила по-своему помочь; она после работы, вечером на руках укачивала меня, и в один из таких вечеров она отозвала папу и сказала: «Абрам, если хочешь, чтоб мальчик жил, купи церковного вина и давай по нескольку капель новатолько старухе».

%%[[page_144]]
117. Родители идею красного вина приняли, и папа каждый вечер макал тряпочный жгутик в вино: я стол поправил, поняв, что, убедившись, это я закрепился в жизни, мое рок зарегистрировали. в загсе 28-го августа. 2 декабря. не могу не сказать о владении 24:28 августа, но девятью годами.

%%[[page_145]]
470
Ами раньше, родился Давид: 2 декабря 1955. Вова родился женичка 24 июля 1944 года, и он тяжело ранен, и через блетк в тот же день родился мальчик.
Уровень медицины начала 20-х годов был ниже возможного уровня: многие врачи не жили, были изувечены в войнах и революциях; другие попали вне-большое число инте.

%%[[page_146]]
146.
интеллектуальной элиты, которую выслали из советской России; некоторые сами уехали в другие страны, другие приспособились к новым условиям. Лекарств и медикаментов не хватало. Многие лечились народными средствами. Помню рассказ мамы о том, как её лечили растопленным собачьим жиром, чтобы обезопасить лёгкие. Но жизнь семьи постепенно нормализовалась. Как вы поняли, речь идет о второй половине 1920-х годов, о которых частично сказано в предыдущем изложении.
Итак, коллективизация, начавшаяся после XIV съезда партии (1927 год), проходила в условиях и по методам, противоположным...

%%[[page_147]]
положением В.И. Ленина по этой проблеме; кроме того, отказ от экономических связей с крестьянством и переход к административно-насильственным методам управления отраслью привел к упадку сельского хозяйства, которое было восстановлено и развито за годы НЭПа. Разгром НЭПа коснулся и города. Зимой 1928/29 моего папу лишили права на магазин, весь товар конфисковали, отца арестовали, судили и сослали на лесоповал в Архангельскую область. Забота о семье легла на мамины плечи — женщины инициативной, но сломленной новой волной до конца её недолгой жизни. Брат Ефим и его жена Лена Маркова предложили маме приютить к ним моего старшего брата Гришу (16 лет); так он стал учеником токаря на заводе "Орг-металл" в районе Воробьевых (Ленинские горы).

%%[[page_148]]
148

ских) гор в Москве; вскоре таким же учеником токаря стал Давид (14 лет). Осенью 1929 года мама с плотом со мной приехала в Москву; мы четверо жили в отдельной комнате в квартире дяди Фимы. Мама начала работать ученицей ткачихи на гардинной фабрике на Плющихе, в отдельную небольшую комнату поселили бабушку, но когда женился дядя Яша, молодым негде было жить; их поселили в эту комнатку; бабушкину кровать перенесли в столовую. Здесь у них родился 2 мая 1933 года Шурик (Александр Яковлевич Величанский); о свадьбе дяди Яши и том, как Шурик чуть было не угодил в минское гетто, вы уже знаете. В ту же квартиру приехал из заключения мой папа. В третьей комнате размещались дядя Фима и тетя Лия; их дочка Розочка, моя младшая двоюродная сестра, спала в столовой, как и бабушка. От тесноты спасала большая кухня, на которой была огромная плита, ее не топили, но использовали как стол для примусов и керосинок; в выходные дни кухня служила семейным клубом.

%%[[page_149]]
Подробно о населении квартиры пишу для того, чтобы показать вам, как высоки были понятия родства, взаимопомощи. Впрочем, если выходной день выпадал в пятницу, устраивали субботний ужин, и вся квартира собиралась за большим обеденным столом; накрывали белоснежную скатерть, выставляли лучшую посуду, создавалась атмосфера праздника; на столе бывало довольно скромно - это были годы карточной системы. Бабушкино место было за серединой стола, у края - подсвечники, молитвенник, спички; когда все рассаживались, бабушка привычным движением поправляла на голове тонкую шаль, вставала, зажигала свечи и начинала громко и внятно читать субботнюю молитву: когда она потом переходила на понятный идиш, ясно было, с какой мольбой и страстностью она просила Бога помочь ей и нам пережить свалившиеся на всех трудности. В бабушкиных словах содержалась...

%%[[page_150]]
150

Жилась такая убежденность, что она зарождала веру и надежду. Наша жизнь в Москве начиналась с серых будней; утром я просыпался в пустой комнате; мама и братья с пораньше уходили на работу, я ни разу не слышал, как они вставали; дома была только бабушка и очень редко — Розочка; бабушка отводила меня на детскую городскую площадку при клубе им. Зуева; здесь меня встретили ребята недружелюбно, потому что меня звали Фроя, и они начали дразнить меня моим именем. Конечно, в некоторых обидных случаях я пытался отстоять свое имя кулаками, но получал в ответ хорошие тумаки; драться я перестал, но и дразнить перестали. Мучительные хождения на площадку вскоре закончились: меня отдали в школу. В Москве я начал быстро взрослеть: впервые подрался, братья научили меня самостоятельно.

%%[[page_151]]
переходить улицу с трамвайным движением; в Москве тогда еще не было размеченных переходов и пешеходных светофоров. Бабушку надо было освобождать от хождений со мной; в недавнем Ульяновске бабушка была вся моя, здесь - нет, из школы домой я не торопился; в начальных классах учился плохо, в нулевке и в первом классе была очень злая учительница. Мама приходила к вечеру уставшая и на короткое время ложилась передохнуть; я стоял у ее изголовья и ждал, когда она откроет глаза, потом она меня крепко обнимала и говорила что-то ласковое. Гриша и Давид приходили позже мамы, помимо работы на заводе они учились в ФЗУ (фабрично-заводское училище); там они повышали свою профессиональную квалификацию и получали общее образование на уровне средней школы.

%%[[page_152]]
появление в квартире молодой четы — дяди Яши и тети Берты внесло струю оптимизма и даже некоторого веселья; при всех сложностях жизни и проявлениях нервозности по отдельным поводам, дядя Яша оставался остроумным и веселым, он легко находил темы для шуток; тетя Берта была выдержанным спокойным человеком, думая о чем-то она всегда напевала, я любил вслушиваться в ее мурлыкание; появившийся крошка Шурик стал центром внимания. Утра выходных дней посвящались ему, его таскали, тискали, качали, кончилось тем, что я выронил его из рук и он упал на пол; моя мама испугалась за него, и тетя Берта сказала примерно так: «Ну, не волнуйтесь, ничего не случилось». В этой квартире дядя Яша с семьей прожили года два, потом он купил небольшую квартиру в доме в Шебашевском переулке, перестроил ее в комнату, пристроил небольшую кухоньку и сени. Там перед войной родилась Туля.

%%[[page_153]]
В один из зимних дней конца 1933 или начала 1934 года мама сказала мне, что мы пойдем встречать папу. Дядя Фима в то время был на высокой должности на крупном московском заводе, у него дома был служебный телефон, в те годы это говорило о многом; видимо, папа сообщил о приезде. Мы вышли на длинный тротуар, вскоре мама показала на едущего навстречу папу; я помчался к нему на привязанных к валенкам деревянных санках - колодках, я подлетел к нему, протянул руки вверх, он поднял меня на уровень своего объятия, подошла мама... Никогда не забуду нас на тротуаре, меня поразил папин вид: поверх шинели, до колен был намотан шарф, а на ногах - валенки.

Через небольшое время папа устроился мелким служащим в какую-то контору, видимо, помог дядя Фима. Наконец, семья была в сборе, все работали, наступило время заиметь свое жилье. Няня Яша подыскала нам неподалеку отдельную комнату и кухню.

%%[[page_154]]
Наш адрес был неочевидный: Князевский переулок, дом 3. Откуда "Князевский"? В облике домиков и хибар переулка я ничего княжеского не находил; возможно, что на этом участке земли когда-то жили княжеские холопы, поэтому сохранилось название "Князевский". Наша семья, все пять человек, поселилась в одной девятнадцатиметровой комнате. Мы жили в обстановке бедного духовного единства; это проявлялось поздними вечерами, когда все уже были дома. Материально жили очень скромно; все зарабатывали мало, продукты выдавали по карточкам. Папа, мама (она оставила работу) были прикреплены к продуктовой лавке на углу улицы Кирова напротив ресторана. Помню достраивавшуюся станцию метро "Пролетарская". Метрополитен еще не функционировал. В субботние выходные дни родители брали меня в магазин, там толпы, очереди по немереным талонам. Зачастую на определенной страничке предлагали книжки или слекнетики.

%%[[page_155]]
155

ный продукт, по другому номеру выдавали что-то другое и т. д. Устав очереди, мы добирались домой все Ру. Я бывал счастлив, когда мама давала мне кусок хлеба с маргарином. Мама доступными ей дешевыми средствами создала уютный дом; поздним вечером, когда над небольшим обеденным столом светился оранжевый абажур, становилось еще уютнее, все сидели за столом, больше деваться было некуда, кто-то рассказывал о событиях прошедшего дня, иногда папа читал вслух что-то из еврейской классики, или в переводе на идиш что-либо из русской литературы. В память о папе и с тех вечеров я привез в Израиль мраморный экземпляр поэмы А.С. Пушкина "Цыганы", в переводе на идиш (изд. "Сгенск", 1934 г., цена Эксп.). Наши единственные соседи-хозяева домика оказались очень симпатичными людьми: когда хозяин дома Константин Казимирович Семашко узнал, что мой папа родился и долго жил в Польше, он обрадовался и перешел с папой на польский язык, он проявил земляческие чувства в том, что подарил нам целую грядку в своем огороде. Мама разделила грядку на три части: укроп, петрушка, помидоры.

%%[[page_156]]
126
А он руководил всеми сельхозработами. Когда наступал период прополки (меня он звал курговы), он учил меня отличать сорняки от нужных ростков. Самое интересное было, когда пилили дрова. У Козлевых была такая же жилплощадь и легкоплита, как у нас, поэтому дрова у нас были общие. Папа, как опытный столяр, сколотил новые козлы и привел в порядок двуручные пилы. Отцы семейств пилили дрова, сыновья кололи, во дворе складывали в поленницы. Через несколько дней, когда дрова подсыхали, ребята перетаскивали их в сарай. В такие дни во дворе было весело: мерно и хорошо колун разваливал бревно, не смолкали шутки. Солнце светило. Жена Козлева, тетя Саша, была тихая работящая женщина. Через несколько лет, когда начали строить метрополитен и все домики Князевского переулка снесли, мы опять встретились.

%%[[page_157]]
17

Мы

Весной 1945 года, когда я уходил в армию, и уже не было ни мамы, ни папы, тетя Саша вышла меня проводить. Она держала меня за руку и сказала: "Ты уходишь под дождь, значит ты вернешься." Я вернулся на Князевский переулок: переезд произвел революцию в моем отношении к учебе, меня перевели в школу №152 (теперь это позади станции метро "Аэропорт"). Новая, построенная по типовому проекту, школа со светлыми просторными классами с паровым отоплением, с впервые появившимися удобными школьными партами, со специально оборудованными классами для занятий физикой и химией, с большим физкультурным залом, была новым словом в школьном образовании; большинство школьников еще жило в старых сырых подслеповатых домишках, и когда мы приходили в такую школу, было ощущение, что мы приходили в новый мир.

Еще важнее было то, что классным...

%%[[page_158]]
158

Руководителем была Клавдия Васильевна — молодая учительница, недавно окончившая педагогический институт; она буквально жила интересами класса, она устраивала дополнительные занятия, на которых оставались и отличники, — настолько было интересно быть с ней. В классе у меня появился первый настоящий друг — Женя Кошелев; он был отличником, я тянулся за ним. Это была середина тридцатых годов; его отец был слушателем Военно-воздушной академии имени Жуковского; Женя сказал, что после окончания академии они уезжают на Дальний Восток; тогда было необыкновенно почетным служить там. Женя обещал мне написать и сообщить свой новый адрес, но ни одного письма я не получил. Приближался 1937 год, поскольку в стране развивалась широкая репрессивная кампания, предполагаю, что его папа пострадал, как и многие другие командиры высоких рангов Красной Армии. В моей памяти 1937 год, помимо сказанного, сохранился ярко.

%%[[page_159]]
ким фактом наглядной агитации: наш переулок выходил на Ленинградское шоссе, а через дорогу — Московский протезный завод. На крыше здания к ноябрьским праздникам установили большой прямоугольный транспарант, ярко освещаемый десятками электролампочек. На транспаранте — две фигуры: Сталин и Ежов, оба в шинелях, застегнутых до верха, на правой руке Ежова рукавица, из которой торчит множество тонких длинных пальцев. Художник тоном цвета сделал так, что именно рукавица привлекала наибольшее внимание. Под фигурами короткий текст с обыгранными словами "вков неживые рукавицы". На фоне темного мрачного осеннего неба и редких слабых фонарей, яркое свечение транспаранта производило гнетущее впечатление. Я невольно представил себе подобное рукопожатие, стало страшно.

На соседнем дворе жила тоже еврейская семья: Немировские в 1931 году бежали с Украины от разразившегося голода, именно бежали, продать что-либо было невозможно.

%%[[page_160]]
они все оставили. Жили они предельно бедно: пятеро детей, работал только отец семейства; старший сын Яня после окончания Московского строительного института им. В.В. Куйбышева стал работать, семья ожила. Второй сын Лева был немного старше меня, он был вдумчивый, серьезный мальчик, мы подружились на всю жизнь (недавно Лева умер в Петах-Тикве). Левин отец Моисей Борисович работал бухгалтером и вечерами дома подрабатывал переплетными работами. Человек он был религиозный, поэтому он никогда не брился, а состригал волосы с лица короткими ножницами. В период ограничений религий в СССР он боялся, что будут утрачены национальные традиции; в кругу знакомых евреев, когда рождался мальчик, он уговаривал молодых родителей соблюсти обряд брит-мила. Будучи потомственным жителем еврейского местечка, Моисей Борисович знал жизнь народа и литературно-увлекательно об этом рассказывал; Лева многое запоминал, здесь в Израиле, он опубликовал это в еженедельнике "Еврейский камертон".

%%[[page_161]]
Яня, параллельно с работой, готовил кандидатскую диссертацию в области технических наук и защитил её. В те годы это было довольно редким явлением, особенно в еврейской среде: старшее поколение только не так давно пересекло черту оседлости.

В жизни Лёвы история страны сыграла иную роль. Он закончил Московский авиационно-технологический институт в самом конце 1940-х годов, когда в СССР разворачивалась большая репрессивная антисемитская кампания, поэтому молодому специалисту, еврею, Льву Моисеевичу Немировскому места в авиационной промышленности не дали; распределительная комиссия института направила его работать в Якутию (г. Пеледуй) на верфь деревянного судостроения; он разобрался в отличиях технологий и начал успешно работать; вскоре он стал уважаемым инженером.

%%[[page_162]]
Тревогу в Москве о положении евреев после сфабрикованного дела врачей помню хорошо; предполагали массовое выселение евреев, в то же время некоторые опровергали такое мнение, но многие вспоминали выселение чеченцев, ингушей, крымских татар и др. Достоверный, неопровержимый ответ на все эти волнения дал мне Лева после своего возвращения из Якутии. В переднюю стали прибывать баржи с грузом колючей проволоки; товарный двор стал быстро наполняться таким грузом; Лева спросил у начальника верфи о назначении проволоки; тот ответил, что это для евреев; если выселение состоится, то у нас, Моисеич, будет возможность найти твоих родственников, взять их сюда и устроить. Четыре года, прожитые в Князевском переулке (1935-1939 годы), были лично для меня единственным осознанным периодом, когда я жил в полноценной, здоровой семье; в это время формировалось мое мировоззрение. Мы очень сблизились с

%%[[page_163]]
семьей дяди Яши, и это прошло через всю нашу жизнь. Дом дяди Яши и тети Берты был своеобразной печкой, возле которой всегда можно было обогреться. Они были отзывчивыми людьми. Вот яркий пример: когда нашему Боречке было четыре годика, сложилось так, что не на кого было оставлять его дома; возить его в детский сад возле типографии, в которой работала Инна, было очень далеко и мучительно для ребенка; тетя Берта и дядя Яша взяли его к себе на всю зиму. После войны они квартирку-кухню поменяли на две комнатки с кухней; в одной из них стояли две кровати - для Шурика и Риты, Боречка посередине спал с каждым из них. Пекельку дядя Яша умел многому придавать шутливые формы, был он остроумен, к ним заходили отдохнуть душой, Яша также любил тех хулиганить, что сказалось и на нашем сыне, мы его получили с набором хулиганских стишков.

%%[[page_164]]
164. Федорино горе и умывальников начальник вытеснили сочинительства моего дорогого дяди. Его юмор, остроумие были с ним настолько органичны, что умер дядя Яша в признанный в мире День розыгрышей и шуток — 1 апреля (1976). Когда мы иной раз подходили к его могиле на Востряковском кладбище, видели его фарфоровый портрет на памятнике, у нас невольно появлялась улыбка; мы всегда вспоминаем дом на Шебашевском переулке. Тетя Берта намного пережила своего мужа, мы были на её могиле в Бостоне. В 1935 году меня отдали в музыкальную школу при моем активном сопротивлении: в те годы это было повальным.

%%[[page_165]]
165 явлением в еврейских семьях; аргументация родительских поколений: при царизме в черте оседлости учить детей музыке было недоступной роскошью, как и с другими формами светского обучения, поэтому родители в советское время хотели видеть в детях то, что было для них недоступно. Конечно, это принесло Советскому Союзу большой успех: вскоре международные музыкальные конкурсы дали такие имена, как Давид Ойстрах, Яков Зак, Миша Фихтенгольц и другие. В этом же ряду, несколько ранее, был вундеркинд Буся Гольдштейн. Музыкальная школа помещалась в старом двухэтажном здании недалеко от станции метро "Аэропорт". Впрочем, мои страдания были недолги: моя учительница Александра Владимировна Шик — интеллигентная, пожилая красивая дама — обнаружила у меня хороший музыкальный слух. Второй скрипичный класс вел Василий Васильевич (фамилию не помню).

%%[[page_166]]
166
Помню, он же руководил и дирижировал школьным скрипичным оркестром; через какое-то время меня включили в оркестр, чтобы ученики Александры Владимировны не подвели её. Она часто по выходным дням устраивала у себя дома репетиции, к ним я готовился с большим старанием.
Я вспомнил, как дорого стоил складной металлический пюпитр, а также футляр для скрипки, поэтому папа сколотил пюпитр из фанеры, а мама сшила из плотной ткани чехол для инструмента.
В 1939 году я сам ушел из музыкальной школы, потому что нас переселили в далекий район, тяжело заболела мама. После войны я собрал свои ноты и сжег их в печке: не они были виноваты передо мной, а время, которое сделало их ненужными для меня.
За годы "князевского сидения" определились жизненные пути моих братьев, Гриши и Павла.

%%[[page_167]]
167, дельные части моего изложения. Хочу подчеркнуть, что этот период совпал с последними годами тринадцатой пятилетки (1933-1937), пятилетки больших контрастов, преступлений и в то же время успехов в индустриализации страны. Гриша связал свою судьбу с техникой, и с завода „Орг-металл" перешёл на авиационный завод №1, одновременно он поступил на вечернее отделение Московского авиационного института. Давид — человек общительный, умевший привлекать к себе людей, принял предложение работать во Фрунзенском Райкоме комсомола Москвы.

Клон Григорий Абрамович (03.01.1913-10.05.1974). Мой родной, старший брат. Незадолго до начала Великой Отечественной войны, все годы войны и до выхода на пенсию (с перерывом на арест) возглавлял СКБ (специализированное конструкторское бюро авиационного завода №1). В советское время человек, проработавший столько лет на одной высокой, руководящей должности, справедливо считался классным профессионалом в своём деле. Требования к профессионализму были высоки, на таких руководителях держалась слаженная организация производства на крупных промышленных предприятиях.

%%[[page_168]]
168

В начале войны завод эвакуировали. Григория Абрамовича оставили в Москве. Когда начала поступать новая техника, группа специалистов занялась ускоренным освоением и монтажом нового оборудования. К заданному сроку фактически новый завод продолжил выпуск боевых самолётов. Когда, спустя годы, предприятие сделали показательным для высокотитулованных гостей, завод получил название и знамя труда.

Григорий Абрамович принадлежал к той части технической интеллигенции и сам был её типичным представителем, для которой интересы работы, бескорыстное отношение к ней входили большой составляющей в жизнь. Эти люди своим умом, знаниями, отношением к долгу и труду подняли страну до уровня промышленно развитых государств, практически эти люди создали военную мощь СССР. Я хорошо помню атмосферу взаимной доброжелательности, добрососедства, интернационализма, которая царила в быту между людьми, несмотря на весьма сложную, а в материальном плане довольно скудную жизнь. Однако после всего сделанного, пережитого запустили термин "безродные космополиты". Термин преподнесли как ясный синоним слову "еврей" в отрицательном смысле. Колесо явного антисемитизма покатилось, к его ободу налипала грязь шовинизма, которая стала разъедать страну. Первоначально думалось, что это месть Сталина евреям за то, что воссозданное еврейское государство Израиль не вошло в орбиту стран советского влияния.

Но в

%%[[page_169]]
На пике оголтелого антисемитизма арестовали моего брата. Пришли за ним ночью, начали с обыска. Утром, когда на кухне заработало не выключаемое на ночь радио (по привычке военных лет), мир впервые услышал о смерти Сталина. К этому часу "улику" нашли: книгу «Михоэлс», советское, открытое издание на русском языке о творчестве актёра, его беседах с видными советскими режиссёрами, артистами и т. п. Оказалось - это националистическая литература.

Наступавшие прогрессивные перемены - оттепель Н.С. Хрущёва не сразу затронула аппарат пожирания граждан своей страны, поэтому моему брату успели, как тогда говорили, припаять 25 лет; его загнали в Магадан. Там тоже ничего к тому времени не изменилось. К пригнанным заключённым вышел начальник лагеря. Он несколько раз важно и медленно прошёлся вдоль линии заключённых и с претензией на глубокомыслие спросил: «Как вы думаете, кто скажет, для чего вас привезли сюда?» Молчание. Пройдясь ещё несколько раз перед заключёнными, сам же громко и угрожающе ответил: «Вас привезли для удобрения Колымского края». Незабываемо содержание фразы, как и время, которое её позволяло.

Реабилитация состоялась, и 18 мая 1956 года мы встречали Гришу на Ярославском вокзале. В Москве приемущему на высоком уровне выразили сожаление по поводу случившегося и, в частности, сказали, что в последующих оформлениях воки

%%[[page_170]]
служебную командировку. Ему вернули партийный билет и предложили прежнюю должность на том же заводе. Он работал в этой должности ещё более семи лет, до выхода на пенсию. Гриша говорил мне, что ему предлагали поработать ещё, но он чувствовал себя выношенным, а работать впожилым не та порода руководителей. И действительно, в качестве пенсионера он прожил немногим более года. Присутствие на прощании с Григорием Абрамовичем большого числа коллег по работе невольно свидетельствовало о его авторитете, об отношении к нему. Гришу угнетали не только жуткая несправедливость всего пережитого, но и судьба его жены. Она оказалась тяжёлой жертвой происходившего. Ольга Ивановна Клоц (Михайлова) была тихой, скромной, малоинициативной, доброй женщиной; когда-то они вместе работали. Для неё понятия "Гриша" и "вина перед государством" были абсолютно несовместимы, как и для всех, кто хорошо знал моего брата. Для Оли арест Гриши оказался совершеннейшей неожиданностью. Её охватило смятение, с ней случилось буйное помешательство, её лечили в психиатрических больницах имени Кащенко П.П., имени Ганушкина П.Б. Освобождение мужа не успокоило её, хотя до последнего дня своей жизни Гриша оберегал Олю от волнений. В приглушённом лекарствами состоянии Оля дожила до конца своих дней; умер и их единственный сын. Не удивительно, что тень мерзкого времени присутствовала в их скромном доме.

%%[[page_171]]
171

Мой второй родной брат Клоц Давид Абрамович. Наша мама умерла 8 октября 1939 года, ей было всего 47 лет. Семья быстро распалась. Давид уехал служить в кадровую армию, Гриша поселился в заводском жилом доме. Давид Абрамович встретил войну на Кавказе, когда немцы рвались к нефтяным источникам. Он служил в бомбардировочной авиации. Когда на территории СССР сформировалась польская народная армия, их часть поддерживала наступление поляков на родине. После падения фашистской Германии мой брат служил в Китае пять лет, потом на Сахалине (Корсаков), оттуда в Ленинград и вскоре на Украину, в город Винницу. Там в звании подполковника ВВС он вышел на военную пенсию (по возрасту). Вскоре он тяжело заболел, его похоронили с воинскими почестями на Старом кладбище.

В 1951 году, незадолго до отъезда на службу на Сахалин, Давид женился. Мария Яковлевна Клоц (Злотник) тогда была молодым специалистом-врачом-терапевтом, на Сахалине она работала в больнице. По бедности медицинских кадров ей пришлось самостоятельно освоить и другие медицинские профессии. Года через два мы с Гришей встречали ее во Внуково: она летела к родителям рожать. Давид впервые увидел своего сына Шурина примерно года через два.

%%[[page_172]]
Более 10 лет семья Давида живет в Израиле. Всеобщей гордостью является Дашенька — внучка моего брата; она отлично закончила школу, отслужила в Армии обороны Израиля, овладела ивритом и английским, поступила в Технион, учится отлично, в настоящее время заканчивает учебу. В рассказе о Давиде хочу подчеркнуть, что судьба неоднократно связывала его с Польшей и оставила посмертную память об этом: в числе одиннадцати правительственных наград есть два польских ордена и медаль. Годы второй пятилетки характерны были тем, что проявлялась ускоренная индустриализация: открывались производства на новых заводах, фабриках, вступали в строй новые доменные печи, линии железных дорог и т.д. Люди, измученные кровавыми передрягами последнего двадцатипятилетия, верили, что придет, наконец, нормальная жизнь. Начали развиваться отдельные отрасли промышленности, например автомобильная, авиационная. На базе этого рос патриотизм: разумно по-

%%[[page_173]]
лагали, что больше стали — это больше тракторов, комбайнов, хлеба и других нужных производств. Люди добросовестно работали, чтобы быстрее приблизить это время; появились ударники производства, мастера своего дела, которые обучали других. Отлично поставленная пропаганда была в струе этих настроений, звала народ только вперед, не оглядываясь назад. Пропаганде развития отраслей народного хозяйства помогло неожиданное обстоятельство.

Популярность авиации началась с неожиданного факта — спасения «Челюскинцев». Напомню отдельные эпизоды. Новый пароход «Челюскин» (капитан В.И. Воронин) с научной экспедицией на борту (руководитель О.Ю. Шмидт) должен был за одну навигацию пройти из морского порта Мурманск (Баренцево море) вдоль Северного морского пути, затем Берингов пролив, спуститься на юг, в Тихий океан,

%%[[page_174]]
174

и войти во Владивостокский морской порт. Но 13 февраля 1934 года льды Чукотского моря раздавили корабль. Весь экипаж судна оказался на плавающих ледяных полях. Спасти людей можно было только самолётами, а они до тех пор на дрейфующие льды не садились. Началась бешеная работа по организации спасения. Всё должно было завершиться до появления талых вод на льдинах. Спасли всех. Затем триумфальная встреча в Москве; утверждение высшего почётного звания за выдающиеся заслуги перед государством «Герой Советского Союза». Первые Герои Советского Союза - лётчики, спасшие экипаж «Челюскина», были легендой времени. И не только по самому факту спасения, но и потому ещё, что они подняли планку возможностей новой в мире промышленной отрасли - авиации. Назову Героев, имена которых помнятся мне с детства: Водопьянов Михаил Васильевич, Каманин Николай Петрович, Леваневский Сигизмунд Александрович, Ляпидевский Анатолий Васильевич, Молоков Василий Сергеевич, их успех приоткрыл дверь в арктическое небо. [Глядя на физическую карту мира, поражаешься грандиозности замысла экспедиции. Мелькнула мысль: организовать бы по тому же маршруту туристический круиз в составе каравана торговых судов под проводкой современного атомного ледокола, впечатлений, понимания труда старших поколений было бы не

%%[[page_175]]
175

В том же 1934 году М.М. Громов установил мировой рекорд дальности полёта на самолёте — 12 тысяч километров. В 1936 году В.П. Чкалов совершил беспосадочный перелёт Москва — остров Удд на Дальнем Востоке, а С.А. Леваневский без посадки перелетел из Лос-Анджелеса в Москву. В 1937 году В.П. Чкалов в экипаже с Г.Ф. Байдуковым и А.В. Беляковым совершил беспосадочный перелёт на одномоторном самолёте «АНТ-25» по маршруту Москва — Северный полюс — Ванкувер (США). Вскоре после В.П. Чкалова М.М. Громов с А.Б. Юмашевым и С.А. Данилиным тоже на «АНТ-25» прошли по маршруту Москва — Северный полюс — США. Мир буквально рукоплескал этим двум экипажам. Чтобы понятнее была степень новизны, которую принесли эти полёты, приведу два факта. Когда в газетах появились фотографии «АНТ-25» с трёхлопастным винтом, это было воспринято как инженерное чудо; с «челюскинского» периода не сходила тема обледенения крыла при полётах в Арктике. Сопоставьте это с возможностями современной авиации, хотя бы гражданской.

Далее. Май 1937 года. В.С. Молоков участвует в воздушной экспедиции на Северный полюс по высадке на дрейфующие льды Северного Ледовитого океана группы полярников под руководством И.Д. Папанина. В 1938 году женский экипаж — М.М. Раскова, П.Д. Осипенко и В.С. Гризодубова — совершил беспосадочный перелёт на двухмоторном самолёте.

%%[[page_176]]
176
моторном самолёте (бомбардировочного типа) по маршруту Севастополь - Архангельск и Москва - Дальний Восток. Были потери. В.П. Чкалов погиб в 1938 году в тренировочном полёте. С.А. Леваневский с большим экипажем на тяжёлом транспортно-десантном самолёте отправился в беспосадочный перелёт из Москвы через Северный полюс в США. В районе полюса самолёт разбился, экипаж погиб.
Помню предстартовую фотографию, опубликованную в газете: шеренга молодых людей, если не ошибаюсь, человек шесть, - в толстых свитерах, обнявшись за плечи, радостно улыбаются на фоне своего самолёта. После потери радиосвязи с экипажем поползли грязные слухи. То было время невиданных контрастов в жизни страны. На сей раз восторжествовала горькая правда: они погибли на пути к мировому рекорду, который приумножил бы авторитет СССР как восходящей индустриальной державы.
Все эти полёты на рекордные показатели имели многофункциональное значение, потому что авиация как новое военно-экономическое явление становилась визитной карточкой страны. Впрочем, как и в США. Там авиация по существу начала развиваться с середины 20-х годов XX века, но на базе самой передовой индустрии. Многофункциональность советских полётов состояла в том, чтобы, во-первых, испытывать авиатехнику в сложных и даже экстремальных условиях; во-вторых, авиация...

%%[[page_177]]
177

В результате прорывов советской авиации к вершинам мировых уровней сложилось так, что авиация стала материальным стержнем подъема советского патриотизма, особенно в молодежной среде: открывались гражданские авиаклубы, где осваивали парашютный и планерный спорт; в парках культуры строили парашютные вышки, и молодые люди стояли в очередях, чтобы разок парить в воздухе. Над всем индустриальным обновлением страны витал образ И.В. Сталина; все знали, и это умело пропагандировалось, что ни один кардинальный вопрос не решался без него; во всем видели Сталина, ему верили, его считали всемогущим. Возвеличивание правящей личности никогда в истории России не было редкостью. Сейчас, оглядываясь на историю СССР, можно сказать, что И.В. Сталин начал создавать культ своей личности с того, что фактически перечеркнул Новую экономическую политику В.И. Ленина. О тяжелых последствиях отказа от экономических методов управления народным хозяйством страны вам уже рассказал.

%%[[page_178]]
178. Несмотря на все тяготы бытовой жизни (я уже сказал), патриотизм лет Второй пятилетки нарастал; спокойное лицо Сталина смотрело на вас с многочисленных портретов; всем успехам, новшествам, достижениям приписывалось слово сталинское: Сталинская Конституция, сталинские соколы (лучшие летчики), сталинские маршруты и т.д.; исключением были только ворошиловские стрелки. Особенности жизни страны были неотъемлемой частью жизни каждой семьи, между тем каждая семья жила своим укладом. К началу 1959 года мы обосновались в новой квартире — две комнаты в трехкомнатной квартире. Приближалась весна, а с ней и Песах; кошерной посуды в те годы ни у кого не было, поэтому пасхальную посуду готовили к празднику сами: во дворе разжигали небольшой костер, в него клали вымытые булыжники, и раскаленные в огне камни клали в кастрюлю с горячей водой, то же делали с ложками, вилками, ножами.

%%[[page_179]]
179, кой процедурой как бы обеспечивалась кошерность. Второй этап - маца. Городская власть разрешила печь ее в какой-либо пекарне на окраине города; в предпасхальные дни к этой пекарне стекались тысячи людей; приезжали утром, уезжали вечером; мацу упаковывали в новые белоснежные наволочки, дома клали ее на крышки шкафов, буфетов, где не лежали другие продукты. Очереди за мацой состояли, в основном, из женщин, а приезжавшие с ними дети весь день играли и веселились. Когда в дом приносили мацу, создавалось настроение праздника. Пасхальный сейдер проходил, конечно, не так, как я потом увидел в Израиле; стол был очень скромный, но всегда были фаршированная рыба и бульон, похальную Агаду не читали, но папа относительно подробно ее пересказывал; на Песах привозили бабушку, приглашали дядю Яшу с семьей. Главным было не соблюдение формы праздника, а понимание его значения, это объединяло нас, семью. Никто не мог предположить, что этот праздник в 1939 году был последним, когда вся наша семья была в сборе.

%%[[page_180]]
После тех самых пасхальных дней мама сказала папе, что все те дни ей трудно было глотать мацу и другую твердую, неразмягченную пищу. Родители пошли к доктору, и с этого визита в поликлинику всё началось: врачи установили диагноз — рак пищевода. Тогда начало внедряться радиоактивное облучение раковых больных; папино начальство уважительно и сочувственно относилось к нему, оно сумело добиться разрешения моей маме пользоваться услугами закрытой поликлиники. На лечение мама ездила раз в несколько дней, но оно обессиливало её и не помогало; время шло, мама явно слабела. Однажды, в выходной день тетя Саша позвала к себе маму (по папиной просьбе); папа пригласил нас троих сесть за стол; он был удручен, манжеты его рубашки были засучены, его сильные руки, сжатые в кулаки, лежали на столе, голова наклонена — это плохой признак; меня впервые пригласили к серьезному разговору взрослых, какое жуткое для меня совпадение: меня впервые признали взрослым, когда речь пошла о конце жизни мамы (даже сейчас это трудно воспринять).

%%[[page_181]]
Папа говорил размеренно, напряженно, в основном на идише, так ему легче было выразить оттенки мысли: после окончания курса лечения врач сказал, что мама безнадежно больна, незачем таскать ее по врачам, дайте ей спокойно умереть. Молчание, довольно долгое. Я сидел рядом с Гришей, мое волнение он, видимо, принял за желание что-то сказать, он своим коленом придавил мою ногу и сжал губы как знак молчания. Но я и не пытался говорить, слишком сильны были мои переживания. Папа не показал своего волнения, когда спазмы в горле прошли, он продолжил: доктор знает, о чем он говорит, но мы не должны с этим смириться, надо сделать все, чтобы продлить мамину жизнь; если вскоре случится так, как сказал доктор, то наша семья в третий раз уже не поднимется. Он сказал, что будет пытаться получить консультации светил медицинского мира того времени — профессора Кончаловского и профессора Юдина. Я впервые услышал эти фамилии.

%%[[page_182]]
Последний месяц школьных каникул 1939 года я провёл в пионерском лагере; встречал меня Давид, мы поехали к нему на работу, в Фрунзенский райком комсомола, который находился в одном из переулков, примыкающих к улице Кропоткина. Через несколько часов мы вышли на Кропоткинскую, в магазине «Цветы» Давид купил букет тёмных астр и сказал мне: «Подаришь маме». Мы приехали на закате солнца. Мама полулежала на диване, прикрыв ноги старым деревенским платком; за день мама намоталась по хозяйству и прилегла, как обычно, отдохнуть, но за время обнаружившейся болезни мама чаще делала это и днём, поэтому я не придал значения факту, что после почти месячной разлуки мама встречает меня лёжа. Произошедшее через мгновение...

%%[[page_183]]
483.
Новое осознание — потрясение навсегда запало в мою душу.
Я подлетел к маме, плюхнулся на край дивана, цветы сунул ей в плечо, прижался к лицу и ощутил непривычное — огрубевшую кожу, скулы. Мама, в свою очередь, ладонями сжала мои уши, приподняла мою голову так, что наши глаза встретились, как никогда близко за 26 дней моего отсутствия (столько длилась смена в пионерском лагере). Мама, которая была изящной невысокой фигурой, очень похудела — болезнь заметно ложилась на её облик: в меня смотрели уменьшившиеся, потерявшие цвет и теплоту глаза. Я приковался к ним, смог только сконцентрировать свой ответ.

%%[[page_184]]
1184. ный взгляд на левом неподвижном глазе, точнее, на правом зрачке через него я улавливал тонкую нить проницательного, тревожного, даже строгого взгляда, идущего из непостижимых глубин, которые мама, очевидно, ощущала в себе; у неё начали проявляться другие понимания контактов с близкими, она часто молчала; уходя из живого мира, мама, опять же очевидно, хотела что-то оставить в себе. Я пытался до конца понять содержание её многосложного взгляда, напряжённо вглядывался в зрачок, а что она увидела в моих глазах? Возле переносицы, в углубившихся ямочках от впавших глаз появились дужицы слёз. Я подтянул платок с маминых ног, промокнул им мамины слёзы и хотел было сказать, чтобы она не плакала, всё будет хорошо, но я не мог врать: хорошо уже не будет.

%%[[page_185]]
один мускул маминого лица не дрогнул, она плакала мольча и продолжала смотреть в мои глаза. Так плачут в состоянии наивысшего душевного и нервного напряжения. Мама прощалась со мной, через сложившиеся знакомства в медицинском мире папа получил согласие профессоров на консультации, но, подчеркну, только в сентябре. Тяжёлое впечатление от рассказанной вам встречи с мамой не увязывалось в моей душе с возникшей в семье надеждой после согласия профессоров. Объяснить настроение семьи можно только тайным желанием благотворного и исторически традиционной верой в особую всевозможность человека высочайшего звания, титула, должности. Между тем, мама слабе...

%%[[page_186]]
186. Да, возникла проблема срочной медицинской помощи на дому, особенно ночью: телефон-автомат находился далеко, да не было уверенности в том, что телефонная трубка цела. Добавлялось и другое: наш новый многоквартирный дом построили в низинке, куда стекала дождевая вода, поэтому нижние ступеньки крыльца заливало, жильцам приходилось прыгать по камням и ящикам, но профессора прыгать не станут. Приближался осенний (и многозначительный с точки зрения истории) конец августа - сентябрь; всё мелко: бытовые неувязки решились перемещением дяди Фимы и тёти Лизы переехать к маме к ним на какое-то время.

%%[[page_187]]
I Подготовку к переезду 187. папа начал с «закрепления тыла»: он пошёл к директору моей школы (603), Кустжанину Ивану Дмитриевичу. Рассказал, почему на какое время я остаюсь дома практически один и попросил у и.о. разрешение позвонить в школу, если возникнет острая необходимость передать мне что-либо важное; понимание и согласие было мгновенным; мне папа дал указание — после занятий обедать в столовой (общественного питания), обязательно кушать суп; отъезд намечался дней на 20, поэтому мне предписывалось быть молодцом: правда, папа сказал это не в назидательно приказном тоне (что случалось нередко), а в тоне некого дружелюбного на-

%%[[page_188]]
поминания об обыденном; когда папа говорил всё это мне, выглядел он уставшим и без того худой, он, казалось, похудел ещё. После отъезда мамы я быстро становился молодцом после нескольких дней проб водянистых супов, как я их назвал, на мясном (в меню столовых писали: суп на мясе, то есть на мясном бульоне) и после слипающихся макарон. Я перешёл на консервы «Сазан в томате» и на пластовый мармелад «облочный». Так я обеспечил себе ежедневное питание. В школу я брал 4-5 толстых кусков хлеба с проложенными не менее толстыми слоями мармелада. Когда Гриша или Давид иногда приезжали с ночёвкой, я потчевал их тем же. Папа домой не приезжал: после работы он ехал к маме, чтобы сменить бабушку и нанятую в помощь женщину. Папа дежурил возле мамы.

Визиты профессоров

%%[[page_189]]
как не улучшили ситуацию, каждый из профессоров приезжал в выходной день. Я тоже навещал маму по выходным, поэтому хорошо помню разнохарактерные облики этих людей: профессор Кончаловский М.П. – интеллигентный, спокойный человек, с добрыми глазами, продолговатым лицом и удлиняющей его остроконечной (казачьей) седой бородкой, профессор Юдин С.С. высокий, грузный, с красным лицом и, кажется, с бритой головой. Несмотря на внешнюю грубоватость, взгляд профессора выражал сопереживание, а это нейтрализовало первое впечатление моё. После окончания знакомства с больным, профессоров, как и в первом случае, пригласили в столовую комнату на стакан чаю. Там проф. Казалось, долго беседовал с папой и дядей Фимой; когда дядя

%%[[page_190]]
190. Пошёл провожать профессора до машины, я впился глазами в папу, помню, как улыбка вежливости сошла с его лица. Он подошёл к кухонному столу и с тихим причитанием швырнул большой конверт с мамиными медицинскими документами. Об этом лучше не спрашивать. Под влиянием папиных усилий добиться визитов двух столпов советской медицины, которые занимались, в частности, проблемами лечения заболеваний пищевода, в нём притаилась слабенькая надежда (видимо, оттянуть время), но папино лицо всегда отражало состояние его души. Папа не был мастером камуфляжа; надежды ни у кого никаких, воцарилась реальность, продиктовавшая бездеятельность. Только разрушать организм продолжало. О возвращении домой уже не говорили. Я по-прежнему навещал маму по выходным дням. В октябре 1938 года, когда...

%%[[page_191]]
Рассвет ещё не пробился до конца сквозь утренний туман. Раздался сильный стук в окно. Я вскочил с постели и увидел пальцы Давида, барабанящие по стеклу... И ожидаемое приходит неожиданно; маме было 47 лет (на памятнике — 45). Моя жизнь с папой складывалась довольно трудно: он разом потерял семью, остался с заботами о младшем сыне и повседневных делах жизни; с последним словом: многие бытовые дела решали вдвоём, вплоть до... "После рода братьев Постарушек, но сами попали в тупик — никто не соглашался постоянно готовить нам. Мы перешли на столовские обеды (на них дотянули до начала войны, когда опять ввели для населения продажу по карточкам недостающего в стране продовольствия). Обедать в столовых стало крайне разорительным по отношению к скудному товарному содержанию продовольственной...

%%[[page_192]]
карточки и низким питательным качеством. 192. Вам обедов: однако долго ломать голову не пришлось: общедоступные столовые общественного питания закрыли, перепрофилировали. После смерти ма- резому мой дом потерял для меня свою притягательность. Всё в нём смотрелось по-иному, отчужденно, поэтому я часто задерживался в школе. Там делал уроки, помогал приводить в порядок оборудование физического и химического кабинетов и пр. Я скучал по времени, в котором мы - все пятеро были вместе. Короткие письма Давида и редкие приезды Гриши подчёркивали тяжесть сложившегося положения. И здесь папа с его свойственной ему решительностью попытался отде

%%[[page_193]]
лить возможное

193. от невозможного: весной 1990 года мы поехали к моей бывшей няне — Ксении. Это с ней я гулял по Ульяновску моего раннего детства; поездке я обрадовался вдвойне: Ксения так же жила в Ульяновске. Она жила в общежитии при какой-то фабрике в районе станции «Планерная», а недалеко был авиазавод и аэродром «Тушино».

До войны ежегодно отмечали день авиации. Возможность увидеть и то и другое вдохновила меня.

— Итак, Ксения — последняя попытка папы сохранить атмосферу привычного ему дома с моей помощницей дотянуть меня. С Ксенией я не виделся 11 лет, она встретила меня с большой радостью, ее широкая улыбка и искрящиеся глаза выражали восторг. В моём подсознании всплыли и очевидно, далекие ассоциации, когда я был в основном между мамой и Ксенией, их

%%[[page_194]]
образы в раннем детстве смешались, поэтому в Ксене я почувствовал почти родное, материнское. Я был взволнован, бывшие в комнате женщины незаметно ушли, мы сели за стол, который стоял посередине большой комнаты. Я огляделся: было очень чисто и светло, по четыре кровати с тумбочкой при каждой стояло перпендикулярно к противоположным стенам. Напротив входной двери, в высокой части простенка между окнами, — портрет И.В. Сталина. Дыхание выходного дня чувствовалось: все кровати застланы белыми покрывалами, удлиненными вязаными подзорами. Только один, связанный из толстых ниток и углами книзу, понравился мне больше остальных.

%%[[page_195]]
Были те годы, когда в качестве корреспондента редакции экономического журнала приезжал в колхозы, совхозы, ночевал в крестьянских избах, у меня была возможность воспринять разнообразие красот орнаментов вышитых и вручную вязанных подзоров. Ксеня принесла чайник, разлила по стаканам чай и молча положила на стол папино письмо. Я почувствовал, что искорка волнения охватила нас. Чай не пился; разговор начала Ксеня с расспросов о моей учёбе, заказано ли надгробие на мамину могилу, как мы справляемся дома. На вопросы односложно отвечал я; потом она обратилась к папе напрямую и сказала, что письмо читала несколько раз, всё себе представляет, очень жалеет нас, но с фабрики уходить не хочет, потому что стала ударницей, сложились хорошие отношения в коллективе.

%%[[page_196]]
196

Подругами; через Ксенины слова пробивались представления о новых общественных ценностях того предвоенного патриотичного времени. Папины предложения не противоречили интересам времени, но звучали архаично; он приехал за ответом на письма, однако при личной встрече пересказал его, пытаясь переубедить Ксеню; безуспешно. На основе давнишнего доверия папа предложил Ксене переехать к нам, быть хозяйкой дома. Мы отдаём ей вторую, изолированную комнату. Папа не в силах быть, как он выделил, «хозяйкой». Убеждён, что подтекста в его словах не было, но в дальнейшем я бы хотел видеть папу неким. Визит к Ксении удручил его. Он явно сник: «Кому довериться?». Я очень жалел папу, поэтому с наигранным мальчишеским задором призвал его не унывать, выплывем. Когда наступили школьные каникулы, я уехал в пионерский лагерь. Попытки смять закончились условиями нашей жизни.

%%[[page_197]]
2. Родственники
197 Если бы родственники были внимательны к нам, усиленно приглашали на выходные дни: отчасти такое внимание объясняю тем, что тётя Соня и братья очень любили свою сестру Гутю, это переложилось в определённой мере на папу и на меня. Жизнь не останавливалась: вскоре после смерти моей мамы тётя Берта родила девочку (23.1.1941), назвали в честь и в память о маме Рита (имя Гита осовременили заменой первой буквы); имя принялось неплохо: Рита всегда красива, эффектна, высока, имела большой успех: сейчас она счастлива тесным общением с семьёй своей дочери Яночки, тремя малолетними внучатами; живет Рита недалеко от своего старшего брата Шуры Великанского, с которым я вас уже познакомил. Да, моральная поддержка родственников имела большое значение, но неприкрашенная действительность оставалась при нас.
Теперь, как и в предыдущих фрагментах воспоминаний, выйду из круга личного и соединю его с жизнью страны.

%%[[page_198]]
1989

Война пришла через весьма прозаичные факты. Сухой, июньский выходной день оказался кстати для навеленца Пореска. В доме папа вынес на улицу зимнюю одежду и развесил её для просушки, как сказала мама, протер где-то в доме и начал мыть пол. Окна были раскрыты, радиотарелка включена, когда диктор сообщил, что будет говорить товарищ Молотов.

- Неживость поля дневной радиопередачи - я насторожился. После первых слов В.М. Молотова я громко крикнул папе, что Гитлер напал на нас. Папа ответил кивком головы; многие жильцы нашего дома в тот момент были на улице. Никакой паники, никакого смятения, только раньше обычного разошлись по домам. Тема войны с германским фашизмом, как я рассказал вам на предыдущих страницах, давно висела в воздухе. Папа предложил поехать в центр города. От станции метро Маяковская по улице Горького.

%%[[page_199]]
мы медленно пошли 199. к Манежной площади: Тоже движение людей, машин, замерли открыты, никаких эмоциональных проявлений после плохой вести; только на противоположной стороне улицы Горького у входа в сберегательную кассу суетно толпилась плотная кучка людей; Поля осуждающе оценил этот факт и медленно, как наш шаг, стал рассказывать мне о своём понимании начавшегося сегодня сверхсложного времени. С того дня сегодня шло более толп, однако первый день войны и наша попытка уловить самое начало понятия её, - особенно мною.

%%[[page_200]]
Помнятся как семи 200. Мутная явь. Папа начал рассуждения с эпизода у сберкассы как одном из первых признаков начала Большой войны: с огромным интересом, присущим неведающему, но любознательному человеку, я, как говорится, разинув рот, слушал папу. Я впервые понял силу маленьких, рыженьких рублёвок и зелёненьких трёшек. На примере Гражданской войны и разрухи, когда цены достигали фантастических размеров, была скрытно разъяснена суть глубокой инфляции и т. д. Папа говорил размеренно, бытовым языком.

%%[[page_201]]
201 ком, да ещё с вкраплениями еврейских слов, в светском понимании. Малограмотный Левец он не знал политэкономической терминологии, но до объяснения многих явлений дошёл своим умом. Это была последняя длительная прогулка, какие мы совершали после смерти мамы: нас гнало из дому её отсутствие, вечерами, когда папа приходил с работы, мы кушали и довольно часто выходили бродить по тропинке.

%%[[page_202]]
вам нашего невзрачного посёлка Коптева я брал папу под руку, за что он всегда благодарил и был рад заслужить папино спасибо; после относительно долгого молчания и под медленный шаг разговор начинал папа с вопроса - на всякий случай, в товарищеском тоне, о школе: он часто просил рассказать что-нибудь интересное, необычное из услышанного на уроках и спрашивал моё мнение о учителе, но сам никогда не пересказывал, однако на родительские собрания не ходил, дневник не смотрел, мою учёбу возложил на меня же. Это была одна из форм доверительных отношений, какие установил между нами после смерти мамы. Он старался воспитать меня ответственным человеком.

%%[[page_203]]
203. Повышение уровня доверительности не означает панибратства, да я к этому и не стремлюсь. После смерти мамы я почувствовал некое утончение слоя защиты и непроизвольно пытался поглубже забиться под папино крыло; папа по-иному видел ситуацию и по-своему стал вводить меня в многообразие реальной жизни. Мои домоводческие функции постепенно расширялись: сперва я стал ответственным за наличие керосина, по своевременную оплату коммунальных услуг; изредка приходилось ездить в домоуправление, где работали в основном женщины; они знали мою маму, поэтому ко мне относились немного внимательнее, заботливее, в инструкциях малозначимых слов пробивалась же недоступное.

%%[[page_204]]
204

для меня материнское тепло. Редкие, душевные эпизоды запомнились, видимо, потому, что они совпадали с моим настроением после смерти мамы и контрастировали с обычным отношением. В таких неродственных случаях, потому-то вспоминая предвоенный период, я вижу в копилке ценностей моей памяти этих милых, обыкновенных женщин. Из того же домоуправления привезли лопаты, грабли вскоре после начала войны, кто-то из приехавших сделал разметку земли: траншея должна вместить всех жильцов восьмиквартирного дома в случае сигнала тревоги.

%%[[page_205]]
205. Взялись за работу. Трудно было снять верхний слой земли, дёрн, потом работа пошла легче, но когда уже прилично углубились в землю, стала проступать обильная вода. Чтобы кто-либо в потемках не угодил в большую яму с водой, мы засыпали её. Пришло распоряжение занавешивать окна так, чтобы с улицы не был виден вечерний свет окон, а на стёкла оконных рам приклеить полосы бумаги, чтобы возможные при взрывах осколки повисали на бумаге. После первой попытки немецких бомбардировщиков прорваться в небо Москвы...

%%[[page_206]]
206.
Жильцы дома решили ещё раз попытаться врезаться (лопатами!) в землю, но, понятно, на другом месте: попробовали, с остервенением и мотом — не помогло, опять вода. Все указания по безопасности населения, и разумные, и наивные, — выполнялись; в поздние часы город погружался в полутьму. На Сталина смотрели с надеждой.
Однако решение главной проблемы все эти меры не помогали, сообщения были тревожные, выработался радиоштамп: «после упорных боёв наши войска шли» (и всё ближе к Москве).

%%[[page_207]]
Когда наступил короткий период воздушных тревог, многие соседи по дому выходили на крыльцо и оставались стоять под навесом от дождя, вспоминая о траншее. Все — с немалым напряжением — смотрели на темное, ночное небо; вдруг, где-то далеко от нас, в облако уперся сильнояркий, голубоватый белый луч прожектора, появился второй луч, они сошлись, видимо, врага нашли. Четвертый этаж и крыша моей школы помешали досмотреть происходившее далеко на горизонте. Так обычно проводили воздушные тревоги в нашем кооперативе на окраине Москвы. Впрочем, было за что переживать и в нашей округе.

%%[[page_208]]
На близком от нас расстоянии (для автомобиля) находились важные объекты: гурт политейный завод имени Войкова; военный завод, путь Рижской железной дорогой и мосты через неё, Братцевская птицефабрика - первое в стране предприятие поточного производства птицеводческой продукции (на базе американской технологии).

И в трагическое иногда вмешивается смешное. В третьей комнате нашей коммуналки жила цыганка Катя с мужем-китайцем. Она работала уборщицей, он - в китайской прачечной (они были такие в Москве). С работы.

%%[[page_209]]
209.
Они приходили вместе и поздно, кухней почему-то не пользовались, даже ведро с водой держали в комнате; люди они были очень скромные и малообщительные. Но когда начались воздушные тревоги, сосед стал совершенно другим человеком: по объявлению воздушной тревоги он раскрывал дверь комнаты и стучал рядом. При первых звуках сообщения об отмене тревоги он выбегал на лестничную клетку и с радостной интонацией выкрикивал: «Тревога миновала, тревога миновала!» Напряжённое ожидание сменилось незаметной улыбкой. Сегодня пронесло. Некоторые, уходившие с крыльца мужики, дружелюбно похлопывали.

%%[[page_210]]
его по плечу, а неко 210. Торые особы (бывало, только бывало!) позволяли себе ущипнуть его.... В один из дней того же периода я откуда-то приехал в конце полуденных часов, соседка сказала, что приходили ребята из школы, меня там ждут, я мигом в школу — она уже закрыта, я — на квартиру к директору (в школах в новостройках была запроектирована квартира для семьи директора школы); открыл дверь сам Иван Дмитриевич, предложил войти, я извинился и объяснил причину прихода в школу; он немного призадумался и с большими паузами между словами сказал: «Клоц. В этой войне тебе еще достанется, иди домой и никого не ищи». Новый учебный год 1.9.41 в школе не начался; я поступил в Московский авиационный техникум.

%%[[page_211]]
211

Занятия в техникуме продолжались всего полтора месяца до 16 октября. Об этом критическом для Москвы, а, возможно, и для всей страны дне, готовившейся к обороне Москвы, я вам рассказал с чувством возродившейся душевной тревоги, которую я испытывал в то время. Передам вам ход моих мыслей, возможно, и путаных, но точных. Когда в тот день, на пути в техникум, я прибыл на площадь станции метро «Сокол», увидел необыкновенное: метрополитен закрыт, тысячи подземных пассажиров одновременно находятся на улице, затененной...

%%[[page_212]]
жёлтыми осенними облаками; Трамваи и троллейбусы обвешаны людьми, спешащими на работу, но транспорт еле ползёт. Серо-черные тона теплых одежд массы передвигающихся фигур резко меняют восприятие родного пейзажа, отчуждают его, сливают с мрачным холодным октябрьским днем. Много народу здесь потому, что «Сокол» — конечная станция одного из радиусов метрополитена, ею пользуется и население примыкающих районов пригорода. Видимо, так и с другими конечными станциями.

%%[[page_213]]
В этой безысходности я как-то растерялся, поэтому мне хотелось услышать совет, мнение солидного, по возрасту, человека, а затем приблизить мнение к моему личному обстоятельству, но кругом суровые, озабоченные лица с крепко сжатыми, мне казалось, губами.

%%[[page_214]]
214. Несмотря на сумятицу момента, принимаю решение идти в техникум, к Мясницкой площади; это очень далеко пешком, но понадеялся, что около крупных объектов трамваи и троллейбусы разгрузятся. Я думал о том, чего же можно ждать в ближайшие дни, начиная с сегодня; проворачивал в памяти самые последние разговоры в кругу семьи об отъезде из Москвы, о дополнительных испытаниях, которые судьба преподнесла впереди в лице Гитлера, но главным было: какое решение принять в складывающейся ситуации. Гришка ясно сказал, что его с группой специалистов оставляют на заводе.

%%[[page_215]]
их эвакуация не 215. предполагается, но если обстановка изменится, то он сможет взять с собой папу и меня. Дядя Яша сказал, что он выжидать ничего не будет, он не хочет рисковать жизнями тети Берты и детей (Шурику было 8 лет, Ритуле 9 месяцев), они поедут на полуторке до гор, а дальше - как сложится (полуторка - грузовичок подъемностью 1,5 т); для меня и папы дядя Яша зарезервировал два места на этом грузовичке. Папа не соблазнился вниманием дяди Яши, но искренне поблагодарил его за заботу. Отказ ехать папа мотивировал рядом причин: он помнит многолетние мытарства всей семьи после изгнания из Брест-Литовска в 1915 году; с опухшими ногами и ослабевшим за

%%[[page_216]]
месяцы войны здо 216. робем он в открытом кузове до Горького не доедет (папа, действительно, очень слаб); по острейшему вопросу — надо ли уезжать из прифронтовой Москвы потому, что мы евреи, папа придерживался оригинальной, рискованной точки зрения, которая схематично звучала так: именно из Москвы, в отличие от другого прифронтового города, нам уезжать не надо, так как Сталин Москву не отдаст без большой войны за город, иначе это будет капитуляция. оба в те дни вопрос о судьбе Москвы на своём месте в этой судьбе стоял остро.

%%[[page_217]]
к Чтобы стать частицей защитной силы, нужно быть лично независимым. А я в свои 17 всё ещё на положении ведомого в обычной жизни. Может, это и не плохо, но... Война создаёт экстремальные ситуации, поэтому надо готовить себя к ним, то есть выработать четкую линию собственного поведения независимо от логического расклада мысли другого человека. Это требование к себе я выработал и закрепил на анализе Гришиного высказывания.

%%[[page_218]]
219. Вопреки Гришиному варианту, ситуация резко ухудшится, и будет принято решение срочно завершить эвакуацию ЦТР (инженерно-технических работников), а через короткое время Гриша с коллегами будет в заводском самолёте лететь на восток. Что будет с нами? Поскольку такое вполне было возможным, я разработал линию нашего поведения: я присоединяюсь к защитникам Москвы, а папа, как и другие старые люди, перейдёт под опеку местных властей. С такими заготовками я, наконец, с большим опозданием добрался до техникума. Впрочем, не опоздал: занятия отменены, техникум готовится к эвакуации.

%%[[page_219]]
220. Секретарь директора техникума предложила записаться на эвакуацию. Я отказался, тогда она посоветовала, как она сказала, «на всякий случай» взять справку о том, что я студент первого курса Московского авиационного техникума. В непринужденной беседе я изложил ей доводы, побудившие меня и отца воздержаться от эвакуации. Мне было очень приятно с ней говорить, её совет как-то расположил меня к ней, и мне хотелось поделиться мыслями ещё до возвращения домой.

%%[[page_221]]
Второй
К концу дня метрополитен начал работать, иначе я бы запомнил дополнительные трудности, которые были бы в связи с моим возвращением домой из техникума. Восстановление работы такого важного стратегического объекта, как метрополитен, означало, что военная обстановка под Москвой как-то стабилизировалась. Однако для меня тот день закончился ещё одним тревожным эпизодом! Я сошёл с трамвая у круга 277 маршрута, то есть напротив нашей «Булочной кондитерской», и увидел непривычно широко раскрытые её двери, только входили и выходили редкие люди. Из персонала — ни души. Витрины прилавков пусты, только валяются марлевые занавески, полки для хлеба пусты, подсобное помещение открыто, и там ни на полках, ни на лотках ни корки хлеба.

%%[[page_222]]
Что же произошло? Утром 1600 прокатилась волна по Мини, перезнал булочной разбежался, а жильцы ближайших домов растащили хлеб. Ночью прошла на редкость тихо, но мы были начеку. Скоро драму прислушивались к тихому шипящему звуку, доносившемуся из радиотарелки. Утром ничего не сказав папе, я пошёл к булочной и... обалдел: входные двери заперты, сквозь стёкла разглядел уборщицу и человека в форменном синем халате. Это работник организации быта москвичей. Быстро наладилась быт, с учётом, конечно, особенностей военного времени, и подобное рассказанному никогда не повторилось. Между тем, эвакуация продолжалась, но в успокоенных домах. Особенно

%%[[page_223]]
223 н.
ности состояли в том, что был комендантский час, в то же время были открыты большие бомбоубежища, станции метрополитена ночью работали в режиме бомбоубежищ. Порой окрест местности патрулировались; не исключалось, что несмотря на тщательно организованную противо-воздушную оборону, о которой я уже говорил, враг попытается, в частности, десантировать небольшие группы диверсантов, все знали об этой опасности и были бдительны.

%%[[page_224]]
2. Попытаюсь дать вам почувствовать Москву того времени. Для этого расскажу о том, как я однажды оказался во власти строгостей комендантского часа. После посещения театра на нем тоже был отблеск войны. Признаки столичного города сохранялись по мере возможного. Например, большинство ведущих театров эвакуировали, но Московский академический музыкальный театр имени Станиславского и Немировича-Данченко оставили в Москве. Я очень любил этот театр и остановлюсь немного на воспоминаниях о нём (в попутчики приглашаю ваше воображение). Гладкий, большой прямоугольник светло-жёлтого фасада здания театра смело вписался в ряд мрачных, богатых домов капиталистической застройки на одной из старейших улиц.

%%[[page_225]]
225.н. и центральных улиц Москвы — на Большой Дмитровке (в советское время улица Пушкинская). Новое здание театра как бы дразнит стариков незамысловатостью своих простых, ясных линий, скромностью главного входа. Этот сдвоенные двери обозначены, с улицы, в меру большими фонарями конусной формы, порог входных дверей — вровень с тротуаром, никаких колонн, парадных лестниц. Идет человек по тротуару, шаг в сторону, перешагнул порог, и он в театре (своеобразная архитектурная форма выражения стремления к демократии); перешагнём порог. В просторном кассовом зале свет приглушен, стоят единичные зрители в ожидании своих пар. Ярко светится только окошко кассы, кассир в зимнем пальто и головном уборе, покупаю самый дешевый билет.

%%[[page_226]]
226 Н. Женщина-контролёр тоже в пледе большим шерстяным платком поверх служебного пиджака. — Спускаюсь в гардеробный зал. Зачем? Я же знаю, что в эту зиму (1941-42 гг.) многие общественные здания в Москве не отапливаются. Я сразу почувствовал и увидел, что театр в их числе, но как можно нарушить традицию — в холодный месяц не зайти в раздевалки, не ощутить зарождения праздничного вечера. Ещё недавно, до войны, в этом зале женщины поспешно меняли уличную обувь на выходную, поспешно, не обращая внимания на чужих мужчин, подправляли прически, подкрашивали губы, а то, условно отвернувшись, быстренько выровняли колготки, тогда ещё носили чулки, нефис выровняли по центру юбки, платья. Мужчины поторапливали дам поскорее выйти, чтобы перекусить после работы, но непривычный галстук оказывался съехавшим вбок, и дома умелой рукой наводили порядок. «Ну, пошли все дружелюбно, в хорошем настроении.»

%%[[page_227]]
зау 227 н поднятое настроение вечера дополняли гардеробщики: они, как пульсирующая жила, протянутая вдоль длинного раздевочного стола, заученно метались между зрителями и вешалками: они старались побыстрее вручить номерок и, по возможности, бинокль. театральная публика, разогретая своим, несколько сумбурным, приготовлением к встрече с искусством, постепенно переместилась в большое фойе, где вход в зрительный зал, медленно прогуливаясь по периметру фойе, зрители с невидимым нетерпением ждали первого звонка. театральный звонок - это запоминающаяся кульминация надежд актёров и зрителей.

%%[[page_228]]
Незабываемы микросценки у выходов из станций метро «Охотный ряд», пл. Свердлова, пл. Революции, недалеко от которых расположены на Театральной площади театры Большой, Малый, Центральный детский и далее вверх по Пушкинской филиал Большого театра, филиал МХАТа, Музыкальный, в котором я мысленно нахожусь, а ещё дальше Ленком. Поднимаешься вечером на улицу и обязательно услышишь многократно повторяемый вопрос: «Есть билетик?», и чем ближе к театру, тем слышнее вопрос. Не есть ли это наивысшая оценка театру?

%%[[page_229]]
я 1229н. затянуть — рассказанное о театре медленно складывалось из отрывочных воспоминаний, когда я долго стоял в том же гардеробном зале, но в зимний вечер 422. В сыровато-холодном зале было безлюдно, изредка хлопала дверь туалета. Раздевалочные стойки, за исключением одной, были невзрачным материалом, и только несколько мужчин-иностранцев сдали на вешалку свои полувоенные покрой пальто. Те гости были приметны высоким ростом, необыкновенной худобой, явно военной выправкой. Я остался созерцать давно знакомый зал — он казался чем-то родным, страдающим от отсутствия прихорашивающейся к празднику публики, от обнаружившейся наготы огромного простора.

Своего

%%[[page_230]]
230 Н.
Неоживлённо было и на втором этаже, в большом фойе: почти все зрители в пальто, кругом по которому пробуждаются зрители в ожидании звонка, не сложился. Зрителей мало. Буфет никого, ничего, стулья составлены стопкой и припёрты к стенам помещения, как говорится, до лучших времён; люди беспорядочно бродят по фойе, смотрят неприметно друг на друга и, видимо, одинаково постигают особенности театра военного времени. Наконец, первый звонок, открываются высокие, нарочито красивые двери зрительного зала. Элегантный рисунок дверей, полагаю, символизирует последний рубеж перед входом в мир прекрасного. В партер я вошёл вместе с теми, у кого были дешёвые билеты, свободных мест не было.

%%[[page_231]]
231.H.

Кресел оказалось много, и галёрочники осели в партере; в зрительном зале было очень зябко, а холод, я заметил, создаёт ложно расширительное представление о расстоянии, кубатуре ёмких помещений. Поэтому знакомый зал впервые показался мне больше, чем всегда; впрочем, оптический обман только усилил впечатление от чётких функциональных деталей архитектуры, не обременённых украшательствами. В таком скромном зале дышится легче.

В составлявшиеся минуты до начала оперы-буфф Ксенофонта Прекрасная Елена подумалось: как в таком холоде артисты, играющие легенду Древней Греции, смогут выступать с оголёнными спинами, руками, да чуть ли...

%%[[page_232]]
Алем себя 232H. и смы не с голыми ногами? Смогли. Содержание слова "смогли" в то время включало в себя организации сопротивления советского народа немецким фашистским захватчикам: война нарастала, грушила условия жизни во всех её сферах, стремление к сохранению нормальной организации труда - это было элементом сопротивления; преодоление постановочных проблем театром - это тоже сложная задача времени, которую успешно решали. На мой зрительский взгляд, спектакль прошёл на высоком уровне: оркестр, исполнители, декорации в реалистическом стиле и впечатление от них, усиленное осветительной техникой - всё это работало на сто и не нуждалось в скидках на военное время. Моё следующее посещение театра состоялось через несколько лет, уже после войны.

%%[[page_233]]
233 н.
- Что давали, не запомнил, как играли - не видел, как звучал оркестр - не слышал: в том же затемнённом зрительном зале я неотрывно смотрел на изящный, с чуть приведёрнутым носиком профиль любимой, моей невесты. Не вернусь к «Прекрасной Елене».
От театра до дому - полтора часа, поэтому в условиях комендантского часа вернуться домой к ночи невозможно. Сработал мальчишеский задор: с одной стороны, хотелось немного побыть в уголке давленной жизни, с другой, посмотреть, ощутить готовность к войне.

%%[[page_234]]
1234 н.
Московского метрополитена — технического чуда эпохи. Такое восприятие метро подкреплялось пейзажем московских улиц того времени. Ещё можно было видеть неухоженных, понурых лошадей, тянувших тяжело нагруженные телеги.
Почему я так много говорю о метрополитене? Почему я выбрал его одним из показателей готовности к войне? Отвечаю на ваш возможный вопрос. Вырывающийся на скорости и с шумом из туннеля сверкающий осветительными лампами, никелем поезд был той новой реальностью, мощью, которая, в числе подобных, создавалась страной.

%%[[page_235]]
Должен сказать, что организация работы метрополитена положительно повлияла на образ жизни деловой Москвы давленного периода, когда метро воспринималось как новинка. Шестидесяти-пятидесяти секундные интервалы в часы пик между уходящим поездом и подъезжающим к платформе поездом долго потрясали пассажиров. Технические новации метрополитена были предметом одобрительных обсуждений, он быстро стал незабываемой частью города. Воспоминания о нём у многих, убеждён, стали лучше понимать цену времени.

Итак, после спектакля я, в соответствии со своим замыслом, пошёл ночевать в бомбоубежище на станцию метро «Маяковская». Воздушной тревоги не было, тихо улицы.

%%[[page_236]]
236н. — Пустынно, людей очень мало, все спешат, скоро начало комендантского часа, машин почти нет; давно знакомые здания главной улицы Москвы смотрятся как-то иначе: вроде бы их фасады одухотворились и насупились, ощетинились даже. Не признаёт меня театр юного зрителя! Во всей этой смешанной провинности наступающей ночи сердце переходит на тревожный ритм, рождает неприятные ощущения и желание поскорее уйти из этой необычной тишины, уйти от этого безлюдья. Когда я подошёл к «Маяковской», комендантский час уже действовал: патруль решал переход на другую сторону площади, улицы, всем предлагали войти в метро на ночь.

%%[[page_237]]
237 н.
Другая функция станции метрополитена изменила её облик: все три эскалатора работали только на спуск, загрузка их была полной; с верхних ступеней эскалатора видно было, что среди публики очень много маленьких детей и старых людей, пестрели матрасики, от кро...
Поток младенцев, складные стульчики, скамеечки. Этой «прослойке общества» отвели огромный, красивый зал, расположенный между платформами; большинство граждан должно было спуститься по деревянным трапам, прикреплённым к платформам на рельсы железнодорожного пути. Мне помнятся тяжёлые картины укладывания детей спать: на каменный пол развернутые листы газет, на них матрасик, рядом — согнутый в коленях родитель, пытающийся...

%%[[page_238]]
в этих напряжённых условиях успокоить и уложить ребёнка. Пристроиться возле него, хотя бы вздремнуть; ничего не получается, а утром на работу. Зачем эти пытания, если нет воздушной тревоги? Она может быть и среди ночи, тогда ещё хуже. Многие живут в старых домах, в полуподвальных комнатах, поэтому взрыв авиабомбы, воздушная волна могут обрушить дом, завалить подвальные помещения, а вот метро — надёга. Да, война проникла во все поры жизни, даже в кроватку младенца. Я спустился по трапу на рельсы. На уровне их высоты лежали, поперёк линий, деревянные настилы, чтобы люди могли лечь. Охватило желание ознакомиться, в туннеле меня бы сказал, с магией скоростей и ритмов движения в метро.

%%[[page_239]]
239 K. диаметр. Труба туннель по всей длине пестрела красным сигналом миниатюрных светофоров, прикреплённых к правой полукруглой стене туннеля. Короткие промежутки между светофорами, видимо, и есть одно из средств регулирования скорости. Отключение высоковольтной линии и достаточная освещённость стимулировали любопытство, и я пошёл немного вперёд, в сторону станции Свердловская. Вскоре я увидел железнодорожную ветку, соединяющую, при необходимости, противоположные пути. А пройдя ещё немного, увидел ярко освещённое помещение. Я оказался в просторном, ухоженном санузле, множество туалетных кабин и раковин с водой.

%%[[page_240]]
239 H. диаметр. Труба туннель по всей длине пестрела красным сигналом миниатюрных светофоров, прикреплённых к правой полукруглой стене туннеля; короткие промежутки между светофорами, видимо, и есть одно из средств регулирования скорости отключения высоковольтной линии и достаточная освещённость стимулировали любопытство, и я пошёл немного вперёд, в сторону станции пл. Свердлова. Вскоре я увидел железнодорожную ветку, соединяющую, при необходимости, противоположные пути, а пройдя ещё немного, увидел ярко освещённое помещение. Оказался в просторном, ухоженном санузле, множество туалетных кабин и раковин с водой.

%%[[page_241]]
241н

Одно дело смотреть на блоки через окно вагона, другое - прикоснуться, вообразить силу, сопротивление стали, мощь её толщины. Всё это применительно к войне, к обеспечению стойкой обороны!

7. Прошедший вечер в Театре, уходящая нога в метро, наполненные сопоставлениями впечатления, размышления, да ещё при пустом желудке - свалили меня. В 5 часов утром те же три эскалатора на ветке работали только на подъём, и через 802 в своём транспортном режиме; Комендантский час закончился. Я убедился в целесообразности своего замысла и в многоликости четкого сопротивления врагу.

%%[[page_242]]
242

Снабжение продовольствием заметно ухудшалось, неслучайно зачастило словечко "недоедание": постоянное физическое ощущение его смыслового значения пригибало плечи. Штрих: когда папа изредка навещал близко живших от нас его друзей Болотиных, мы брали с собой два кусочка сахара для двух стаканов чаю. Покушаться на сахар хозяина дома было кощунственно с подобными контурами жизни. Мы вошли в морозную, голодную, тяжёлую в военном отношении зиму 1941/1942 года.

- Итак, после тревожных дней середины октября Красная Армия вошла в тяжёлые,

%%[[page_243]]
243. жертвенные бои за каждый метр подмосковной земли; как потом стало известно, это было продолжением Московской битвы, начатой ещё 30 сентября 1941 года, закончилась она нашей победой 20 апреля 1942 года. В результате, немецкие войска были отогнаны от Москвы на десятки километров; в отдельных случаях до 200 км. Из сопоставлений дат и расстояний можно понять, как велики были потери, каково значение московской битвы для жизни столицы; значение также и в том, что это было первое поражение наземных немецко-фашистских войск в Европе после начала Второй мировой войны. До этого поражения они входили в столицы захватываемых стран в параде.

%%[[page_244]]
ных мундиров. 1244 - в Московской честь можно сравнить Победу Красной Армии абзац нить с ярким лучом прожектора во мраке ночи: свет — это надежда на победу, мрак — это продолжающееся наползание немецко-фашистских войск на территорию СССР. воспринималось как одно из форм идеологического давления фашизма. Сознание не покидало реальное ощущение этого давления: захватывались дети твоей расы и других стран, порабощались народы для принуждения их работать на фашизм; из порабощенных народов фашизм выделял и собирал евреев для специально организованного их (нас) физического уничтожения.

%%[[page_245]]
245. н.
Из рассказанного вам о довоенной антифашистской пропаганде и об убийстве в Минском гетто родственной мне еврейской семьи можете - хотя бы по названным фактам - почувствовать, чем был в реальной жизни нацизм. Мировая документалистика содержит огромный материал о преступлениях германского национал-социализма.
В наши дни, когда отдельные последыши фашизма отрицают Холокост, не лишне напомнить о документе Международного суда. Он издан под названием Нюрнбергский процесс (два тома, Москва, 1954 г.).

%%[[page_246]]
246.
Когда всё, ныне документально зафиксированное,tragическим действием на захваченных немцами территориях, то ещё тогда стало совершенно ясной чрезвычайная, смертельная — в буквальном смысле слова — опасность, которую нес германский фашизм каждому еврею.
Смириться с положением обреченного на смерть — исключалось напрочь. Наоборот, чем больше было фактов бесчинств нацистов на захватываемых территориях Советского Союза, тем яростнее становилась ненависть к врагу. Так было не только в моей душе.

%%[[page_247]]
Обзавто 247

Послевоенная статистика даёт конкретное представление об участии евреев в Великой Отечественной войне. Например, по количеству Героев Советского Союза евреи заняли пятое место (1 - русские, 2 - украинцы, 3 - белорусы, 4 - татары, 5 - евреи и т.д.). В условиях зыбкого положения на значительной части линии фронта первой половины 1942 года повышался градус патриотизма: усиленно развивалось партизанское движение в тылу врага, шире становились ряды тех, кто добровольно уходил в армию. Я тоже решил уйти в армию.

%%[[page_248]]
246

По условиям военного времени юноши допризывного возраста проходили курс всеобщего военного обучения при местных районных военных комиссариатах; я - при Тимирязевском РВК г. Москвы. Моё окончание курса совпало по времени с завершением Московской битвы и с очередным весенним призывом в армию (апрель 1942 года). Всеобуч заключался в том, что нас знакомили с устройством, методами ухода и использования стрелкового оружия по учебным плакатам. Артиллерию малого и среднего калибров было интересно изучать. Строевой подготовки не было, видимо, потому что в округе домов по Дмитровскому шоссе, где помещался РВК, не оказалось необходимой площадки.

* * *

%%[[page_249]]
На всеобуч я смотрел как на техническую подготовку к решённому мною вопросу - досрочному уходу в армию; об этом я сказал руководителю курса, уже после окончания всеобуча 9 мая 1942 года я стоял перед призывной комиссией. После того, как я назвал себя, председатель комиссии спросил, в каких войсках хотел бы я служить. Мой ответ был краток: хочу быть на войне лётчиком-истребителем. Председатель внимательно посмотрел мне в лицо, взял листок из стопочки бумаг на столе, долго читал, потом сказал: "Молодой человек, для лётчика-истребителя вы ещё слишком молоды, давайте-ка поедем во второе ЛАУ, Хорошев Ленинградское."

%%[[page_250]]
Саба 1250 артиллерийское училище. Согласны? Огорчённый такой неожиданностью, я кивнул головой в знак согласия. В г. Кострому, где находилось эвакуированное из Ленинграда училище, я прибыл в составе небольшой группы призывников под командой ефрейтора. Поезд прибыл ближе к полдню, погода была отменная, и мы с удовольствием прошли по улицам незнакомого, старой архитектуры, города. На территории училища мы остались «щипать травку» возле КПП (контрольного пропускного пункта). Ефрейтор ушел с папочкой к начальству. Он очень долго отсутствовал, а когда вернулся, удивил.

%%[[page_251]]
В училище нас не приняли, 24 причина — новички не нагонят пройденного материала. Вот те на! Кто из нас мог подумать о такой нестыковке — время тяжёлое, да ещё во время войны. — В Москве, в военкомате, никто, похоже, не удивился нашему возвращению. Нам сказали, что когда понадобимся, вызовут повесткой, можем разъезжаться по домам. Опять пришла весна, опять весенний призыв в армию. Я получил повестку как призывник призывного возраста, и уже без всяких собеседований меня включили в список лиц, направляемых в Тульское пулемётно-миномётное училище (ТПУ).

%%[[page_252]]
9

За время, прошедшее между двумя названными призывами, у меня произошла личная трагедия. 1 октября 1942 года умер мой отец. Ему стало плохо на работе, на «Скорой помощи» его отвезли в больницу на площади Восстания, рядом с территорией планетария, куда мы, кстати, нередко ходили слушать популярные лекции по астрономии и смотреть тематические спектакли; папа умер как бы возле близкого душе его дома. В то утро я приехал в больницу около 11 утра, чтобы после обхода поговорить с врачом. Я взбежал на второй этаж, открыл дверь длинного коридора и увидел перед собой няню.

%%[[page_253]]
253

чку среди лужи воды: она мыла полы. аккуратненько обошёл лужу и направился в палату. Вдруг слышу: — Мальчик! Ты куда? Твой отец утром помер. Иди вон туда, — и протянула рукой, в которой держала обильно мокрую тряпку, показала на противоположный конец коридора. Я впервые услышал это страшное, моё сердце тормознуло, как тормозит машина на скорости. Куда? Зачем? Я растерялся, остановился. Нянечка всё поняла, бросила тряпку в ведро, быстро обтерла руки о халат, взяла меня за руку и тихо сказала: — Пойдём к доктору. Этому эпизоду придаю некое философское значение: в столкновении смерти и человечности побеждает человечность. Вспоминая знаковые события, я часто ссылался на папу, поэтому, думаю, представление о нём у вас сложилось. Папу

%%[[page_254]]
254. Мы — Гриша и я, похоронили, как и маму, по еврейскому обряду. Это было возможным и в тяжёлом 1942 году, так как на Востряковском еврейском кладбище с довоенного времени продолжала официально работать еврейская ритуальная служба. Но главное, думаю, в этом то, что Москва прилагала большие усилия для сохранения своего столичного статуса. Уместно сказать о папе. Аврум Бенцион — полное его имя, данное новорожденному в 1880 году в еврейской семье Польши. После Первой мировой войны семья осела, как я вам уже рассказал, в Симбирске. Имя 28-

%%[[page_255]]
235.
рум не сближало человека с окружающей русскоговорящей средой, и он перевёл символическое значение своего имени на многозначимое на русском Абрам. Отчеством стала вторая половина полного имени Аврум бен Циен, Бен — сын, Циен — имя папиного отца от слова Сион. Слово бен (или бат) в подобном словообразовании выполняет функцию грамматической приставки и по-русски получается отчество Бенцинович — правильно, но сложно произносимо. Из близких вариантов папа выбрал отчество Бенцианович.
Итак, в паспорте и документах он значился как Абрам Бенцианович; в быту, чтобы не переспрашивали, папа представлялся как Абрам Борисович. Вся старшая родня по-прежнему называла его Аврум. Чехарда с именами.

%%[[page_256]]
256. Особенно с отчествами, была не только у евреев, но и у граждан других национальностей. Однако вскоре это превратилось в обязательный процесс, так как в конце декабря 1932 года в Советском Союзе началась паспортизация, с тех пор в советских паспортах закрепилась благородная традиция — указывать отчество гражданина. Но своеобразная интернационализация паспортов не исключала противодействий. Вот печальный факт. Вы, конечно, помните об аресте моего старшего брата Гриши. Так вот, в один из будничных вечеров, когда нашего трехлетнего Боречку еще не уложили спать, ко мне пришел элегантный молодой человек, представился сотрудником КГБ, показал удостоверение.

%%[[page_257]]
251

Ранее и сказал, чтобы завтра, к 10 утра, я был в приёмной КГБ в Лубянском переулке. В небольшой приземистой прихожей было много людей: скорее всего, справлялись об арестованных. От внутренних помещений прихожую отделяла широкая стеклянная дверь, за которой стоял часовой; ровно в 10 часов подошёл вчерашний гость, увидев меня возле двери, жестом руки предложил пройти. Мы пошли на верхний этаж. Я был в большом напряжении, в неменьшем недоумении. Это состояние, видите, диктовало...

%%[[page_258]]
258.
Моему подсознанию желание поскорее увидеть точку цели вызова, чтобы самому предварительно что-то уловить. Мы вошли в просторную, сезон с заметно высоким потолком, комнату: два высоких, узких, стрельчатых как в костёлах, окна, из них вид на церковную колокольню (в том же переулке, всё это подчёркивало высоту и пустоту комнаты), в которой стоял только один большой письменный стол; за столом капитан-следователь, напротив него, на стуле я; за моей спиной медленно прохаживался вызвавший меня сотрудник. Разумеется, я ничего не уловила из окружавшего меня.

%%[[page_259]]
259. Быстро освободившись от чтения бумаг, капитан-следователь спросил меня, знаю ли я еврейское имя своего брата Гриши. Я ответил, что знаю. "Конечно, назовите еврейское имя Гриши," — назвал: Эршэл. "Кто в семье называл Гришу еврейским именем?" — "В общем-то никто. Я и Давид никогда, но бабушка, которая временами приезжала к нам на выходной день, по-старинке, ласкательно называла Гришу Эршэл." Закончив установлением принадлежности имени, следователь спросил у меня, могу ли я написать на еврейском языке еврейское имя Гриши. Я ответил, что писать и читать на еврейском языке я не умею. Неожиданный был вопрос следователя о состоянии Оли. Я ответил и решил.

%%[[page_260]]
узнать что же о Грише. Мы 260 Тут даём ему белый лебедь, маслом, был ответ следователя, я понял бессмысленность желания узнать подробно о Грише. Часу в новом Пашином паспорте после реабилитации отец значился: Григорий (он же Гершель) Абрамович. Какой смысл был или должен был быть в этом "он же"? И это в условиях сложившейся тенденции на русификацию имен. Впрочем, вернёмся к воспоминаниям о моём друге Льве Моисеевиче Немировском. Однако благодаря Никите Сергеевичу Хрущеву предстояло десятилетие политической оттепели в стране.

%%[[page_261]]
261. Воспоминания имеют свойство обрастать родственными фактами, а то и приводят к неожиданным открытиям. Мог ли я предположить, что моя публикация о той памятной атаке приведет к тому, что узнаем в нашей Лене прямого потомка (по женской линии) семьи известного философа, писателя Мартина (Мордехая) Бубера (1878-1965). Его единоутробная сестра Софья — прабабушка мамы нашей Лены Марины Владимировны Ивановой (Гринкруг). Какова же цепочка фактов? Свою публикацию я послал Владимиру Владимировичу Иванову — дяде Лены; он тоже инвалид войны.

%%[[page_262]]
1262.
Ответ ошеломил. Спустя десятилетия оказалось, что мы одновременно прибыли в Тульское миномётное пулемётное училище, но были в разных воинских подразделениях. Месяца через 3 его подразделение отправили на фронт восполнять потери после Орловской битвы. 13 октября 1943 года пулемётчик Владимир Иванов был ранен в нос. Его старший брат Сергей Владимирович, отец Лены, был солдатом Сталинградской битвы и тоже был ранен в ноги. Участвуя в обсуждении оригинальных эпизодов истории семьи, Лена пересказала слышанное от своей мамы о Победе.

%%[[page_263]]
263

Перечитав последние странички рукописи, я невольно удивился тому, как далеко стоящие друг от друга во времени и месте действия факты бывают родственны по личностным духовным связям, и само движение жизни ведёт человека, независимо от его сознания, воли, ведёт от одного состояния к другому. Жизни прибавилось, меня к моему 90-летию. Я должен радоваться, но не сопутствующие этому чувства наполняют меня. Ой.

%%[[page_264]]
264
Скажем так, сговорились поздравить меня к юбилею. И Инну не оставили в покое, с Инной я с 30 декабря 1945 года; мы стали одним человеком по восприятию каждого из нас. Вот, эти гадючки чувства навалились на меня. Почему обострились ощущения недугов, что вызвало их? Когда я решил сам разобраться в перегруженности ощущений, я собрал все свои нервы, в том числе и не железные, и спокойно обдумал все и пришёл к выводу, что я испугался послеюбилейного времени; страх обострил ощущения недугов. К такому же

%%[[page_265]]
265 обобщающему выводу пришла и интуиция. Здесь я должен сказать, что одна из опор нашего Союза — это совпадение мнений по серьезным вопросам. Страх, испуг, как и вспышка гнева, опасные категории, так как обладают свойством мгновенного влияния. Поэтому принятые под их влиянием решения ошибочны. Чтобы не ошибиться, надо выполнить тяжёлую работу: взять себя в руки, блокировать эти качества. Только тогда можно поразмыслить над ситуацией.

%%[[page_266]]
Понимаю, что не сделать открытия, сказав в схеме построения рассуждения, но твёрдо могу сказать, чем дольше находишься в жизни, тем чаще обращаешься к этой теме. В этом периоде очень важно не потерять себя в хаосе лекарств и недугов, которые складываются в понятие беспомощности. Такая тяжесть душевного состояния поражает не каждого человека, но я не в том числе. Я никогда не выставлял напоказ, как бельё на верёвочке, свои...

%%[[page_267]]
267

Особенности, вызывающие согласие.

1

Из воспоминаний о госпитале вы уже знаете, как нацелил я себя на тренировку левой руки, чтобы как-то восполнить потерю правой. За годы я достиг многого совершенства.

Я всегда был опрятен, подтянут, самодисциплинирован. Это придавало мне силы. Я могу многое рассказать из жизни без руки, но не надо. Хотя к своему юбилею я пришел в состоянии старческой беспомощности, но, кажется, при сохранившейся способности думать.

%%[[page_268]]
268. Надо спасать свое сознание воспоминаниями, пытаться войти в круг созерцаний окружающей реальности, пытаться быть нужным не только себе. Иной подумает: «Легко сказать, да трудно сделать!» Верно. Однако я рассказал вам о пережитом мною и происходящем сейчас. Хотел было вернуться в годы молодости, но беспомощность — тема цепкая и разноплановая: помимо нарастающих недугов медицинского свойства, она ещё влияет на моральное состояние. По мере углубления беспомощности возрастает зависимость от другого человека; и здесь уже, как правило, обратного хода нет. На эту реальность надо смотреть открытыми глазами. От этой печальной тематики отойду далеко назад, во времена послевоенных надежд.

%%[[page_269]]
по отдельным рассказам и вам эпизодам вы поняли, что те времена быстро прошли. Из госпиталя я вышел уже после окончания войны. Все родственники проявили побольше внимания и думали, как мне помочь, чтобы как-то смягчить проблему. Гришина теща Варвара Васильевна и ее дочь Валентина настояли на том, чтобы какое-то начальное время я жил у них. Они оказались исключительно душевными, заботливыми людьми. Они жили в доме многокомнатных коммунальных квартир. Каждое утро Варвара Васильевна звала меня на кухню и пыталась научить меня что-то делать одной рукой. Не получалось или получалось плохо у нас обоих. В те времена с едой было очень скудно! Весь дом был на карточках.

%%[[page_270]]
270

В случае, да ещё чай настоящий, то это было уже хорошо. Я пытался приспособиться к нахоженной работе и придумал нарезанную, но не чищенную картошку, зажав её между двумя тяжёлыми предметами; рационализация была одобрена, но применения она, конечно, не получила. Так, с трудом я приспособился к новой жизни. У Михайловых я прожил полтора месяца. По сей день я с величайшей благодарностью вспоминаю тот дом. Война не обошла и этот дом: Иван Иванович Михайлов - муж Варвары Васильевны, погиб на фронте, защищая Родину в 1944 году. Его портрет висел около иконы с лампадой. Тоску по погибшему мужу Варвара Васильевна выражала в том, что по воскресеньям утром ходила на молитву в сельский собор. Только долго

%%[[page_271]]
2
271
Общаясь с ней, можно было понять крепость ее характера.
Итак, пришло мое время войти в реальную жизнь из тепличных условий военного госпиталя и дома Михайловых.
Начался самый долгий период моей жизни, который длится по сей день; расскажу вам, что было и есть определяющим в этом движении жизни; расскажу о руке судьбы, которая соединила меня с Инной.
Главную цель моего ближайшего будущего я определил, еще будучи в госпитале: получить институтское образование. Технические институты я исключал потому, что такой важный предмет, как черчение, был для меня недоступен. Меня огорчала только причина невозможности.
История и художественная литература также были объектами моего внимания, но я считал себя недостаточно подготовленным для успешной сдачи экзаменов на филологический факультет МГУ.

%%[[page_272]]
Выбрать специфику высшего учебного заведения помогла мне художница Анна Изакова, соседка моей тёти по коммунальной квартире. Аня отнеслась ко мне с большим вниманием и посоветовала поступить на редакционно-издательский факультет Полиграфического института. Предварительно она рассказала о специфике факультета и о педагогическом составе. Когда я услышал фамилии Светлаев и Крючков — авторов школьного учебника "Русский язык", мой выбор определился: пахнуло школой, довоенным временем, мне захотелось быть близко к тем, кто представляет прошлое.

%%[[page_273]]
Эпизодом из общения со Св. Светлаевым хочу похвастать. Он раздавал контрольные работы после проверки их; когда я подошёл, чтобы получить свою работу, Светлаев сказал мне, что я хорошо чувствую язык и посоветовал сосредоточиться на этом. Итак, основное направление моей будущей профессии определилось, и я поступил на подготовительное отделение института. Там поселился в тех же двух комнатах, в которых жила наша семья до моего ухода в армию. Пришлось научить себя колоть дрова, собирать их в вязанку, зажигать лучины, штопать брюки, завязывать шнурки на ботинках.

%%[[page_274]]
274

В ботинках и мостов пречеву каждо есть освоение какого-либо действия вызывало во мне внутреннюю, никому не сообщаемую радость. «Не сообщаемую» — это потому, что я старался держать себя так, чтобы по стечению руковой не привлекать излишнего внимания. Моё поступление в институт обрадовало родственников, и они начали давать советы. Например, Яша и тётя Берта предложили заходить к ним покушать по пути из института домой (по тем временам — царский подарок); дядя Фима и тётя Лия взяли круче — советовали жениться. С этой целью познакомили меня с семьей своих друзей.

%%[[page_275]]
275

В представлениях о не очень далеком будущем женитьба была у меня на последнем месте. Верьте, я не оригинальничаю: мне противно лицемерие. Добросердечность родственников очень приятна, но и хорошие советы теряют свою привлекательность, потому что они затрагивают одну проблему из ряда. И каково мое отношение к ним, советчик не знает. Главное - я сам тоже не всегда знал. Однако по опыту жизни моих родителей и тех же наставников я имел представление об ответственности за семью. С такой, может быть, даже скептической, подход к доброжелательности был уместен в моих условиях, так как насущным был проблемный вопрос: как и на что мне жить? Никто за меня это сказать не мог, я и сам не сразу нашёл ответ. Пришлось крепко задуматься.

абзац

%%[[page_276]]
на Отправной точкой в рассуждениях о себе была незыблемость институтского образования. Стипендии и пенсии инвалида войны, которая в первые послевоенные годы была на низком уровне, не хватало для скромной жизни. Тогда, может, учиться и одновременно работать? Но для этого я не чувствовал достатка физических сил, работа победила бы учебу, и я, в итоге, остался бы без образования (после третьего курса я учился и работал). Триумф Победы вселял надежды, они складывались из больших и малых фактов: армейские части демобилизовывались, а предприятия наполнялись рабочей силой, начали завершаться последствия эвакуации. Еще применительно к моим личным интересам, студентам Полиграфинститута разрешили пользоваться столовыми министерства торговли и министерства сельского хозяйства.

%%[[page_277]]
1277

Все названные заведения расположены на перекрёстке Садовое Кольцо - Орликов переулок. Удобство места сочеталось с хорошим качеством обедов, и это надолго решило проблему моего питания, ну а слова «завтрак», «ужин» сохраняли ещё реальное по тем трудным временам содержание. Это называлось одним словом «часк». Помнится и такое: поставил человек на стол поднос со своим обедом, достал из него посудинку и положил в неё часть обеда для кого-то дома. Подобное можно было видеть нередко. Трудности жизни проявлялись на городском уровне: малое, личное перемежалось с крупномасштабным, государственным, и всё это входило в понятие послевоенное восстановление.

Мои родные братья Гриша и Давид изначально остро приняли изменения в моей жизни с большим сочувствием и восприятием. Они одобрили мой выбор образования.

%%[[page_278]]
27
и стали материально поддерживать мою учёбу в институте.
Две проблемы — ежедневные обеды и гарантированный ежемесячный доход (весьма скромный, но гарантированный) — решились довольно легко. Остались проблемы быта, то есть приспособление к жизни с одной рукой; сюда же отношу освоение каллиграфии.
По мере наработки практических навыков менялся ритм моей жизни, её уклад. Вот пример. В квартирах у Гриши и дяди Фимы были ванные комнаты, я мог мыться, но это привязывало меня к месту и условиям; я пытался быть способным мыться капитально при разных обстоятельствах. Быть, как все. Потом, сколько можно ездить мыться в чужой, хотя и родственный, дом. Эта проблема оказалась для меня довольно неловкой: никто не возражает, а неудобно. Варианты размышлений на эту тему привели меня к чуду русского быта — к бане.

%%[[page_279]]
279. С папой и братьями с детства посещал баню, знал её удобства, некоторые из них обернулись для меня своеобразными препятствиями, которые я пытался преодолеть. Итак, представьте большой моечный зал, уставленный рядами длинных, достаточно широких мраморных скамей. В конце каждой небольшая раковина и два крана, обращённые во внешнюю сторону. Посетитель наливает в таз воду, берёт его за две ручки, обходит угол скамьи и идёт к своему месту. В последней фразе выражена моя проблема: как это, приспособленное для двух рук, ранее делавшееся мною, повторить мне сейчас? Ответ я искал и нашёл. Он оказался несложным для выполнения, но чрезвычайно важным для моего собственного самоощущения в жизни.

%%[[page_280]]
Ну, а теперь в баню, проверить свой метод взятия таза». В банный зал я вошёл в приподнятом настроении с надеждой на успех. Жара, плохая видимость из-за обилия пара; вот деревянная лесенка вверх, ближе к потолку, где крепче пар. Здесь можно видеть голых мужиков в фетровых шляпах; любители паровой бани мочили свои шляпы в холодной воде, отжимали на голову: своеобразная гарантия от теплового удара. Если с человеком что-либо случалось на скамьях под потолком, то причиной могло быть всё, но не пар. В моечном зале я быстро нашёл свободное место, закрепил его за собой, положив на него кусок мыла и мочалку, и пошёл к кранам, к кульминации моего замысла. Обычный банный таз с двумя крепкими ручками я поставил на раковину, наполнил его водой.

%%[[page_281]]
наконец, надо брать таз. 281. заранее продумав действия, я боком, левым, встал к раковине, присел. Настолько, чтобы верхний край таза и лись на мокрую часть живота - ручка приходила, прижал это покрепче к телу, обхватил рукой округлость таза, и ещё крепче прижимая таз к телу, поднял таз самостоятельно; потом пару неуверенных шагов в сторону. ещё шажок другой, и я огибаю заветный угол с комби; осторожно, чтобы таз не выскользнул, я дотянул до своего места, подобными уроками я добивался физической независимости.

Не только этим памятна мне баня. Её назначение, всё устройство, как бы одухотворившись, дали мне возможность уберечь руку от частых физических перегрузок, когда надо было купать сыночков в домашних условиях.

%%[[page_282]]
Небольшим штрихом покажу вам ту действительность. Чтобы искупать в домашних условиях Боречку и Женечку, надо было принести не менее трёх вёдер воды и нагреть её. По мере взросления детей я брал их с собой в баню, но ходили мы только по выходным дням. Однако и это было для меня большим облегчением. Далее, много воды уходило на ежедневные хозяйственные и личные нужды. И всю эту воду носил я. Инулю к ношению воды я не допускал. У неё было полно хлопот по дому, но главное заключалось в том, что я старался сберечь красоту её рук, её пальцев, в меру удлинённых и наполненных изяществом.

%%[[page_283]]
283. К сожалению, сказанным не могу закончить благодарственные воспоминания о бане, они подгажены тоже незабываемым фактом. Были, очевидно, сохранился давний обычай, когда моющийся человек намыливал мочалку и просил кого-либо крепенько потереть спину. Это было столь обычным, что даже вертолетый иконкий человек, окажись в бане, не чванились бы. Тоже сделал я: намылил мочалку и обратился к соседу по скамье; он выпрямился, развернулся анфасом в мою сторону и тихо сказал: "Авон видишь, человек моется, он тоже еврей, иди к нему, он тебя потрёт." Я не нашёлся с ответом, какая-то горечь заставляла меня непрерывно глотать слюну. То было в самом начале пятидесятых годов. В ноябре 1960 года руководство издательства "Колос" выдало мне ордер на...

%%[[page_284]]
284. двухкомнатную квартиру в современном доме. Мы ожили. Баня ушла в прошлое, у нас изменился порядок жизни выходного дня: стало больше времени для интеллектуального воспитания детей. На отдельных фактах я показал вам, как преодолевались последствия ранения, — радость, которую я испытывал в случае успеха восстановления функций, глубокой стойкости и багажа навыков, начавших расти естественно, пополняться новыми. Недуги, неудобства и прочие болезненные для души и здоровья особенности угнетают носителя их и отнимают больше сил.

%%[[page_285]]
285. у того, кто ухаживает. Страдания душевные и физические часто уводят в мир размышлений с грустным выводом о несовместимости пути. Это ещё больше ухудшает состояние носителя недугов и отнимает больше сил у обслуживающих. Не лучше ли, собрав моральные и физические силы, запрятав в глубины своего сознания тяжело пережитое, уйти в воспоминания о возвышенном, радостном, обнадеживающем, что было в личной жизни? Вправду, почему бы не пожить в мире воспоминаний, если реальность ясна? Я принимаю своё же предложение и расскажу вам, родные мои дети, о зажегшейся однажды в Фенере встрече с шиной.

%%[[page_286]]
Боре 1286. закон едыдущий абзац я рассказать обещании • встрече с Инной Написал это в последних числах мая, имея в виду, что продолжу работу после свадьбы Гриши и Рахели (08.06.2015). Ход времени семейных событий совпадал с моими намерениями начать писать о знакомстве и первых встречах с Инной, но вдруг на нашу семью обрушилось огромное горе - умер самый близкий, самый любимый человек ушел из жизни мой Инуля. Это случилось 15 июля 2015 года. Получилось так, что рассказ о первых шагах нашей жизни я вынужден начать с рассказа о её последнем дне. Не дай Бог никому такого сочетания фактов! Вот моё письмо Душе Инны.

%%[[page_287]]
Инуля, родная!
Вспоминается совсем недавнее. Мы стоим, слегка обнявшись, у раскрытого в ночь окна и в который раз восторгаемся любимым тобою пейзажем: яркая голубая Луна на фоне чёрного, бескрайнего неба. Мощь вида завораживает и будит мысль, это предположения, утверждения, отрицания, сомнения, и мы с тобой в минуты духовного насыщения красотой пейзажа внутренне ощущали силу вечности луны и пришли к ненаучному, а эмоциональному выводу: Луна — это корабль Вселенной в свой безостановочный путь в вечность, он принимает только чистые души.
С луной оказались на нас эти мгновения. Уния "Встречи" бой роковыми.

%%[[page_288]]
благо- 2 одночасье 15 июня 2015 года было обычным для всей нашей семьи. Мы с тобой были дома, обсуждали разное, в частности, предстоящее твоё девяностолетие. День протекал в обычном ритме для нас. После обеда ты, как обычно, помыла посуду, прибрала кухню. С обычным ворчанием по поводу мытья всяких посудин пошла в спальню отдохнуть, почитать. Прошло какое-то время, я услышал, как ты встала, возилась в спальне. Вдруг крик: "Фима, мне плохо!" и с криком Одновременно раздался шум от падающего предмета. В те мгновения невозможно было предположить, что упала ты. Я был близко, за стенкой спальни, но пока я проковылял несколько шагов и оказался у дверного проёма, я увидел роковое: ты лежала на каменном полу, головой к ковру, на правом боку.

%%[[page_289]]
3

Я стал громко звать тебя — бездележно сразу же позвонил Жене, рассказав о маме. Он связался с медицинскими службами, вскоре приехала реанимация, следом и Женя. Большая комната быстро наполнилась специальной аппаратурой, тебя вынесли из спальни в ту же комнату и начали работать. Вся эта подготовка была такой чёткой и оперативной, что возникла надежда, что всё кончится только большим испугом. Медицинская бригада работала долго, но в итоге пришлось посадить тебя в медицинское кресло, зафиксировать ремнями безопасности и повезти в больницу; Женя сопровождал тебя в больницу.

Когда тебя проверили, мимо

Так как я тоже снова. Твоя голова была опущена, глаза закрыты, белое лицо скорее выражало глубокий сон, чем уходящую жизнь в эти последние минуты жизни.

%%[[page_290]]
Судьба дала тебе возможность предстать перед живыми в своём повседневно элегантном, изящном виде: после вставания ты успела "вылезти из жены" и опять стать притягательной в ярко-жёлтой одежде. Такой ты лежала на кленовом полу.

О произошедшем с тобой Женя сразу сообщил Боре и держал его на связи. Ирочку Клоц Женя попросил побыть со мной дома до возвращения из больницы. Так и было. Он вернулся часа через два, вплотную подошёл ко мне и сказал: «Мама умерла». Прошли первые минуты скорби, и Женя добавил, что диагноз больницы — остановка сердца. Меня одного на ночь в осиротевшей квартире Женя не оставил: он ночевал у меня.

%%[[page_291]]
Боря, Митя и Лёва прилетели на следующий день. Накануне похорон, которые были назначены на полдень 17 июня 2015 г., на похороны съехалось много родственников, друзей наших и наших сыновей, знакомых. Женя приложил большие усилия, чтобы за одни сутки охватить всех желаемых, подсчитать количество свободных мест для "безлошадных" и в машинах. Оригинально выразил скорбь по маме Боря: он принес большой букет крупных роз, передал через Женю мне, чтобы я возложил цветы на могилу мамы. Эпизодом с букетом роз Боря напомнил известную в нашей семье историю о том, как я однажды выразил свою любовь к Инне веру в неё. Инна заканчивала...

%%[[page_292]]
Московский Юридический институт, шла подготовка к экзамену по курсу ГУБС (Государственное устройство буржуазных стран). Курс вёл проф. Гурвич — человек очень старый, ехидный, но в реальных условиях никогда не ставил двойки, а назначал переэкзаменовку после дополнительных занятий; таким образом он спасал студента от лишения стипендии на целый семестр. Многие боялись этого экзамена, но я был убеждён, что Инна его сдаст. Я хотел выразить свою уверенность ярким, неопровержимым фактом.

Инна жила у своей тётушки, Виктории Исаковны, на Маросейке, поэтому я проходил мимо гостиницы "Метрополь", вдоль которой стояли продавцы цветов. Я всегда любовался морем цветов и решил тем же ошеломить Инну. Подчёркиваю, не зная ещё итога экзамена, я подошёл к женщине, которая начала разворачивать мешковину со свежими цветами.

%%[[page_293]]
7.
Моё предложение купить всю охапку вызвало интерес, торг, в результате которого женщина упаковала всю охапку цветов. Инна из института ещё не пришла, а тётя Витя, увидев такое, ахнула и схватилась обеими ладонями за лицо. Она быстро, по-хозяйски оценила обстановку: пошла на кухню, принесла ведро, поставила в него цветы, и этот необычный букет мы поставили на пол посередине Инниной комнаты. Инна пришла с экзамена с оценкой "отлично", когда мы втроём вошли в её комнату, перед ней предстал ошеломляющее зрелище, она обомлела, потом крепко обняла меня и стала целовать. Вкус тех поцелуев помню и сейчас: Инна сразу поняла смысл такого подношения.

%%[[page_294]]
Вот какие воспоминания пробудил букет роз. Всё это и близкое к этому промчалось за минуты. Перед глазами могила, поглотившая всё это. Слышны последние постукивания лопат, которыми завершилось погребение, потом - вторая молитва. Боря, Женя и я стояли рядом. Женя, завершая эстафету роз, дал цветы мне. Я, ощущая всем содержанием своим наступивший момент разрыва наших жизней, поцеловал розы как живую Инну и возложил цветы. Я хотел сказать много славного об Инне, начал говорить, но слёзы победили слова. Цветов, а также добрых слов в общениях между провожавшими Инну, было много. Постепенно, не торопясь, лица печали стали собираться к выходу. Так, единой группой, вышли на площадку прощания.

%%[[page_295]]
9.
В этой площадке начинается ритуал похорон, поэтому хочу оставить в памяти нашей семьи запись части похорон. Площадка прощания. Так я назвал её для себя. Является фактическим центром нового, оригинального кладбища "Дяркон". Представьте: огромное, открытое, глубоко запечённое пространство под внушительного вида шатром. Площадка радиофицирована, есть кафедра и микрофон. Здесь собираются группы людей, приехавшие на кладбище. Бывает многолюдно, но всегда господствует понимание значения этого места. Понимание ощущается и зрительно, когда присутствующие на площадке люди как бы притормаживают себя при звуках начинающейся первой молитвы по умершему.

%%[[page_296]]
10. Такое уважительное отношение к молитве воспринимается как участие всех присутствующих на площадке в скорбном молчании. В такой прискорбно-печальной обстановке прошла первая часть похорон дорогой Инны. Некоторые уточнения: похороны были назначены на 17.06.15, на 12 часов дня; все близкие приехали заблаговременно к жене, а потом все поехали на кладбище. Там все собрались на площадке прощания, возле кафедры. Вскоре подошёл священник и сказал, что пора начинать. Он взошёл на кафедру и прекрасным баритоном стал читать молитву: эмоциональные всплески и искренность мольбы за Инну доходили с болью до глубин души.

%%[[page_297]]
Закончив первую молитву на площадке прощания, Яров предложил сказать слово от семьи. Боря и Маша поднялись на кафедру. Боря от себя, Жени, семьи говорил с сдерживаемым волнением, всеохватывающе проникновенно. Маша переводила на иврит.

%%[[page_298]]
12. 15 июня, 2015. Как ни раздумываешь, не ожидаешь смерти, это всегда удар под дых. Дыхание перехватывает, хотя никто из нас еще не осознал реальности происшедшего. Мамы не стало. Ребенок, который остался в каждом из нас, потерял половину мира, и эти фантомные боли на месте разорвавшейся связи остаются навсегда. Мы знаем, что любая жизнь подходит к концу, и каждый из нас думает об этом, и подводит промежуточные итоги. Чем старше, тем чаще. Сегодня для мамы этот итог превращается в окончательный. Какой же он? По-моему, феноменальный. Во-первых, сами обстоятельства смерти. Она встала, позавтракала, помыла посуду, и ее нет. Кто не позавидует такой легкой смерти? Во-вторых, она создала и оставила после себя огромный клан, поразительно дружный во всех, уже многих, поколениях. У нее был папа, муж, который ее буквально обожал и боготворил до последней секунды. У нее была жизнь, которая хотя была очень трудной, но эта ее жизнь все время шла по восходящей линии, а это самое важное. У нее была работа в одном из самых интересных мест на свете. Ее там любили и она любила эту работу. Это была жизнь, которой, несмотря на трудности, можно позавидовать. Но все имеет свой конец. Время делает свое дело. Но вот что удивительно. Деда как раз на этой неделе остановился в мемуарах, которые он пишет, на моменте, как он увидел Инну в первый раз, и как он преподнес ей большой букет красных роз. Так вот, круг замыкается: как их совместная жизнь началась с букета роз, так она и заканчивается большим букетом красных роз. Давайте возложим его на могилу. Это очень тяжелый день, и это начинает еще более тяжелый период для всех нас, а для папы в особенности. Но давайте всегда помнить, какая богатая, необыкновенная, феноменально успешная жизнь закончилась сегодняшней церемонией, когда мы предаем маму земле.

%%[[page_299]]
После похорон Инны начался семидневный траур — шива (слово шива — семь). С утра до позднего вечера была открыта наша квартира, в вестибюле висело типографски выполненное сообщение о смерти Инны. Любой человек, знающий нас, мог зайти и выразить соболезнования. Приходили родственники, друзья сыновей, коллеги по работе, причём коллеги самых высоких должностей. Пришедшие выражали искренние сочувствия, создавалась обстановка тепла, душевной поддержки: никто не спешил уходить, наоборот, группки собравшихся беседовали между собой. Вся эта обстановка как-то уравновешивала ситуацию. Велико значение шивы. Убедимся на отношении к Жене его коллег. В отдельные дни шивы квартира заполнялась в основном коллегами Жени. Машины и Мышины сотрудники также выразили соболезнования.

%%[[page_300]]
На тридцатый день после похорон отмечается Хаскара — день памяти (от слова зиккарон — память). На этот заключительный обряд прощания, как правило, съезжаются в основном самые близкие люди, хотя о хаскаре сообщается всем. Традиционным стало то, что ко дню Хаскары заканчивается изготовление надгробного камня и его установка на могиле. Женя проследил, чтобы традиция не нарушилась.

Как и на предыдущих обрядах похорон, была прочитана молитва, все были под впечатлением появления именной плиты. Все горестно созерцали этот факт. Постояв немного среди нас, раввин обратился ко мне на идише и сказал: «Эршт кент ир кимен аэр ибер аюр» (теперь вы можете прийти сюда через год). Этой фразой раввин как бы завершил месячный период ритуалов похорон.

%%[[page_301]]
и я закончил письмо моей умершей Инуле: письмо адресовано но также всем близким как память об Инуле. Предложенный вариант изложения - не моя выдумка, а результат желания противостоять гигантскому давлению обрушившейся на меня беды. Я имею в виду смерть Инны. Её смерть - смертельный удар по мне. Однако пытаюсь выстоять. Это выражается в том, что я неизменно, по-прежнему глубоко всегда любил и люблю Инулю. Её смерть не погасила ни одного чувства. Я разговариваю с ней, советуюсь, слышу её реплики. Она не уходит из моего сознания. Я не даю этому совершиться, поэтому, считаю, я вправе напрямую сказать ей: - Инулик, родная! Столько, сколько будет существовать моё сознание, столько буду тебя любить и дорожить памятью о тебе. 15.9.2015 Фильм

%%[[page_302]]
16.
Как понять? Письмо я закончил 15-9; Инна умерла 15-6. Прошу поверить, что совпадение чисел 15 совершенно произвольное, у меня и мысли не было, чтобы совместить числа. Но, очевидно, удивлю Вас фактом: в ночь на 15-9 мне приснилась Инна. Она была в осеннем пальто и в шляпе с широкими полями, которую мы купили после нашей женитьбы. Инна улыбалась и смотрела мне в глаза. Сон был очень чёткий, ясный. Что бы значило сведение трёх взаимосвязанных явлений к одному числу 15? Впрочем, кто бы что ни сказал, грусть, тоска по Инне будет всегда со мной. Обратить внимание на число мне посоветовал Женя. 22-9-15.

%%[[page_303]]
17.

Вместе с Ваней ещё раз прочитал письмо. Оно по-прежнему созвучно с моим состоянием тоски по Инне. Тоска угнетает, убеждает в беспомощности, но она же вынимает из памяти былое; тоска читается им. Но былое разное по эмоциональному настрою, содержанию есть и радостное. Вернусь к тем страницам, которые я написал до вашего прочтения моего письма. Возвращаюсь к рассказу об Инне.

Итак, 1945 год. Москва. Мне 21 год. Инне 20. Мы ещё не знакомы, но шум победы над фашистской Германией кружит головы, рождает надежды; после больших людских потерь за время войны быстрее стали завязываться знакомства, и жизнь кипела.

Я в своей личной жизни придерживался принципов, с которыми вы уже...

%%[[page_304]]
Знакомы.
Приближался первый послевоенный новогодний вечер в моём Полиграфическом институте. Это 30 декабря 1945 года. Предшествовавшие вечеру дни были суматошные и интересные для меня: сами студенты украшали зал, готовили номера, сложился приличный самодеятельный оркестр, были и свои декламаторы, пианисты.
Сложилось так, что именно в эти дни моему брату Грише его соседка по дому Сара Мазина предложила познакомить меня с её родственницей студенткой. Договорились. Вдруг оказалось, что в армейский отпуск приехал брат девушки, и предновогодние дни она проведет с братом.
30 декабря я пошёл на институтский новогодний вечер, получил огромное удовольствие, был в приподнятом настроении. Да, трамвай лимитировал, с запасом времени я стал медленно спускаться по парадной лестнице из Актового зала в вестибюль. Внизу тоже было людно, но мой взгляд остановился на девушке.

%%[[page_305]]
которая всем своим естеством источала скромность и обаяние. Как-то сразу я почувствовал особенности моей незнакомки. Все угаданные штрихи детали сложились в действие: меня сильно потянуло к ней, к её обаянию. Обилие, а главное значение впечатлений пробудило во мне особую решительность. Я сам себе сказал: «Эта девушка будет моей женой».

20. Она и её подруга беседовали с моей однокурсницей Жанной Совниной, лучше момента не найти! Я подошёл, обобщающим новогодним приветствием поздравил, и Жанна представила мне своих подруг — Зину и Инну.

Редко в то время имя Инна в моём воображении мгновенно слилось с образами впечатлений и ещё больше облагородило.

%%[[page_306]]
Здороваясь, я обратил внимание на быстроту реакции Инны на обстоятельства: она раньше меня протянула левую руку при рукопожатии. Многие терялись от непривычки. Впоследствии Инна говорила, чтобы я вообще не протягивал руку, а приветствовал только голосом, что и есть по сей день. Здороваясь, мы невольно посмотрели в глаза друг друга. В её глазах я разглядел ум и спокойствие. Девушки тоже собирались домой, и мы все вместе вышли на Садовое Кольцо, прошлись, поговорили и направились к станции метро.

%%[[page_307]]
Прогулка подходила к концу. Я судорожно думал над тем, как в ближайшие дни увидеть Инну. В помощники взял себе пропагандистскую работу МГУ. В корпусе напротив манежа по выходным дням читались научно-популярные лекции по отраслям наук. Было очень познавательно и интересно. Я предложил Инне пойти со мной в ближайший выходной день. Она, после паузы, ответила согласием. Я был счастлив! Но был сдержан. Мы замечательно провели первый совместный выходной день. Были на лекции, потом решили пройти пешком от МГУ до Маросейки, там жила Инна. Хорошая и долгая беседа получилась задушевная.

%%[[page_308]]
В таких беседах задушевность возникает, можно сказать, всегда, потому что хочешь понять духовный мир понравившегося человека, в том числе и такого, с которым готов соединить свою жизнь. Через сказанное, через рассуждения раскрывается логика мышления. И себя раскрываешь таким же образом. Встречались мы довольно часто, но по настрою чувств, ожиданий хотелось большего. Расставаясь, всего больше сожалели о следующей встрече.

%%[[page_309]]
Речь уже шла не её интерес об организации встречи, о том, как сделать её. Москва в области культуры всегда была представительным городом. У многих был свой памятный уголок. Для нас это был лекционный зал, где мы впервые и осознанно встретились. Мы там бывали не часто, но всегда с чувством трепета. Именно там я первый раз букет цветов неожиданным образом повлиял на наши отношения. Сара Мазино. Уже далеко за новогодние дни она решила довести замысел знакомства до конца. Позвонила родственнице и из долгого телефонного разговора составился мой портрет. Мы обрадовались необыкновенно: судьба дважды сводила нас (институтский вечер, Сара). Нам суждено быть вместе.

\fi