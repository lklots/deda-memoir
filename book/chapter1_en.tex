\pageimage{page_001}
\chapter{Introduction}
\label{1-1}The passage of years has brought me to that stage of life when a person can say about themselves that they are the oldest by age and the only one still alive who knew many representatives of what was once a large family. Unfortunately, I do not know everything and everyone. I am ashamed that I do not even know the name of my grandmother - my father's mother. More about such phenomena - later. Here, in the introduction, I consider it necessary to note that I wrote only about what I remember myself and what I heard in the family; assessments of some socio-political events in the understanding of the time to which they belonged, and some details for a better understanding of the situations.

\label{1-2}I wrote the memoirs for many reasons. The chains of life of most Jewish families are broken by forced diaspora, many links of the chain are irretrievably lost, so it is necessary to pass on what is still known to descendants; fortunately, mine are interested in this. I have long wanted to engage in family topics, but current work never ends.

\pageimage{page_002}
\label{2-1}The sequence of work was corrected by my youngest grandson Lyovochka: he showed great interest in the history of the family. However, the sixtieth anniversary of the Victory over Nazi Germany was approaching. My friends suggested responding to this very important date, and I began my family memoirs with myself, writing "Meters of Infantry". The theme's concept is not violated: the war passed through the lives of people from our family with relentless steel scrapers.
