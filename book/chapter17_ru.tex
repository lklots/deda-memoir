\pageimage{page_268}
\label{268-1}
Надо спасать свое сознание воспоминаниями, пытаться войти в круг созерцаний окружающей реальности, пытаться быть нужным не только себе. Иной подумает: «Легко сказать, да трудно сделать!» Верно. Однако я рассказал вам о пережитом мною и происходящем сейчас. Хотел было вернуться в годы молодости, но беспомощность — тема цепкая и разноплановая: помимо нарастающих недугов медицинского свойства, она ещё влияет на моральное состояние. По мере углубления беспомощности возрастает зависимость от другого человека; и здесь уже, как правило, обратного хода нет. На эту реальность надо смотреть открытыми глазами. От этой печальной тематики отойду далеко назад, во времена послевоенных надежд.

\pageimage{page_269}
\label{269-1}
по отдельным рассказам и вам эпизодам вы поняли, что те времена быстро прошли. Из госпиталя я вышел уже после окончания войны. Все родственники проявили побольше внимания и думали, как мне помочь, чтобы как-то смягчить проблему. Гришина теща Варвара Васильевна и ее дочь Валентина настояли на том, чтобы какое-то начальное время я жил у них. Они оказались исключительно душевными, заботливыми людьми. Они жили в доме многокомнатных коммунальных квартир. Каждое утро Варвара Васильевна звала меня на кухню и пыталась научить меня что-то делать одной рукой. Не получалось или получалось плохо у нас обоих. В те времена с едой было очень скудно! Весь дом был на карточках.

\pageimage{page_270}
\label{270-1}
В случае, да ещё чай настоящий, то это было уже хорошо. Я пытался приспособиться к нахоженной работе и придумал нарезанную, но не чищенную картошку, зажав её между двумя тяжёлыми предметами; рационализация была одобрена, но применения она, конечно, не получила. Так, с трудом я приспособился к новой жизни. У Михайловых я прожил полтора месяца. По сей день я с величайшей благодарностью вспоминаю тот дом. Война не обошла и этот дом: Иван Иванович Михайлов - муж Варвары Васильевны, погиб на фронте, защищая Родину в 1944 году. Его портрет висел около иконы с лампадой. Тоску по погибшему мужу Варвара Васильевна выражала в том, что по воскресеньям утром ходила на молитву в сельский собор. Только долго

\pageimage{page_271}
\label{271-1}
Общаясь с ней, можно было понять крепость ее характера.
Итак, пришло мое время войти в реальную жизнь из тепличных условий военного госпиталя и дома Михайловых.
Начался самый долгий период моей жизни, который длится по сей день; расскажу вам, что было и есть определяющим в этом движении жизни; расскажу о руке судьбы, которая соединила меня с Инной.
Главную цель моего ближайшего будущего я определил, еще будучи в госпитале: получить институтское образование. Технические институты я исключал потому, что такой важный предмет, как черчение, был для меня недоступен. Меня огорчала только причина невозможности.
История и художественная литература также были объектами моего внимания, но я считал себя недостаточно подготовленным для успешной сдачи экзаменов на филологический факультет МГУ.

\pageimage{page_272}
\label{272-1}
Выбрать специфику высшего учебного заведения помогла мне художница Анна Изакова, соседка моей тёти по коммунальной квартире. Аня отнеслась ко мне с большим вниманием и посоветовала поступить на редакционно-издательский факультет Полиграфического института. Предварительно она рассказала о специфике факультета и о педагогическом составе. Когда я услышал фамилии Светлаев и Крючков — авторов школьного учебника "Русский язык", мой выбор определился: пахнуло школой, довоенным временем, мне захотелось быть близко к тем, кто представляет прошлое.

\pageimage{page_273}
\label{273-1}
Эпизодом из общения со Св. Светлаевым хочу похвастать. Он раздавал контрольные работы после проверки их; когда я подошёл, чтобы получить свою работу, Светлаев сказал мне, что я хорошо чувствую язык и посоветовал сосредоточиться на этом. Итак, основное направление моей будущей профессии определилось, и я поступил на подготовительное отделение института. Там поселился в тех же двух комнатах, в которых жила наша семья до моего ухода в армию. Пришлось научить себя колоть дрова, собирать их в вязанку, зажигать лучины, штопать брюки, завязывать шнурки на ботинках.

\pageimage{page_274}
\label{274-1}
В ботинках и мостов пречеву каждо есть освоение какого-либо действия вызывало во мне внутреннюю, никому не сообщаемую радость. «Не сообщаемую» — это потому, что я старался держать себя так, чтобы по стечению руковой не привлекать излишнего внимания. Моё поступление в институт обрадовало родственников, и они начали давать советы. Например, Яша и тётя Берта предложили заходить к ним покушать по пути из института домой (по тем временам — царский подарок); дядя Фима и тётя Лия взяли круче — советовали жениться. С этой целью познакомили меня с семьей своих друзей.

\pageimage{page_275}
\label{275-1}
В представлениях о не очень далеком будущем женитьба была у меня на последнем месте. Верьте, я не оригинальничаю: мне противно лицемерие. Добросердечность родственников очень приятна, но и хорошие советы теряют свою привлекательность, потому что они затрагивают одну проблему из ряда. И каково мое отношение к ним, советчик не знает. Главное - я сам тоже не всегда знал. Однако по опыту жизни моих родителей и тех же наставников я имел представление об ответственности за семью. С такой, может быть, даже скептической, подход к доброжелательности был уместен в моих условиях, так как насущным был проблемный вопрос: как и на что мне жить? Никто за меня это сказать не мог, я и сам не сразу нашёл ответ. Пришлось крепко задуматься.

абзац

\pageimage{page_276}
\label{276-1}
на Отправной точкой в рассуждениях о себе была незыблемость институтского образования. Стипендии и пенсии инвалида войны, которая в первые послевоенные годы была на низком уровне, не хватало для скромной жизни. Тогда, может, учиться и одновременно работать? Но для этого я не чувствовал достатка физических сил, работа победила бы учебу, и я, в итоге, остался бы без образования (после третьего курса я учился и работал). Триумф Победы вселял надежды, они складывались из больших и малых фактов: армейские части демобилизовывались, а предприятия наполнялись рабочей силой, начали завершаться последствия эвакуации. Еще применительно к моим личным интересам, студентам Полиграфинститута разрешили пользоваться столовыми министерства торговли и министерства сельского хозяйства.

\pageimage{page_277}
\label{277-1}
Все названные заведения расположены на перекрёстке Садовое Кольцо - Орликов переулок. Удобство места сочеталось с хорошим качеством обедов, и это надолго решило проблему моего питания, ну а слова «завтрак», «ужин» сохраняли ещё реальное по тем трудным временам содержание. Это называлось одним словом «часк». Помнится и такое: поставил человек на стол поднос со своим обедом, достал из него посудинку и положил в неё часть обеда для кого-то дома. Подобное можно было видеть нередко. Трудности жизни проявлялись на городском уровне: малое, личное перемежалось с крупномасштабным, государственным, и всё это входило в понятие послевоенное восстановление.
