

%%[[page_187]]
I Подготовку к переезду 187. папа начал с «закрепления тыла»: он пошёл к директору моей школы (603), Кустжанину Ивану Дмитриевичу. Рассказал, почему на какое время я остаюсь дома практически один и попросил у и.о. разрешение позвонить в школу, если возникнет острая необходимость передать мне что-либо важное; понимание и согласие было мгновенным; мне папа дал указание — после занятий обедать в столовой (общественного питания), обязательно кушать суп; отъезд намечался дней на 20, поэтому мне предписывалось быть молодцом: правда, папа сказал это не в назидательно приказном тоне (что случалось нередко), а в тоне некого дружелюбного на-

%%[[page_188]]
поминания об обыденном; когда папа говорил всё это мне, выглядел он уставшим и без того худой, он, казалось, похудел ещё. После отъезда мамы я быстро становился молодцом после нескольких дней проб водянистых супов, как я их назвал, на мясном (в меню столовых писали: суп на мясе, то есть на мясном бульоне) и после слипающихся макарон. Я перешёл на консервы «Сазан в томате» и на пластовый мармелад «облочный». Так я обеспечил себе ежедневное питание. В школу я брал 4-5 толстых кусков хлеба с проложенными не менее толстыми слоями мармелада. Когда Гриша или Давид иногда приезжали с ночёвкой, я потчевал их тем же. Папа домой не приезжал: после работы он ехал к маме, чтобы сменить бабушку и нанятую в помощь женщину. Папа дежурил возле мамы.

Визиты профессоров

%%[[page_189]]
как не улучшили ситуацию, каждый из профессоров приезжал в выходной день. Я тоже навещал маму по выходным, поэтому хорошо помню разнохарактерные облики этих людей: профессор Кончаловский М.П. – интеллигентный, спокойный человек, с добрыми глазами, продолговатым лицом и удлиняющей его остроконечной (казачьей) седой бородкой, профессор Юдин С.С. высокий, грузный, с красным лицом и, кажется, с бритой головой. Несмотря на внешнюю грубоватость, взгляд профессора выражал сопереживание, а это нейтрализовало первое впечатление моё. После окончания знакомства с больным, профессоров, как и в первом случае, пригласили в столовую комнату на стакан чаю. Там проф. Казалось, долго беседовал с папой и дядей Фимой; когда дядя

%%[[page_190]]
190. Пошёл провожать профессора до машины, я впился глазами в папу, помню, как улыбка вежливости сошла с его лица. Он подошёл к кухонному столу и с тихим причитанием швырнул большой конверт с мамиными медицинскими документами. Об этом лучше не спрашивать. Под влиянием папиных усилий добиться визитов двух столпов советской медицины, которые занимались, в частности, проблемами лечения заболеваний пищевода, в нём притаилась слабенькая надежда (видимо, оттянуть время), но папино лицо всегда отражало состояние его души. Папа не был мастером камуфляжа; надежды ни у кого никаких, воцарилась реальность, продиктовавшая бездеятельность. Только разрушать организм продолжало. О возвращении домой уже не говорили. Я по-прежнему навещал маму по выходным дням. В октябре 1938 года, когда...

%%[[page_191]]
Рассвет ещё не пробился до конца сквозь утренний туман. Раздался сильный стук в окно. Я вскочил с постели и увидел пальцы Давида, барабанящие по стеклу... И ожидаемое приходит неожиданно; маме было 47 лет (на памятнике — 45). Моя жизнь с папой складывалась довольно трудно: он разом потерял семью, остался с заботами о младшем сыне и повседневных делах жизни; с последним словом: многие бытовые дела решали вдвоём, вплоть до... "После рода братьев Постарушек, но сами попали в тупик — никто не соглашался постоянно готовить нам. Мы перешли на столовские обеды (на них дотянули до начала войны, когда опять ввели для населения продажу по карточкам недостающего в стране продовольствия). Обедать в столовых стало крайне разорительным по отношению к скудному товарному содержанию продовольственной...

%%[[page_192]]
карточки и низким питательным качеством. 192. Вам обедов: однако долго ломать голову не пришлось: общедоступные столовые общественного питания закрыли, перепрофилировали. После смерти ма- резому мой дом потерял для меня свою притягательность. Всё в нём смотрелось по-иному, отчужденно, поэтому я часто задерживался в школе. Там делал уроки, помогал приводить в порядок оборудование физического и химического кабинетов и пр. Я скучал по времени, в котором мы - все пятеро были вместе. Короткие письма Давида и редкие приезды Гриши подчёркивали тяжесть сложившегося положения. И здесь папа с его свойственной ему решительностью попытался отде

%%[[page_193]]
лить возможное

193. от невозможного: весной 1990 года мы поехали к моей бывшей няне — Ксении. Это с ней я гулял по Ульяновску моего раннего детства; поездке я обрадовался вдвойне: Ксения так же жила в Ульяновске. Она жила в общежитии при какой-то фабрике в районе станции «Планерная», а недалеко был авиазавод и аэродром «Тушино».

До войны ежегодно отмечали день авиации. Возможность увидеть и то и другое вдохновила меня.

— Итак, Ксения — последняя попытка папы сохранить атмосферу привычного ему дома с моей помощницей дотянуть меня. С Ксенией я не виделся 11 лет, она встретила меня с большой радостью, ее широкая улыбка и искрящиеся глаза выражали восторг. В моём подсознании всплыли и очевидно, далекие ассоциации, когда я был в основном между мамой и Ксенией, их

%%[[page_194]]
образы в раннем детстве смешались, поэтому в Ксене я почувствовал почти родное, материнское. Я был взволнован, бывшие в комнате женщины незаметно ушли, мы сели за стол, который стоял посередине большой комнаты. Я огляделся: было очень чисто и светло, по четыре кровати с тумбочкой при каждой стояло перпендикулярно к противоположным стенам. Напротив входной двери, в высокой части простенка между окнами, — портрет И.В. Сталина. Дыхание выходного дня чувствовалось: все кровати застланы белыми покрывалами, удлиненными вязаными подзорами. Только один, связанный из толстых ниток и углами книзу, понравился мне больше остальных.

%%[[page_195]]
Были те годы, когда в качестве корреспондента редакции экономического журнала приезжал в колхозы, совхозы, ночевал в крестьянских избах, у меня была возможность воспринять разнообразие красот орнаментов вышитых и вручную вязанных подзоров. Ксеня принесла чайник, разлила по стаканам чай и молча положила на стол папино письмо. Я почувствовал, что искорка волнения охватила нас. Чай не пился; разговор начала Ксеня с расспросов о моей учёбе, заказано ли надгробие на мамину могилу, как мы справляемся дома. На вопросы односложно отвечал я; потом она обратилась к папе напрямую и сказала, что письмо читала несколько раз, всё себе представляет, очень жалеет нас, но с фабрики уходить не хочет, потому что стала ударницей, сложились хорошие отношения в коллективе.

%%[[page_196]]
196

Подругами; через Ксенины слова пробивались представления о новых общественных ценностях того предвоенного патриотичного времени. Папины предложения не противоречили интересам времени, но звучали архаично; он приехал за ответом на письма, однако при личной встрече пересказал его, пытаясь переубедить Ксеню; безуспешно. На основе давнишнего доверия папа предложил Ксене переехать к нам, быть хозяйкой дома. Мы отдаём ей вторую, изолированную комнату. Папа не в силах быть, как он выделил, «хозяйкой». Убеждён, что подтекста в его словах не было, но в дальнейшем я бы хотел видеть папу неким. Визит к Ксении удручил его. Он явно сник: «Кому довериться?». Я очень жалел папу, поэтому с наигранным мальчишеским задором призвал его не унывать, выплывем. Когда наступили школьные каникулы, я уехал в пионерский лагерь. Попытки смять закончились условиями нашей жизни.

%%[[page_197]]
2. Родственники
197 Если бы родственники были внимательны к нам, усиленно приглашали на выходные дни: отчасти такое внимание объясняю тем, что тётя Соня и братья очень любили свою сестру Гутю, это переложилось в определённой мере на папу и на меня. Жизнь не останавливалась: вскоре после смерти моей мамы тётя Берта родила девочку (23.1.1941), назвали в честь и в память о маме Рита (имя Гита осовременили заменой первой буквы); имя принялось неплохо: Рита всегда красива, эффектна, высока, имела большой успех: сейчас она счастлива тесным общением с семьёй своей дочери Яночки, тремя малолетними внучатами; живет Рита недалеко от своего старшего брата Шуры Великанского, с которым я вас уже познакомил. Да, моральная поддержка родственников имела большое значение, но неприкрашенная действительность оставалась при нас.
Теперь, как и в предыдущих фрагментах воспоминаний, выйду из круга личного и соединю его с жизнью страны.