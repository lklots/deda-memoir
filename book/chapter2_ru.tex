\label{2-2}
Чем дальше уходят годы Великой Отечественной войны, тем важнее, мне представляется, показывать связь военных эпизодов с решениями высшего командования, с фактами жизни страны на всех стадиях войны. В период определившегося всеобщего наступления Советской Армии действия даже взвода были частью стратегически-тактического плана победы. В такой связи - один из залогов успеха и более ясного представления о происходившем. Поэтому приведу фрагменты из книги Мартына Мержанова «Солдат, генерал, маршал (о Баграмяне и др.)». Из-во полит. литературы, М., 1974.

\label{2-3}
В июле 1944 года «Правда» писала: «По центральным улицам Москвы под конвоем советских солдат прошли 57 600 человек пленных.

\pageimage{page_003}
\label{3-1}
хваченных в Белоруссии. Впереди гигантской колонны, опустив головы, двигались германские генералы и офицеры... Они победно промаршировали через многие столицы Европы — Варшаву, Париж, Прагу, Белград, Афины, Амстердам, Брюссель и Копенгаген. Их мечтой было так пройти по Москве. И вот они шагали по ней, но не как победители, а как побежденные... Шагали пленные мимо молчавших, гневных москвичей, плотными рядами стоявших на тротуарах. с. 81. 3. Прошедшие пленные, по численности, составляли более четырёх стрелковых дивизий. Внушительная сила. Так что июльское шествие было зримым свидетельством грядущей победы, моральным стимулом измученному войной советскому народу. Тогда 9 мая было не более, чем число в календаре. Несмотря на определившийся явный перевес в пользу Советской Армии, война оставалась ожесточённой, враг пытался переломить ситуацию. 4. Баграмян исходил из того, что Прибалтику гитлеровцы будут удерживать до последних возможностей. Эта их задача имела не только стратегический, но и политический смысл. Опираясь на созданные здесь хорошо оборудованные в инженерном отношении рубежи, командование группы армий «Север» могло высвободить достаточное количество войск для противодействия войскам 1-го Прибалтийского фронта. с. 82. 14. Предстоящая операция была связана с трудностями, которые возникли в связи с контратаками гитлеровцев. 20 июля 1944 года началось наступление ударной группировки фронта на Шяуляйском направлении. Войска сравнительно быстро преодолели первый оборонительный рубеж и двинулись вперед. с. 83, 85. В книге М. Мошинов. с. 86.

\pageimage{page_004}
\label{4-1}
Обратим внимание ещё на один значительный факт, произошедший в тот же день, 20 июля, но по ту сторону фронта. В ту ночь, когда командующий фронтом генерал Баграмян и начальник штаба генерал Курасов окончательно отшлифовали все детали шяуляйской операции, в Восточной Пруссии, в районе Растенбурга, в ставке Гитлера - так называемом «Вольфшанце» («волчье логово») - в 300 километрах от штаба 1-го Прибалтийского фронта, тоже над картой сидел начальник оперативного отдела генерального штаба генерал А. Хойзингер. Он готовился к докладу фюреру, назначенному на 20 июля в полдень... Именно в эту секунду раздался оглушительный взрыв. Гитлер вылез из-под упавшего на него стола. Он получил ожоги и легкие ранения. (с. 85-86) Наши войска шли на запад. Гитлеровцы чувствовали близкую Восточную Пруссию и край земли - берег Балтийского моря. На второй день после начала наступления был освобожден город Паневежис, вскоре и Шяуляй. В тот же день после упорных боев войска 2-го Прибалтийского фронта и 6-й гвардейской армии генерала Чистякова овладели Даугавпилсом. После освобождения Шяуляя противник лихорадочно усиливал сопротивление. (с. 86) Череда масштабных событий приближала мою особую дату, 24 июля 1944 года, о которой рассказываю. Раннее утро. Передовая. Вернулась разведка. Она проверила, в частности, положение на нейтральном участке, откуда предстояло продолжить наступление. Незадолго до этого 6-я гвардейская армия входила в состав 1-го Прибалтийского фронта. В данной операции она взаимодействовала с войсками 2-го Прибалтийского фронта.

\pageimage{page_005}
\label{5-1}
Возвращения разведки нам принесли в наплечных термосах горячую вкусную кашу с маслом, сладкий чай, хлеб. Мы не торопясь наелись, ещё раз проверили оружие, кто-то осторожно закурил, все замолчали; ждали приказа на выдвижение к исходным позициям. Я был младшим лейтенантом (после окончания годичного Тульского пулемётно-миномётного училища), командиром стрелкового взвода - 517 стрелковый Краснознамённый полк, 166 гвардейская дивизия, 8-я гвардейская армия. Обстановка на уровне стрелкового-пехотного взвода была такова: Нейтральная полоса проходила как под углом. Линия нашего переднего края возвышалась слегка над узким полем, которое упиралось противоположной стороной в молоденькое редколесье. Предстояло спуститься в лесок, пройти его поперёк и залечь за низким земляным валом - рубежом атаки. Впритык к валу, с внешней от нас стороны, узкая мелиоративная канава. За валом, напрямую в сторону немцев, тоже было небольшое поле длиной - по направлению нашего наступления - метров, примерно 120.

\pageimage{page_006}
\label{6-1}
Это поле было под слабым наклоном, но в нашу сторону (для наступающей пехоты это плохо). На немецкой стороне поле граничило с кустарником, старыми деревьями, боковой стеной длинного бревенчатого хуторского строения (амбар или сарай), а немного правее — с большой ёмкостью на высоких металлических опорах. Два объекта — строение и ёмкость — можно было превратить за короткое время в хорошо защищённые огневые точки. Это входило в защитную полосу немецкой обороны. Такова была диспозиция, место нашего взвода в бою, показываю всё в плане.

\pageimage{page_007}
\label{7-1}
Выдвижение на исходные позиции прошло быстро, спокойно. В молоденький лесок вошли в полной боевой готовности. Когда приближались к земляному валу, противник открыл огонь, как слышалось, по широкому фронту наступавшей армии. Всем взводом мы нырнули к валу. Никто не пострадал. Мгновенно открыли ответный огонь. Сила нажатия на спусковой курок, убойная сила пули, казалось, умножались на силу ненависти к фашизму, Гитлеру. Автоматными очередями мы били по местам наиболее вероятных, надёжных для врага, огневых точек. По обстановке мы хорошо подготовились к атаке. В соседнем взводе находился командир роты (значит, там участок был сложнее). Они тоже как следует прополоскали позицию врага, и мы одновременно поднялись в атаку: мигом перемахнули через земляной вал, канаву и устремились вперёд. Особенность боя состояла в том, что расстояние между нашими и немецкими позициями, повторяю, было минимальное. Это значило, что нам надо было сбить уверенность врага в надёжности обороны, а главное — усилением огня ближнего боя выбить его с занимаемых позиций. Причины такого вида пехотных атак возникали потому,


\pageimage{page_008}
\label{8-1}

Что если враг выдерживал ад ударов сгруппированной огневой мощи дивизий, армий, то он закреплялся на промежуточных линиях обороны и пытался сдержать натиск наступающей Советской Армии. Для этого он поспешно выстраивал защиту, и на переднюю линию своей обороны выдвигал (обычно стрелковые подразделения) человека с ружьём. Немцы отступали, но не бежали, поэтому — не дать закрепиться врагу на промежуточных позициях, выбить человека с ружьём и отогнать противника к рубежам и ко времени, установленным в оперативных разработках высшего командования. Такова задача. В подобных операциях последнюю точку ставила пехота. В этом была её незаменимость, сила и жертвенность. Мы атаковали и на бегу вели автоматный огонь короткими очередями, чтобы ограничить ответный огонь противника и не снижать темпа атаки. (Пояснение: на коротком расстоянии, да ещё на открытой местности под наклоном даже слабым, но...


\pageimage{page_009}
\label{9-1}
Выходным противнику, "классическая" атака — залегания и короткие перебежки с ведением огня — практически исключена. Атакующие хорошо видны, малоподвижны. Нужен быстрый бросок и сконцентрированный огонь по неширокой полосе атакуемой взводом линии. Так и было. По мере быстрого преодоления короткого расстояния огонь нарастал, но хуже тому, возле их позиций находится атакующая пехота. Однако в общем шуме боя пули выводили из строя наших бойцов. На каком-то десятке метров что-то плоское, жёсткое очень сильно толкнуло меня справа, сбило с ног. Толчок был такой силы, что в моём сознании не зафиксировался момент падения.

\label{9-2}
Самоощущение начало возвращаться непрерывным гулом, мраком и своеобразным ветром в голове. Всё это как бы вытягивало меня в какую-то даль, к светлому пятну. В какой-то миг открылись глаза. Я лежал ничком, в луже крови. Жгущая боль во рту, на лице, в правом плече. У головы...

\pageimage{page_010}
\label{10-1}
Лежал мой автомат ППШ (пистолет-пулемёт Шпагина), боковым зрением увидел куски рваной щеки, с которых стекали сгустки крови; правого предплечья не было: месиво мяса и крови держало лоб на куске разорванного рукава гимнастёрки, и на темно-синей жилке ужас охватил меня: неужели умираю, неужели не увижу мою Москву?!

\label{10-2}
Москва не представлялась ни родственниками, ни родным домом — мне было 19 лет, и мои родители к тому времени умерли. Москва представилась тогда уютным уголком Пушкинской площади, где стояли памятник великому поэту (на исконном месте) и гипсовая статуя балерины на крыше нового углового дома на улице Горького.

\label{10-3}
Растерянность, шок сменились восстановлением сознания, хотя в голове сильно шумело и стучало в висках, сгустки крови давили горло, я беспрерывно заглатывал их. Однако страх смерти сменился стремлением выжить. Я осторожно развернулся на левом плече, опёрся на левую часть спины, левой рукой подтянул правую руку, потом протащил полевую сумку и ремнём обмотал оторванную руку; в поиске некоторого облегчения боли я попытался снять с лица жгущую тяжесть, но пальцы коснулись языка. Я решил выбираться своими силами. Не успел поеду.

\pageimage{page_011}
\label{11-1}
свои действия, как через секунды я увидел возле себя медицинскую сестру. С колен, нагнувшись без слов, она начала бинтовать моё лицо, потом дважды пыталась наложить жгут на правое предплечье, жгут не держался. «Придётся тугую повязку», — сказала она тихо ни мне, ни себе и опять открыла санитарную сумку. Пока сестра бинтовала, я пытался внятно произнести слово «пить». Она, конечно, всё понимала и без моего мычания, и когда кончила бинтовать, приподняла две фляги, слегка потрясла ими и тихо, с большим сочувствием сказала: «Раненые выпили. Потерпи, родной, скоро заберём тебя». По-прежнему оставаясь низко пригнувшейся к земле, сестра отползла. Тугие повязки приглушили ощущения острой боли, а жажда мучала: казалось, что во рту у меня провонявшая, перепревшая портянка, которую я вынужден отсасывать... Полулёжа на спине, я подтянул повыше, к груди, полевую сумку с примотанной рукой, зацепил ремень автомата и, опираясь на левый локоть, стал ползти в сторону лесочка воз-

\pageimage{page_012}
\label{12-1}
Можно, если бы я попил воды, остался бы ждать помощи. Близкий лесочек уже казался далёким, однако я дополз. Помню себя уже в санроте: значит, сестра вернулась, меня вынесли. В санроте наполнили незабываемой прелести сырой, прохладной водой, сделали обязательный для всех раненых противостолбнячный укол. Потом операционная медсанбата.

\label{12-2}
\footnotetext{Санитарная рота - ближайший к боевым действиям медицинский пункт; располагалась на передовых позициях}
\footnotetext{её функция - неотложная помощь раненому и отправка его в медсанбат.}

\label{12-3}
\footnotetext{Медицинско-санитарный батальон - развёрнутый в боевых условиях полноценный, по мере продвижения армии стационарный больничный комплекс; располагался в глубине фронтовой полосы на уровне вспомогательных подразделений дивизий, армии.}

Из медсанбата прооперированных тяжело раненых бойцов отправляли по цепочке эвакогоспиталей (ЭГ) до ближайшей восстановленной железной дороги, куда подходили санитарные поезда. Они развозили раненых по госпиталям страны; в ЭГ раненые находились по 2-4 дня, в зависимости от процесса лечения и их физического состояния. Перевозили раненых и самолётами.

\pageimage{page_013}
\label{13-1}
Была ночь, когда я проснулся от наркоза. Увидел себя в большой армейской палатке. На центральном опорном шесте слабо светилась лампочка. За небольшой тумбочкой сидела медицинская сестра. Много коек, на всех раненые. Полог входа в палатку отброшен, тянет запахом леса. Особая тишина ночи. Ни выстрела, как будто нет войны. Моя челюсть стянута бинтами, по горло накрыт простыней. Рука?! Я резко сдёрнул простыню. Широкие бинты опоясывали грудь, а правое плечо было замотано вкруговую толстой бинтовой повязкой. «Всё. Руки не будет», — холодно сказал я себе и понял свою неновую будущность. Сестра услышала какое-то движение, подошла, заново накрыла простыней, говорила что-то обнадеживающее, но я уже не вслушивался, как утром. Оказалось, я не всё знал тогда об этом дне — была и третья пуля. В городе Иваново, где я лежал в челюстном госпитале, впервые после ранения взял свою полевую сумку и ахнул: в краешке её левого уголка был пулевой вход (попросту говоря, маленькая дырка), а на вылете, в задней утолщённой стенке — большой разрез, как ножом; на донышке сумки лежали разороченная латунная оболочка пули и...

\pageimage{page_014}
\label{14-1}
Рассплющенный кусочек свинца, так рвёт разрывная пуля. По сей день храню эти предметы, ставшие для меня реликвиями Великой Отечественной войны. Дороги к победе над германским нацизмом были крутыми. Слушая чистые звуки фанфар победителям, отдадим честь и тем, чья молодость навечно застыла в той тяжёлой войне. Я награждён двумя орденами Отечественной войны, первой и второй степени, а также медалью «За Победу над Германией» и памятными юбилейными медалями СССР, РФ и \mbox{Израиля.\footnotemark}

\label{14-2}
\footnotetext{Сокращённо опубликовано в еженедельнике «Кстати» (Калифорния) 1-7 июня 2006 года. №585, стр. 35. http://www.kstati.net}

\label{14-3}
Продолжение начну с важного для меня эпизода из госпитального периода. Я задался вопросом: как мне жить без руки? Сложности были очевидны, да суровые условия жизни военных лет не утешали. Были и такие инвалиды войны, которые не выдерживали свалившихся на них бед и скатывались на дно жизни; это вовсе не была закономерность, следовавшая из их прошлого. Я понял, что помощь, силу надо искать в себе же; уже в госпитале надо начинать готовить себя к новой жизни. Но как? Первое испытание подсказалось необходимостью написать братьям о своём ранении.

\pageimage{page_015}
\label{15-1}
Я побуквы Когда решил, что достиг успеха в каллиграфии, попросил медицинскую сестру поискать твёрдую доску или фанерку, чтобы на согнутых ногах мог написать два коротких письма. Как сейчас помню её удивление и предложение написать письма под мою диктовку. Я поблагодарил и отказался, объяснив, что у меня никого нет, надо учиться всё делать самому. Буквы получились корявые, адреса написала сестра. Главное было в том, что я не принял соблазнительную, иногда разрушающую волю человека помощь. Это был мой первый шаг к самоутверждению в сложившейся реальности.
