
\pageimage{page_157}
\label{157-1}
Мы

Весной 1945 года, когда я уходил в армию, и уже не было ни мамы, ни папы, тетя Саша вышла меня проводить. Она держала меня за руку и сказала: "Ты уходишь под дождь, значит ты вернешься." Я вернулся на Князевский переулок: переезд произвел революцию в моем отношении к учебе, меня перевели в школу №152 (теперь это позади станции метро "Аэропорт"). Новая, построенная по типовому проекту, школа со светлыми просторными классами с паровым отоплением, с впервые появившимися удобными школьными партами, со специально оборудованными классами для занятий физикой и химией, с большим физкультурным залом, была новым словом в школьном образовании; большинство школьников еще жило в старых сырых подслеповатых домишках, и когда мы приходили в такую школу, было ощущение, что мы приходили в новый мир.

Еще важнее было то, что классным...

\pageimage{page_158}
\label{158-1}
Руководителем была Клавдия Васильевна — молодая учительница, недавно окончившая педагогический институт; она буквально жила интересами класса, она устраивала дополнительные занятия, на которых оставались и отличники, — настолько было интересно быть с ней. В классе у меня появился первый настоящий друг — Женя Кошелев; он был отличником, я тянулся за ним. Это была середина тридцатых годов; его отец был слушателем Военно-воздушной академии имени Жуковского; Женя сказал, что после окончания академии они уезжают на Дальний Восток; тогда было необыкновенно почетным служить там. Женя обещал мне написать и сообщить свой новый адрес, но ни одного письма я не получил. Приближался 1937 год, поскольку в стране развивалась широкая репрессивная кампания, предполагаю, что его папа пострадал, как и многие другие командиры высоких рангов Красной Армии. В моей памяти 1937 год, помимо сказанного, сохранился ярко.

\pageimage{page_159}
\label{159-1}
ким фактом наглядной агитации: наш переулок выходил на Ленинградское шоссе, а через дорогу — Московский протезный завод. На крыше здания к ноябрьским праздникам установили большой прямоугольный транспарант, ярко освещаемый десятками электролампочек. На транспаранте — две фигуры: Сталин и Ежов, оба в шинелях, застегнутых до верха, на правой руке Ежова рукавица, из которой торчит множество тонких длинных пальцев. Художник тоном цвета сделал так, что именно рукавица привлекала наибольшее внимание. Под фигурами короткий текст с обыгранными словами "вков неживые рукавицы". На фоне темного мрачного осеннего неба и редких слабых фонарей, яркое свечение транспаранта производило гнетущее впечатление. Я невольно представил себе подобное рукопожатие, стало страшно.

На соседнем дворе жила тоже еврейская семья: Немировские в 1931 году бежали с Украины от разразившегося голода, именно бежали, продать что-либо было невозможно.

\pageimage{page_160}
\label{160-1}
они все оставили. Жили они предельно бедно: пятеро детей, работал только отец семейства; старший сын Яня после окончания Московского строительного института им. В.В. Куйбышева стал работать, семья ожила. Второй сын Лева был немного старше меня, он был вдумчивый, серьезный мальчик, мы подружились на всю жизнь (недавно Лева умер в Петах-Тикве). Левин отец Моисей Борисович работал бухгалтером и вечерами дома подрабатывал переплетными работами. Человек он был религиозный, поэтому он никогда не брился, а состригал волосы с лица короткими ножницами. В период ограничений религий в СССР он боялся, что будут утрачены национальные традиции; в кругу знакомых евреев, когда рождался мальчик, он уговаривал молодых родителей соблюсти обряд брит-мила. Будучи потомственным жителем еврейского местечка, Моисей Борисович знал жизнь народа и литературно-увлекательно об этом рассказывал; Лева многое запоминал, здесь в Израиле, он опубликовал это в еженедельнике "Еврейский камертон".

\pageimage{page_161}
\label{161-1}
Яня, параллельно с работой, готовил кандидатскую диссертацию в области технических наук и защитил её. В те годы это было довольно редким явлением, особенно в еврейской среде: старшее поколение только не так давно пересекло черту оседлости.

В жизни Лёвы история страны сыграла иную роль. Он закончил Московский авиационно-технологический институт в самом конце 1940-х годов, когда в СССР разворачивалась большая репрессивная антисемитская кампания, поэтому молодому специалисту, еврею, Льву Моисеевичу Немировскому места в авиационной промышленности не дали; распределительная комиссия института направила его работать в Якутию (г. Пеледуй) на верфь деревянного судостроения; он разобрался в отличиях технологий и начал успешно работать; вскоре он стал уважаемым инженером.

\pageimage{page_162}
\label{162-1}
Тревогу в Москве о положении евреев после сфабрикованного дела врачей помню хорошо; предполагали массовое выселение евреев, в то же время некоторые опровергали такое мнение, но многие вспоминали выселение чеченцев, ингушей, крымских татар и др. Достоверный, неопровержимый ответ на все эти волнения дал мне Лева после своего возвращения из Якутии. В переднюю стали прибывать баржи с грузом колючей проволоки; товарный двор стал быстро наполняться таким грузом; Лева спросил у начальника верфи о назначении проволоки; тот ответил, что это для евреев; если выселение состоится, то у нас, Моисеич, будет возможность найти твоих родственников, взять их сюда и устроить. Четыре года, прожитые в Князевском переулке (1935-1939 годы), были лично для меня единственным осознанным периодом, когда я жил в полноценной, здоровой семье; в это время формировалось мое мировоззрение. Мы очень сблизились с

\pageimage{page_163}
\label{163-1}
семьей дяди Яши, и это прошло через всю нашу жизнь. Дом дяди Яши и тети Берты был своеобразной печкой, возле которой всегда можно было обогреться. Они были отзывчивыми людьми. Вот яркий пример: когда нашему Боречке было четыре годика, сложилось так, что не на кого было оставлять его дома; возить его в детский сад возле типографии, в которой работала Инна, было очень далеко и мучительно для ребенка; тетя Берта и дядя Яша взяли его к себе на всю зиму. После войны они квартирку-кухню поменяли на две комнатки с кухней; в одной из них стояли две кровати - для Шурика и Риты, Боречка посередине спал с каждым из них. Пекельку дядя Яша умел многому придавать шутливые формы, был он остроумен, к ним заходили отдохнуть душой, Яша также любил тех хулиганить, что сказалось и на нашем сыне, мы его получили с набором хулиганских стишков.

\pageimage{page_164}
\label{164-1}
164. Федорино горе и умывальников начальник вытеснили сочинительства моего дорогого дяди. Его юмор, остроумие были с ним настолько органичны, что умер дядя Яша в признанный в мире День розыгрышей и шуток — 1 апреля (1976). Когда мы иной раз подходили к его могиле на Востряковском кладбище, видели его фарфоровый портрет на памятнике, у нас невольно появлялась улыбка; мы всегда вспоминаем дом на Шебашевском переулке. Тетя Берта намного пережила своего мужа, мы были на её могиле в Бостоне. В 1935 году меня отдали в музыкальную школу при моем активном сопротивлении: в те годы это было повальным.

\pageimage{page_165}
\label{165-1}
165 явлением в еврейских семьях; аргументация родительских поколений: при царизме в черте оседлости учить детей музыке было недоступной роскошью, как и с другими формами светского обучения, поэтому родители в советское время хотели видеть в детях то, что было для них недоступно. Конечно, это принесло Советскому Союзу большой успех: вскоре международные музыкальные конкурсы дали такие имена, как Давид Ойстрах, Яков Зак, Миша Фихтенгольц и другие. В этом же ряду, несколько ранее, был вундеркинд Буся Гольдштейн. Музыкальная школа помещалась в старом двухэтажном здании недалеко от станции метро "Аэропорт". Впрочем, мои страдания были недолги: моя учительница Александра Владимировна Шик — интеллигентная, пожилая красивая дама — обнаружила у меня хороший музыкальный слух. Второй скрипичный класс вел Василий Васильевич (фамилию не помню).

\pageimage{page_166}
\label{166-1}
Помню, он же руководил и дирижировал школьным скрипичным оркестром; через какое-то время меня включили в оркестр, чтобы ученики Александры Владимировны не подвели её. Она часто по выходным дням устраивала у себя дома репетиции, к ним я готовился с большим старанием.
Я вспомнил, как дорого стоил складной металлический пюпитр, а также футляр для скрипки, поэтому папа сколотил пюпитр из фанеры, а мама сшила из плотной ткани чехол для инструмента.
В 1939 году я сам ушел из музыкальной школы, потому что нас переселили в далекий район, тяжело заболела мама. После войны я собрал свои ноты и сжег их в печке: не они были виноваты передо мной, а время, которое сделало их ненужными для меня.
За годы "князевского сидения" определились жизненные пути моих братьев, Гриши и Павла.

\pageimage{page_167}
\label{167-1}
дельные части моего изложения. Хочу подчеркнуть, что этот период совпал с последними годами тринадцатой пятилетки (1933-1937), пятилетки больших контрастов, преступлений и в то же время успехов в индустриализации страны. Гриша связал свою судьбу с техникой, и с завода „Орг-металл" перешёл на авиационный завод №1, одновременно он поступил на вечернее отделение Московского авиационного института. Давид — человек общительный, умевший привлекать к себе людей, принял предложение работать во Фрунзенском Райкоме комсомола Москвы.

Клон Григорий Абрамович (03.01.1913-10.05.1974). Мой родной, старший брат. Незадолго до начала Великой Отечественной войны, все годы войны и до выхода на пенсию (с перерывом на арест) возглавлял СКБ (специализированное конструкторское бюро авиационного завода №1). В советское время человек, проработавший столько лет на одной высокой, руководящей должности, справедливо считался классным профессионалом в своём деле. Требования к профессионализму были высоки, на таких руководителях держалась слаженная организация производства на крупных промышленных предприятиях.

\pageimage{page_168}
\label{168-1}
В начале войны завод эвакуировали. Григория Абрамовича оставили в Москве. Когда начала поступать новая техника, группа специалистов занялась ускоренным освоением и монтажом нового оборудования. К заданному сроку фактически новый завод продолжил выпуск боевых самолётов. Когда, спустя годы, предприятие сделали показательным для высокотитулованных гостей, завод получил название и знамя труда.

Григорий Абрамович принадлежал к той части технической интеллигенции и сам был её типичным представителем, для которой интересы работы, бескорыстное отношение к ней входили большой составляющей в жизнь. Эти люди своим умом, знаниями, отношением к долгу и труду подняли страну до уровня промышленно развитых государств, практически эти люди создали военную мощь СССР. Я хорошо помню атмосферу взаимной доброжелательности, добрососедства, интернационализма, которая царила в быту между людьми, несмотря на весьма сложную, а в материальном плане довольно скудную жизнь. Однако после всего сделанного, пережитого запустили термин "безродные космополиты". Термин преподнесли как ясный синоним слову "еврей" в отрицательном смысле. Колесо явного антисемитизма покатилось, к его ободу налипала грязь шовинизма, которая стала разъедать страну. Первоначально думалось, что это месть Сталина евреям за то, что воссозданное еврейское государство Израиль не вошло в орбиту стран советского влияния.

Но в

\pageimage{page_169}
\label{169-1}
На пике оголтелого антисемитизма арестовали моего брата. Пришли за ним ночью, начали с обыска. Утром, когда на кухне заработало не выключаемое на ночь радио (по привычке военных лет), мир впервые услышал о смерти Сталина. К этому часу "улику" нашли: книгу «Михоэлс», советское, открытое издание на русском языке о творчестве актёра, его беседах с видными советскими режиссёрами, артистами и т. п. Оказалось - это националистическая литература.

Наступавшие прогрессивные перемены - оттепель Н.С. Хрущёва не сразу затронула аппарат пожирания граждан своей страны, поэтому моему брату успели, как тогда говорили, припаять 25 лет; его загнали в Магадан. Там тоже ничего к тому времени не изменилось. К пригнанным заключённым вышел начальник лагеря. Он несколько раз важно и медленно прошёлся вдоль линии заключённых и с претензией на глубокомыслие спросил: «Как вы думаете, кто скажет, для чего вас привезли сюда?» Молчание. Пройдясь ещё несколько раз перед заключёнными, сам же громко и угрожающе ответил: «Вас привезли для удобрения Колымского края». Незабываемо содержание фразы, как и время, которое её позволяло.

Реабилитация состоялась, и 18 мая 1956 года мы встречали Гришу на Ярославском вокзале. В Москве приемущему на высоком уровне выразили сожаление по поводу случившегося и, в частности, сказали, что в последующих оформлениях воки

\pageimage{page_170}
\label{170-1}
служебную командировку. Ему вернули партийный билет и предложили прежнюю должность на том же заводе. Он работал в этой должности ещё более семи лет, до выхода на пенсию. Гриша говорил мне, что ему предлагали поработать ещё, но он чувствовал себя выношенным, а работать впожилым не та порода руководителей. И действительно, в качестве пенсионера он прожил немногим более года. Присутствие на прощании с Григорием Абрамовичем большого числа коллег по работе невольно свидетельствовало о его авторитете, об отношении к нему. Гришу угнетали не только жуткая несправедливость всего пережитого, но и судьба его жены. Она оказалась тяжёлой жертвой происходившего. Ольга Ивановна Клоц (Михайлова) была тихой, скромной, малоинициативной, доброй женщиной; когда-то они вместе работали. Для неё понятия "Гриша" и "вина перед государством" были абсолютно несовместимы, как и для всех, кто хорошо знал моего брата. Для Оли арест Гриши оказался совершеннейшей неожиданностью. Её охватило смятение, с ней случилось буйное помешательство, её лечили в психиатрических больницах имени Кащенко П.П., имени Ганушкина П.Б. Освобождение мужа не успокоило её, хотя до последнего дня своей жизни Гриша оберегал Олю от волнений. В приглушённом лекарствами состоянии Оля дожила до конца своих дней; умер и их единственный сын. Не удивительно, что тень мерзкого времени присутствовала в их скромном доме.

\pageimage{page_171}
\label{171-1}
Мой второй родной брат Клоц Давид Абрамович. Наша мама умерла 8 октября 1939 года, ей было всего 47 лет. Семья быстро распалась. Давид уехал служить в кадровую армию, Гриша поселился в заводском жилом доме. Давид Абрамович встретил войну на Кавказе, когда немцы рвались к нефтяным источникам. Он служил в бомбардировочной авиации. Когда на территории СССР сформировалась польская народная армия, их часть поддерживала наступление поляков на родине. После падения фашистской Германии мой брат служил в Китае пять лет, потом на Сахалине (Корсаков), оттуда в Ленинград и вскоре на Украину, в город Винницу. Там в звании подполковника ВВС он вышел на военную пенсию (по возрасту). Вскоре он тяжело заболел, его похоронили с воинскими почестями на Старом кладбище.

В 1951 году, незадолго до отъезда на службу на Сахалин, Давид женился. Мария Яковлевна Клоц (Злотник) тогда была молодым специалистом-врачом-терапевтом, на Сахалине она работала в больнице. По бедности медицинских кадров ей пришлось самостоятельно освоить и другие медицинские профессии. Года через два мы с Гришей встречали ее во Внуково: она летела к родителям рожать. Давид впервые увидел своего сына Шурина примерно года через два.

\pageimage{page_172}
\label{172-1}
Более 10 лет семья Давида живет в Израиле. Всеобщей гордостью является Дашенька — внучка моего брата; она отлично закончила школу, отслужила в Армии обороны Израиля, овладела ивритом и английским, поступила в Технион, учится отлично, в настоящее время заканчивает учебу. В рассказе о Давиде хочу подчеркнуть, что судьба неоднократно связывала его с Польшей и оставила посмертную память об этом: в числе одиннадцати правительственных наград есть два польских ордена и медаль. Годы второй пятилетки характерны были тем, что проявлялась ускоренная индустриализация: открывались производства на новых заводах, фабриках, вступали в строй новые доменные печи, линии железных дорог и т.д. Люди, измученные кровавыми передрягами последнего двадцатипятилетия, верили, что придет, наконец, нормальная жизнь. Начали развиваться отдельные отрасли промышленности, например автомобильная, авиационная. На базе этого рос патриотизм: разумно по-

\pageimage{page_173}
\label{173-1}
лагали, что больше стали — это больше тракторов, комбайнов, хлеба и других нужных производств. Люди добросовестно работали, чтобы быстрее приблизить это время; появились ударники производства, мастера своего дела, которые обучали других. Отлично поставленная пропаганда была в струе этих настроений, звала народ только вперед, не оглядываясь назад. Пропаганде развития отраслей народного хозяйства помогло неожиданное обстоятельство.

Популярность авиации началась с неожиданного факта — спасения «Челюскинцев». Напомню отдельные эпизоды. Новый пароход «Челюскин» (капитан В.И. Воронин) с научной экспедицией на борту (руководитель О.Ю. Шмидт) должен был за одну навигацию пройти из морского порта Мурманск (Баренцево море) вдоль Северного морского пути, затем Берингов пролив, спуститься на юг, в Тихий океан,

\pageimage{page_174}
\label{174-1}
и войти во Владивостокский морской порт. Но 13 февраля 1934 года льды Чукотского моря раздавили корабль. Весь экипаж судна оказался на плавающих ледяных полях. Спасти людей можно было только самолётами, а они до тех пор на дрейфующие льды не садились. Началась бешеная работа по организации спасения. Всё должно было завершиться до появления талых вод на льдинах. Спасли всех. Затем триумфальная встреча в Москве; утверждение высшего почётного звания за выдающиеся заслуги перед государством «Герой Советского Союза». Первые Герои Советского Союза - лётчики, спасшие экипаж «Челюскина», были легендой времени. И не только по самому факту спасения, но и потому ещё, что они подняли планку возможностей новой в мире промышленной отрасли - авиации. Назову Героев, имена которых помнятся мне с детства: Водопьянов Михаил Васильевич, Каманин Николай Петрович, Леваневский Сигизмунд Александрович, Ляпидевский Анатолий Васильевич, Молоков Василий Сергеевич, их успех приоткрыл дверь в арктическое небо. [Глядя на физическую карту мира, поражаешься грандиозности замысла экспедиции. Мелькнула мысль: организовать бы по тому же маршруту туристический круиз в составе каравана торговых судов под проводкой современного атомного ледокола, впечатлений, понимания труда старших поколений было бы не

\pageimage{page_175}
\label{175-1}
В том же 1934 году М.М. Громов установил мировой рекорд дальности полёта на самолёте — 12 тысяч километров. В 1936 году В.П. Чкалов совершил беспосадочный перелёт Москва — остров Удд на Дальнем Востоке, а С.А. Леваневский без посадки перелетел из Лос-Анджелеса в Москву. В 1937 году В.П. Чкалов в экипаже с Г.Ф. Байдуковым и А.В. Беляковым совершил беспосадочный перелёт на одномоторном самолёте «АНТ-25» по маршруту Москва — Северный полюс — Ванкувер (США). Вскоре после В.П. Чкалова М.М. Громов с А.Б. Юмашевым и С.А. Данилиным тоже на «АНТ-25» прошли по маршруту Москва — Северный полюс — США. Мир буквально рукоплескал этим двум экипажам. Чтобы понятнее была степень новизны, которую принесли эти полёты, приведу два факта. Когда в газетах появились фотографии «АНТ-25» с трёхлопастным винтом, это было воспринято как инженерное чудо; с «челюскинского» периода не сходила тема обледенения крыла при полётах в Арктике. Сопоставьте это с возможностями современной авиации, хотя бы гражданской.

Далее. Май 1937 года. В.С. Молоков участвует в воздушной экспедиции на Северный полюс по высадке на дрейфующие льды Северного Ледовитого океана группы полярников под руководством И.Д. Папанина. В 1938 году женский экипаж — М.М. Раскова, П.Д. Осипенко и В.С. Гризодубова — совершил беспосадочный перелёт на двухмоторном самолёте.

\pageimage{page_176}
\label{176-1}
моторном самолёте (бомбардировочного типа) по маршруту Севастополь - Архангельск и Москва - Дальний Восток. Были потери. В.П. Чкалов погиб в 1938 году в тренировочном полёте. С.А. Леваневский с большим экипажем на тяжёлом транспортно-десантном самолёте отправился в беспосадочный перелёт из Москвы через Северный полюс в США. В районе полюса самолёт разбился, экипаж погиб.
Помню предстартовую фотографию, опубликованную в газете: шеренга молодых людей, если не ошибаюсь, человек шесть, - в толстых свитерах, обнявшись за плечи, радостно улыбаются на фоне своего самолёта. После потери радиосвязи с экипажем поползли грязные слухи. То было время невиданных контрастов в жизни страны. На сей раз восторжествовала горькая правда: они погибли на пути к мировому рекорду, который приумножил бы авторитет СССР как восходящей индустриальной державы.
Все эти полёты на рекордные показатели имели многофункциональное значение, потому что авиация как новое военно-экономическое явление становилась визитной карточкой страны. Впрочем, как и в США. Там авиация по существу начала развиваться с середины 20-х годов XX века, но на базе самой передовой индустрии. Многофункциональность советских полётов состояла в том, чтобы, во-первых, испытывать авиатехнику в сложных и даже экстремальных условиях; во-вторых, авиация...

\pageimage{page_177}
\label{177-1}
В результате прорывов советской авиации к вершинам мировых уровней сложилось так, что авиация стала материальным стержнем подъема советского патриотизма, особенно в молодежной среде: открывались гражданские авиаклубы, где осваивали парашютный и планерный спорт; в парках культуры строили парашютные вышки, и молодые люди стояли в очередях, чтобы разок парить в воздухе. Над всем индустриальным обновлением страны витал образ И.В. Сталина; все знали, и это умело пропагандировалось, что ни один кардинальный вопрос не решался без него; во всем видели Сталина, ему верили, его считали всемогущим. Возвеличивание правящей личности никогда в истории России не было редкостью. Сейчас, оглядываясь на историю СССР, можно сказать, что И.В. Сталин начал создавать культ своей личности с того, что фактически перечеркнул Новую экономическую политику В.И. Ленина. О тяжелых последствиях отказа от экономических методов управления народным хозяйством страны вам уже рассказал.
