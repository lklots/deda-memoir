

\label{253-2}
Вспоминая знаковые события, я часто ссылался на папу, поэтому, думаю, представление о нём у вас сложилось. Папу

\pageimage{page_254}
\label{254-1}
254. Мы — Гриша и я, похоронили, как и маму, по еврейскому обряду. Это было возможным и в тяжёлом 1942 году, так как на Востряковском еврейском кладбище с довоенного времени продолжала официально работать еврейская ритуальная служба. Но главное, думаю, в этом то, что Москва прилагала большие усилия для сохранения своего столичного статуса. Уместно сказать о папе. Аврум Бенцион — полное его имя, данное новорожденному в 1880 году в еврейской семье Польши. После Первой мировой войны семья осела, как я вам уже рассказал, в Симбирске. Имя 28-

\pageimage{page_255}
\label{255-1}
рум не сближало человека с окружающей русскоговорящей средой, и он перевёл символическое значение своего имени на многозначимое на русском Абрам. Отчеством стала вторая половина полного имени Аврум бен Циен, Бен — сын, Циен — имя папиного отца от слова Сион. Слово бен (или бат) в подобном словообразовании выполняет функцию грамматической приставки и по-русски получается отчество Бенцинович — правильно, но сложно произносимо. Из близких вариантов папа выбрал отчество Бенцианович.
Итак, в паспорте и документах он значился как Абрам Бенцианович; в быту, чтобы не переспрашивали, папа представлялся как Абрам Борисович. Вся старшая родня по-прежнему называла его Аврум. Чехарда с именами.

\pageimage{page_256}
\label{256-1}
256. Особенно с отчествами, была не только у евреев, но и у граждан других национальностей. Однако вскоре это превратилось в обязательный процесс, так как в конце декабря 1932 года в Советском Союзе началась паспортизация, с тех пор в советских паспортах закрепилась благородная традиция — указывать отчество гражданина. Но своеобразная интернационализация паспортов не исключала противодействий. Вот печальный факт. Вы, конечно, помните об аресте моего старшего брата Гриши. Так вот, в один из будничных вечеров, когда нашего трехлетнего Боречку еще не уложили спать, ко мне пришел элегантный молодой человек, представился сотрудником КГБ, показал удостоверение.

\pageimage{page_257}
\label{257-1}
Ранее и сказал, чтобы завтра, к 10 утра, я был в приёмной КГБ в Лубянском переулке. В небольшой приземистой прихожей было много людей: скорее всего, справлялись об арестованных. От внутренних помещений прихожую отделяла широкая стеклянная дверь, за которой стоял часовой; ровно в 10 часов подошёл вчерашний гость, увидев меня возле двери, жестом руки предложил пройти. Мы пошли на верхний этаж. Я был в большом напряжении, в неменьшем недоумении. Это состояние, видите, диктовало...

\pageimage{page_258}
\label{258-1}
Моему подсознанию желание поскорее увидеть точку цели вызова, чтобы самому предварительно что-то уловить. Мы вошли в просторную, сезон с заметно высоким потолком, комнату: два высоких, узких, стрельчатых как в костёлах, окна, из них вид на церковную колокольню (в том же переулке, всё это подчёркивало высоту и пустоту комнаты), в которой стоял только один большой письменный стол; за столом капитан-следователь, напротив него, на стуле я; за моей спиной медленно прохаживался вызвавший меня сотрудник. Разумеется, я ничего не уловила из окружавшего меня.

\pageimage{page_259}
\label{259-1}
Быстро освободившись от чтения бумаг, капитан-следователь спросил меня, знаю ли я еврейское имя своего брата Гриши. Я ответил, что знаю. "Конечно, назовите еврейское имя Гриши," — назвал: Эршэл. "Кто в семье называл Гришу еврейским именем?" — "В общем-то никто. Я и Давид никогда, но бабушка, которая временами приезжала к нам на выходной день, по-старинке, ласкательно называла Гришу Эршэл." Закончив установлением принадлежности имени, следователь спросил у меня, могу ли я написать на еврейском языке еврейское имя Гриши. Я ответил, что писать и читать на еврейском языке я не умею. Неожиданный был вопрос следователя о состоянии Оли. Я ответил и решил.

\pageimage{page_260}
\label{260-1}
узнать что же о Грише. Мы 260 Тут даём ему белый лебедь, маслом, был ответ следователя, я понял бессмысленность желания узнать подробно о Грише. Часу в новом Пашином паспорте после реабилитации отец значился: Григорий (он же Гершель) Абрамович. Какой смысл был или должен был быть в этом "он же"? И это в условиях сложившейся тенденции на русификацию имен. Впрочем, вернёмся к воспоминаниям о моём друге Льве Моисеевиче Немировском. Однако благодаря Никите Сергеевичу Хрущеву предстояло десятилетие политической оттепели в стране.

\pageimage{page_261}
\label{261-1}
Воспоминания имеют свойство обрастать родственными фактами, а то и приводят к неожиданным открытиям. Мог ли я предположить, что моя публикация о той памятной атаке приведет к тому, что узнаем в нашей Лене прямого потомка (по женской линии) семьи известного философа, писателя Мартина (Мордехая) Бубера (1878-1965). Его единоутробная сестра Софья — прабабушка мамы нашей Лены Марины Владимировны Ивановой (Гринкруг). Какова же цепочка фактов? Свою публикацию я послал Владимиру Владимировичу Иванову — дяде Лены; он тоже инвалид войны.

\pageimage{page_262}
\label{262-1}
Ответ ошеломил. Спустя десятилетия оказалось, что мы одновременно прибыли в Тульское миномётное пулемётное училище, но были в разных воинских подразделениях. Месяца через 3 его подразделение отправили на фронт восполнять потери после Орловской битвы. 13 октября 1943 года пулемётчик Владимир Иванов был ранен в нос. Его старший брат Сергей Владимирович, отец Лены, был солдатом Сталинградской битвы и тоже был ранен в ноги. Участвуя в обсуждении оригинальных эпизодов истории семьи, Лена пересказала слышанное от своей мамы о Победе.

\pageimage{page_263}
\label{263-1}
Перечитав последние странички рукописи, я невольно удивился тому, как далеко стоящие друг от друга во времени и месте действия факты бывают родственны по личностным духовным связям, и само движение жизни ведёт человека, независимо от его сознания, воли, ведёт от одного состояния к другому. Жизни прибавилось, меня к моему 90-летию. Я должен радоваться, но не сопутствующие этому чувства наполняют меня. Ой.

\pageimage{page_264}
\label{264-1}
Скажем так, сговорились поздравить меня к юбилею. И Инну не оставили в покое, с Инной я с 30 декабря 1945 года; мы стали одним человеком по восприятию каждого из нас. Вот, эти гадючки чувства навалились на меня. Почему обострились ощущения недугов, что вызвало их? Когда я решил сам разобраться в перегруженности ощущений, я собрал все свои нервы, в том числе и не железные, и спокойно обдумал все и пришёл к выводу, что я испугался послеюбилейного времени; страх обострил ощущения недугов. К такому же

\pageimage{page_265}
\label{265-1}
обобщающему выводу пришла и интуиция. Здесь я должен сказать, что одна из опор нашего Союза — это совпадение мнений по серьезным вопросам. Страх, испуг, как и вспышка гнева, опасные категории, так как обладают свойством мгновенного влияния. Поэтому принятые под их влиянием решения ошибочны. Чтобы не ошибиться, надо выполнить тяжёлую работу: взять себя в руки, блокировать эти качества. Только тогда можно поразмыслить над ситуацией.

\pageimage{page_266}
\label{266-1}
Понимаю, что не сделать открытия, сказав в схеме построения рассуждения, но твёрдо могу сказать, чем дольше находишься в жизни, тем чаще обращаешься к этой теме. В этом периоде очень важно не потерять себя в хаосе лекарств и недугов, которые складываются в понятие беспомощности. Такая тяжесть душевного состояния поражает не каждого человека, но я не в том числе. Я никогда не выставлял напоказ, как бельё на верёвочке, свои...

\pageimage{page_267}
\label{267-1}
Особенности, вызывающие согласие.

1

Из воспоминаний о госпитале вы уже знаете, как нацелил я себя на тренировку левой руки, чтобы как-то восполнить потерю правой. За годы я достиг многого совершенства.

Я всегда был опрятен, подтянут, самодисциплинирован. Это придавало мне силы. Я могу многое рассказать из жизни без руки, но не надо. Хотя к своему юбилею я пришел в состоянии старческой беспомощности, но, кажется, при сохранившейся способности думать.