\label{285-2}
Страдания душевные и физические часто уводят в мир размышлений с грустным выводом о неотвратимости пути. Это ещё больше ухудшает состояние носителя недугов и отнимает больше сил у обслуживающих. Не лучше ли, собрав моральные и физические силы, запрятав в глубины своего сознания тяжело пережитое, уйти в воспоминания о возвышенном, радостном, обнадеживающем, что было в личной жизни? И вправду, почему бы не пожить в мираже воспоминаний, если реальность ясна? Я принимаю своё же предложение и расскажу вам, родные мои дети, о зажегшемся однажды фонаре - о встрече с Инной.

\pageimage{page_286}
\label{286-1}
Боре:  Предыдущий абзац я закончил обещанием рассказать о встрече с Инной. Написал я это в последних числах мая, имея в виду, что продолжу работу после свадьбы Гриши и Рахели (08.06.2015). Ход времени семейных событий совпадал с моими намерениями начать писать о знакомстве и первых встречах с Инной, но вдруг на нашу семью обрушилось огромное горе - умер самый близкий, самый любимый человек - ушла из жизни моя Инуля. Это случилось 15 июля 2015 года. Получилось так, что рассказ о первых шагах нашей жизни я вынужден начать с рассказа о её последнем дне. Не дай Бог никому такого сочетания фактов! Вот моё письмо адрессованное Душе Инны.

\pageimage{page_287}
\label{287-1}
Инуля, родная!
Вспоминается совсем недавнее. Мы стоим, слегка обнявшись, у раскрытого в ночь окна и в который раз восторгаемся любимым тобою пейзажем: яркая голубая Луна на фоне чёрного, бескрайнего неба. Мощь вида завораживает и будит мысль, это предположения, утверждения, отрицания, сомнения, и мы с тобой в минуты духовного насыщения красотой пейзажа внутренне ощущали силу вечности луны и пришли к ненаучному, а эмоциональному выводу: Луна — это корабль Вселенной. В свой безостановочный путь в вечность, он принимает только чистые души.
 Эти недавние, Инуля, 'встречи' с луной оказались для нас с тобой роковыми.

\pageimage{page_288}
\label{288-1}
Ты умерла в одночасье. 15 июня 2015 года было обычным для всей нашей семьи. Мы с тобой были дома, обсуждали разное, в частности, предстоящее твоё девяностолетие. День протекал в обычном блогостном  для нас режиме. После обеда ты, как обычно, помыла посуду, прибрала кухню и с обычным ворчанием по поводу мытья всяких посудин пошла в спальню отдохнуть, почитать. Прошло какое-то время, я услышал, как ты встала, возилась в спальне. Вдруг крик: "Фима, мне плохо!" Одновременно с криком раздался шум от падающего предмета. В те мгновения невозможно было предположить, что упала ты. Я был близко, за стенкой спальни, но пока я проковылял несколько шагов и оказался у дверного проёма, я увидел роковое: ты лежала на каменном полу, головой к кдвери, на правом боку.

\pageimage{page_289}
\label{289-1}
Я стал громко звать тебя — безуспешно. Сразу же позвонил Жене, рассказав о маме. Он связался с медицинскими службами, вскоре приехала реанимация, следом и Женя. Большая комната быстро наполнилась специальной аппаратурой, тебя вынесли из спальни в ту же комнату и начали работать. Вся эта подготовка была такой чёткой и оперативной, что возникла надежда, что всё кончится только большим испугом. Медицинская бригада работала долго, но в итоге пришлось посадить тебя в медицинское кресло, зафиксировать ремнями безопасности и повезти в больницу; Женичка сопровождал тебя в больницу.

Когда тебя провозили мимо меня,наши лица почти сомкнулись
так как я тоже сидел. Твоя голова была опущена, глаза закрыты, белое лицо скорее выражало глубокий сон, чем уходящую жизнь, и в эти последние минуты жизни

\pageimage{page_290}
\label{290-1}
судьба дала тебе возможность предстать перед живыми в своём повседневно элегантном, изящном виде: после вставания ты успела "вылезти из халата" и опять стать притягательной в ярко-жёлтой одежде. Такой ты лежала на 
каменном полу.

О произошедшем с тобой Женя сразу сообщил Боре и держал его на связи. Ирочку Клоц Женя попросил побыть со мной дома до возвращения из больницы. Так и было. Он вернулся часа через два, вплотную подошёл ко мне и сказал: «Мама умерла». Прошли первые минуты скорби, и Женя добавил, что диагноз больницы — остановка сердца. Меня одного на ночь в осиротевшей квартире Женя не оставил: он ночевал у меня.

\pageimage{page_291}
\label{291-1}
Боря, Митя и Лёва прилетели на следующий день. Накануне похорон, которые были назначены на полдень 17 июня 2015 г., на похороны съехалось много родственников, друзей наших и наших сыновей, знакомых. Женя приложил большие усилия, чтобы за одни сутки охватить всех желаемых, подсчитать количество свободных мест в машинах для "безлошадных". Оригинально выразил скорбь по маме Боря: он принес большой букет крупных роз, передал через Женю мне, чтобы я возложил цветы на могилу мамы. Эпизодом с букетом роз Боря напомнил известную в нашей семье историю о том, как я однажды выразил свою любовь к Инне, веру в неё. Инна заканчивала...

\pageimage{page_292}
\label{292-1}
Московский Юридический институт. Шла подготовка к экзамену по курсу ГУБС (Государственное устройство буржуазных стран). Курс вёл проф. Гурвич — человек очень старый, ехидный, но в реальных условиях никогда не ставил двойки, а назначал переэкзаменовку после дополнительных занятий; таким образом он спасал студента от лишения стипендии на целый семестр. Многие боялись этого экзамена, но я был убеждён, что Инна его сдаст. Я хотел выразить свою уверенность ярким, неопровержимым фактом.

\label{292-2}
Инна жила у своей тётушки, Виктории Исаковны, на Маросейке, поэтому я проходил мимо гостиницы "Метрополь", вдоль которой стояли продавцы цветов. Я всегда любовался морем цветов и решил тем же ошеломить Инну. Подчёркиваю, не зная ещё итога экзамена. Я подошёл к женщине, которая начала разворачивать мешковину со свежими цветами.

\pageimage{page_293}
\label{293-1}
Моё предложение купить всю охапку вызвало интерес, торг, в результате которого женщина упаковала всю охапку цветов. Инна из института ещё не пришла, а тётя Витя, увидев такое, ахнула и схватилась обеими ладонями за лицо. Она быстро, по-хозяйски оценила обстановку: пошла на кухню, принесла ведро, поставила в него цветы, и этот необычный букет мы поставили на пол посередине Инниной комнаты. Инна пришла с экзамена с оценкой "отлично", когда мы втроём вошли в её комнату, перед ней предстал ошеломляющее зрелище, она обомлела, потом крепко обняла меня и стала целовать. Вкус тех поцелуев помню и сейчас: Инна сразу поняла смысл такого подношения.

\pageimage{page_294}
\label{294-1}
Вот какие воспоминания пробудил букет роз. Всё это и близкое к этому промчалось за минуты. А перед глазами могила, поглотившая всё это. Слышны последние постукивания лопат, которыми завершилось погребение, потом - вторая молитва. Боря, Женя и я стояли рядом. Женя, завершая эстафету роз, дал цветы мне. Я, ощущая всем содержанием своим наступивший момент разрыва наших жизней, поцеловал розы как живую Инну и возложил цветы. Я хотел сказать много славного об Инне, начал говорить, но слёзы победили слова. Цветов, а также добрых слов, в общениях между провожавшими Инну, было много. Постепенно, не торопясь, с лицами печали, стали собираться к выходу. Так, единой группой, вышли на площадку прощания.

\pageimage{page_295}
\label{295-1}
9.
С этой площадки начинается ритуал похорон, поэтому хочу оставить в памяти нашей семьи запись части похорон Инны. Площадка прощания (так я назвал её для себя) является фактическим центром нового, оригинального кладбища  
"Аяркон". Представьте: огромное, открытое, глубоко затенённое пространство под внушительного вида шатром. Площадка радиофицирована, есть кафедра и микрофон. Здесь собираются группы людей, приехавшие на кладбище. Бывает многолюдно, но всегда тихо: господствует понимание значения этого места. Понимание ощущается и зрительно, когда присутствующие на площадке люди как бы притормаживают себя при звуках начинающейся первой молитвы по умершему.

\pageimage{page_296}
\label{296-1}
Такое уважительное отношение к молитве воспринимается как участие всех присутствующих на площадке в скорбном молчании. В такой прискорбно-печальной обстановке прошла первая часть похорон дорогой Инны. Некоторые уточнения: похороны были назначены на 17.06.15, на 12 часов дня; все близкие приехали заблаговременно к Жене, а потом все поехали на кладбище. Там все собрались на площадке прощания, возле кафедры. Вскоре подошёл рав и сказал, что пора начинать. Он взошёл на кафедру и прекрасным баритоном стал читать молитву: эмоциональные всплески и искренность мольбы за Инну доходили с болью до глубин души.

\pageimage{page_297}
\label{297-1}
Закончив первую молитву на площадке прощания, рав предложил сказать слово от семьи. Боря и Маша поднялись на кафедру. Боря от себя, Жени, семьи говорил с сдерживаемым волнением, всеохватывающе и проникновенно. Маша переводила на иврит.

\pageimage{page_298}
\label{298-1}
Следующий абзац это текст Бориной речи на похоронах Инны:
 15 июня, 2015. Как ни раздумываешь, не ожидаешь смерти, это всегда удар под дых. Дыхание перехватывает, хотя никто из нас еще не осознал реальности происшедшего. Мамы не стало. Ребенок, который остался в каждом из нас, потерял половину мира, и эти фантомные боли на месте разорвавшейся связи остаются навсегда. Мы знаем, что любая жизнь подходит к концу, и каждый из нас думает об этом, и подводит промежуточные итоги. Чем старше, тем чаще. Сегодня для мамы этот итог превращается в окончательный. Какой же он? По-моему, феноменальный. Во-первых, сами обстоятельства смерти. Она встала, позавтракала, помыла посуду, и ее нет. Кто не позавидует такой легкой смерти? Во-вторых, она создала и оставила после себя огромный клан, поразительно дружный во всех, уже многих, поколениях. У нее был папа, муж, который ее буквально обожал и боготворил до последней секунды. У нее была жизнь, которая хотя была очень трудной, но эта ее жизнь все время шла по восходящей линии, а это самое важное. У нее была работа в одном из самых интересных мест на свете. Ее там любили и она любила эту работу. Это была жизнь, которой, несмотря на трудности, можно позавидовать. Но все имеет свой конец. Время делает свое дело. Но вот что удивительно. Деда как раз на этой неделе остановился в мемуарах, которые он пишет, на моменте, как он увидел Инну в первый раз, и как он преподнес ей большой букет красных роз. Так вот, круг замыкается: как их совместная жизнь началась с букета роз, так она и заканчивается большим букетом красных роз. Давайте возложим его на могилу. Это очень тяжелый день, и это начинает еще более тяжелый период для всех нас, а для папы в особенности. Но давайте всегда помнить, какая богатая, необыкновенная, феноменально успешная жизнь закончилась сегодняшней церемонией, когда мы предаем маму земле.

\pageimage{page_299}
\label{299-1}
После похорон Инны начался семидневный траур — шива (слово шева — семь). С утра до позднего вечера была открыта наша квартира, в вестибюле висело типографски выполненное сообщение о смерти Инны. Любой человек, знающий нас, мог зайти и выразить соболезнования. Приходили родственники, друзья сыновей, коллеги по работе, причём коллеги самых высоких должностей. Пришедшие выражали искренние сочувствия, создавалась обстановка тепла, душевной поддержки: никто не спешил уходить, наоборот, группки собравшихся беседовали между собой. Вся эта обстановка как-то уравновешивала ситуацию. Сколь велико значение шивы убедимся на отношении к Жене его коллег. В отдельные дни шивы квартира заполнялась в основном коллегами Жени. Машины и Мишины сотрудники также выразили соболезнования.