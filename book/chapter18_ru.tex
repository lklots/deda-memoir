
\label{277-2}
Мои родные братья Гриша и Давид изначально остро приняли изменения в моей жизни с большим сочувствием и восприятием. Они одобрили мой выбор образования.

\pageimage{page_278}
\label{278-1}
и стали материально поддерживать мою учёбу в институте.
Две проблемы — ежедневные обеды и гарантированный ежемесячный доход (весьма скромный, но гарантированный) — решились довольно легко. Остались проблемы быта, то есть приспособление к жизни с одной рукой; сюда же отношу освоение каллиграфии.
По мере наработки практических навыков менялся ритм моей жизни, её уклад. Вот пример. В квартирах у Гриши и дяди Фимы были ванные комнаты, я мог мыться, но это привязывало меня к месту и условиям; я пытался быть способным мыться капитально при разных обстоятельствах. Быть, как все. Потом, сколько можно ездить мыться в чужой, хотя и родственный, дом. Эта проблема оказалась для меня довольно неловкой: никто не возражает, а неудобно. Варианты размышлений на эту тему привели меня к чуду русского быта — к бане.

\pageimage{page_279}
\label{279-1}
С папой и братьями с детства посещал баню, знал её удобства, некоторые из них обернулись для меня своеобразными препятствиями, которые я пытался преодолеть. Итак, представьте большой моечный зал, уставленный рядами длинных, достаточно широких мраморных скамей. В конце каждой небольшая раковина и два крана, обращённые во внешнюю сторону. Посетитель наливает в таз воду, берёт его за две ручки, обходит угол скамьи и идёт к своему месту. В последней фразе выражена моя проблема: как это, приспособленное для двух рук, ранее делавшееся мною, повторить мне сейчас? Ответ я искал и нашёл. Он оказался несложным для выполнения, но чрезвычайно важным для моего собственного самоощущения в жизни.

\pageimage{page_280}
\label{280-1}
Ну, а теперь в баню, проверить свой метод взятия таза». В банный зал я вошёл в приподнятом настроении с надеждой на успех. Жара, плохая видимость из-за обилия пара; вот деревянная лесенка вверх, ближе к потолку, где крепче пар. Здесь можно видеть голых мужиков в фетровых шляпах; любители паровой бани мочили свои шляпы в холодной воде, отжимали на голову: своеобразная гарантия от теплового удара. Если с человеком что-либо случалось на скамьях под потолком, то причиной могло быть всё, но не пар. В моечном зале я быстро нашёл свободное место, закрепил его за собой, положив на него кусок мыла и мочалку, и пошёл к кранам, к кульминации моего замысла. Обычный банный таз с двумя крепкими ручками я поставил на раковину, наполнил его водой.

\pageimage{page_281}
\label{281-1}
наконец, надо брать таз. 281. заранее продумав действия, я боком, левым, встал к раковине, присел. Настолько, чтобы верхний край таза и лись на мокрую часть живота - ручка приходила, прижал это покрепче к телу, обхватил рукой округлость таза, и ещё крепче прижимая таз к телу, поднял таз самостоятельно; потом пару неуверенных шагов в сторону. ещё шажок другой, и я огибаю заветный угол с комби; осторожно, чтобы таз не выскользнул, я дотянул до своего места, подобными уроками я добивался физической независимости.

Не только этим памятна мне баня. Её назначение, всё устройство, как бы одухотворившись, дали мне возможность уберечь руку от частых физических перегрузок, когда надо было купать сыночков в домашних условиях.

\pageimage{page_282}
\label{282-1}
Небольшим штрихом покажу вам ту действительность. Чтобы искупать в домашних условиях Боречку и Женечку, надо было принести не менее трёх вёдер воды и нагреть её. По мере взросления детей я брал их с собой в баню, но ходили мы только по выходным дням. Однако и это было для меня большим облегчением. Далее, много воды уходило на ежедневные хозяйственные и личные нужды. И всю эту воду носил я. Инулю к ношению воды я не допускал. У неё было полно хлопот по дому, но главное заключалось в том, что я старался сберечь красоту её рук, её пальцев, в меру удлинённых и наполненных изяществом.

\pageimage{page_283}
\label{283-1}
К сожалению, сказанным не могу закончить благодарственные воспоминания о бане, они подгажены тоже незабываемым фактом. Были, очевидно, сохранился давний обычай, когда моющийся человек намыливал мочалку и просил кого-либо крепенько потереть спину. Это было столь обычным, что даже вертолетый иконкий человек, окажись в бане, не чванились бы. Тоже сделал я: намылил мочалку и обратился к соседу по скамье; он выпрямился, развернулся анфасом в мою сторону и тихо сказал: "Авон видишь, человек моется, он тоже еврей, иди к нему, он тебя потрёт." Я не нашёлся с ответом, какая-то горечь заставляла меня непрерывно глотать слюну. То было в самом начале пятидесятых годов. В ноябре 1960 года руководство издательства "Колос" выдало мне ордер на...

\pageimage{page_284}
\label{284-1}
двухкомнатную квартиру в современном доме. Мы ожили. Баня ушла в прошлое, у нас изменился порядок жизни выходного дня: стало больше времени для интеллектуального воспитания детей. На отдельных фактах я показал вам, как преодолевались последствия ранения, — радость, которую я испытывал в случае успеха восстановления функций, глубокой стойкости и багажа навыков, начавших расти естественно, пополняться новыми. Недуги, неудобства и прочие болезненные для души и здоровья особенности угнетают носителя их и отнимают больше сил.

\pageimage{page_285}
\label{285-1}
у того, кто ухаживает.
