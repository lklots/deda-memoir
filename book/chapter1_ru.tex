\pageimage{page_001}
\label{1-1}
Течение лет вывело меня на ту ступень жизни, когда человек может сказать о себе, что он самый старший по возрасту и единственный ещё живой, кто хорошо знал многих представителей когда-то большого рода. К сожалению, я знаю не всё и не всех. Мне стыдно за то, что не знаю даже имени своей бабушки — мамы моего папы. О подобных явлениях — дальше. Здесь, в вводной части, считаю нужным отметить, что написал только о том, что помню сам и о слышанном в семье; оценки некоторым общественно-политическим событиям в понимании времени, к которому они относились, а отдельные подробности для лучшего представления о ситуациях.

\label{1-2}
Воспоминания написал по многим причинам. Цепочки жизни большинства еврейских семей разорваны вынужденной диаспорой, многие звенья цепочки безвозвратно утеряны, поэтому надо передавать потомкам ещё известное; мои, к счастью, этим интересуются.Я давно хотел заняться семейной тематикой, но текущие работы никак не кончаются.

\pageimage{page_002}
\label{2-1}
Коррективы в последовательность работ внёс мой самый младший внучок Лёвочка: он проявил большой интерес к истории семьи. Однако приближалась шестидесятая годовщина Победы над фашистской Германией. Мои друзья предложили откликнуться на эту очень важную дату, и я начал воспоминания о семье с себя, написав «Метры пехоты». Замысел темы не нарушен: война неумолимыми стальными скребками прошлась по жизням людей и из нашего рода.
