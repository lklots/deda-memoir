\pageimage{page_211}
\label{211-1}
Занятия в техникуме продолжались всего полтора месяца до 16 октября. Об этом критическом для Москвы, а, возможно, и для всей страны дне, готовившейся к обороне Москвы, я вам рассказал с чувством возродившейся душевной тревоги, которую я испытывал в то время. Передам вам ход моих мыслей, возможно, и путаных, но точных. Когда в тот день, на пути в техникум, я прибыл на площадь станции метро «Сокол», увидел необыкновенное: метрополитен закрыт, тысячи подземных пассажиров одновременно находятся на улице, затененной...

\pageimage{page_212}
\label{212-1}
жёлтыми осенними облаками; Трамваи и троллейбусы обвешаны людьми, спешащими на работу, но транспорт еле ползёт. Серо-черные тона теплых одежд массы передвигающихся фигур резко меняют восприятие родного пейзажа, отчуждают его, сливают с мрачным холодным октябрьским днем. Много народу здесь потому, что «Сокол» — конечная станция одного из радиусов метрополитена, ею пользуется и население примыкающих районов пригорода. Видимо, так и с другими конечными станциями.

\pageimage{page_213}
\label{213-1}
В этой безысходности я как-то растерялся, поэтому мне хотелось услышать совет, мнение солидного, по возрасту, человека, а затем приблизить мнение к моему личному обстоятельству, но кругом суровые, озабоченные лица с крепко сжатыми, мне казалось, губами.

\pageimage{page_214}
\label{214-1}
Несмотря на сумятицу момента, принимаю решение идти в техникум, к Мясницкой площади; это очень далеко пешком, но понадеялся, что около крупных объектов трамваи и троллейбусы разгрузятся. Я думал о том, чего же можно ждать в ближайшие дни, начиная с сегодня; проворачивал в памяти самые последние разговоры в кругу семьи об отъезде из Москвы, о дополнительных испытаниях, которые судьба преподнесла впереди в лице Гитлера, но главным было: какое решение принять в складывающейся ситуации. Гришка ясно сказал, что его с группой специалистов оставляют на заводе.

\pageimage{page_215}
\label{215-1}
их эвакуация не 215. предполагается, но если обстановка изменится, то он сможет взять с собой папу и меня. Дядя Яша сказал, что он выжидать ничего не будет, он не хочет рисковать жизнями тети Берты и детей (Шурику было 8 лет, Ритуле 9 месяцев), они поедут на полуторке до гор, а дальше - как сложится (полуторка - грузовичок подъемностью 1,5 т); для меня и папы дядя Яша зарезервировал два места на этом грузовичке. Папа не соблазнился вниманием дяди Яши, но искренне поблагодарил его за заботу. Отказ ехать папа мотивировал рядом причин: он помнит многолетние мытарства всей семьи после изгнания из Брест-Литовска в 1915 году; с опухшими ногами и ослабевшим за

\pageimage{page_216}
\label{216-1}
месяцы войны здо 216. робем он в открытом кузове до Горького не доедет (папа, действительно, очень слаб); по острейшему вопросу — надо ли уезжать из прифронтовой Москвы потому, что мы евреи, папа придерживался оригинальной, рискованной точки зрения, которая схематично звучала так: именно из Москвы, в отличие от другого прифронтового города, нам уезжать не надо, так как Сталин Москву не отдаст без большой войны за город, иначе это будет капитуляция. оба в те дни вопрос о судьбе Москвы на своём месте в этой судьбе стоял остро.

\pageimage{page_217}
\label{217-1}
к Чтобы стать частицей защитной силы, нужно быть лично независимым. А я в свои 17 всё ещё на положении ведомого в обычной жизни. Может, это и не плохо, но... Война создаёт экстремальные ситуации, поэтому надо готовить себя к ним, то есть выработать четкую линию собственного поведения независимо от логического расклада мысли другого человека. Это требование к себе я выработал и закрепил на анализе Гришиного высказывания.

\pageimage{page_218}
\label{218-1}
219. Вопреки Гришиному варианту, ситуация резко ухудшится, и будет принято решение срочно завершить эвакуацию ЦТР (инженерно-технических работников), а через короткое время Гриша с коллегами будет в заводском самолёте лететь на восток. Что будет с нами? Поскольку такое вполне было возможным, я разработал линию нашего поведения: я присоединяюсь к защитникам Москвы, а папа, как и другие старые люди, перейдёт под опеку местных властей. С такими заготовками я, наконец, с большим опозданием добрался до техникума. Впрочем, не опоздал: занятия отменены, техникум готовится к эвакуации.

\pageimage{page_219}
\label{219-1}
Секретарь директора техникума предложила записаться на эвакуацию. Я отказался, тогда она посоветовала, как она сказала, «на всякий случай» взять справку о том, что я студент первого курса Московского авиационного техникума. В непринужденной беседе я изложил ей доводы, побудившие меня и отца воздержаться от эвакуации. Мне было очень приятно с ней говорить, её совет как-то расположил меня к ней, и мне хотелось поделиться мыслями ещё до возвращения домой.

\pageimage{page_221}
\label{221-1}
Второй
К концу дня метрополитен начал работать, иначе я бы запомнил дополнительные трудности, которые были бы в связи с моим возвращением домой из техникума. Восстановление работы такого важного стратегического объекта, как метрополитен, означало, что военная обстановка под Москвой как-то стабилизировалась. Однако для меня тот день закончился ещё одним тревожным эпизодом! Я сошёл с трамвая у круга 277 маршрута, то есть напротив нашей «Булочной кондитерской», и увидел непривычно широко раскрытые её двери, только входили и выходили редкие люди. Из персонала — ни души. Витрины прилавков пусты, только валяются марлевые занавески, полки для хлеба пусты, подсобное помещение открыто, и там ни на полках, ни на лотках ни корки хлеба.

\pageimage{page_222}
\label{222-1}
Что же произошло? Утром 1600 прокатилась волна по Мини, перезнал булочной разбежался, а жильцы ближайших домов растащили хлеб. Ночью прошла на редкость тихо, но мы были начеку. Скоро драму прислушивались к тихому шипящему звуку, доносившемуся из радиотарелки. Утром ничего не сказав папе, я пошёл к булочной и... обалдел: входные двери заперты, сквозь стёкла разглядел уборщицу и человека в форменном синем халате. Это работник организации быта москвичей. Быстро наладилась быт, с учётом, конечно, особенностей военного времени, и подобное рассказанному никогда не повторилось. Между тем, эвакуация продолжалась, но в успокоенных домах. Особенно

\pageimage{page_223}
\label{223-1}
ности состояли в том, что был комендантский час, в то же время были открыты большие бомбоубежища, станции метрополитена ночью работали в режиме бомбоубежищ. Порой окрест местности патрулировались; не исключалось, что несмотря на тщательно организованную противо-воздушную оборону, о которой я уже говорил, враг попытается, в частности, десантировать небольшие группы диверсантов, все знали об этой опасности и были бдительны.
