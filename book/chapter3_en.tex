\label{10-2}
  Moscow did not appear to me as family or a childhood home — I was 19 years old, and by that time, my parents had passed away. Moscow appeared at that moment as a cozy corner of Pushkin Square, where the monument to the great poet stood (in its original place), along with a plaster statue of a ballerina on the roof of a then-new corner building on Gorky Street.

\label{10-3}
Confusion, shock were replaced by the restoration of consciousness, although there was a loud noise in my head and pounding in my temples, clots of blood pressed on my throat, I continuously swallowed them. However, the fear of death was replaced by the desire to survive. I carefully turned on my left shoulder, leaned on the left part of my back, pulled my right hand with my left hand, then dragged the field bag and wrapped the torn hand with the strap; in search of some relief from the pain, I tried to remove the burning weight from my face, but my fingers touched my tongue. I decided to get out on my own. I didn't have time to...

\pageimage{page_011}
\label{11-1}
my actions, when seconds later I saw a nurse next to me. Kneeling, bending over without a word, she began to bandage my face, then twice tried to apply a tourniquet to my right forearm, the tourniquet did not hold. "We'll have to use a tight bandage," she said quietly, neither to me nor to herself, and opened the medical bag again. While the nurse was bandaging, I tried to clearly say the word "drink." She, of course, understood everything without my mumbling, and when she finished bandaging, she lifted two flasks, shook them slightly, and quietly, with great sympathy, said: "The wounded drank. Hold on, dear, we'll take you soon." Still staying low to the ground, the nurse crawled away. The tight bandages muffled the sensations of sharp pain, but the thirst tormented me: it seemed that there was a stinking, overripe footcloth in my mouth, which I was forced to suck... Half-lying on my back, I pulled the field bag with the strapped hand higher to my chest, hooked the strap of the submachine gun, and, leaning on my left elbow, began to crawl towards the forest...

\pageimage{page_012}
\label{12-1}
Maybe if I had drunk water, I would have stayed to wait for help. The nearby forest already seemed far away, but I crawled. I remember myself already in the medical company: it means the nurse returned, they carried me out. In the medical company, they filled me with the unforgettable charm of raw, cool water, gave the mandatory tetanus shot for all the wounded. Then - the operating room of the medical battalion.

\label{12-2}
- The medical company is the nearest medical point to the combat actions; it was located at the front positions; its function was to provide emergency assistance to the wounded and send them to the medical battalion.

\label{12-3}
The medical battalion is a full-fledged hospital complex deployed in combat conditions, becoming stationary as the army advanced; it was located deep in the front line at the level of auxiliary units of divisions and the army. From the medical battalion, operated-on severely wounded soldiers were sent through a chain of evacuation hospitals (EH) to the nearest restored railway, where hospital trains arrived. They transported the wounded to hospitals across the country; in the EH, the wounded stayed for 2-4 days, depending on the treatment process and their physical condition. The wounded were also transported by planes.

\pageimage{page_013}
\label{13-1}
It was night when I woke up from anesthesia. I saw myself in a large army tent. A dim light bulb glowed on the central support pole. A nurse sat at a small bedside table. Many cots, all with wounded. The flap of the tent entrance was thrown back, the smell of the forest wafted in. A special silence of the night. Not a shot, as if there was no war. My jaw was bandaged, covered with a sheet up to my neck. My arm?! I sharply pulled off the sheet. Wide bandages wrapped around my chest, and my right shoulder was wrapped in a thick bandage. "That's it. There will be no arm," I coldly said to myself and understood my new future. The nurse heard some movement, approached, covered me with the sheet again, said something encouraging, but I was no longer listening, as in the morning. It turned out I didn't know everything about that day - there was a third bullet. In the city of Ivanovo, where I was lying in a maxillofacial hospital, I first took my field bag after the injury and gasped: in the corner of its left edge was a bullet entry (simply put, a small hole), and at the exit, in the back thickened wall - a large cut, like with a knife; at the bottom of the bag lay a torn brass bullet casing and...

\pageimage{page_014}
\label{14-1}
a flattened piece of lead, that's how an explosive bullet tears. To this day, I keep these items, which have become relics of the Great Patriotic War for me. The roads to victory over German Nazism were steep. Listening to the pure sounds of fanfares for the victors, let us honor those whose youth was forever frozen in that heavy war. I was awarded two Orders of the Patriotic War, first and second class, as well as the medal "For Victory over Germany" and commemorative jubilee medals of the USSR, RF, and Israel.

\label{14-2}
Abbreviated published in the weekly "Kstati" (California) June 1-7, 2006. No. 585, p. 35. http://www.kstati.net
