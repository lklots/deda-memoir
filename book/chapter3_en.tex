\label{15-2}
Among the dead was my friend - Zaslavsky Yakov Mikhailovich, born in 1923, a Muscovite, lived on Sretenka, Khmeleva St., 14. We met long before the war in a pioneer camp. Yasha graduated (if I'm not mistaken, from the Samarkand) tank school. He was wounded, after the hospital he returned to the front, but not as a tanker. As a senior lieutenant, on the 2nd Baltic Front, on July 25, 1944, that is, the day after my injury, Yasha was seriously wounded in the stomach, and on the same day he died.

\label{15-3}
After the war, Yasha's front-line comrade sent his parents a field map on which Yasha's grave was precisely marked. With the assistance of military commissariats, the parents were allowed to reburial their only son in the Vostryakovskoye Jewish cemetery near Moscow. Yasha's grave is in the right corner from the main gate. A selection of his letters from the front was published in the magazine "Znamya". I pay tribute to my friend's cousin - Lyusya Zimonenko, who did a lot to preserve the memory of her deceased brother.

\pageimage{page_016}
\label{16-1}
In the same grave rests his mother — Milda Yakovlevna. A memory associated with Yasha is about the last pre-war pioneer summer, in 1940. You could always stay out of town for two shifts, and parents were freed from the worries of children. The level of interest in the camp was determined by the qualifications of school teachers who led club activities. The clubs included: literary, botanical, handicrafts, military-oriented — model aircraft, geographical. In the evening, when it was already getting dark, a campfire was lit. The children, mostly girls, read — some from memory, some from books — poems, short stories by Russian-Soviet classics. Then the club leader helped to understand the true content of what was read, to find the semantic emphasis. Such optional theoretical lessons were easily remembered and helped us get good grades in literature at school. In the pre-war years, suburban electric trains on the Kiev railway did not yet run. Only on the section from the Northern (Yaroslavsky) station to the city of Alexandrov did the trains run on electric locomotives. Those who came back told with delight about the different impressions of the ride. Well, if such a thing happened to familiar guys, you became an envious witness to the flow of boasting; the guys felt like heroes, as if they were those very electric locomotives.

\pageimage{page_017}
\label{17-1}
So, in the summer of 1940, we traveled to the Sukovo station on the Kiev railway, where the pioneer camp was located, in old, rattling, swaying side-to-side carriages. The train was also pulled by an old steam locomotive adapted for suburban traffic. But it was polished to a sparkling shine, which added some charm to the journey. The steam locomotive whistled intermittently, released steam with noise; the wheels monotonously and frequently clattered at the rail joints; coal soot from the steam locomotive's chimney flew into the open windows of the carriages; when braking, the carriages collided with a clang and force, so the children who stood unsteadily or sat on the edges of the seats could fall. The travel conditions amused us, we compared them with the relatively recently (summer of 1935) opened first line of the Moscow Metro named after L.M. Kaganovich ("Sokolniki — Gorky Park"). Despite the puffing of the steam locomotive, the landscape slowly floated past the train windows. Sukovo seemed an incredible distance from Moscow, and few people came on parents' days, even to the younger students. The remoteness united everyone. We easily accepted the leaders' suggestions. For example, the botany teacher suggested joining the collection.

\pageimage{page_018}
\label{18-1}
and preparation of herbariums for middle school students. Yasha photographed me doing this: with a bouquet of wildflowers in my right hand (this turned out to be her last photo). We went on azimuth hikes with great interest. The geographer gave two groups of pioneers diagrams of different routes to a common meeting point. After training on the camp grounds, we often went on exciting day-long hikes. After breakfast, we received dry rations, the geographer joined the group, which was mostly made up of newcomers, and off we went. The attitude towards long hikes was serious for both boys and girls. Even in cases when we went to familiar terrain, we still compared the given route with our movement, there was a special navigator's farce in this. We converged at the meeting point by the end of the day. Those who arrived first took care of the campfire. By the fire, we finished our food supplies, discussed the highlights of the routes, asked questions, and listened attentively to the club leader, who usually added some story to the direct answer. It was fun and relaxed, but the culmination of everything was... potatoes. We placed them on the coals, at the edge of the fire: at the limit of patience...

\pageimage{page_019}
\label{19-1}
Passing the hot potato from hand to hand, burning ourselves, we devoured the queen of the hike with pleasure. The routes were completed, the smoldering campfire was extinguished under the watchful eye of the leader, backpacks on our backs, and home to the Pioneer camp, where dinner awaited. If we returned very late, and the Big Dipper and the North Star shone in the sky, they were our guiding stars on some parts of the way. The geographer also told interesting stories about countries where these stars are poorly visible, and about navigation at sea. Filled with impressions of the big day, completely exhausted, we fell asleep soundly. In the worst nightmare, no one could have dreamed that some of the older guys would be trying to escape from the German encirclement using the skills of pioneer hikes in just over a year.

\pageimage{page_020}
\label{20-1}
But we are still in the pioneer summer of the difficult year 1940. Besides school, Yasha attended a painting studio on Sretenka. He was most interested in portraiture. Unfortunately, I did not keep his drawing; it would be interesting to see how psychological his portraits were. I still remember the characteristic expressions of the eyes in some images of our mutual acquaintances. Drawing never attracted me. The only time, in elementary school, I drew my cat with all my effort, and even then my mother had to add fluffiness to the animal. Clearly, Yasha drew a lot; I attended model aircraft building. My choice was not accidental. Even now, when I see a flying airplane, I follow it with my eyes until it merges with the depth of the sky. Of course, aviation romance has been killed in air battles, by terrorists. But I was always fascinated, and I easily understood the physical laws that are combined in the word airplane, so school physics was my favorite subject. In addition, my older brother Grisha worked at an aircraft factory and studied in the evening department of the Moscow Aviation Institute; we had fascinating conversations. After the signing of the Non-Aggression Pact between Nazi Germany and the USSR (August 1939), Grisha sometimes brought me well-illustrated books.

\pageimage{page_021}
\label{21-1}
A technical journal of German aviation industry. At that time, I first saw the Junkers and Messerschmitts, and learned some of their technical characteristics. The Germans apparently did not value the Soviet aviation and its defense, if they sent a journal from which specialists could understand a lot, or maybe it was just a threat.

\label{21-2}
An interesting pioneer summer was not detached from big events in the country and the world, but was a flood of them. By the summer of 1940, the German army, bypassing the French defensive line of Maigno from Belgium, invaded France without any obstacles and on June 14, without a fight, entered Paris. In the same month, the Soviet Union joined Estonia, Latvia, Lithuania, Bessarabia.

\label{21-3}
In the youth scientific-popular publications there was a lot of interesting, exciting information from different areas of knowledge. I remember reading an article in the magazine "Science and Life" about the reliability of the Maigno line. This was when the Soviet Union was conducting powerful, exposing fascism propaganda. In such articles, it was subtly said about the military might of our potential solid allies in the upcoming struggle with fascism. But such potential - France was put on the feet of the Germans from the first capture. We, schoolchildren, also had our milestones of definitions of changes: our bellies were bursting with laughter on the

\pageimage{page_022}
\label{22-1}
We watched Charlie Chaplin's films "New Times" and "Lights of the Big City", and we already looked forward to the fun on the announced Charlie Chaplin film "Dictator"; suddenly, the film about it was silent, there were no anti-fascist content cartoons by Bor. Efimova, Kukrynikovs (abbreviation of the surnames of three famous Soviet artists Kupriyanov M.V., Krylov P.N., Sokolov N.A.), there were no other anti-fascist propaganda materials. But soon in the central newspapers there appeared a photograph of Joachim Ribbentrop, the foreign minister of Nazi Germany, arriving to sign the Non-Aggression Pact, and indeed, new times had come.