\label{10-2}
  Москва не представлялась ни родственниками, ни родным домом — мне было 19 лет, и мои родители к тому времени умерли. Москва представилась тогда уютным уголком Пушкинской площади, где стояли памятник великому поэту (на исконном месте) и гипсовая статуя балерины на крыше нового углового дома на улице Горького.

\label{10-3}
Растерянность, шок сменились восстановлением сознания, хотя в голове сильно шумело и стучало в висках, сгустки крови давили горло, я беспрерывно заглатывал их. Однако страх смерти сменился стремлением выжить. Я осторожно развернулся на левом плече, опёрся на левую часть спины, левой рукой подтянул правую руку, потом протащил полевую сумку и ремнём обмотал оторванную руку; в поиске некоторого облегчения боли я попытался снять с лица жгущую тяжесть, но пальцы коснулись языка. Я решил выбираться своими силами. Не успел поеду.

\pageimage{page_011}
\label{11-1}
свои действия, как через секунды я увидел возле себя медицинскую сестру. С колен, нагнувшись без слов, она начала бинтовать моё лицо, потом дважды пыталась наложить жгут на правое предплечье, жгут не держался. «Придётся тугую повязку», — сказала она тихо ни мне, ни себе и опять открыла санитарную сумку. Пока сестра бинтовала, я пытался внятно произнести слово «пить». Она, конечно, всё понимала и без моего мычания, и когда кончила бинтовать, приподняла две фляги, слегка потрясла ими и тихо, с большим сочувствием сказала: «Раненые выпили. Потерпи, родной, скоро заберём тебя». По-прежнему оставаясь низко пригнувшейся к земле, сестра отползла. Тугие повязки приглушили ощущения острой боли, а жажда мучала: казалось, что во рту у меня провонявшая, перепревшая портянка, которую я вынужден отсасывать... Полулёжа на спине, я подтянул повыше, к груди, полевую сумку с примотанной рукой, зацепил ремень автомата и, опираясь на левый локоть, стал ползти в сторону лесочка воз-

\pageimage{page_012}
\label{12-1}
Можно, если бы я попил воды, остался бы ждать помощи. Близкий лесочек уже казался далёким, однако я дополз. Помню себя уже в санроте: значит, сестра вернулась, меня вынесли. В санроте наполнили незабываемой прелести сырой, прохладной водой, сделали обязательный для всех раненых противостолбнячный укол. Потом - операционная медсанбата.

\label{12-2}
- Санитарная рота - ближайший к боевым действиям медицинский пункт; располагалась на передовых позициях; её функция - неотложная помощь раненому и отправка его в медсанбат.

\label{12-3}
Медицинско-санитарный батальон - развёрнутый в боевых условиях полноценный, по мере продвижения армии стационарный больничный комплекс; располагался в глубине фронтовой полосы на уровне вспомогательных подразделений дивизий, армии. Из медсанбата прооперированных тяжело раненых бойцов отправляли по цепочке эвакогоспиталей (ЭГ) до ближайшей восстановленной железной дороги, куда подходили санитарные поезда. Они развозили раненых по госпиталям страны; в ЭГ раненые находились по 2-4 дня, в зависимости от процесса лечения и их физического состояния. Перевозили раненых и самолётами.

\pageimage{page_013}
\label{13-1}
Была ночь, когда я проснулся от наркоза. Увидел себя в большой армейской палатке. На центральном опорном шесте слабо светилась лампочка. За небольшой тумбочкой сидела медицинская сестра. Много коек, на всех раненые. Полог входа в палатку отброшен, тянет запахом леса. Особая тишина ночи. Ни выстрела, как будто нет войны. Моя челюсть стянута бинтами, по горло накрыт простыней. Рука?! Я резко сдёрнул простыню. Широкие бинты опоясывали грудь, а правое плечо было замотано вкруговую толстой бинтовой повязкой. «Всё. Руки не будет», — холодно сказал я себе и понял свою неновую будущность. Сестра услышала какое-то движение, подошла, заново накрыла простыней, говорила что-то обнадеживающее, но я уже не вслушивался, как утром. Оказалось, я не всё знал тогда об этом дне — была и третья пуля. В городе Иваново, где я лежал в челюстном госпитале, впервые после ранения взял свою полевую сумку и ахнул: в краешке её левого уголка был пулевой вход (попросту говоря, маленькая дырка), а на вылете, в задней утолщённой стенке — большой разрез, как ножом; на донышке сумки лежали разороченная латунная оболочка пули и...

\pageimage{page_014}
\label{14-1}
Рассплющенный кусочек свинца, так рвёт разрывная пуля. По сей день храню эти предметы, ставшие для меня реликвиями Великой Отечественной войны. Дороги к победе над германским нацизмом были крутыми. Слушая чистые звуки фанфар победителям, отдадим честь и тем, чья молодость навечно застыла в той тяжёлой войне. Я награждён двумя орденами Отечественной войны, первой и второй степени, а также медалью «За Победу над Германией» и памятными юбилейными медалями СССР, РФ и Израиля.

\label{14-2}
Сокращённо опубликовано в еженедельнике «Кстати» (Калифорния) 1-7 июня 2006 года. №585, стр. 35. http://www.kstati.net
