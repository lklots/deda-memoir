\label{15-2}
Среди погибших мой друг — Заславский Яков Михайлович, р. 1923 года, москвич, жил на Сретенке, ул. Хмелева, 14. Мы познакомились задолго до войны в пионерском лагере. Яша закончил (если не ошибаюсь, Самаркандское) танковое училище. Был ранен, после госпиталя вернулся на фронт, но не танкистом. В звании старшего лейтенанта, на 2-м Прибалтийском фронте, 25 июля 1944 года, то есть на следующий день после моего ранения, Яшу тяжело ранило в живот, и в тот же день он скончался.

\label{15-3}
После войны фронтовой товарищ Яши прислал его родителям полевую карту, на которой точно помечена могила Яши. При содействии военкоматов родителям разрешили перезахоронить своего единственного сына на подмосковном еврейском кладбище «Востряковское». Яшина могила в правом углу от главных ворот. Подборка его писем с фронта опубликована в журнале «Знамя». Отдаю честь двоюродной сестре моего друга — Люсе Зимоненко, которая многое сделала для сохранения памяти о погибшем брате.

\pageimage{page_016}
\label{16-1}
В той же могиле покоится его мать — Мильда Яковлевна. С Ящей связано воспоминание о последнем довоенном пионерском лете, о 1940 годе. Всегда можно было пробыть за городом две смены, да родители освобождались от забот о детях. Уровень интереса определялся в лагере квалификацией школьных педагогов, которые вели кружковую работу. Кружки были: художественного слова, ботанический, умелые руки, военизированного направления — авиамодельный, географический. Вечером, когда уже смеркалось, разжигали костёр. Ребята, преимущественно девочки, читали — кто наизусть, кто по книге — стихи, небольшие рассказы русско-советских классиков. Потом педагог-руководитель кружка помогал разобраться в подлинном содержании прочитанного, найти смысловые ударения. Такие необязательные теоретизированные уроки легко запоминались и помогали нам в школе получать хорошие отметки по литературе.
В предвоенные годы пригородные электрички по Киевской железной дороге ещё не ходили. Только на участке Северный (Ярославский) вокзал — город Александров поезда тянули электровозы. Приезжавшие с восторгом рассказывали о разности впечатлений от езды. Ну, а если такое выпадало на счастье знакомых ребят, то ты становился завистливым свидетелем потока хвастовства; ребята чувствовали себя героями, будто они и есть те самые электровозы.

\pageimage{page_017}
\label{17-1}
Так что и летом 1940 года до станции Суково по Киевской, где размещался пионерский лагерь, мы ехали в старых, громыхавших, переваливавшихся сбоку на бок вагонах. Состав тащил тоже старый, приспособленный для пригородного движения, паровоз. Зато он был надраен до сверкающего блеска, и это придавало какой-то шарм путешествию. Паровоз прерывисто гудел, с шумом выпускал пар; колёса монотонно и часто стучали на стыках рельс; угольная гарь из трубы паровоза летела в открытые окна вагонов; при торможении вагоны с лязгом и силой сталкивались буферами, поэтому ребята, которые неустойчиво стояли или сидели на краешках сидений, могли упасть. Условия езды веселили нас, мы сравнивали их с относительно недавно (лето 1935 года) открывшейся впервые в СССР линией Московского метрополитена имени Л.М. Кагановича («Сокольники — Парк культуры имени А. М. Горького»). Несмотря на пыхтение паровоза, пейзаж медленно проплывал мимо окон поезда. Суково казалось по отношению к Москве несусветной далью, и на родительские дни мало кто приезжал, даже к младшеклассникам. Отдалённость объединяла всех. Мы с лёгкостью принимали предложения руководителей. Так, педагог по ботанике предложила присоединиться к сбору

\pageimage{page_018}
\label{18-1}
и оформлению гербариев для среднеклассников. За таким занятием меня сфотографировал Яша: с букетиком полевых цветов в правой руке (это оказалось её последним фото). С большим интересом мы ходили в походы по азимуту. Ту двум группам пионеров географ давал схемы разных маршрутов движения к общему сборному пункту. После тренировок на территории лагеря мы частенько уходили на весь день в увлекательные походы. После завтрака получали сухие пайки, географ присоединялся к группе, в которой преобладали новички, и в путь. Отношение к длительным походам было по-мальчишески серьёзное, у девочек тоже. Даже в тех случаях, когда выходили на знакомую местность, всё равно сверяли заданное в схеме с нашим движением, в этом был особый штурманский фарс. На сборный пункт сходились к концу дня. Пришедшие первыми занимались костром. У огня доедали продзапасы, обсуждали примечательности маршрутов, задавали вопросы, внимательно слушали руководителя кружка, который к прямому ответу обычно добавлял какую-нибудь байку. Было весело, непринужденно, но кульминацией всему была... картошка. Мы клали её на угольки, с краю костра: на пределе терпения вынимали

\pageimage{page_019}
\label{19-1}
и перебрасывали печенную картофелину с руки на руку, обжигаясь, с наслаждением пожирали королеву похода. Маршруты завершены, недогоревший костёр залит под пристальным оком руководителя, вещмешочки - за спины, и домой, в пионерский лагерь, где ждёт ужин. Если возвращались очень поздно, и на небе светились Большая Медведица и Полярная звезда, то на каких-то участках пути они были нашими путеводными звёздами. Географ интересно рассказывал также о странах, где эти звёзды видны плохо, об ориентировании на море. От насыщенности впечатлениями большого дня, вконец уставшие, мы крепко засыпали. В самом страшном сне никому не могло присниться, что некоторые из старших ребят немногим более чем через год будут пытаться выйти из немецкого окружения, используя навыки пионерских походов.

\pageimage{page_020}
\label{20-1}
Но мы пока в пионерском лете сложного 1940 года. Помимо школы Яша занимался в студии живописи на Сретенке. Больше всего он увлекался портретом. К сожалению, у меня не сохранился его рисунок; было бы интересно посмотреть, насколько психологичны его портреты. Мне ещё помнятся характерные выражения глаз на отдельных изображениях некоторых общих знакомых. Меня рисование никогда не привлекало. Единственный раз, в начальной школе, я со всем старанием нарисовал свою кошку, и то маме пришлось добавить зверю пушистости. Понятно, Яша много рисовал; я ходил на авиамоделирование. Мой выбор не был случайным. Даже сейчас, когда вижу летящий самолёт, провожаю его взглядом до тех пор, пока он не сольётся с глубиной неба. Конечно, авиационная романтика убита в воздушных боях, террористами. Но меня всегда восхищали, и я легко понимал физические законы, которые объединяю словом самолёт, поэтому школьная физика была моим любимым предметом. Кроме того, мой старший брат Гриша работал на авиационном заводе и учился на вечернем отделении Московского авиационного института; у нас бывали увлекательные беседы. После заключения Пакта о ненападении между фашистской Германией и СССР (август 1939 года) Гриша иногда приносил мне хорошо иллюстрированные книги.

\pageimage{page_021}
\label{21-1}
иллюстрированный технический журнал германской авиационной промышленности. Тогда я впервые увидел «Юнкерсы», «Мессершмиты» и узнал некоторые их технические характеристики. Немцы, очевидно, невысоко оценивали советскую авиацию и её защиту, если присылали журнал, из данных которого специалисты могли многое понять, а может быть, это было элементарным запугиванием.

\label{21-2}
Интересное пионерское лето вовсе не отрывало нас от больших событий в стране и мире, а было их навалом. К лету 1940 года войска фашистской Германии через Бельгию, обойдя с севера французскую оборонительную линию Мажино, беспрепятственно вторглись во Францию и 14 июня без боя вошли в Париж. В том же июне Советский Союз присоединил Эстонию, Латвию, Литву, Бессарабию.

\label{21-3}
В молодёжных научно-популярных изданиях было много интересной, увлекательной информации из разных областей знаний. Помнится, в журнале «Наука и жизнь» читал статью о надёжности линии Мажино. Это было, когда Советский Союз вёл мощную, разоблачающую фашизм пропаганду. В подобных статьях ненавязчиво говорилось о военной мощи наших потенциальных солидных союзников по предстоящей борьбе с фашизмом. Но такой потенциал — Францию немцы положили на лопатки с первого захвата. У нас, школьников, были и свои вехи определения перемен: наши животы надрывались от смеха на 

\pageimage{page_022}
\label{22-1}
просмотрах фильмов Чарли Чаплина «Новые времена» и «Огни большого города», мы уже предвкушали веселье на анонсировавшемся фильме Чарли Чаплина «Диктатор»; вдруг об этом фильме замолчали, не стало в газетах антифашистского содержания карикатур Бор. Ефимова, Кукрыниксов (аббревиатура фамилий трёх известных советских художников Куприянова М.В., Крылова П.Н., Соколова Н.А.), не стало и других агитационных антифашистских материалов. Но вскоре в центральных газетах появилась фотография Иоахима Риббентропа, министра иностранных дел фашистской Германии, приехавшего на подписание Пакта о ненападении. И впрямь, наступили новые времена.