%%[[page_054]]
На фронте погиб ещё один мой двоюродный брат Величанский Лев Наумович. Он родился в городе Ульяновске в конце 1925 года (или в начале 1926 года), он был моложе меня значительно, как мне казалось. Старший брат моей мамы, Наум (ноях так называла его вся родня), из Ульяновска перевёз свою семью в Смоленск, а потом в Тамбов, где жил самый младший брат Мотя. Из города Тамбова Лёвочка ушел в армию, он был танкистом; «Погиб в бою», — сказано в книге памяти воинов-евреев, павших в боях с нацизмом. 1941-1945», том 5, стр. 376. Строкой ниже сказано о дяде Си...

Вышло девять томов книги памяти..., предполагается десятый. Сколько же погибло воинов-евреев, если понадобилось издать пять томов, чтобы дойти всего лишь до третьей буквы русского алфавита, с которой начинаются фамилии с буквенного сочетания "Велич"? Оказалось, что в это многотомье не включены...

%%[[page_055]]
Все имена павших воинов-евреев, например, нет сведений о моих братьях Розенбаумах. Данное издание - израильско-российское: Публикации в девяти томах подтверждены Центральным архивом Советской Армии при Министерстве обороны Российской Федерации. По вопросам гибели советских воинов в годы Великой Отечественной войны следует обращаться в этот архив. Лёвиному отцу - Науму Абрамовичу Величанскому к началу Великой Отечественной войны было уже за 50 лет, однако его мобилизовали, служил он во вспомогательных прифронтовых частях; вернулся с войны невредимым. Его старший сын Абрам Наумович Величанский (Котик) был старше меня, служил офицером на Дальнем Востоке, затем в Северной Корее; после демобилизации жил в Москве. Нарицательное имя Котик дала своему первенцу мать; оно прижилось в кругу родни и утратило сущностное значение. Жена Мила пыталась переименовать мужа в Костю, но безуспешно. С Лёвочкой я общался редко и только в довоенные годы: один раз летом ездил с бабушкой в Смоленск, потом дядя Ноях несколько раз приезжал в Москву с сыновьями. В моей памяти Лёвочка остался любознательным, очень подвижным, добродушным взрослым мальчиком.

%%[[page_056]]
В 1931 году наша семья породнилась с большой семьей Хаима и Эстер Нибощиков. Мы жили на Новосущёвской улице, 29, а Нибощики — почти напротив, в старой двухэтажной развалюхе, на втором этаже, куда вела мрачная, не освещавшаяся лестница. (На месте того дома и подобных ему на той стороне улицы ещё до войны построили один из корпусов расширявшегося Московского института железнодорожного транспорта.) С двумя сыновьями и тремя дочерьми Нибощики переехали в Москву из Минска; там остались самые старшие дети — Рафаил и Рахель. Хаим работал наборщиком в типографии какой-то крестьянской газеты; Эстер, будучи профессиональной дамской портнихой, подрабатывала дома. На этой почве моя бабушка познакомилась с ней, и вскоре обе старушки (в моём тогдашнем понимании) стали близкими подружками. Я очень хорошо помню мадам Эстер, потому что у моей бабушки было плохое зрение, она остерегалась...

%%[[page_057]]
57. Тёмной лестницы и часто просила меня проводить её. Нас встречала немного суетливая, радушная, высокая, полноватая женщина в больших круглых очках и с портновским сантиметром на шее. Швейная машина была открыта, лежали какие-то материалы; машина стояла между окном и маленьким буфетиком. Вскоре мадам Эстер, так называли её моя мама и тётя Лия - жена младшего маминого брата Ефима, открывала малюсенькую дверцу буфетика, клала на блюдце несколько карамельных конфеток «Подушечка» и добрым жестом руки угощала меня. Если Люся - самая младшая дочка Нибощиков, моя ровесница, была дома, мы отходили в недалёкий угол небольшой комнаты, о чём-то болтали и слопывали подушечки - роскошь бедноты. (Люся доживает свой век в доме для престарелых; я бы хотел, чтобы ей прочитали всё, что относится к жизни семьи её родителей. Думаю, она подтвердит достоверность моих воспоминаний. Единственное, что уточнил имя Эстер: с десятилетиями оно трансформировалось в моём сознании в имя Эсфирь. Однако это разная транскрипция одного и того же имени одной и той же женщины еврейки, персидской царицы, которой мы обязаны праздником Пурим).

%%[[page_058]]
Итак, старушки решили женить своих детей. Моя бабушка опасалась, что её двадцатидвухлетний сын Янкель — весёлый, работящий парень — попадёт в плохие руки, а мадам Эстер беспокоилась, что у её старшей дочери Берты — элегантной и красивой девушки — нет достойного жениха. Старушки "взвесили" каждого и пришли к согласию. Семьи познакомились поближе, бабушка запустила в ход свою команду: мою маму как средство внимания к слову любимой сестры, тетю Лию — искусного мастера тонкой женской дипломатии. Как рассказывала мама, ребятам долго гулять не дали. И вот, тёплым летним вечером 1931 года я присутствую на настоящей еврейской свадьбе в старой деревянной синагоге в Марьиной Роще. Старушки скроили замечательную семью, с которой мы прошли всю жизнь, о чём непременно расскажу. А пока из послесвадебного периода отмечу факты: 2 мая 1933 года у молодых родился мальчик. К началу войны в Германии Шурику было 8 лет. Соня, младшая сестра Берты, за пару лет до войны вышла замуж, родила мальчика и в первой половине июня 1941 года повезла его в Минск показать родственникам. Берта тоже собирала Шурика в поездку с Соней, но у Берты не было денег на покупку сыну нового костюмчика, и она отменила его поездку.

%%[[page_059]]
покунученика у тинькой родне. Небожчик Рафаил, р. 1905 года - старший брат моей тёти, жил в Минске, был призван в Красную Армию. В 1941 году пропал без вести на фронте Великой Отечественной. О сёстрах Рафаила в частности, я писал в январе 2003 года в Институт мемориал «Яд ва-Шем». Фрагмент письма: Отделу регистрации евреев, погибших от нацизма в годы Второй мировой войны, сообщаю имена моих близких и дальних родственников: Рахель Гольдблат, рождения 1907 года, минчанка. Погибла в Минске 28 июля 1942 года; Соня Гольдблат, рождения 1918 года, москвичка. Погибла в Минске 26 июля 1942 года; её годовалый младенец Сашенька Гольдблат, рождения 1940 года, москвич. Погиб в Минске 28 июля 1942 года. За несколько дней до начала войны Соня приехала в Минск показать своего первенца родителям мужа, сестре и др. родственникам минчанам. Сёстры-однофамилицы потому, что их мужья были между собой родными братьями. Бабушка и дедушка Сашеньки тоже погибли в Минске 28 июля 1942 года... После проверки фактов имена погибших внесли в сайт «Яд ва-Шем»: www.yadvashem.org.il names@yad-vashem.org.il Спустя месяцы после этого звонит мне женщина из Канады. После объяснений трудностей в поиске моего телефона...

%%[[page_060]]
на и выяснения цепочки родственных связей она рассказала, что является дочерью от второго брака бывшего мужа Сони: он вернулся с войны, 17 лет жил один, потом женился. В разговорах новой семьи бытуют воспоминания о Соне, Сашеньке. Сейчас папа очень стар, дочь не хочет тревожить его полученной информацией. Элла сообщила мне свой номер телефона и уточнила: малышу было тогда два годика; родной племянник Сони - Шурик Величанский созвонился с Эмой. На этом ограничить воспоминания о Минской трагедии нельзя, потому что о бесчеловечном завершении её мы узнаём, даже трудно представить, от Хайма Нибощика. Однако сперва несколько слов в память

%%[[page_061]]
на и выяснения цепочки родственных связей она рассказала, что является дочерью от второго брака бывшего мужа Сони: он вернулся с войны, 17 лет жил один, потом женился. В разговорах новой семьи бытуют воспоминания о Соне, Сашеньке. Сейчас папа очень стар, дочь не хочет тревожить его полученной информацией; Элла сообщила мне свой номер телефона и уточнила: малышу было тогда два годика; родной племянник Сони - Шурик Величанский созвонился с Эллой. На этом ограничить воспоминания о Минской трагедии нельзя, потому что о бесчеловечном завершении её мы узнаём, даже трудно представить, от Хайма Нибощика. Однако сперва несколько слов в память

%%[[page_062]]
об этом милейшем старом человеке: Своим тактом, всегда к месту найденным словом, советом, почерпнутым из традиционной еврейской мудрости, ощущением покоя, которое появлялось при разговоре с ним, - всё это сделало Хаима Нибощика безусловно уважаемым человеком всей новой родней. Худой, можно сказать, - отощавший, сгорбленный старик, в неизменно старом тёмно-сером костюме с узелком галстука, завязанным, очевидно, ещё живой Эстер, нёс в себе груз больших утрат. Через несколько лет после войны, когда миграция населения как-то отрегулировалась, он решил поехать в Минск попытаться узнать о гибели близких. Ему удалось разыскать дальнего родственника Лазара Чарного, который

%%[[page_063]]
скать дальнего род 62 служил в партизанском отряде, базировавшемся в пригородах Минска. Рассказ Лазаря дедушка Хаим пересказал своему родному внуку Мурику Велиганскому, когда он стал уже зрелым юношей, а Шурик — мне, 217. Лазарь проникал в еврейское гетто Минска, чтобы организовывать единичные побеги в партизанские отряды. Он встретился с Рахелью и Соней, предложил им попытку побега в партизанский отряд (Рахели было 35 лет, Соне — 24), но при условии, что сына Соня оставит в гетто: плач или крик малыша могут помочь немцам обнаружить отряд. Соня сказала, что не сможет своими руками отдать ребёнка фашистам; Лазарь предложил побег Рахели одной.

%%[[page_064]]
Ответ: в этой беде она Соне мать и останется с нею до конца. Он наступил в виде грузовиков-фургонов. Укрупненные команды нацистов и их пособников насилием загоняли евреев в металлические фургоны, набивали их людьми до отказа, захлопнутые двери ставили на запоры с внешней стороны, включали двигатели грузовиков и душили евреев выхлопным газом, поступавшим в кузова фургонов по шлангам, протянутым от выхлопных труб. Можно полагать, что бывший партизан пожалел старика и не рассказал ему, что облазело людьми, когда они воочию увидели распахнутые перед собой двери в Смерть. Считаю справедливым сперва сказать о погибших, а потом хронологии жизни рода; и здесь,

%%[[page_065]]
не без войн, причем, всякого толка.
Клоц Давид Абрамович (28.8.1915-08.10.1971). Мой родной, старший брат. На факте рождения младенца отразилась история евреев Российской империи. Местом традиционного проживания моих предков был город Брест-Литовск. [город известен по Брестской унии 1596 года - о признании православной церковью Украины и Беларуси своим главой Папы римского, но при сохранении православной обрядности. Уния расторгнута в 1946 году; по Брестскому миру - 1918 год; Брестской крепости = герою - начало Вел. Отеч. войны.]
По тем понятиям город был большой. Весьма значительную часть его составляло польско-литовское еврейство. Мои родители женились в 1911 году и счастливо прожили до конца лета 1914 года, до начала Первой мировой войны.
Ряду неудач на русско-германском фронте предшествовали относительно недавнее поражение России в русско-японской войне 1904/5 года, а в связи с поражением - обострившаяся революционная обстановка в 1905-1907 годах и последующие рецидивные события.
Трудно понять, как можно было после потрясений в Российской Империи за период после 1904 года дать втянуть себя (через короткое время!) в новую, изначально большую, войну; протянуть руку своей гибели. О том времени впечатляюще рассказано в забытом, по существу документальном, романе А. С. Новикова-Прибоя "Цусима". За потрясения,

%%[[page_066]]
политические провалы власти взыскали с евреев в многочисленных погромах. И в неблагоприятной военной обстановке 1914-1915 годов тоже зацепили евреев тем, которые жили или работали в прифронтовой полосе, куда входил и Брест-Литовск, приказано было покинуть места обитания. На сей раз без насилия обошлось.

В придирках к евреям первая причина была величиной постоянной - никем не защищаемая нация; вторая величиной переменной - повод, над поводом власть не задумывалась, его не назвали и в требовании покинуть Брест-Литовск. В еврейской среде предполагали, как рассказывал мой папа, что поводом стал бытовой еврейский язык - идиш, который сродни немецкому. Следовательно, евреи могут передавать немцам сведения о русской армии... (Религиозным языком является древнееврейский, иврит). Среди местных жителей - поляков, литовцев, русских и даже полуоседлых цыган некоторые знали идиш. Большие базары, деловые и личные общения были в те времена своеобразными инязами. Однако выселяли евреев. Предполагалось также, что слух об идише подкинула сама власть.

Наступило время собираться в дорогу. Продать имущество - дом, дело (т.е. бизнес), хозяйство, корову, козу, всякую утварь было почти невозможно: уезжала большая часть города и пригорода; всех настораживал факт войны, некоторые не торопились покупать, понимая, что евреям так или иначе придётся всё бросить. В то же время еврейские семьи начали покупать лошадей, фуры, телеги, большие брички; многие стали учиться у извозчиков обращаться с лошадьми.

%%[[page_067]]
Детьми, упряжью и т.п. В один из августовских дней 1915 года семья Величанских, к которой по рождению принадлежала моя мама — Гита Абрамовна Клон (1892-8.10.1939), тронулась в неведомый путь, на восток, вглубь Российской империи. Плетёные корзины — сундучки (прообразы чемоданов); скатерти, занавеси, простыни, превращённые в узлы с домашним скарбом, положили в подводы. Поверх клали узлы с мягкими вещами, чтобы на них посадить слабых. Таких в большой семье было немало: мой прадед Ефраим (по маминой линии), которому было далеко за 90; старшая мамина сестра Соня (Сура) с маленьким мальчиком и с несколькомесячным младенцем (Шурой Розенбаумом); два самых младших, шестилетних, сына-близнеца моей бабушки — Янкэлэ и Мотя; моя мама — двадцатитрёхлетняя Гита на последнем месяце беременности и с двухлетним сыночком Эршэлэ (Тришей). Дождаться родов в своём брест-литовском доме уже не представлялось возможным.

Мобильной силой семьи были мои дедушка и бабушка, папа; Яков Розенбаум — муж тёти Сони, старший брат моей мамы — Ноях (Наум) и ещё два младших родных брата моей мамы — Шимэн (Семён) и Хаим (Ефим). Для большей безопасности и взаимопомощи в дорогу двинулись большими группами. В них входили семьи родственников, друзей. Вы поняли, что о финале жизни некоторых названных здесь людей сказано на предыдущих страницах.

%%[[page_068]]
купить изразцы помнили составы групп. За многие дни пути выработался ритм движения: в жаркие августовские дни группы двигались с утра до полуденных часов, потом большой привал, где как, а с вечера до ночи — дальше, как позволяли свет луны, дорога, потом опять остановка до утра. Неизвестность не манила беженцев. Они двигались со скоростью лошадиного шага и с длительными остановками. Тлелась надежда, что может быть ещё разрешат вернуться в Брест-Литовск, домой. Под конец одного из ночных переходов у Гиты начались родовые схватки. Хотя в принципе это ожидалось и родные как-то готовились, но в ночи, на краю леса, без света и тёплой воды? К счастью, кто-то из ребят разглядел контуры крестьянских изб. Папа и Ноях помчались в ту сторону. Встретившийся им человек указал избу, в которую надо обратиться; там две женщины выслушали папу и сказали, чтобы скорее везли роженицу. Оказалось, это были повитухи (крестьянки, которые в далёких деревнях выполняли функции акушерок). С роженицей оставили в избе только бабушку. На рассвете 28 августа 1915 года моя мама благополучно родила мальчика. В этой избе они прожили три дня, дальше — опять телеги. На восьмой день после рождения устроили брис, под открытым небом, мальчику дали имя Дувэд (Давид). И представьте, через 30 лет, уже после разгрома нацистской Германии, воинская часть, в которой служил Давид Абрамович, базировалась в Польше, не очень далеко от мест, откуда изгнали его предков, родителей, да, собственно, и его самого. Надежда вернуться в свой Брест-Литовск всем угасла, и семья двигалась всё дальше на вос-

%%[[page_069]]
Ток. На базарах и мелким перекупщикам продавали содержимое корзин, узлов; клади становилось всё меньше, а лошадям — всё легче. Политическая ситуация в стране осложнялась, а положение бездомных беженцев — в ещё большей мере: чем дальше на восток, тем меньше можно было пользоваться польским и литовским языками, а собственно в России оставался слабый русский. В обществе слышались пораженческие настроения, Россия проигрывала Первую мировую войну. Население было взбудоражено и политической борьбой внутри страны. Наступало время двух революций — февральской и октябрьской; Гражданской войны; проигрышной советско-польской войны 1920 года. К Польше отошли (до 1939 года) Западная Украина, Западная Белоруссия, в том числе город Брест-Литовск. Была еще интервенция. Большие потрясения, произошедшие за короткое время на территории бывшей Российской империи, привели к разрухе, голоду, сиротству, гибели людей. В этом переборе осязаемых бед сказалась, как щепка в штормовом море, большая, традиционная, приближённая к патриархально-религиозному укладу жизни, еврейская семья; в тех условиях она острее ощущала значение насильственного изгнания из своего дома, что усиливало тоску по нему, но, разумеется, не по старой власти. Осложнённая малыми детьми, моя родня искала приемлемого пристанища. Выбор пал на небольшой, расположенный на высоком берегу Волги, красивый город Симбирск (Ульяновск).

%%[[page_070]]
Последовательность изложения прерываю потому, что Люся Величанская прочитала предшествовавший текст. С волнением, признательностью пожелала успешного завершения моего замысла и сообщила факт, которому я придаю значение документального свидетельства начала войны. С фронта дядя Сима прислал домой две записки, о которых Рая рассказала Люсе. В последней, уже из Вязьмы, отец писал: «Положение наших ополченцев плохое, у нас одна винтовка на трёх человек, вины. В таких условиях, наверное, не вернусь. Дора, сбереги детей». Записка-предчувствие скорой гибели: ладонью пу-

%%[[page_071]]
ле не противостоишь; записка - прощание мужественного человека не сохранится. Уместно вспомнить глубокий смысл афоризма М. Булгакова: «Рукописи не горят»; значительное сохраняет память, она же воссоздаёт сказанное. Шура Величенский тоже сделал дополнения. После войны Лазарь Сарный остался в Минске, Шура приезжал к нему, а он - в Москву, конец жизни прожил в Израиле. О партизане Лазаре Чарном рассказано в книге «Мстители гетто». Через московских друзей попытаюсь найти эту книгу.