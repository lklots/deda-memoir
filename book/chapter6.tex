\label{071-2}
Вернёмся на несколько десятилетий назад, в город Симбирск. В глухой провинциальный город все потрясе...

\pageimage{page_072}
\label{072-1}
ния в стране доходили +1. в виде ослабленной волны. Это облегчало обустройство в новой, во всех смыслах, жизни. Свободного жилья в городе было много, проблема - в его оплате в условиях обесценивания денег. Впрочем, это обесценивание - одна из причин рынка свободного жилья: владельцы его, фэкспроприированные новым государством их собственности, сдавали или продавали жилье и таким образом справлялись с трудностями. Вся взрослая и юношеская часть родни бралась за любую работу: жили скученно, напряженно. С началом НЭПа все начало меняться, как и во всей стране, в лучшую сторону. Незавершенный план В.И. Ленина «Новая экономическая политика (НЭП)» был уникальным явлением в управ-

\pageimage{page_073}
\label{073-1}
Влении народным хозяйством. Мои ранние детские впечатления об окружающей жизни начались именно в этот период; они закрепились в памяти потому, что мой папа примкнул к НЭПу. В итоге это печально сказалось на судьбе семьи; впрочем, с ней произошло то же, что со всей новой экономической политикой — разгром. Поскольку участие в НЭПе — звено в цепи жизни семьи, и это малоизвестная вам часть истории государства, остановлюсь на связке проблем. К концу 1920 г. - началу 1921 годов как бы определились последствия сложных общественно-политических, межнациональных и череды военных событий, начавшихся всего 6 лет назад, в 1914 году — Первая Мировая война.

\pageimage{page_074}
\label{074-1}
Бойна. События сотрясали бывшую Российскую империю, а затем молодую Советскую Россию; на предыдущих страницах названы эти события; каждое из них явление в тяжёлой истории народа. О многом из этого рассказано в высокой российско-советской литературе 1920-ых-1930-ых годов; широкий доступ к ней читатель впервые получил в годы «оттепели» Н.С. Хрущёва. В частности, имею в виду произведения М.А. Булгакова, К.П. Платонова: отдельно говорю о шедевре киноискусства, о художественном фильме «Бег» и фильме «Белое солнце пустыни». Многие произведения названных писателей переведены на языки иврит и английский. Что же привело В.И. Ленина менее, чем через 4 года...

\pageimage{page_075}
\label{075-1}
Волюции 1917 года к решению использовать буржуазно-капиталистические методы управления экономикой страны? Как и другая новая власть, установившаяся Советская власть делала всё, чтобы удержаться. В тех условиях главным было снять проблему затянувшегося голода, которая сфокусировалась своей остротой к концу Гражданской войны (1918 - 1920 годы). Произошло это по ряду причин; начну в определённой мере с объективно-естественной - засухи. В журнале «Вопросы экономики» (1969 год, № 7; издание Института экономики АН СССР) опубликована моя статья, в которой даю таблицу, составленную по собранным мною данным о засухах, начиная с 1885 года - первого года введе-

\pageimage{page_076}
\label{076-1}
сной службы в России - по 1400 год; привожу часто таблицы: 1905-1909 гг. 1906 г. 1925-1929 гг. нет 1920-1924 гг. 1920 г., 1921 г., 1924 г. 1910-1914 гг. 1911 г., 1914 г. 1930-1934 гг. 1915-1919 гг. 1917 г. 1935-1939 гг. 1936 г., 1938 г., 1940-1944 гг. 1939 г. 1943 г. 1931 г., 1934 г.

\label{076-2}
* Указана засуха только в Поволжье, значение которого в обеспечении страны хлебом всегда остается большим. В годы Великой Отечественной войны роль этого района необычайно возросла, так как из-за временной немецко-фашистской оккупации части нашей страны и разрухи в освобожденных районах посевные площади в стране в 1943 г. составляли 94,1 млн. га, или всего 63% посевной площади 1940 г.

\label{076-3}
Как видим, события интересующего нас периода совпали с годами засух (1914, 1917, 1920, 1924). Прямой связи между явлением природы - засухой и политическими явлениями в стране - революцией, войной, конечно, нет, но есть, или вполне возможна, прямая зависимость между силой воздействия засухи на общество и происходящими в нём процессами. Когда страна живёт в своих привычных условиях, то убытки урожая в засушливые годы проходят по установившимся (Стр. 81).

\pageimage{page_077}
\label{077-1}
традициям: в крестьянской стране, какой была Россия в те времена, когда малочисленный рабочий класс сам был вчерашним крестьянином с большой деревенской родней, существовал тёплый, надёжный обычай - в годы засух родственники из городов приезжали в деревню помочь поскорее убрать урожай. Ударным является слово «поскорее». Дело в том, что при благоприятных погодных условиях уборка урожая, возьмём зерновые, которые, исходя из их значимости, называют также хлебами – уборка урожая, например, пшеницы после полной её спелости должна завершиться за 8-10 дней: потом начинается естественное осыпание зерна - выпадение зёрен из колосьев.

\pageimage{page_078}
\label{078-1}
са; иными словами, начинается безвозвратная потеря урожая. Столько дней, всего 8-10, поля полноценных колосьев пшеницы восхищают вас густотой стояния, поклонами массы поля от дуновения ветра, отблеском Солнцу родственным ему цветом, поклонами колосья как бы благодарят Солнце за данную им энергию, а Человека - за приложенный труд, который поднял их из почвы, колосья как бы говорят: возьми нас человек, не дай погибнуть, наберись нашей силы, чтобы под Солнцем будущего года мы опять наполнили друг друга. Советую: съездите на зерно-производящую ферму, поживите там, попросите хозяина рассказать о циклах производства.

\pageimage{page_079}
\label{079-1}
серое кольцо ишелищ, полути те его на ладонь, всмотритесь: помимо покалывающего тепла колоса вы с неожиданным восторгом воспримите изящество его контура, Гармонию Геометрических симметрий, линий, форм. Эта строгая выразительность одного из источников энергии нашей жизни не оставляла меня равнодушным, когда во время командировок доводилось бывать на таких полях; убеждён, что и вы испытаете такое же чувство, поймёте, что рядом с вами есть возвышенный мир Гармонии и дела, который не менее интересен далёких стран. В годы засух всё не так: колос слабый, поэтому зерно начинает выпадать раньше обычного срока, чтобы убрать без потерь хотя бы то, что выросло, надо спешить с уборкой; в те годы, о которых мы говорим, ситуация ослож-

\pageimage{page_080}
\label{080-1}
часть крестьян и тех, кто приезжал помогать, были в солдатских шинелях разных армий: кто в красной, кто в белой, кто в бандитских формированиях. А хлеб нужен всем, и его брали у крестьян в меру своей морали. Как нетрудно понять, главного деятеля крестьянского двора не было, всё легло на плечи женщин, подростков, родителей, управиться с уборкой урожая в отведенный природой срок, особенно в засуху, практически невозможно. А из собранного надо оставить на питание семьи, на семена и сколько-то продать, чтобы купить необходимое для семьи и в хозяйство. При таких условиях в массе крестьянских хозяйств всё на минимуме. Итак, первая причина, думаю, прояснилась.

\pageimage{page_081}
\label{081-1}
свежую продукцию по согласованным ценам, а само пыталось обеспечить запросы деревни в различных промышленных товарах тоже по ценам, приемлемым для крестьян; однако обменное равновесие власть поддерживала с большим трудом, потому что боевые действия, точнее, результаты участия Российской империи в Первой мировой войне перетекли в Февральскую революцию, из неё - Октябрьскую, а затем в Гражданскую войну, тогда к тому же противостояния политических партий, различных организованных сил, стремления каждого - победить, дало быстрый, первый, самый ощутимый результат для всех - разрушение структуры управления экономикой страны, воцарилась хозяйственная разруха: промышленность еле дышала, и то - не вся.

\pageimage{page_082}
\label{082-1}
главный показатель 82. состояния экономики - провал торговли пришёл в упадок: товаров всё меньше, а потому цены — всё выше, происходит обесценивание денег, понятно, что в условиях недействующей экономики страны в казну её не поступают законные отчисления от различных отраслей, бюджет крайне ограничен. При таких условиях Советское правительство предложило твёрдые цены (фактически ниже рыночных или согласованных договорных) на закупаемую у крестьян продукцию, но крестьянство отказалось, оно само находилось по всем названным причинам в тяжёлом хозяйственном и материальном положении. Вопрос о ценах осложнился тем, что в марте 1918 года началась интервенция стран-союзников Российской империи по первой мировой.

\pageimage{page_083}
\label{083-1}
войне, их цель была — поддержать борьбу против советской власти сил, верных старому режиму, поэтому Гражданская война обострялась, а с ней проблемы продовольствия, особенно зерна (хлеба), и советское руководство, которое возглавлял В. И. Ленин, в июне того же 1918 года пошло на первую насильственную меру по отношению ко всему советскому крестьянству — на создание в деревнях комитетов бедноты (комбедов); из данных комбедам прав следует, что Советская власть доверила свою функцию взаимоотношений с крестьянством этим комитетам.

\pageimage{page_084}
\label{084-1}
ношении с крестьянством его низшему бездеятельному слою: в комбеды входили действительно бедные, несчастные по тем или иным причинам люди, бывшие батраки. Примазывались лодыри, горлопаны и всякая подлость; комбеды распределяли средства производства и отдельные виды товаров между крестьянскими семьями, определяли за них их личную потребность в хлебе, в семенах, а остальное зерно предлагали продать по твёрдым ценам государству; к экономическому прессу добавлялось и то, что отдельные комбедовцы, пользуясь положением, сводили личные счёты с односельчанами.

\pageimage{page_085}
\label{085-1}
Давление Советской власти на крестьянство продолжилось потому, что летом 1918 года 3/4 территории Советской страны было в руках её врагов; положение сложилось гибельное; на местах уже начали создавать белогвардейские и националистические правительства; чтобы спасти положение, Советское правительство принимает решительные меры, оно вводит всеобщее военное обучение (всеобуч), всеобщую трудовую повинность, в сельском хозяйстве - с января 1919 года вводит продовольственную

\pageimage{page_086}
\label{086-1}
разверстку и организуем продовольственные отряды для облегчения изъятия продовольствия; это был период военного коммунизма в деревне, когда, исходя из расчётных данных, каждому крестьянскому двору заранее предопределяли обязательный объём сдачи (продажи) государству сельскохозяйственной продукции, и «излишки» изымали; и эти условия ужесточались: в зависимости от обстоятельств, власть уменьшала семенной фонд, убавляла количество оставляемого продовольствия на личное питание крестьянской семьи;

\pageimage{page_087}
\label{087-1}
Потом крестьянам запретили торговать своей продукцией. В ответ на жесткие меры ленинского правительства они сократили посевные площади (всё равно отнимут, сидеть голодными), в результате деревня нищала ещё больше. По мере приближения исхода Гражданской войны напряжение нарастало. В.И. Ленин тоже объяснял потом причины усиленного давления на крестьян в годы Гражданской войны; он, как увидим, понимал, судя по его последующим оценкам первых результатов Кэпа, что репрессивными мерами невозможно за короткое время вывести страну из состояния голода и хозяйственной разрухи, а времени было мало: ситуация угрожала существованию советской власти, и В.И. Ленин решает послевоенные проблемы с позиций новой экономической политики.

\pageimage{page_088}
\label{088-1}
разработал; программу утвердил десятый съезд партии в марте 1921 года. Главная цель НЭПа — вернуть крестьянина к материальной заинтересованности в своём труде и таким образом преодолеть голод была достигнута грамотными экономическими решениями: вместо насильственных изъятий хлеба власть ввела натуральный подоходный налог на количество произведённых основных видов сельскохозяйственной продукции (900 баллов), остальной, наибольшей частью её крестьянин распоряжался сам — сколько оставить на потребление и воспроизводство, сколько продать; деловая жизнь в деревне пробуждалась; в ряде случаев крестьяне кооперировались по видам отдельных работ, по использо-

\pageimage{page_089}
\label{089-1}
ванию средств производства, а позже - и по их собственному приобретению: значение этих фактов и поощрения государством расширения посевных площадей станет вам понятным, если узнаете, что к 1924 году они сократились до 90,3 млн. га пашни, а в 1913 году было 105 млн. Процент снижения урожайности был ещё выше, плюс падёж скота (продовольственные трудности в стране утяжелялись бескомпромиссным, истребительным отношением советской власти к кулачейству; в городе разрешили, говоря современным языком, малый и средний бизнесы, а также ограниченное использование частного капитала в рамках НЭПа, но естественно, государство контролировало соблюдение условий новой политики и оберегало свои монопольные интересы; результат

\pageimage{page_090}
\label{090-1}
Получился хороший: действующие на то время промышленные предприятия страны увеличили выпуск средств производства для сельского хозяйства и таким образом Государство обеспечило себе господствующее положение в экономических связях с деревней (уровни цен, объём и направления поставок и т.п.); частные мелкие предприятия, артели, кустарная частная торговля тоже были вовлечены в оборот с деревней, потому что она нуждалась во всём - от мелкого инвентаря до иголки; новые условия взаимоотношений продолжились в небывалом росте сельскохозяйственного производства. Ожившую трудовую сельскую жизнь того периода, охвативший людей энтузиазм, показал Андрей Платонов в рассказе «Лампочка Ильича»; рассказ датирован 1926 годом (через два года после

\pageimage{page_091}
\label{091-1}
смерти Ленина), писатель, видимо уловил тогда другие веяния, потому финал рассказа символичен. Причиной его написания? Уже в 1923 году острота голода спала. Об этом я и по отрывочным разговорам в семье, то есть и на бытовом уровне в среде простых людей, подтверждалось положительное значение начатого процесса. Всего лишь за первые два года осуществления НЭПа Советская власть убедилась в преимуществе экономических динамичных методов управления; после первых значительных успехов в сельскохозяйственном производстве Советское правительство видоизменило в пользу крестьян условия переходного периода и в 1923 году отменило продналог, государство перешло на закупки зерна и другой сельхозпродукции.

\pageimage{page_092}
\label{092-1}
рынке; получив дополнительные производственные свободы, крестьяне в том же (1923) году расширили посевные площади на 12,4 миллиона гектаров, или на 18,7\%. [Чтобы вы не запутались в цифрах: пашня — составная часть посевных площадей, а посевные площади — составная часть сельскохозяйственных угодий]; результаты экономического стимулирования выразились в прогрессирующем развитии сельского хозяйства вплоть до ликвидации НЭПа в 1928 году. Вот «узловые» данные: довоенный, благополучный 1913 год примем по всем показателям за 100\%, в критическом 1921 году производство валовой продукции сельского хозяйства упало до 60\%, продукции земледелия — до 55\%, а в 1928 году подошло (соответственно показателям к

\pageimage{page_093}
\label{093-1}
уровни даже 1913 года, эти соотношения наглядны в графике, 1928 130 120 1913 год. 100\% до 65 1991 год 50 Валовая продукция сельского хозяйства - Продукция земледелия Легко понять, в какую яму падало сельское хозяйство, а с ним и вся экономика России, в период с осени 1914 по 1921 годы (при ускорении в годы Гражданской войны).

\pageimage{page_094}
\label{094-1}
P63. 94. Отмечу: до утверждения партийным съездом Новой экономической политики она обсуждалась на различных форумах. Несмотря на горький опыт военного коммунизма, были мнения о продолжении принудительной массовой организации производства. Ленин, ссылаясь на тот же опыт, опровергал подобные взгляды. Полагаю, что уверенность в правильности альтернативной позиции придавали ему ещё два факта: подавление возмущения, или восстания Тамбовских крестьян в 1920-1921 годах действиями Советской власти в деревне. Руководил восстанием начальник уездного (районного) отделения милиции А.С. Антонов (1888-1922), отсюда — «антоновщина». По датам восстания и по высокой должности А.С. Антонова можно предположить, что он был коммунистом-большевиком; восстания по той же причине были и в других местах.

\pageimage{page_095}
\label{095-1}
95. губерниях; что самый факт - мятеж на военной морской базе «Кронштадт», и здесь звучала крестьянская тема, поскольку подавляющее число восставших моряков и солдат - это были крестьянские парни, тесно связанные с деревней; углублять разрыв с крестьянством - вовсе не было задачей, как вы уже знаете, нового государства, наоборот, требовалось восстановить значение декрета о земле, чтобы крестьяне быстрее наладили производство товарной продукции; так и получилось: единоличные крестьянские хозяйства, поддержанные государственным экономическим регулированием, учитывавшим интересы обеих сторон, сняли за два года проблему голода и накормили народ; этот факт свидетельствует о том, что никакой необычности во взаимоотношениях социалистического

\pageimage{page_096}
\label{096-1}
96. Государства с единоличным, частнособственническим крестьянством не было, именно в таком соотношении В.И. Ленин видел развитие сельского хозяйства страны на длительную перспективу. Объяснял он это так: крестьяне составляют гигантскую часть всего населения и всей экономики; преобладает мелкотоварная форма ведения крестьянского хозяйства, она будет существовать в течение длительного исторического периода в рамках переходной экономики, потому что для социалистического преобразования мелкотоварного крестьянского хозяйства нужна мощная материально-техническая база. В те и во многие последующие годы такой базы не было.

\pageimage{page_097}
\label{097-1}
Крестьянин с его фотопными орудиями труда, работой по старинке, с психологией собственника, пересаженный в коллективное хозяйство (колхоз), не будет заинтересован в труде, так как не ощутит его конкретного результата. Поэтому длительным путём к колхозу является кооперация - не административная, а добровольная. Разъяснительная фантазёры те коммунисты, которые думали, что за три года можно переделать экономическую базу, экономические корни мелкого земледелия.

Двухлетняя практика действия Новой экономической политики В.И. Ленина подтвердила, воп.

\pageimage{page_098}
\label{098-1}
реки высказывавшимся несогласиям, правильность его взглядов на аграрное развитие страны: рассуждая о государственном капитализме в условиях социалистического строительства, он пришёл к важному выводу о том, что кооперация выполняет роль государственного контроля и надзора за частным капиталом, кооперация помогает государственному регулированию овладеть стихией мелкотоварного производства, поэтому необходимо направить развитие капитализма в деревне в русло кооперативного capitalismo» (формулировка В.И. Ленина). Чтобы вы поняли значение приведенного текста, объясню различие между членством в кооперации и членством в коллективном хозяйстве.

\pageimage{page_099}
\label{099-1}
ОК (Коллод), крестьянскую кооперацию. Крестьянник: единоличник вступал добровольно, чтобы совместно, а, как правило, и легче, решить однозначную и для дружных односельчан задачу. В условиях ускоренного роста сельскохозяйственного производства расширялись хозяйственные потребности крестьян, и специфика коопераций соответствовала этому, она охватывала всю инфраструктуру сельской жизни. Всё это делалось при активном, преимущественно экономическом, участии государства; главная цель его была в том, чтобы через кооперацию вытащить крестьянина-единоличника из его психологической и хозяйственной замкнутости, показать ему преимущества совместных решений текущих, повседневных задач.

\pageimage{page_100}
\label{100-1}
лилии но: был ли крестьянин постоянным или временным членом кооперации, или не входил в неё вообще, - в любом случае он оставался собственником участка земли, который закрепила за ним Советская Власть после национализации всей земли и в соответствии с Декретом о земле, то есть: Крестьянин-единоличник вёл частное хозяйство (в этом смысле он оставался «капиталистом») на государственной, социалистической земле; в то же время он ощущал «локоть» кооперации и знал о преимуществах вхождения в неё (некоторые более выгодные условия от совместных дел, цены, финансовые кредиты, также консультации агрономов, ветеринаров и пр.); исто-

\pageimage{page_101}
\label{101-1}
Источник возможностей для внутрикооперационной взаимопомощи было отчасти само кооперационное объединение людей, средств, но главным источником было государство: организационно, пропагандой, своевременно принятыми экономическими решениями оно поддерживало и укрепляло «кооперативный капитализм»; таким образом государство пыталось на широкой, доказательной практике «переделать экономическую базу, экономические корни мелкого земледелия». В результате такой политики сельское хозяйство развивалось активно, чтобы усилить этот процесс, крестьянам в 1925 году разрешили использовать наёмный труд для работ на арендованных ими землях.

\pageimage{page_102}
\label{102-1}
кто-либо из вас, - Вскоре после революции возрождение отдельных условий, из-за которых громили помещичьи и кулацкие хозяйства. Отступление? Никакого отступления не было, дело в том, какие экономические и социальные условия стоят за понятиями аренда земли и наёмный труд. К успеху НЭПа надо отнести и тот факт, что в те годы кулацкие хозяйства, где - разорённые, где - полуразорённые, а кое-где - ещё нетронутые советской властью, получили права в занятии сельскохозяйственным производством. Ленинское толкование кооперации, очередность направлений аграрного развития; что НЭП — не экономический манёвр, а долгосрочная стратегия и вводится всерьёз и надолго — эта концепция просуществовала всего

\pageimage{page_103}
\label{103-1}
1927 года, когда пятнадцатый съезд партии, вопреки предостережениям В.И. Ленина, принял решение о развертывании на селе коллективизации, то есть фактически речь шла о создании в деревне таких условий, при которых экономически поднявшееся, неповское крестьянство вынуждено было бы вступать в колхозы, и это в условиях, когда более половины площадей под хлебами крестьяне засеивали вручную и также убирали урожай, ручной труд преобладал и в животноводстве. Этому, исходя из ленинских оценок ситуации ещё в 1925-1927 годах, преобладало мнение, что не колхозы являются столбовой дорогой к социализму, а кооперация. (По книге В.В. Милосердова - академика Россельхозакадемии).

\pageimage{page_104}
\label{104-1}
после смерти
В.И. Ленина (24 января 1924 года) защиту его идей на партийных обсуждениях и в полемике с И.В. Сталиным взял на себя известный партийный и государственный деятель того времени Бухарин Николай Иванович. Суть расхождений во мнениях между ними на перспективные решения в следующем. В.И. Ленин умер на взлёте НЭПа. С 1920 года осуществлялась также инициированная В.И. Лениным электрификация страны (ГОЭЛРО), которая, значимая сама по себе, являлась одной из опор индустриализации СССР; развивать равнозначно три направления народного хозяйства государство не могло: не было ни материальных, ни финансовых средств.

\pageimage{page_105}
\label{105-1}
тому выбрали самое жизненно важное - сельское хозяйство, чтобы ликвидировать голод. Как вы уже знаете, эта проблема решалась успешно: продовольствия было полно, цены - низкие. В деревне накапливались заработанные деньги, крестьяне стали больше покупать разных товаров, спрос деревни опережал предложение города (промышленности), нарушился паритет вздутых низких цен, промышленные товары дорожали. Вопрос о паритете цен - обычный и решаемый (и тогда была возможность благополучно его снять: например, даже в медленно восстанавливаемой после разрухи промышленности можно было перераспределить номенклатуру выпускаемых товаров, в том числе и средств производства для села) - вопрос завис, вместо его решения.

\pageimage{page_106}
\label{106-1}
106

на уровне государственного экономического регулирования. Крестьяне уже восприняли за несколько лет НЭПа, они ощутили влияние начинающегося ускорения темпов индустриализации в виде требований продавать зерно и другую сельхозпродукцию государству по сложившимся низким рыночным ценам. Крестьяне, как и перед НЭПом, ответили сдерживанием объёмов продаж хлеба, сокращением посевных площадей. С индустриализацией возник, на неопытный взгляд, заколдованный круг: ускорение её развития вовлекало всё больше промышленных центров, увеличивалось городское население, всех надо было кормить, возрастала

\pageimage{page_107}
\label{107-1}
Потребность в расширении сельскохозяйственного производства, особенно зерна, восстановленное и хорошо развитое за годы НЭПа сельское хозяйство в силах было справиться с этой задачей даже в тех условиях, когда государство большую часть средств вкладывало в индустриализацию; средства, деньги на ускорение сельскохозяйственного производства были в самой отрасли: приведенный условный пример можно тематически продолжить вариантом, по которому государство могло скупить у крестьян через торгово-сбытовые кооперации удерживаемое зерно по согласованным ценам, но в кредит под Госбанковскую гарантию на повышенную часть цены; примерно также можно было договориться.

\pageimage{page_108}
\label{108-1}
106. Мы кооперируемся по поводу урожая предстоящего года и спокойно, не нарушая производственного цикла, увеличиваем экспортную массу зерна, валютная прибыль от которого шла на закупки техники для ускоренно развивавшейся тяжёлой промышленности. Дело, конечно, не в поиске решения возникших противоречий из-за цен, всё заключалось в позиции И.В. Сталина. В статье «Сталин» в «Советском энциклопедическом словаре» (М., «Советская энциклопедия», 1980) сказано, в частности, «допускал теоретические и политические ошибки» (с.1275). В составляющие эту оценку вошло, можно считать, и отношение И.В. Сталина к новой экономической политике.

\pageimage{page_109}
\label{109-1}
тупал за ускоренную индустриализацию страны, но против сохранения тех экономических связей, методов управления и их результатов, которые обеспечивал НЭП, поскольку они не удовлетворяли растущих запросов индустриализации; исходя из этого своего взгляда, а также имея сторонников в партии, И.В. Сталин считал возможным применять насильственные меры по отношению к крестьянству, чтобы изъять у него дополнительные количества зерна в целях поддержания высоких темпов индустриализации (острота вопросов концентрировалась вокруг зерна потому, что в те годы зерно было единственным экспортным товаром СССР); И.В. Сталин считал, что неповские капиталисты — средний и малый бизнесы — будут мешать социалистическому развитию.

\pageimage{page_110}
\label{110-1}
по-алистической реконструкции страны, в частности расширению производственных связей между предприятиями на основе государственного планирования, и он призывал «к активному наступлению на капиталистические элементы» (надуманность повода для наступления состояла в том, что все заводы, фабрики, земля, железные дороги, источники сырья и другие стержневые ветви экономики были после Октябрьской революции в руках государства; неповские капиталисты — это разрешённый правительством ремесленно-артельный уровень частного производства, который работал на купленных отходах заводских, фабричных производств, на отходах других типов предприятий и на сельскохозяйственном сырье. Торговцы — напманы — того же экономического ряда, но они знали бытовые запросы населения и заказывали производителям нужные товары. Весь этот ремесленный и торговый люд платил установленные налоги, прилично жили, избавлял правительство от забот по обеспечению населения большим набором.

\pageimage{page_111}
\label{111-1}
ром бытовых товаров; 131, никакой политической силы эти люди не представляли, так как никаких партийных групп не создавали, это были обычные обыватели, перенесшие две войны, две революции, разруху, голод, они ухватились за возможность работать и наладить свою жизнь; вся политическая и экономическая сила была у руководства страны, но громкая, модная тогда фразеология понадобилась, чтобы обосновать ликвидацию НЭПа. На партийных форумах Н.И. Бухарин выступал против насильственных мер по отношению к крестьянству, они ведут,

\pageimage{page_112}
\label{112-1}
как известно, к сопротивлению и падению производства (Так, в приведенном выше графике на данных за 1928 год уже сказались вновь начавшиеся протестные меры крестьянства); причины нарастающих потребностей, особенно в зерне, Бухарин видел в неимоверных темпах индустриализации. Как получать дополнительное зерно для всесторонних нужд индустриализации он предлагал не путём насилия над крестьянством, а путём расширения аренды земли и прав на наём рабочей силы, дальнейшего развития товарно-денежных отношений. Главными принципами в аграрных отношениях Бухарин считал нормальные хозрасчетные отношения, эквивалентный обмен между городом и деревней.

\pageimage{page_113}
\label{113-1}
ом и деревней, у него 115. были предложения и по текущим вопросам, но он возражал против преждевременной коллективизации, которая, утверждал Н.И. Бухарин, не сможет оказать существенного влияния на развитие сельского хозяйства. И.В. Сталин и его единомышленники придерживались другого мнения: экономический механизм нэпа не обеспечит перекачку необходимых средств в развитие тяжёлой индустрии, поэтому при возникающем неэквивалентном обмене невозможен рынок (в широком политэкономическом смысле: речь идет о преимущественно равновыгодном торговом обмене продукцией разных сфер производства).

\pageimage{page_114}
\label{114-1}
льку в те годы только аграрная отрасль могла стать основным донором индустриализации, то при отрицании возможности попытаться снять возникшие производственно-экономические разногласия с крестьянством начали складываться экономические неравенства отраслей. Можно полагать, что идея неизбежности неэквивалентного обмена и сохранение условий, способствующих этому, укрепляли позиции сторонников преждевременной коллективизации: правительство получало фактически неограниченное право брать из сельского хозяйства продукции столько и так, как посчитает нужным руководство страны, и таким образом перекачивать средства в индустриализацию, лишая крестьян возможности выражать действенными мерами.

\pageimage{page_115}
\label{115-1}
отдельными решениями правительства. В партийной полемике верх одержала позиция И.В. Сталина, и в 1929 году началась, повторюсь, коллективизация. Точки зрения взглядов в члени- на материально-технически и морально (без влияния кооперации) неподготовленная, а потому преждевременная, коллективизация. В реальной жизни это означало отказ от управления экономическими методами и замена их пропагандой призывно: карающими лозунгами типа: «Кто не идет в колхоз, тот враг Советской власти». узаконенное партийным решением 6 ноября 1929 года соревнование по наращиванию темпов коллективизации, затем соревнование рец.

\pageimage{page_116}
\label{116-1}
116. онов за скорейшее завершение коллективизации; переход к применению репрессивных, карательных мер, в отдельных случаях и военной силы, для выполнения «темпов», «соревнований», о которых благополучное в массе своей неповское крестьянство и понятия не имело, привел к тому, что насильно включённые в колхозы люди обязаны были передать колхозам свои права на пользование землёй, отдать рабочий и продуктивный скот, орудия сельскохозяйственного труда, семена и т.п.; крестьянин-индивидуалист, лишаемый своей производственной опоры и товарно-денежных отношений,

\pageimage{page_117}
\label{117-1}
в которые укладывались его хозяйственные и личные интересы, в то же время обязанный принять новые формы организации труда, сбитый с толку навалившимися на него переменами, оставшись без поддержки правительства и видя теперь в нём своего противника (на фоне безобразных фактов коллективизации), стал резко сокращать свою хозяйственную деятельность, в частности сверхмерно забивать скот, чтобы получить какие-то деньги за отбираемую собственность. Как начавшиеся перемены быстро сказались на разорении единоличных крестьянских хозяйств, в данном случае на снижении численности скота, видно из сопоставимых

\pageimage{page_118}
\label{118-1}
1928 года по 1 января 1954. Гораздо меньше голов во всех категориях хозяйства при значительном преобладании индивидуальных хозяйств:

1928 год

Крупный рогатый скот Свиньи Овцы

8,9 27,7 104,2 36,1

50,6 14,2 85,5 31,0

Численность поголовья заметно снижалась и в последующие несколько лет; тенденция падения производства по отношению к уровням 1928 года была явная во всех главных отраслях сельского хозяйства: Итак, принудительная коллективизация, мародёрские методы изъятий дополнительного зерна.

\pageimage{page_119}
\label{119-1}
нарастающих потребностей Экспорта — опять, как в годы Гражданской войны, насильно отнятое зерно, которое было предназначено для питания семьи и на семена, усиление административного произвола в деревне, вся эта беззащитность привела к массовому выражению протеста тысячами крестьянских восстаний, зарегистрированных только в одном 1929 году: чтобы неповадно было бунтовать, усилили темпы коллективизации, и к середине 1931 года в колхозы уже было включено 52,7\% единоличных крестьянских хозяйств, преднамеренный отказ послеленинского руководства от Новой экономической политики завершился крахом.

\pageimage{page_120}
\label{120-1}
хозяйства: восстановленное и давшее замечательные производственные результаты в годы НЭПа, оно развалилось под воздействием названных притеснений крестьянства. Коллективизация не обеспечила необходимого количества зерна для нужд индустриализации, в том числе экспорта, поэтому подключили продажу за рубеж произведений искусства. В розах стали открывать магазины «Торгсин» (торговля с иностранцами - там обменивали на специальные талоны драгоценности и иностранную валюту, на талоны можно было приобрести продовольствие и промтовары); в то же время власть ещё сильнее придавила крестьянство.

\pageimage{page_121}
\label{121-1}
и низовой аппарат управления на селе в результате несоразмерных с возможностями требований власть допустила голод в деревнях и селах, унесший не менее трёх миллионов жизней крестьян в благоприятные для сельского хозяйства годы — 1932-1933; на Украине это голод называют голодомором. Таковы финальные аккорды реквиема по новой экономической политике в деревне, диезы и бемоли в условиях городского исполнения покажу на примере моего отца, а потом поделюсь своими личными обобщениями.