\translatechapter{Ранения и Восстановление}{Injuries and Recovery}
\begin{paracol}{2}

% continued from page_010
\translate{
Москва представилась не родственниками и не родным домом - мне было 19 лет, мои родители к тому времени умерли. Москва представилась в тот миг уютным уголком Пушкинской площади, где стояли памятник великому поэту (на исконном месте) и гипсовая статуя балерины на крыше нового тогда углового дома по улице Горького.
}{
Moscow did not appear as relatives or a native home - I was 19 years old, my parents had died by that time. Moscow appeared at that moment as a cozy corner of Pushkin Square, where the monument to the great poet (in its original place) and the plaster statue of a ballerina on the roof of the then new corner house on Gorky Street stood.
}

\translate{
Растерянность, шок сменились восстановлением сознания, хотя в голове сильно шумело и стучало в висках, сгустки крови давили горло, я беспрерывно заглатывал их. Однако страх смерти сменился стремлением выжить. Я осторожно развернулся на левом плече, опёрся на левую часть спины, левой рукой подтянул правую руку, потом протащил полевую сумку и ремнём обмотал оторванную руку; в поиске некоторого облегчения боли я попытался снять с лица жгущую тяжесть, но пальцы коснулись языка. Я решил выбираться своими силами. Не успел поеду.
}{
Confusion, shock were replaced by the restoration of consciousness, although there was a loud noise in my head and pounding in my temples, clots of blood pressed on my throat, I continuously swallowed them. However, the fear of death was replaced by the desire to survive. I carefully turned on my left shoulder, leaned on the left part of my back, pulled my right hand with my left hand, then dragged the field bag and wrapped the torn hand with the strap; in search of some relief from the pain, I tried to remove the burning weight from my face, but my fingers touched my tongue. I decided to get out on my own. I didn't have time to...
}

\pageimage{page_011}{
\translate{
свои действия, как через секунды я увидел возле себя медицинскую сестру. С колен, нагнувшись без слов, она начала бинтовать моё лицо, потом дважды пыталась наложить жгут на правое предплечье, жгут не держался. «Придётся тугую повязку», — сказала она тихо ни мне, ни себе и опять открыла санитарную сумку. Пока сестра бинтовала, я пытался внятно произнести слово «пить». Она, конечно, всё понимала и без моего мычания, и когда кончила бинтовать, приподняла две фляги, слегка потрясла ими и тихо, с большим сочувствием сказала: «Раненые выпили. Потерпи, родной, скоро заберём тебя». По-прежнему оставаясь низко пригнувшейся к земле, сестра отползла. Тугие повязки приглушили ощущения острой боли, а жажда мучала: казалось, что во рту у меня провонявшая, перепревшая портянка, которую я вынужден отсасывать... Полулёжа на спине, я подтянул повыше, к груди, полевую сумку с примотанной рукой, зацепил ремень автомата и, опираясь на левый локоть, стал ползти в сторону лесочка воз-
}{
my actions, when seconds later I saw a nurse next to me. Kneeling, bending over without a word, she began to bandage my face, then twice tried to apply a tourniquet to my right forearm, the tourniquet did not hold. "We'll have to use a tight bandage," she said quietly, neither to me nor to herself, and opened the medical bag again. While the nurse was bandaging, I tried to clearly say the word "drink." She, of course, understood everything without my mumbling, and when she finished bandaging, she lifted two flasks, shook them slightly, and quietly, with great sympathy, said: "The wounded drank. Hold on, dear, we'll take you soon." Still staying low to the ground, the nurse crawled away. The tight bandages muffled the sensations of sharp pain, but the thirst tormented me: it seemed that there was a stinking, overripe footcloth in my mouth, which I was forced to suck... Half-lying on my back, I pulled the field bag with the strapped hand higher to my chest, hooked the strap of the submachine gun, and, leaning on my left elbow, began to crawl towards the forest...
}
}

\pageimage{page_012}{
\translate{
Можно, если бы я попил воды, остался бы ждать помощи. Близкий лесочек уже казался далёким, однако я дополз. Помню себя уже в санроте: значит, сестра вернулась, меня вынесли. В санроте наполнили незабываемой прелести сырой, прохладной водой, сделали обязательный для всех раненых противостолбнячный укол. Потом - операционная медсанбата.
}{
Maybe if I had drunk water, I would have stayed to wait for help. The nearby forest already seemed far away, but I crawled. I remember myself already in the medical company: it means the nurse returned, they carried me out. In the medical company, they filled me with the unforgettable charm of raw, cool water, gave the mandatory tetanus shot for all the wounded. Then - the operating room of the medical battalion.
}

\translate{
- Санитарная рота - ближайший к боевым действиям медицинский пункт; располагалась на передовых позициях; её функция - неотложная помощь раненому и отправка его в медсанбат.
}{
- The medical company is the nearest medical point to the combat actions; it was located at the front positions; its function was to provide emergency assistance to the wounded and send them to the medical battalion.
}

\translate{
Медицинско-санитарный батальон - развёрнутый в боевых условиях полноценный, по мере продвижения армии стационарный больничный комплекс; располагался в глубине фронтовой полосы на уровне вспомогательных подразделений дивизий, армии. Из медсанбата прооперированных тяжело раненых бойцов отправляли по цепочке эвакогоспиталей (ЭГ) до ближайшей восстановленной железной дороги, куда подходили санитарные поезда. Они развозили раненых по госпиталям страны; в ЭГ раненые находились по 2-4 дня, в зависимости от процесса лечения и их физического состояния. Перевозили раненых и самолётами.
}{
The medical battalion is a full-fledged hospital complex deployed in combat conditions, becoming stationary as the army advanced; it was located deep in the front line at the level of auxiliary units of divisions and the army. From the medical battalion, operated-on severely wounded soldiers were sent through a chain of evacuation hospitals (EH) to the nearest restored railway, where hospital trains arrived. They transported the wounded to hospitals across the country; in the EH, the wounded stayed for 2-4 days, depending on the treatment process and their physical condition. The wounded were also transported by planes.
}
}

\pageimage{page_013}{
\translate{
Была ночь, когда я проснулся от наркоза. Увидел себя в большой армейской палатке. На центральном опорном шесте слабо светилась лампочка. За небольшой тумбочкой сидела медицинская сестра. Много коек, на всех раненые. Полог входа в палатку отброшен, тянет запахом леса. Особая тишина ночи. Ни выстрела, как будто нет войны. Моя челюсть стянута бинтами, по горло накрыт простыней. Рука?! Я резко сдёрнул простыню. Широкие бинты опоясывали грудь, а правое плечо было замотано вкруговую толстой бинтовой повязкой. «Всё. Руки не будет», — холодно сказал я себе и понял свою неновую будущность. Сестра услышала какое-то движение, подошла, заново накрыла простыней, говорила что-то обнадеживающее, но я уже не вслушивался, как утром. Оказалось, я не всё знал тогда об этом дне — была и третья пуля. В городе Иваново, где я лежал в челюстном госпитале, впервые после ранения взял свою полевую сумку и ахнул: в краешке её левого уголка был пулевой вход (попросту говоря, маленькая дырка), а на вылете, в задней утолщённой стенке — большой разрез, как ножом; на донышке сумки лежали разороченная латунная оболочка пули и...
}{
It was night when I woke up from anesthesia. I saw myself in a large army tent. A dim light bulb glowed on the central support pole. A nurse sat at a small bedside table. Many cots, all with wounded. The flap of the tent entrance was thrown back, the smell of the forest wafted in. A special silence of the night. Not a shot, as if there was no war. My jaw was bandaged, covered with a sheet up to my neck. My arm?! I sharply pulled off the sheet. Wide bandages wrapped around my chest, and my right shoulder was wrapped in a thick bandage. "That's it. There will be no arm," I coldly said to myself and understood my new future. The nurse heard some movement, approached, covered me with the sheet again, said something encouraging, but I was no longer listening, as in the morning. It turned out I didn't know everything about that day - there was a third bullet. In the city of Ivanovo, where I was lying in a maxillofacial hospital, I first took my field bag after the injury and gasped: in the corner of its left edge was a bullet entry (simply put, a small hole), and at the exit, in the back thickened wall - a large cut, like with a knife; at the bottom of the bag lay a torn brass bullet casing and...
}
}

\pageimage{page_014}{
\translate{
Рассплющенный кусочек свинца, так рвёт разрывная пуля. По сей день храню эти предметы, ставшие для меня реликвиями Великой Отечественной войны. Дороги к победе над германским нацизмом были крутыми. Слушая чистые звуки фанфар победителям, отдадим честь и тем, чья молодость навечно застыла в той тяжёлой войне. Я награждён двумя орденами Отечественной войны, первой и второй степени, а также медалью «За Победу над Германией» и памятными юбилейными медалями СССР, РФ и Израиля.
}{
a flattened piece of lead, that's how an explosive bullet tears. To this day, I keep these items, which have become relics of the Great Patriotic War for me. The roads to victory over German Nazism were steep. Listening to the pure sounds of fanfares for the victors, let us honor those whose youth was forever frozen in that heavy war. I was awarded two Orders of the Patriotic War, first and second class, as well as the medal "For Victory over Germany" and commemorative jubilee medals of the USSR, RF, and Israel.
}

\translate{
Сокращённо опубликовано в еженедельнике «Кстати» (Калифорния) 1-7 июня 2006 года. №585, стр. 35. http://www.kstati.net
}{
Abbreviated published in the weekly "Kstati" (California) June 1-7, 2006. No. 585, p. 35. http://www.kstati.net
}
}
\end{paracol}