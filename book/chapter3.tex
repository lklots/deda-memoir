% continue from page_014

\begin{paracol}{2}
\translate{
Продолжение начну с важного для меня эпизода из госпитального периода. Я задался вопросом: как мне жить без руки? Сложности были очевидны, да суровые условия жизни военных лет не утешали. Были и такие инвалиды войны, которые не выдерживали свалившихся на них бед и скатывались на дно жизни; это вовсе не была закономерность, следовавшая из их прошлого. Я понял, что помощь, силу надо искать в себе же; уже в госпитале надо начинать готовить себя к новой жизни. Но как? Первое испытание подсказалось необходимостью написать братьям о своём ранении.
}{
  I will continue with an important episode from the hospital period. The first test was prompted by the need to write to my brothers about my injury. I asked myself: how do I live without an arm? The difficulties were obvious, and the harsh living conditions of the war years offerred no consolation. There were also such war invalids who could not withstand the misfortunes that befell them and slid to the bottom of life; this was not at all a pattern that followed from their past. I realized that help, strength must be sought within oneself; already in the hospital, one must begin to prepare oneself for a new life. But how?
}

\pageimage{page_015}{
\translate{
  Я побуквы Когда решил, что достиг успеха в каллиграфии, попросил медицинскую сестру поискать твёрдую доску или фанерку, чтобы на согнутых ногах мог написать два коротких письма. Как сейчас помню её удивление и предложение написать письма под мою диктовку. Я поблагодарил и отказался, объяснив, что у меня никого нет, надо учиться всё делать самому. Буквы получились корявые, адреса написала сестра. Главное было в том, что я не принял соблазнительную, иногда разрушающую волю человека помощь. Это был мой первый шаг к самоутверждению в сложившейся реальности.
}{
 I decided that I had achieved success in calligraphy, and asked the nurse to find a hard board or plywood so that I could write two short letters on my bent legs. I still remember her surprise and her offer to write the letters under my dictation. I thanked her and refused, explaining that I had no one, I had to learn to do everything myself. The letters turned out crooked, the nurse wrote the addresses. The main thing was that I did not accept the tempting, sometimes destructive help that can undermine a person's will. This was my first step towards self-affirmation in the new reality.
}
\translate{
Среди погибших мой друг — Заславский Яков Михайлович, р. 1923 года, москвич, жил на Сретенке, ул. Хмелева, 14. Мы познакомились задолго до войны в пионерском лагере. Яша закончил (если не ошибаюсь, Самаркандское) танковое училище. Был ранен, после госпиталя вернулся на фронт, но не танкистом. В звании старшего лейтенанта, на 2-м Прибалтийском фронте, 25 июля 1944 года, то есть на следующий день после моего ранения, Яшу тяжело ранило в живот, и в тот же день он скончался.
}{
Among the dead was my friend - Zaslavsky Yakov Mikhailovich, born in 1923, a Muscovite, lived on Sretenka, Khmeleva St., 14. We met long before the war in a pioneer camp. Yasha graduated (if I'm not mistaken, from the Samarkand) tank school. He was wounded, after the hospital he returned to the front, but not as a tanker. As a senior lieutenant, on the 2nd Baltic Front, on July 25, 1944, that is, the day after my injury, Yasha was seriously wounded in the stomach, and on the same day he died.
}
\translate{
После войны фронтовой товарищ Яши прислал его родителям полевую карту, на которой точно помечена могила Яши. При содействии военкоматов родителям разрешили перезахоронить своего единственного сына на подмосковном еврейском кладбище «Востряковское». Яшина могила в правом углу от главных ворот. Подборка его писем с фронта опубликована в журнале «Знамя». Отдаю честь двоюродной сестре моего друга — Люсе Зимоненко, которая многое сделала для сохранения памяти о погибшем брате.
}{
After the war, Yasha's front-line comrade sent his parents a field map on which Yasha's grave was precisely marked. With the assistance of military commissariats, the parents were allowed to reburial their only son in the Vostryakovskoye Jewish cemetery near Moscow. Yasha's grave is in the right corner from the main gate. A selection of his letters from the front was published in the magazine "Znamya". I pay tribute to my friend's cousin - Lyusya Zimonenko, who did a lot to preserve the memory of her deceased brother.
}
}


\pageimage{page_016}{
\translate{
В той же могиле покоится его мать — Мильда Яковлевна. С Ящей связано воспоминание о последнем довоенном пионерском лете, о 1940 годе. Всегда можно было пробыть за городом две смены, да родители освобождались от забот о детях. Уровень интереса определялся в лагере квалификацией школьных педагогов, которые вели кружковую работу. Кружки были: художественного слова, ботанический, умелые руки, военизированного направления — авиамодельный, географический. Вечером, когда уже смеркалось, разжигали костёр. Ребята, преимущественно девочки, читали — кто наизусть, кто по книге — стихи, небольшие рассказы русско-советских классиков. Потом педагог-руководитель кружка помогал разобраться в подлинном содержании прочитанного, найти смысловые ударения. Такие необязательные теоретизированные уроки легко запоминались и помогали нам в школе получать хорошие отметки по литературе.
В предвоенные годы пригородные электрички по Киевской железной дороге ещё не ходили. Только на участке Северный (Ярославский) вокзал — город Александров поезда тянули электровозы. Приезжавшие с восторгом рассказывали о разности впечатлений от езды. Ну, а если такое выпадало на счастье знакомых ребят, то ты становился завистливым свидетелем потока хвастовства; ребята чувствовали себя героями, будто они и есть те самые электровозы.
}{
In the same grave rests his mother — Milda Yakovlevna. A memory associated with Yasha is about the last pre-war pioneer summer, in 1940. You could always stay out of town for two shifts, and parents were freed from the worries of children. The level of interest in the camp was determined by the qualifications of school teachers who led club activities. The clubs included: literary, botanical, handicrafts, military-oriented — model aircraft, geographical. In the evening, when it was already getting dark, a campfire was lit. The children, mostly girls, read — some from memory, some from books — poems, short stories by Russian-Soviet classics. Then the club leader helped to understand the true content of what was read, to find the semantic emphasis. Such optional theoretical lessons were easily remembered and helped us get good grades in literature at school.
In the pre-war years, suburban electric trains on the Kiev railway did not yet run. Only on the section from the Northern (Yaroslavsky) station to the city of Alexandrov did the trains run on electric locomotives. Those who came back told with delight about the different impressions of the ride. Well, if such a thing happened to familiar guys, you became an envious witness to the flow of boasting; the guys felt like heroes, as if they were those very electric locomotives.
}
}

%%[[page_017]]
\pageimage{page_017}{
\translate{
Так что и летом 1940 года до станции Суково по Киевской, где размещался пионерский лагерь, мы ехали в старых, громыхавших, переваливавшихся сбоку на бок вагонах. Состав тащил тоже старый, приспособленный для пригородного движения, паровоз. Зато он был надраен до сверкающего блеска, и это придавало какой-то шарм путешествию. Паровоз прерывисто гудел, с шумом выпускал пар; колёса монотонно и часто стучали на стыках рельс; угольная гарь из трубы паровоза летела в открытые окна вагонов; при торможении вагоны с лязгом и силой сталкивались буферами, поэтому ребята, которые неустойчиво стояли или сидели на краешках сидений, могли упасть. Условия езды веселили нас, мы сравнивали их с относительно недавно (лето 1935 года) открывшейся впервые в СССР линией Московского метрополитена имени Л.М. Кагановича («Сокольники — Парк культуры имени А. М. Горького»). Несмотря на пыхтение паровоза, пейзаж медленно проплывал мимо окон поезда. Суково казалось по отношению к Москве несусветной далью, и на родительские дни мало кто приезжал, даже к младшим классникам. Отдалённость объединяла всех. Мы с лёгкостью принимали предложения руководителей. Так, педагог по ботанике предложила присоединиться к обору.
}
{
So, in the summer of 1940, we traveled to the Sukovo station on the Kiev railway, where the pioneer camp was located, in old, rattling, swaying side-to-side carriages. The train was also pulled by an old steam locomotive adapted for suburban traffic. But it was polished to a sparkling shine, which added some charm to the journey. The steam locomotive whistled intermittently, released steam with noise; the wheels monotonously and frequently clattered at the rail joints; coal soot from the steam locomotive's chimney flew into the open windows of the carriages; when braking, the carriages collided with a clang and force, so the children who stood unsteadily or sat on the edges of the seats could fall. The travel conditions amused us, we compared them with the relatively recently (summer of 1935) opened first line of the Moscow Metro named after L.M. Kaganovich ("Sokolniki — Gorky Park"). Despite the puffing of the steam locomotive, the landscape slowly floated past the train windows. Sukovo seemed an incredible distance from Moscow, and few people came on parents' days, even to the younger students. The remoteness united everyone. We easily accepted the leaders' suggestions. For example, the botany teacher suggested joining the collection.
}
}

%%[[page_018]]
\pageimage{page_018}{
\translate{
и оформлению гербариев для среднеклассников. За таким занятие меня сфотографировал Яша: с букетиком полевых цветов в правой руке (это оказалось её последним фото). С большим интересом мы ходили в походы по азимуту. Ту двум группам пионеров географ давал схемы разных маршрутов движения к общему сборному пункту. После тренировок на территории лагеря мы частенько уходили на весь день в увлекательные походы. После завтрака получали сухие пайки, географ присоединялся к группе, в которой преобладали новички, и в путь. Отношение к длительным походам было по-мальчишески серьёзное, у девочек тоже. Даже в тех случаях, когда выходили на знакомую местность, всё равно сверяли заданное в схеме с нашим движением, в этом был особый штурманский фарс. На сборный пункт сходились к концу дня. Пришедшие первыми занимались костром. У огня доедали продзапасы, обсуждали примечательности маршрутов, задавали вопросы, внимательно слушали руководителя кружка, который к прямому ответу обычно добавлял какую-нибудь байку. Было весело, непринужденно, но кульминацией всему была... картошка. Мы клали её на угольки, с краю костра: на пределе терпения...
}{
and preparation of herbariums for middle school students. Yasha photographed me doing this: with a bouquet of wildflowers in my right hand (this turned out to be her last photo). We went on azimuth hikes with great interest. The geographer gave two groups of pioneers diagrams of different routes to a common meeting point. After training on the camp grounds, we often went on exciting day-long hikes. After breakfast, we received dry rations, the geographer joined the group, which was mostly made up of newcomers, and off we went. The attitude towards long hikes was serious for both boys and girls. Even in cases when we went to familiar terrain, we still compared the given route with our movement, there was a special navigator's farce in this. We converged at the meeting point by the end of the day. Those who arrived first took care of the campfire. By the fire, we finished our food supplies, discussed the highlights of the routes, asked questions, and listened attentively to the club leader, who usually added some story to the direct answer. It was fun and relaxed, but the culmination of everything was... potatoes. We placed them on the coals, at the edge of the fire: at the limit of patience...
}
}

%%[[page_019]]
\pageimage{page_019}{
\translate{
Ченную картофелину с руки на руку, обжигаясь, с наслаждением пожирали королеву похода. Маршруты завершены, недогоревший костёр залит под пристальным оком руководителя, вещмешочки - за спины и домой, в Пионерский лагерь, где ждёт ужин. Если возвращались очень поздно, и на небе светились Большая Медведица и Полярная звезда, то на каких-то участках пути они были нашими путеводными звёздами. Географ интересно рассказывал также о странах, где эти звёзды видны плохо, об ориентировании на море. От насыщенности впечатлениями большого дня, вконец уставшие, мы крепко засыпали. В самом страшном сне никому не могло присниться, что некоторые из старших ребят немногим более чем через год будут пытаться выйти из немецкого окружения, используя навыки пионерских походов.
}{
Passing the hot potato from hand to hand, burning ourselves, we devoured the queen of the hike with pleasure. The routes were completed, the smoldering campfire was extinguished under the watchful eye of the leader, backpacks on our backs, and home to the Pioneer camp, where dinner awaited. If we returned very late, and the Big Dipper and the North Star shone in the sky, they were our guiding stars on some parts of the way. The geographer also told interesting stories about countries where these stars are poorly visible, and about navigation at sea. Filled with impressions of the big day, completely exhausted, we fell asleep soundly. In the worst nightmare, no one could have dreamed that some of the older guys would be trying to escape from the German encirclement using the skills of pioneer hikes in just over a year.
}
}

%%[[page_020]]
\pageimage{page_020}{
\translate{
Но мы пока в пионерском лете сложного 1940 года. Помимо школы Яша занимался в студии живописи на Сретенке. Больше всего он увлекался портретом. К сожалению, у меня не сохранился его рисунок; было бы интересно посмотреть, насколько психологичны его портреты. Мне ещё помнятся характерные выражения глаз на отдельных изображениях некоторых общих знакомых. Меня рисование никогда не привлекало. Единственный раз, в начальной школе, я со всем старанием нарисовал свою кошку, и то маме пришлось добавить зверю пушистости. Понятно, Яша много рисовал; я ходил на авиамоделирование. Мой выбор не был случайным. Даже сейчас, когда вижу летящий самолёт, провожаю его взглядом до тех пор, пока он не сольётся с глубиной неба. Конечно, авиационная романтика убита в воздушных боях, террористами. Но меня всегда восхищали, и я легко понимал физические законы, которые объединяются словом самолёт, поэтому школьная физика была моим любимым предметом. Кроме того, мой старший брат Гриша работал на авиационном заводе и учился на вечернем отделении Московского авиационного института; у нас бывали увлекательные беседы. После заключения Пакта о ненападении между фашистской Германией и СССР (август 1939 года) Гриша иногда приносил мне хорошо иллюстрированные книги.
}{
But we are still in the pioneer summer of the difficult year 1940. Besides school, Yasha attended a painting studio on Sretenka. He was most interested in portraiture. Unfortunately, I did not keep his drawing; it would be interesting to see how psychological his portraits were. I still remember the characteristic expressions of the eyes in some images of our mutual acquaintances. Drawing never attracted me. The only time, in elementary school, I drew my cat with all my effort, and even then my mother had to add fluffiness to the animal. Clearly, Yasha drew a lot; I attended model aircraft building. My choice was not accidental. Even now, when I see a flying airplane, I follow it with my eyes until it merges with the depth of the sky. Of course, aviation romance has been killed in air battles, by terrorists. But I was always fascinated, and I easily understood the physical laws that are combined in the word airplane, so school physics was my favorite subject. In addition, my older brother Grisha worked at an aircraft factory and studied in the evening department of the Moscow Aviation Institute; we had fascinating conversations. After the signing of the Non-Aggression Pact between Nazi Germany and the USSR (August 1939), Grisha sometimes brought me well-illustrated books.
}
}

%%[[page_021]]
\pageimage{page_021}{
\translate{
рованный технический журнал германской авиационной промышленности. Тогда я впервые увидел «Юнкерсы», «Мессершмиты» и узнал некоторые их технические характеристики. Немцы, очевидно, невысоко оценивали советскую авиацию и её защиту, если присылали журнал, из данных которого специалисты могли многое понять, а может быть, это было элементарным запугиванием.
}{
A technical journal of German aviation industry. At that time, I first saw the Junkers and Messerschmitts, and learned some of their technical characteristics. The Germans apparently did not value the Soviet aviation and its defense, if they sent a journal from which specialists could understand a lot, or maybe it was just a threat.
}
}

\translate{
Интересное пионерское лето вовсе не отрывало нас от больших событий в стране и мире, а было их навалом. К лету 1940 года войска фашистской Германии через Бельгию, обойдя с севера французскую оборонительную линию Мажино, беспрепятственно вторглись во Францию и 14 июня без боя вошли в Париж. В том же июне Советский Союз присоединил Эстонию, Латвию, Литву, Бессарабию.
}{
An interesting pioneer summer was not detached from big events in the country and the world, but was a flood of them. By the summer of 1940, the German army, bypassing the French defensive line of Maigno from Belgium, invaded France without any obstacles and on June 14, without a fight, entered Paris. In the same month, the Soviet Union joined Estonia, Latvia, Lithuania, Bessarabia.
}

\translate{
В молодёжных научно-популярных изданиях было много интересной, увлекательной информации из разных областей знаний. Помнится, в журнале «Наука и жизнь» читал статью о надёжности линии Мажино. Это было, когда Советский Союз вёл мощную, разоблачающую фашизм пропаганду. В подобных статьях ненавязчиво говорилось о военной мощи наших потенциальных солидных союзников по предстоящей борьбе с фашизмом. Но такой потенциал — Францию немцы положили на лопатки с первого захвата. У нас, школьников, были и свои вехи определения перемен: наши животы надрывались от смеха на про-
}{
  In the youth scientific-popular publications there was a lot of interesting, exciting information from different areas of knowledge. I remember reading an article in the magazine "Science and Life" about the reliability of the Maigno line. This was when the Soviet Union was conducting powerful, exposing fascism propaganda. In such articles, it was subtly said about the military might of our potential solid allies in the upcoming struggle with fascism. But such potential - France was put on the feet of the Germans from the first capture. We, schoolchildren, also had our milestones of definitions of changes: our bellies were bursting with laughter on the
}

%%[[page_022]]
\pageimage{page_022}{
  \translate{
смотрах фильмов Чарли Чаплина «Новые времена» и «Огни большого города», мы уже предвкушали веселье на анонсировавшемся фильме Чарли Чаплина «Диктатор»; вдруг об этом фильме замолчали, не стало в газетах антифашистского содержания карикатур Бор. Ефимова, Кукрыниксов (аббревиатура фамилий трёх известных советских художников Куприянова М.В., Крылова П.Н., Соколова Н.А.), не стало и других агитационных антифашистских материалов. Но вскоре в центральных газетах появилась фотография Иоахима Риббентропа, министра иностранных дел фашистской Германии, приехавшего на подписание Пакта о ненападении, и впрямь, наступили новые времена.
}{
  We watched Charlie Chaplin's films "New Times" and "Lights of the Big City", and we already looked forward to the fun on the announced Charlie Chaplin film "Dictator"; suddenly, the film about it was silent, there were no anti-fascist content cartoons by Bor. Efimova, Kukrynikovs (abbreviation of the surnames of three famous Soviet artists Kupriyanov M.V., Krylov P.N., Sokolov N.A.), there were no other anti-fascist propaganda materials. But soon in the central newspapers there appeared a photograph of Joachim Ribbentrop, the foreign minister of Nazi Germany, arriving to sign the Non-Aggression Pact, and indeed, new times had come.
}
}
\end{paracol}