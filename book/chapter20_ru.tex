\pageimage{page_300}
\label{300-1}
На тридцатый день после похорон отмечается Хаскара — день памяти (от слова зиккарон — память). На этот заключительный обряд прощания, как правило, съезжаются в основном самые близкие люди, хотя о хаскаре сообщается всем. Традиционным стало то, что ко дню Хаскары заканчивается изготовление надгробного камня и его установка на могиле. Женя проследил, чтобы традиция не нарушилась.

Как и на предыдущих обрядах похорон, была прочитана молитва, все были под впечатлением появления Инниного имени на надгробной каменной плите. Все горестно созерцали этот факт. Постояв немного среди нас, раввин обратился ко мне на идише и сказал: «Эршт кент ир кимен аэр ибер аюр» (теперь вы можете прийти сюда через год). Этой фразой раввин как бы завершил месячный период ритуалов похорон Инны.

\pageimage{page_301}
\label{301-1}
И я закончил письмо моей умершей Инуле; письмо адресовано также всем близким как память об Инне. Предложенный здесь стиль изложения - не моя выдумка, а результат желания противостоять гигантскому давлению обрушившейся на меня беды. Я имею в виду смерть Инны. Её смерть - смертельный удар по мне. Однако пытаюсь выстоять. Это выражается в том, что я неизменно, по-прежнему глубоко всегда любил и люблю Инулю. Её смерть не погасила ни одного чувства. Я разговариваю с ней, советуюсь, слышу её реплики. Она не уходит из моего сознания. Я не даю этому совершиться, поэтому, считаю, я вправе напрямую сказать ей: - Инулик, родная! Столько, сколько будет существовать моё сознание, столько буду тебя любить и дорожить памятью о тебе. 
  15.9.2015 Фима

\pageimage{page_302}
\label{302-1}
Как понять? Письмо я закончил 15 сентября; Инна умерла 15 июня. Прошу поверить, что совпадение чисел 15 совершенно произвольное, у меня и мысли не было, чтобы совместить числа. Но, очевидно, удивлю Вас фактом: в ночь на 15 сентября мне приснилась Инна. Она была в осеннем пальто и в шляпе с широкими полями, которую мы купили после нашей женитьбы. Инна улыбалась и смотрела мне в глаза. Сон был очень чёткий, ясный. Что бы значило сведение трёх взаимосвязанных явлений к одному числу 15? Впрочем, кто бы что бы ни сказал, грусть, тоска по Инне будет всегда со мной. Обратить внимание на число мне посоветовал Женя. 
 22 сентября 2015.

\pageimage{page_303}
\label{303-1}
Вместе с вами  яещё раз прочитал письмо. Оно по-прежнему созвучно с моим состоянием тоски по Инне. Тоска угнетает, убеждает в беспомощности, но она же вынимает из памяти былое; тоска питается им. Но былое разное по эмоциональному настрою, содержанию. Есть и радостное. Вернусь к тем страницам, которые я написал до вашего прочтения моего письма. Возвращаюсь к рассказу об Инне.

Итак, 1945 год. Москва. Мне 21 год. Инне 20. Мы ещё не знакомы, но триуф  победы над фашистской Германией кружит головы и рождает надежды; после больших людских потерь за время войны быстрее стали завязываться знакомства, и жизнь кипела.

Я в своей личной жизни придерживался принципов, с которыми вы уже...

\pageimage{page_304}
\label{304-1}
знакомы.
Приближался первый послевоенный новогодний вечер в моём Полиграфическом институте. Это 30 декабря 1945 года. Предшествовавшие вечеру дни были суматошные и интересные для меня: сами студенты украшали зал, готовили номера, сложился приличный самодеятельный оркестр, были и свои декламаторы, пианисты.
Сложилось так, что именно в эти дни моему брату Грише его соседка по дому Сара Мазина предложила познакомить меня с её родственницей студенткой. Договорились. Вдруг оказалось, что в армейский отпуск приехал брат девушки, и предновогодние дни она проведет с братом.
30 декабря я пошёл на институтский новогодний вечер, получил огромное удовольствие, был бы до конца, да трамвай лимитирует. С запасом времени я стал медленно спускаться по парадной лестнице из Актового зала в вестибюль. Внизу тоже было людно, но мой взгляд остановился на девушке,

\pageimage{page_305}
\label{305-1}
которая всем своим естеством источала скромность и обаяние. Как-то сразу я почувствовал особенности моей незнакомки. Все угаданные штрихи и детали сложились в действие: меня сильно потянуло к ней, к её обаянию. Обилие, а главное значение впечатлений пробудило во мне особую решительность. Я сам себе сказал: «Эта девушка будет моей женой».

20. Она и её подруга беседовали с моей однокурсницей Жанной Савкиной. 
Лучшего момента не найти! Я подошёл, обобщающим новогодним приветствием поздравил их, и Жанна представила мне своих подруг — Зину и Инну.

Редкое в то время имя Инна в моём воображении мгновенно слилось с образами впечатлений и ещё больше облагородило ее образ.

\pageimage{page_306}
\label{306-1}
Здороваясь, я обратил внимание на быстроту реакции Инны на обстоятельства: она раньше меня протянула левую руку при рукопожатии. Многие терялись от непривычки. Впоследствии Инна говорила, чтобы я вообще не протягивал руку, а приветствовал только голосом, что и есть по сей день. Здороваясь, мы невольно посмотрели в глаза друг друга. В её глазах я разглядел ум и спокойствие. Девушки тоже собирались домой, и мы все вместе вышли на Садовое Кольцо, прошлись, поговорили и направились к станции метро.

\pageimage{page_307}
\label{307-1}
Прогулка подходила к концу. Я судорожно думал над тем, как в ближайшие дни увидеть Инну. В помощники взял себе пропагандистскую работу МГУ. В корпусе напротив манежа по выходным дням читались научно-популярные лекции по отраслям наук. Было очень познавательно и интересно. Я предложил Инне пойти со мной в ближайший выходной день. Она, после паузы, ответила согласием. Я был счастлив! Но был сдержан. Мы замечательно провели первый совместный выходной день. Были на лекции, потом решили пройти пешком от МГУ до Маросейки, там жила Инна. Хорошая и долгая беседа получилась задушевной.

\pageimage{page_308}
\label{308-1}
В таких беседах задушевность возникает, можно сказать, всегда, потому что хочешь понять духовный мир понравившегося человека, в том числе и такого, с которым готов соединить свою жизнь. Через сказанное, через рассуждения раскрывается логика мышления. И себя раскрываешь таким же образом. Встречались мы довольно часто, но по настрою чувств, ожиданий хотелось большего. Расставаясь, всегда договаривались о следующей встрече.

\pageimage{page_309}
\label{309-1}
Речь уже шла не об организации встречи, о том, как сделать её интересной. Москва в области культуры всегда была представительным городом. У многих был свой памятный уголок. Для нас это был лекционный зал, где мы впервые и осознанно встретились. Мы там бывали не часто, но всегда с чувством трепета. Именно там я  преподнес Инне первый букет цветов. Неожиданным образом повлиял на наши отношения Сара Мазина. Уже далеко за новогодние дни она решила довести замысел знакомства до конца. Позвонила родственнице и из долгого телефонного разговора составился мой портрет. Мы обрадовались необыкновенно: судьба дважды сводила нас (институтский вечер, Сара). Нам суждено быть вместе.
