\pageimage{page_179}
\label{179-1}
Taкой процедурой как бы обеспечивалась кошерность. Второй этап - маца. Городская власть разрешила печь ее в какой-либо пекарне на окраине города; в предпасхальные дни к этой пекарне стекались тысячи людей; приезжали утром, уезжали вечером; мацу упаковывали в новые белоснежные наволочки, дома клали ее на крышки шкафов, буфетов, где не лежали другие продукты. Очереди за мацой состояли, в основном, из женщин, а приезжавшие с ними дети весь день играли и веселились. Когда в дом приносили мацу, создавалось настроение праздника. Пасхальный сейдер проходил, конечно, не так, как я потом увидел в Израиле; стол был очень скромный, но всегда были фаршированная рыба и бульон. Пасхальную Агаду не читали, но папа относительно подробно ее пересказывал; на Песах привозили бабушку, приглашали дядю Яшу с семьей. Главным было не соблюдение формы праздника, а понимание его значения, это объединяло нас, семью. Никто не мог предположить, что этот праздник в 1939 году был последним, когда вся наша семья была в сборе.

\pageimage{page_180}
\label{180-1}
После тех самых пасхальных дней мама сказала папе, что все те дни ей трудно было глотать мацу и другую твердую, 
неразмоченную пищу. Родители пошли к доктору, и с этого визита в поликлинику всё началось: врачи установили диагноз — рак пищевода. Тогда начало внедряться радиоактивное облучение раковых больных; папино начальство уважительно и сочувственно относилось к нему, оно сумело добиться разрешения моей маме пользоваться услугами закрытой поликлиники. На лечение мама ездила раз в несколько дней, но оно обессиливало её и не помогало; время шло, мама явно слабела. Однажды, в выходной день тетя Саша позвала к себе маму (по папиной просьбе); папа пригласил нас троих сесть за стол; он был удручен, манжеты его рубашки были засучены, его сильные руки, сжатые в кулаки, лежали на столе, голова наклонена — это плохой признак; меня впервые пригласили к серьезному разговору взрослых, какое жуткое для меня совпадение: меня впервые признали взрослым, когда речь пошла о конце жизни мамы (даже сейчас это трудно воспринять).

\pageimage{page_181}
\label{181-1}
Папа говорил размеренно, напряженно, в основном на идише, так ему легче было выразить оттенки мысли: после окончания курса лечения врач сказал, что мама безнадежно больна, незачем таскать ее по врачам, дайте ей спокойно умереть. Молчание, довольно долгое. Я сидел рядом с Гришей, мое волнение он, видимо, принял за желание что-то сказать, он своим коленом придавил мою ногу и сжал губы как знак молчания. Но я и не пытался говорить, слишком сильны были мои переживания. Папа не показал своего волнения, когда спазмы в горле прошли, он продолжил: доктор знает, о чем он говорит, но мы не должны с этим смириться, надо сделать все, чтобы продлить мамину жизнь; если вскоре случится так, как сказал доктор, то наша семья в третий раз уже не поднимется. Он сказал, что будет пытаться получить консультации светил медицинского мира того времени — профессора Кончаловского и профессора Юдина. Я впервые услышал эти фамилии.

\pageimage{page_182}
\label{182-1}
Последний месяц школьных каникул 1939 года я провёл в пионерском лагере; встречал меня Давид. Мы поехали к нему на работу, в Фрунзенский райком комсомола, который находился в одном из переулков, примыкающих к улице Кропоткина. Через несколько часов мы вышли на Кропоткинскую, в магазине «Цветы» Давид купил букет тёмных астр и сказал мне: «Подаришь маме». Домой мы приехали на закате солнца. Мама полулежала на диване, прикрыв ноги старым деревенским платком; за день мама намоталась по хозяйству и прилегла, как обычно, отдохнуть, но за время обнаружившейся болезни мама чаще делала это и днём, поэтому я не придал значения факту, что после почти месячной разлуки мама встречает меня лёжа. Произошедшее через мгновение

\pageimage{page_183}
\label{183-1}
новое осознание — потрясение навсегда запало в мою душу.
Я подлетел к маме, плюхнулся на край дивана, цветы сунул ей в плечо, прижался к лицу и ощутил непривычное — огрубевшую кожу, скулы. Мама, в свою очередь, ладонями сжала мои уши, приподняла мою голову так, что наши глаза встретились, как никогда близко за 26 дней моего отсутствия (столько длилась смена в пионерском лагере). Мама, которая была изящной невысокой фигурой, очень похудела — болезнь заметно ложилась на её облик: в меня смотрели уменьшившиеся, потерявшие цвет и теплоту глаза. Я приковался к ним, смог только сконцентрировать свой ответный

\pageimage{page_184}
\label{184-1}
взгляд на левом мамином глазе, точнее, на неподвижном зрачке; через него я улавливал тонкую нить проницательного, тревожного, даже строгого взгляда, идущего из непостижимых глубин, которые мама, очевидно, ощущала в себе; у неё начали проявляться другие понимания контактов с близкими, она часто молчала; уходя из живого мира, мама, опять же очевидно, хотела что-то оставить в себе. Я пытался до конца понять содержание её многосложного взгляда, напряжённо вглядывался в зрачок, а что она увидела в моих глазах? Возле переносицы, в углубившихся ямочках от впавших глаз появились дужицы слёз. Я подтянул платок с маминых ног, промокнул им мамины слёзы и хотел было сказать, чтобы она не плакала, всё будет хорошо, но я не мог врать: хорошо уже не будет; выступили новые слезы,

\pageimage{page_185}
\label{185-1}
ни один мускул маминого лица не дрогнул, она плакала мольча и продолжала смотреть в мои глаза. Так плачут в состоянии наивысшего душевного и нервного напряжения. Мама прощалась со мной.
Через сложившиеся знакомства в медицинском мире папа получил согласие профессоров на консультации, но, подчеркну, только в сентябре. Тяжёлое впечатление от рассказанной вам встречи с мамой не увязывалось в моей душе с возникшей в семье надеждой после согласия профессоров. Объяснить настроение семьи можно только тайным желанием благотворного чуда и исторически традиционной верой в особую всевозможность человека высочайшего звания, титула, должности. Между тем, мама слабела,

\pageimage{page_186}
\label{186-1}
 возникла проблема срочной медицинской помощи на дому, особенно ночью: телефон-автомат находился далеко, да не было уверенности в том, что телефонная трубка цела. Добавлялось и другое: наш новый многоквартирный дом построили в низинке, куда стекала дождевая вода, поэтому нижние ступеньки крыльца заливало, жильцам приходилось прыгать по камням и ящикам, но профессора прыгать не станут. Приближался осенний (и многозначительный с точки зрения истории) конец августа - сентябрь; всё медико-бытовые неувязки решились предложением дяди Фимы и тёти Лии переехать маме к ним на какое-то время.