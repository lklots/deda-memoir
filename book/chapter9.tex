% continued from page_146
Итак, коллективизация, начавшаяся после XIV съезда партии (1927 год), проходила в условиях и по методам, противоположным...

%%[[page_147]]
положением В.И. Ленина по этой проблеме; кроме того, отказ от экономических связей с крестьянством и переход к административно-насильственным методам управления отраслью привел к упадку сельского хозяйства, которое было восстановлено и развито за годы НЭПа. Разгром НЭПа коснулся и города. Зимой 1928/29 моего папу лишили права на магазин, весь товар конфисковали, отца арестовали, судили и сослали на лесоповал в Архангельскую область. Забота о семье легла на мамины плечи — женщины инициативной, но сломленной новой волной до конца её недолгой жизни. Брат Ефим и его жена Лена Маркова предложили маме приютить к ним моего старшего брата Гришу (16 лет); так он стал учеником токаря на заводе "Орг-металл" в районе Воробьевых (Ленинские горы).

%%[[page_148]]
148

ских) гор в Москве; вскоре таким же учеником токаря стал Давид (14 лет). Осенью 1929 года мама с плотом со мной приехала в Москву; мы четверо жили в отдельной комнате в квартире дяди Фимы. Мама начала работать ученицей ткачихи на гардинной фабрике на Плющихе, в отдельную небольшую комнату поселили бабушку, но когда женился дядя Яша, молодым негде было жить; их поселили в эту комнатку; бабушкину кровать перенесли в столовую. Здесь у них родился 2 мая 1933 года Шурик (Александр Яковлевич Величанский); о свадьбе дяди Яши и том, как Шурик чуть было не угодил в минское гетто, вы уже знаете. В ту же квартиру приехал из заключения мой папа. В третьей комнате размещались дядя Фима и тетя Лия; их дочка Розочка, моя младшая двоюродная сестра, спала в столовой, как и бабушка. От тесноты спасала большая кухня, на которой была огромная плита, ее не топили, но использовали как стол для примусов и керосинок; в выходные дни кухня служила семейным клубом.

%%[[page_149]]
Подробно о населении квартиры пишу для того, чтобы показать вам, как высоки были понятия родства, взаимопомощи. Впрочем, если выходной день выпадал в пятницу, устраивали субботний ужин, и вся квартира собиралась за большим обеденным столом; накрывали белоснежную скатерть, выставляли лучшую посуду, создавалась атмосфера праздника; на столе бывало довольно скромно - это были годы карточной системы. Бабушкино место было за серединой стола, у края - подсвечники, молитвенник, спички; когда все рассаживались, бабушка привычным движением поправляла на голове тонкую шаль, вставала, зажигала свечи и начинала громко и внятно читать субботнюю молитву: когда она потом переходила на понятный идиш, ясно было, с какой мольбой и страстностью она просила Бога помочь ей и нам пережить свалившиеся на всех трудности. В бабушкиных словах содержалась...

%%[[page_150]]
150

Жилась такая убежденность, что она зарождала веру и надежду. Наша жизнь в Москве начиналась с серых будней; утром я просыпался в пустой комнате; мама и братья с пораньше уходили на работу, я ни разу не слышал, как они вставали; дома была только бабушка и очень редко — Розочка; бабушка отводила меня на детскую городскую площадку при клубе им. Зуева; здесь меня встретили ребята недружелюбно, потому что меня звали Фроя, и они начали дразнить меня моим именем. Конечно, в некоторых обидных случаях я пытался отстоять свое имя кулаками, но получал в ответ хорошие тумаки; драться я перестал, но и дразнить перестали. Мучительные хождения на площадку вскоре закончились: меня отдали в школу. В Москве я начал быстро взрослеть: впервые подрался, братья научили меня самостоятельно.

%%[[page_151]]
переходить улицу с трамвайным движением; в Москве тогда еще не было размеченных переходов и пешеходных светофоров. Бабушку надо было освобождать от хождений со мной; в недавнем Ульяновске бабушка была вся моя, здесь - нет, из школы домой я не торопился; в начальных классах учился плохо, в нулевке и в первом классе была очень злая учительница. Мама приходила к вечеру уставшая и на короткое время ложилась передохнуть; я стоял у ее изголовья и ждал, когда она откроет глаза, потом она меня крепко обнимала и говорила что-то ласковое. Гриша и Давид приходили позже мамы, помимо работы на заводе они учились в ФЗУ (фабрично-заводское училище); там они повышали свою профессиональную квалификацию и получали общее образование на уровне средней школы.

%%[[page_152]]
появление в квартире молодой четы — дяди Яши и тети Берты внесло струю оптимизма и даже некоторого веселья; при всех сложностях жизни и проявлениях нервозности по отдельным поводам, дядя Яша оставался остроумным и веселым, он легко находил темы для шуток; тетя Берта была выдержанным спокойным человеком, думая о чем-то она всегда напевала, я любил вслушиваться в ее мурлыкание; появившийся крошка Шурик стал центром внимания. Утра выходных дней посвящались ему, его таскали, тискали, качали, кончилось тем, что я выронил его из рук и он упал на пол; моя мама испугалась за него, и тетя Берта сказала примерно так: «Ну, не волнуйтесь, ничего не случилось». В этой квартире дядя Яша с семьей прожили года два, потом он купил небольшую квартиру в доме в Шебашевском переулке, перестроил ее в комнату, пристроил небольшую кухоньку и сени. Там перед войной родилась Туля.

%%[[page_153]]
В один из зимних дней конца 1933 или начала 1934 года мама сказала мне, что мы пойдем встречать папу. Дядя Фима в то время был на высокой должности на крупном московском заводе, у него дома был служебный телефон, в те годы это говорило о многом; видимо, папа сообщил о приезде. Мы вышли на длинный тротуар, вскоре мама показала на едущего навстречу папу; я помчался к нему на привязанных к валенкам деревянных санках - колодках, я подлетел к нему, протянул руки вверх, он поднял меня на уровень своего объятия, подошла мама... Никогда не забуду нас на тротуаре, меня поразил папин вид: поверх шинели, до колен был намотан шарф, а на ногах - валенки.

Через небольшое время папа устроился мелким служащим в какую-то контору, видимо, помог дядя Фима. Наконец, семья была в сборе, все работали, наступило время заиметь свое жилье. Няня Яша подыскала нам неподалеку отдельную комнату и кухню.

%%[[page_154]]
Наш адрес был неочевидный: Князевский переулок, дом 3. Откуда "Князевский"? В облике домиков и хибар переулка я ничего княжеского не находил; возможно, что на этом участке земли когда-то жили княжеские холопы, поэтому сохранилось название "Князевский". Наша семья, все пять человек, поселилась в одной девятнадцатиметровой комнате. Мы жили в обстановке бедного духовного единства; это проявлялось поздними вечерами, когда все уже были дома. Материально жили очень скромно; все зарабатывали мало, продукты выдавали по карточкам. Папа, мама (она оставила работу) были прикреплены к продуктовой лавке на углу улицы Кирова напротив ресторана. Помню достраивавшуюся станцию метро "Пролетарская". Метрополитен еще не функционировал. В субботние выходные дни родители брали меня в магазин, там толпы, очереди по немереным талонам. Зачастую на определенной страничке предлагали книжки или слекнетики.

%%[[page_155]]
155

ный продукт, по другому номеру выдавали что-то другое и т. д. Устав очереди, мы добирались домой все Ру. Я бывал счастлив, когда мама давала мне кусок хлеба с маргарином. Мама доступными ей дешевыми средствами создала уютный дом; поздним вечером, когда над небольшим обеденным столом светился оранжевый абажур, становилось еще уютнее, все сидели за столом, больше деваться было некуда, кто-то рассказывал о событиях прошедшего дня, иногда папа читал вслух что-то из еврейской классики, или в переводе на идиш что-либо из русской литературы. В память о папе и с тех вечеров я привез в Израиль мраморный экземпляр поэмы А.С. Пушкина "Цыганы", в переводе на идиш (изд. "Сгенск", 1934 г., цена Эксп.). Наши единственные соседи-хозяева домика оказались очень симпатичными людьми: когда хозяин дома Константин Казимирович Семашко узнал, что мой папа родился и долго жил в Польше, он обрадовался и перешел с папой на польский язык, он проявил земляческие чувства в том, что подарил нам целую грядку в своем огороде. Мама разделила грядку на три части: укроп, петрушка, помидоры.

%%[[page_156]]
126
А он руководил всеми сельхозработами. Когда наступал период прополки (меня он звал курговы), он учил меня отличать сорняки от нужных ростков. Самое интересное было, когда пилили дрова. У Козлевых была такая же жилплощадь и легкоплита, как у нас, поэтому дрова у нас были общие. Папа, как опытный столяр, сколотил новые козлы и привел в порядок двуручные пилы. Отцы семейств пилили дрова, сыновья кололи, во дворе складывали в поленницы. Через несколько дней, когда дрова подсыхали, ребята перетаскивали их в сарай. В такие дни во дворе было весело: мерно и хорошо колун разваливал бревно, не смолкали шутки. Солнце светило. Жена Козлева, тетя Саша, была тихая работящая женщина. Через несколько лет, когда начали строить метрополитен и все домики Князевского переулка снесли, мы опять встретились.
